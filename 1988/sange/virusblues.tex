\documentclass[a4paper,11pt]{article}

\usepackage{revy}
\usepackage[utf8]{inputenc}
\usepackage[T1]{fontenc}
\usepackage[danish]{babel}


\revyname{DIKUrevy}
\revyyear{1988}
\version{0.1}
\eta{? minutter}
\status{Ikke faerdig}

\title{Virus Blues}
\author{?}
\melody{Virus Blues}

\begin{document}
\maketitle

\begin{roles}
\role{S}[?] Sanger
\end{roles}

\begin{song}

\sings{S}
Den er gået ned,
angstens  kolde  sved,
hele kataloget klistret til.
Skærmen flyder ud,
der findes ingen gud,
mørke kræfter driver deres spil.
Det er selve disken, der trues,
derfor synger jeg virus-blues.

\sings{S}
Drevet snurrer vildt.
Fyrre timer spildt.
Afleveres klokken el've prik
Hele mit program,
skide isenkram,
otte hundred liniers slam, der nu er væk.
Jeg havd lige fundet den allersidste lus
Nu har jeg kun min virus-blues.

\sings{S}
Den satans virus,
har taget min sjæl,
vogt jer for virus-invasion.
Det' djævlens værk,
jeg, som var så stærk,
tager nu tælling til en million (og 48.575)
Det' slut med at gi flere interviews
I må nøjes med min virus-blues.

\sings{S}
Bare læg mig ned
i ubemærkethed
glem kun hvad jeg fabled om engang:
``Færdig på normeret tid'',
``uden særlig flid''.
Nu må jeg ta dat-0 endnu en gang.
Et helt år til med ``false'' og ``true'''s.
Derfor hulker jeg virus-blues.

\end{song}
\end{document}
