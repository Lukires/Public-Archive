\documentclass[a4paper,11pt]{article}

\usepackage{revy}
\usepackage[utf8]{inputenc}
\usepackage[T1]{fontenc}
\usepackage[danish]{babel}


\revyname{DIKUrevy}
\revyyear{1988}
\version{1.0}
\eta{$n$ minutter}
\status{Færdig}

\title{3 små grise}
\author{Peter Johansen}
\melody{Shu-bi-dua: ``De 3 små grise''}

\begin{document}
\maketitle

\begin{roles}
\role{S}[] Sanger
\end{roles}

\begin{song}
Vi er / tre studenter og vi / har det svært /
bange for den næste faste / dead line /
retter i programmet men vi / synes det er sært, det /
kører bar' løs men komme r/ Ingn vegn' rap- /
portopgaven p ådat1 den / løser vi nok bedst i mor- /
gen / rap /
portopgaven på dat1 den / løser vi nok bedst i mor- /
gen / morgen / morgen / morgen.  Da vi /

fik den stillet så den / enkel ud /
solen var på himlen da vi / fik den /
tiden næsten gået, vi er / tæt på sammenbrud og /
tordenvejr er vejrud / sigten. Rap /
portopgaven på dat1 den løser vi nok bedst i mor- /
gen / rap /
portopgaven på dat1 den / løser vi nok bedst i mor- /
gen / rap /
portopgaven på dat1 den / løser vi nok bedst i mor- /
gen / morgen / morgen / morgen.  Næste /

gang så kør vi efter / arbejdsplan /
den er nemlig større end den / første /
starter gruppen op imens vi / læser en roman, i /
solskin i græs foran / Ørsted. Rap /
portopgaven på dat1 den / løser vi nok bedst i mor- /
gen / rap /
portopgaven på dat1 den / løser vi nok bedst i mor- /
gen / rap /
portopgaven på dat1 den / løser vi nok bedst i mor- /
gen / rap /
portopgaven på dat1 den / løser vi nok bedst i mor- /
gen / morgen / morgen / morgen.  Onsdag /

torsdag, fredag, lørdag, søndag, / mandag med kan vi /
sammen spille A D / og D'e /
tirsdag går det løs, så skal vi / styrte afsted, og /
bare vi nu kan holde / til de' de kan /
stille den eksamen de vil / vi ved vi /
altid kan bestå en til / vi ved sgu /
godt / at dat1 står på spil / / / /
\end{song}

\end{document}

