\documentclass[a4paper,11pt]{article}

\usepackage{revy}
\usepackage[utf8]{inputenc}
\usepackage[T1]{fontenc}
\usepackage[danish]{babel}


\revyname{DIKUrevy}
\revyyear{1988}
\version{1.0}
\eta{$n$ minutter}
\status{Færdig}

\title{Kandidaten}
\author{?}
\melody{Pierre-Alexandre Monsigny - ``Le Deserteur''}

\begin{document}
\maketitle

\begin{roles}
\role{S}[] Sanger
\end{roles}

\begin{song}
  \sings{S}
  Min kære hr. Dekan
  Jeg kommer for at klage
  Ja -- har man nu hørt mage
  Jeg er blev't kandidat
  Det måtte ikke ske
  Jeg sagde pænt og høfligt "`NEJ!"'
  da Institutet spurgte mig
  ved mødet sidste jul.
  De skrev mig på hos Jens
  Jeg kom der godt nok ikke
  men de var mere kvikke
  og gav mig blot kredit
  og så var løbet kørt
  beviset kom forleden dag
  jeg havde bestået sidste fag
  var blevet kandidat

  Så var det altså slut
  de femten gode studieår
  den hyggelige kaffetår
  om morg'nen klokken tolv
  Tilbage er revy'en
  og festerne hvert halve år
  Jeg holder op - men de består
  Nej -- det kan ikke gå.
  Min kære hr. dekan
  det er jeg ikke nødt til
  for jeg blev ikke født til
  at være kandidat
  Et fyrre timers job
  skal det vær' hvad der venter mig
  så går jeg hjem og hænger mig
  men du kan sige STOP

  (RÅBES:) gør mig til licenciat
  åh gi' mig blot et halvt kontor
  men helst mit eget skrivebord
  så vil jeg blive glad
  en lille flot P.C.
  med farveskærm og HARDDISK til
  en mus - måske et enkelt spil
  så er jeg ovenpå
  og efter fire år
  som et af DIKU's esser
  så kan jeg bli' professor
  og gøre hvad jeg ivl.
  Min kære hr Dekan
  nu må du ikke skuffe mig
  der må da være plads til mig
  ved dette institut.
\end{song}

\end{document}

