\documentclass[a4paper,11pt]{article}

\usepackage{revy}
\usepackage[utf8]{inputenc}
\usepackage[T1]{fontenc}
\usepackage[danish]{babel}

\revyname{DIKUrevy}
\revyyear{1988}
\version{0.1}
\eta{? minutter}
\status{Ikke færdig}

\title{Datalogen}
\author{Niels Ull Jacobsen}
\melody{Eddie Skoller: ``Blyanten''}

\begin{document}
\maketitle

\begin{roles}
\role{S1}[] Mandlig sanger
\role{S2}[] Kvindelig sanger
\end{roles}

\begin{song}
\sings{S1}Kan han kode Fortran?
Nej det kan han ikke.

\sings{S1}Kan han rapportere?
Nej det kan han ikke.

\sings{S1}Kan han skrive COBOL?
Nej det kan han ikke.

\sings{S1}Kan hans kerner køre?
Kun i teorien.

\sings{S1}Kan han skrive brugervejledninger?
Nej det kan han ikke.

\sings{S1}Kan han skrive vers med versefødder?
Nej det kan han ikke.

\sings{S1} Kan han \dots

\sings{S2} Nej, Nu kan det være nok. Nu har der været fire han-vers.
Nu må vi have et hun-vers.

\scene{Den mandlige sanger (Kigger på en lap papir)}

\sings{S1}Hmm. Du vil ikke synes om det.

\scene{S2 svinger knytnæven.}

\sings{S2}Du vil ikke synes om at lade være.

\sings{S1}Får hun på den dumme?
Nej kun alt for sjældent.

\sings{S1}Kan han tage sit bifag?
Nej det kan han ikke.

\sings{S1}Kan han blive færdig?
Nej det kan han ikke.

\sings{S1}$\parallel \dots \parallel$
Kan han gå i løkke
$\parallel \dots \parallel$\\

\end{song}

\end{document}
