\documentclass[a4paper,11pt]{article}

\usepackage{revy}
\usepackage[utf8]{inputenc}
\usepackage[T1]{fontenc}
\usepackage[danish]{babel}


\revyname{DIKUrevy}
\revyyear{1988}
\version{1.0}
\eta{? minutter}
\status{Færdig}

\title{Hvem sidder der bag Skærmen}
\author{Jacob Marquard}
\melody{``Hvem sidder der bag skærmen''}

\begin{document}
\maketitle

\begin{roles}
\role{S}[?] Sanger
\end{roles}

\begin{song}

\sings{S}
Hvem sidder der bag skærmen
og taster dagen lang,
bag femten kopper kaffe
og synger denne sang?
Det er såmænd Karl Koder
Ak intet har han lært
at skrive go'e rapporter
er nemlig temlig svært

\sings{S}
Han banede for andre
den vanskelige vej
Skrev masser af programmer
``Det er lige noget for mig''
og resten af rapporten
den skrev de andre nok.
Karl dumped til eksamen
han blev nummer sjok.

\sings{S}
At kode i assembler
kan være ganske flot,
men i stystemarbejde
der er det ikk' så godt.
En stakkels mand fra ``klatten''
i Lyngby han blev skør
på grund af databasen
som Karl fik til at kør'.

\sings{S}
- Mikroprogrammering
det var det rene svir,
Karl koded' hele natten
fik aldrig tid til lir.
Men med lidt tekstbehandling
så ku' han skrive hex.
At se programmet køre
er li'så godt som sex.

\sings{S}
Karl Koder gik og syntes
at livet var lidt surt,
men så på vandreturen
fik han et råd af Kurt.
Nu har han nok en chance
for det er ganske vist.
Nu koder Karl kun Unix
så alt blev godt til sidst.

\end{song}
\end{document}
