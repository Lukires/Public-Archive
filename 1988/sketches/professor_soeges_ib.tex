\documentclass[a4paper,11pt]{article}

\usepackage{revy}
\usepackage[utf8]{inputenc}
\usepackage[T1]{fontenc}
\usepackage[danish]{babel}


\revyname{DIKUrevy}
\revyyear{1988}
% HUSK AT OPDATERE VERSIONSNUMMER
\version{1.0}
\eta{$n$ minutter}
\status{Færdig}

\title{Professor søges Ib}
\author{CKH, NCKH}

\begin{document}
\maketitle

\begin{roles}
  \role{A}[] Studerende
  \role{C}[] Studerende
\end{roles}

\begin{sketch}

  \says{A}[står og læser på opslagstavlen] Jeg forstår det altså ikke
  ,dette er et bestyrelsesmødereferat fra den 2/1, i hvilket de vil
  slå en professorstilling op med ansøgningsfrist den 4/1.

  \says{C} Det er der da intet odiøst i - hvis man vil være professor,
  så må man være hutig i vendingen.

  \says{A} Jah - det ka' jeg da godt se - men 2 dage?  Det vil da også
  gøre det urimeligt svært for udenlandske ansøgere.

  \says{C} Nej, ved du nu hvad, udenlandske ansøgere, der ikke
  allerede har fået øjnene op for mulighederne i dette land fotjener
  sandelig heller ikke en chance.  I øvrigt har vi jo allerede
  rigeligt med udenlandske ansatte her på instituttet.  Vi er jo
  immervæk et dansk universitet.

  \says{A} Tjo - men at opslaget kun findes på institutbestyrerens
  dør?  Hvad med alle dem, der ikke har deres daglige gang på
  instituttet?

  \says{C} Hvis folk ikke har deres daglige gang på instituttet, så
  interesserer de sig nok ingengang for instituttets ve og vel.  Og
  den slags folk er vi vel ikke interesserede i at få indenfor dørene.

  \says{A} Nej, det er klart, men alligevel - på inersiden af døren!
  \act{Er nu, på trods af C's argumenter, en anelse indigneret}

  \says{C} Ja, du vil da vel ikke lukke folk ind på instituttet, hvis
  ikke Stig vil lukke dem ind på sit kontor.  Tænk på alle de
  skænderier, vi allerede har i institutrådet, lærerne kan jo knap nok
  snakke sammen. \act{C hidser sig lidt op}

  \says{A} Men jeg forstår nu altså ikke at opslaget er skrevet med
  pitch 32 - man skal jo bruge lup for at læse det.

  \says{C} Det er nu skrevet med pitch 10 - og så kopieret ned.  Du må
  tænke på, at det, vi har brug for, er een med sans for de små ting i
  tilværelsen.

  \says{A} Ja, ellers ville ha nvel ikke søge hertil.

  \says{C} Pjat med dig, se nu bare ham den sidste, alt for STOR, de
  andre kunne slet ikke finde ud af det med ham \act{gestikulerer},
  han brugte alt for store ord om store ting, gik i for store sko.
  Nej, hvad vi har brug for er en lille mand.

  \says{A}[viser med hånden] Det er måske også derfor, at opslaget
  befinder sig 10 centimeter over gulvet.

  \says{C} Ja, vi skulle jo nødigt risikere, at ansøgeren ragede op
  over de øvrige herinde.  Vi har brug for en ydmyg mand, der kender
  sin plads.  Som ikke kommer og laver alting om. IKK'.

  \says{A}[Lidt betyttet over C's følelsesudbrud, går ufortrødent
  videre] Jamen det er da at gøre sagen temmeligt kompliceret at
  placere opslaget med teksten ind mod døren.

  \says{C} Jaja, men som Peter Naur siger "`det er hverken opslag,
  ansøgere eller bedømmelsesudvalg, der er interessant, men hvem, der
  får stillingen.  Giv slip på formaliteterne - gå efter manden (som
  de siger i fodbold), hvad ingen ser har ingen ondt af.  Og jo færre
  der søger, jo færre er der til at klage."'

  \scene{Tæppe}

\end{sketch}
\end{document}
