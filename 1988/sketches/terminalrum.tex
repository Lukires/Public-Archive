\documentclass[a4paper,11pt]{article}

\usepackage{revy}
\usepackage[utf8]{inputenc}
\usepackage[T1]{fontenc}
\usepackage[danish]{babel}


\revyname{DIKUrevy}
\revyyear{1988}
% HUSK AT OPDATERE VERSIONSNUMMER
\version{1.0}
\eta{? minutter}
\status{Ikke færdig}

\title{Terminalrum-sketch}
\author{Torben Mogensen}
\begin{document}
\begin{roles}
  \role{Journalist}[?]fra fra radioprogrammet ``En arbejdsplads i Danmark''
  \role{Student}[?]
\end{roles}

\begin{sketch}
  \scene Terminalrummet på 1. sal; en studerende sidder ved en
  terminal.  Journalisten kommer ind.
  
  \says{Journalisten}[til publikum] Vi har nu bevæget os op fra
  maskinrummet i kælderen og befinder os nu i terminalrummet på
  1.\ sal.  Her er flere rækker af borde med skærme, hvor der sidder
  folk og arbejder.  Vi går nu hen og snakker med en af disse.

  \says{Journalisten}[til den studerende] Goddag.  Hvad er det
  egentlig der foregår her?

  \says{Studenten} Øh, goddag.  Ikke ret meget lige nu, for maskinen er
  nede.

  \says{Journalisten} Ja.  Vi var lige nede og se på den i kælderen.
  Men vil det sige at den også kommer her op?

  \says{Studenten} Øh, ja.  Vi har jo nettet, som sørger for det.

  \says{Journalisten} Det lyder som et stærkt net.  Kan det ikke ske
  at det bryder sammen ved den hårde belastning?

  \says{Studenten} Ork, jo.  Men denne gang er det maskinen selv, der
  er gået ned.

  \says{Journalisten} Øh.  Kan den sådan gøre det helt af sig selv?

  \says{Studenten} Ja, men det er nu mest når der er mange der kører
  på den, så bliver bussen overbelastet.

  \says{Journalisten}[efter en lille pause] Aha!  Når bussen er
  overbelastet, kan den ikke køre, og så går den ned!

  \says{Studenten} Øh, ja.  Noget i den retning.

  \says{Journalisten} Er de maskiner vi så nede i kælderen de eneste
  der er her?

  \says{Studenten} Nej, nej.  Der er for eksempel Piccoloerne her inde
  ved siden af, som Dat-0'erne bruger.

  \says{Journalisten} Mener du ikke at det er piccoloerne, der bruger
  Dat-0'erne.  

  \says{Studenten} Sådan kunne man måske godt udtrykke det.

  \says{Journalisten} Har i så ikke nogen piccoliner?

  \says{Studenten} Nej, da.  Det ville bare give problemer.

  \says{Journalisten} Det har måske noget at gøre med den lille
  forskel?

  \says{Studenten} Ja, den er nu slet ikke så lille endda.  Det er
  også af den grund at Mac'erne er et helt andet sted.

  \says{Journalisten} Makkerne?

  \says{Studenten} Ja, de arbejder helt anderledes.  De har vinduer og
  rullegardiner, som man kan åbne og lukke med en mus.

  \says{Journalisten} Med en mus??  Hvordan fungerer det?

  \says{Studenten} Jo, man peger på vinduet med musen og trykker på
  den.

  \says{Journalisten} Det lyder godt nok avanceret.

  \says{Studenten} Ja, det er ved at blive meget moderne.  Sådan noget
  kan man slet ikke med Thor og Freja.

  \says{Journalisten} Thor og Freja?  Det lyder lidt oldnordisk.

  \says{Studenten} Det er de også ved at være.

  \scene (evt. mere snak)

  \says{Journalisten} Jeg siger mange tak for den interessante
  samtale.  Hvilken musik kunne du tænke dig at høre?

  \says{Studenten} Tjoe, jeg vil da gerne høre \emph{(titel på næste
    revysang)}.

  \scene Spotten på journalisten og den studerende fader ud, og
  sangerne til det følgende nummer kommer ind og begynder at synge.
\end{sketch}
\end{document}
