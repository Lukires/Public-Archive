\documentclass[a4paper,11pt]{article}

\usepackage{revy}
\usepackage[utf8]{inputenc}
\usepackage[T1]{fontenc}
\usepackage[danish]{babel}


\revyname{DIKUrevy}
\revyyear{1988}
% HUSK AT OPDATERE VERSIONSNUMMER
\version{1.0}
\eta{$n$ minutter}
\status{Færdig}

\title{Teknisk tikamp}
\author{?}

\begin{document}
\maketitle

\begin{roles}
  \role{S}[] Speaker
  \role{K}[] Kommentator
  \role{E}[] Ekspertn
\end{roles}


\begin{sketch}

  \scene{Speaker (S) sidder ved et bord med vandglas osv.}

  \says{S} Vi byder seerne velkommen her til de danske mesterskaber i
  teknisk tikamp.

  \says{S} Som navnet antyder har vi 2 discipliner i teknisk tikamp,
  og vi starter med at bruge et uddrag fra formiddagens øvelser i
  indkørsel.

  \scene{Lys på et bord, hvor der sidder to personer ved en terminal:
    Kommentatoren (K) og exbert, (E)}.

  \says{K} Og vi byder seerne velkommen til første runde af de danske
  mesterskaber i teknisk tikamp her på Datalogisk Institut.  Vi er i
  gang med den første disciplin, indkring, hvor Karl Koder er ved at
  loade.  Han kører på en 8MHz Motorola med 512K RAM og
  dobbeltforniklede bespændinger.  Og nu starter han...

  \scene{Lydefekt: Uii-beep crash!}

  \says{K} Ja, det var jo ikke helt til at se, hvad der skete der.
  Lad os tage det en gang til i Trace Mode?

  \says{E} Ja, bemærk hvordan han starter.  Man ser tydeligt, at det
  er en professionel.  Han initialiserer registrene, checker at
  stakpilen sidder rigtigt... og så kører han!

  \says{K} Det går godt, vi har kørt den første kilobyte, stakken er
  ved at være fyldt... Men se så der!  Et indirekte hop!  Men det er
  det forkerte register!!  Lykkes det?  Lykkes det?

  \says{E} Hvaffen løkke?  Det' da et hop.

  \says{K} Og hvor havner vi nu?

  \says{E} Øh, det er da skærmen det der.

  \says{K} Uha da da, det ser grimt ud.  Men se så, uret kommer ind og
  afbryder.  Det kan man da kalde at blive reddet af klokken!

  \says{E} Ja, han var heldig der.

  \says{K} Og nu, nu sætter han prioritetsniveauet ned igen.  Og så...

  \says{E} Ja, hvis jeg lige må afbryde her.  Det var måske ikke så
  heldigt.  Uret kommer igen, og nu har han to afbrydelsesrutiner i
  gang.

  \says{K} Hvordan vil du vurdere hans situation nu?

  \says{E} Ja, hans stak er jo håbløs, og registrene ser usikre ud.
  Men han holder i det mindste flagene højt.  Men at nå tilbage til
  det oprindelige program vil jo nok være som et finde en hob i en
  stak.

  \says{K} Og se så.  Han prøver at... formattere... printeren?

  \says{E} Ja, det er der ikke mange, der tør, men Karl kaster sig
  lige ud i det, uden at initialisere eller lukke for afbrydelser!
  Sikke et mod!  Hvad bliver det næste?

  \says{K} Han hopper... og nu igen!  Han hopper igen!  Er det ikke fantastisk!

  \says{E} Jooh, men...

  \says{K} Hvor længe kan han blive i det forrygende tempo?  Det er næsten ligesom hjemme i Italien.

  \says{E} Næeeh, han hoppede ud over lagertoppen...

  \says{K} Ægte spaghetti-kode.  Er det ikke... hvad sagde du?

  \says{E} Han hoppede ud over toppen.

  \says{K} Vi ldet sige, det er slut nu?

  \says{E} Ja, det hop kommer han ikke over.

  \says{K} Jamen... kan der ikke komme en afbrydelse!

  \says{E} Nej, se engang på hans prioritetsniveau.

  \says{K} 7?!?.. Nå, ja så er det vel slut.  Men det var nu en
  strålende præstation.

  \says{E} Tjah... Trist, at han ikke fik rodet på disketten.

  \says{K} Og hvilken karakter giver dommerne?  00101.  Det var ikke
  meget.  Men tilbage til studiet.

  \scene{Lys på speaker, exbert og kommentator lister ud med deres bord.}

  \says{S} Efter denne reportage stiller vi nu direkte om til aftenens
  disciplin: 10Kb bughunt.  Værsgå!  \act{Trykker på knap} Kommentator
  kommer ind, gestikulerende og mimende.  Speakeren skynder sig at
  trykke på en knap til.

  \scene{Dommeren kommer ind}

  \says{K}[midt i en sætning] "= har vi så dommeren, der kommer på
  banen.  Til så stor en begivenhed som denne er der selvfølgelig en
  anerkendt dommer, den nu atter udlandsprofessionelle På-Spring
  Hansen.  Han er jo kendt fra sin tid som fejlretter på den første
  GIER.  På det seneste har han deltaget i konstruktionen af Danmarks
  Radios velkendte DORA projekt.

  Dommerens første opgave er at undersøge, om udskrifterne overholder
  de internationale normer for 10kb bughunt.  Han vejer dem \act{de
    vejes} - som bekendt må de jo ikke afbive fra 10KB grænsen med
  mere end 100 byte - Og nu er den første undersøgt, den er i orden,
  dommeren vinker afværgende.  Nu går han igang med den anden udskrift
  - men hvad er det!  En ulovlig fejl!  Han virker helt ophidset,
  tilkalder operatørvagten - tilkalder operatørvagten - TILKALDER
  OPERATØRVAGTEN!, ******* og får straks stukket en ny frisk udskrift
  i hånden.  Det bliver spændende, om denne udskrift er i orden, den
  klarer vægtgrænsen, og der er heller ikke ulovlige fejl.  Jeg tror,
  at arrangørerne ånder lettet op, det kunne blive en alvorlig plet på
  dansk datalogis rygte i udlandet med ulovlige fejl i mere end 1
  udskrift.

  Nu hvor udskrifterne er i orden kommer de to deltagere på banen:

  \scene{Deltagerne kommer på banen}

  \says{K} Vi har på bane 0 sidste års vinder, repræsenterende DIKUs
  EDB-afdeling, Karl Koder.  Karl har virkelig trænet i år, han har
  haft adgang til UNI*C's OS1100 anlæg, hvor der er virkelig gode
  muligheder for bughunt.  Man kan nok sige, at Karl Koder er
  storfavorit, han er nærmest et naturtalent som afluser.  Han har jo
  en fortid som sommerfuglejæger, og allerede fra 12-års alderen var
  det klart, at han var et endog meget stort talent, da han hackede
  sig ind på en amerikansk hospitalscomputer og debuggede et
  hjerneelektroencefalogramanalyseprogram pr. modem - on line - i
  løbet af 14 dage.  Meget imponerende - særligt telefonselskabet var
  glade for hans indsats.  Det forlyder dog ikke noget om
  hjernepatienternes reaktion - men angiveligt er der ikke nogen der
  har klaget.  Men det er dog en respektindgydende modstander, han
  står overfor.

  Hvis Karl Koder har en dårlig dag, kan man være sikker på at
  Kode-Kurt på bane 1 står på spring for at overtage hans trone.  Vi
  ved jo alle, at Kode-Kurt er tidl. mester i kerne-kodning på dat1,
  der går stadig frasagn om hans utroligt dårlige rapporter, inklusive
  kerneopgaven selvfølgelig, selvom kernen kørte.  Manden der koder
  ekspertsystemer i HEX kan også debugge, men han har ikke haft
  mulighed for at forberede sig helt så professionelt som Karl Koder
  har.  Derimod har Kode-Kurt sin ungdom og sit vovemod at tage med på
  banen.  Han er ikke bange for at hoppe til en ukendt adresse med
  smalede ben, og det kan komme til at afgøre konkurrencen.  Kurt
  mangler jo stadig sit helt store gennembrud på trods af de mange
  store præstationer han har ydet.

  Nu beder dommeren om at få deltagerne på plads, og han giver dem
  hver deres udskrift.  De to ekspertdebuggere spidser deres blyanter,
  kontrollerer indholdet i kaffekanderne og ølkasserne.  Ja-ha- Karl
  Koder brokker sig højlydt over at der ikke er Blå Nykøbing i
  kasserne, men man kan jo ikke yde nogen særbehandling - end ikke til
  den forsvarende mester.  Og nu smøger Kode-Kurt ærmerne op... Karl
  Koder tager brillerne af... Og starten er gået!  Karl Koder lægger
  stærkt ud, i et meget højt tempo, man skulle tro det var
  sprinterdistancen 1KB bughunt, det drejede sig om, men Kode-Kurt
  følger sit eget tempo.  Det er fantastisk spændende, skal det
  virkelig lykkes for Karl Koder at distancere Kode-Kurt allerede fra
  starten?  Men hvad er det?  Dommeren er henne ved Karl Koder - hvad
  er der sket?  Jeg kan ikke rigtig se - men jeg tror, ja jeg kan nu
  se, at dommeren trækker en debugger frem fra Karls ærme, en
  fantastisk udvikling og et tarveligt snyderi.  Dommeren har kun én
  mulighed, han trækker den røde blyant frem fra brystlommen, og Karl
  Koder bliver bortvist fra banen.  En meget ukollegial optræden af
  Karl Koder.

  Vi har nu kun én deltager tilbage, og han er jo sikker vinder, tager
  det stille og roligt, holder sit eget tempo, men hvad er det?  Han
  trækker en lup frem fra brystlommen, han begynder at læse direkte i
  mikrokoden, han rykker og han rykker hurtigt, en forrygende
  slutspurt.  En helt utrolig fart han har på, det er oven i købet
  hurtigere end Karl Koder, da han brugte debugger - det er
  fantastisk, har vi mon en verdensrekord på vej, og nu går han i mål,
  han har tangeret verdensrekorden i 10KB bughunt, det er det helt
  store gennembrud for denne fine sportsmand.  Og med denne flotte
  præstation siger vi tak herfra og stiller tilbage til studiet.

  \says{S} Ja, tak til ?? Og vi kan lige nå et enkelt telegram inden
  vi slutter.

  Vi har resultatet fra institutrådsmødet:
  mangler\\
  mangler\\
  mangler\\
\hfill\\
  mangler\\
  mangler\\
  mangler\\
\hfill\\
  mangler\\
  mangler\\
  mangler\\
\hfill\\
  mangler\\
  mangler\\
  mangler\\
\hfill\\
for

Vi kan tilføje at det sidste resultat er fundet ved
tendenslodtrækning.  Og det var slut på sporten, opløst af n.n og
redigeret af k.k.

\end{sketch}
\end{document}
