\documentclass[a4paper,11pt]{article}

\usepackage{revy}
\usepackage[utf8]{inputenc}
\usepackage[T1]{fontenc}
\usepackage[danish]{babel}


\revyname{DIKUrevy}
\revyyear{1988}
% HUSK AT OPDATERE VERSIONSNUMMER
\version{1.0}
\eta{$n$ minutter}
\status{Færdig}

\title{Professor søges 1a}
\author{CKH, NCKH}

\begin{document}
\maketitle

\begin{roles}
  \role{A}[] Studerende
  \role{B}[] Studerende
  \role{C}[] Studerende
  \role{JC}[]
  \role{SS}[] Stig Skelboe
  \role{K}[] Kor
\end{roles}

\begin{sketch}

  \scene{A er en lettere undrende studerende, der prøver at forstå
    denne verden og dens professoropslag.  B er ikke nærmere
    beskrevet.  C er en rigtig DIKU-pamper af den slags, der vil sælge
    sit eksamensbevis bare for at spise af samme aflagte tallerken som
    en af instituttets brilliante VIPere.  JC og SS behøver vel ikke
    nogen nærmere introduktion.

    Man hører en dør smække med et ordentligt brag.

    Musikken spiller "`Born in the USA"', et kor synger "`Back to the
    USA"' et par gange.

    På scenen står A og B.  De vinker med hvide lommetørklæder i
    retning af den smækkede dør, snøfter højlydt og tørrer
    (krokodille?)tårer af kinderne på sig selv og hinanden.  B stiller
    sig på tæer og stirrer efter dør-smækkeren.}

  \says{A} Er han væk nu?

  \says{B} Nej, han sidder fast i hoveddøren.

  \says{B} Åh, pyha.  Nu kom han fri.

  \says{A} Ved du, hvad det betyder?

  \says{A+B}[nikker samstemmende og udbryder] Nu skal vi ha' os en ny
  professor!

  \says{A} Han skal være noget særligt!

  \says{B} Han skal forske!

  \says{A} Han skal kunne undervise!

  \says{B} Han skal GIDE at undervise!

  \says{A} Han skal gøre DIKU til en visdommens højborg, summende af
  virkelyst!  Han skal med sin ånd og kraft lede os alle ud af dvalen og
  ind i åndernes rige.

  \says{B} Han skal nok ikke være datalog.

  \says{A} Han skal bare have brede skuldre.

  \says{B} Eller også skal han være en hun.

  \says{A} Dette er vort instituts chance.  Vi skal have verdens største - nå nej - verdens bedste forsker hetil!  Vi må finde ham!

  \says{B} Der er kun én vej frem!  Vi må avertere efter ham overalt: (alle de største tidsskrifter)

  \scene{A og B begynder at løbe rundt på scenen og råbe: PROFESSOR
    SØGES.  PROFESSOR SØGES!

    Ind på scenen kommer SS og JC med fingrene på læberne.  Syyh, I vækker jo hele institutet!

    A og B ser naturligvis særdeles forundrede ud, men af bar
    forbløfffelse tier de stille, indtil SS og JC er gået igen.  Så råber
    A og B videre.

    Kort efter kommer SS og JC ind igen.  SS er bevæbnet med
    "`bruger-battet"' og JC med "`debuggeren"'.  A bliver truet med
    battet, og B får et velrettet af med debuggeren.  SS og JC genner vore
    helte ud, men bliver selv på scenen.

    Lidt efter kommer A og B på banen igen.  Denne gang har de to store
    plakater, hvorpå der står "`PROFESSOR SØGES"' med sig.  De viser
    glædesstrålende deres værk til SS og JC.}

  \says{A} Se, hvad vi har lavet!

  \says{SS} Ja, det ser mig lidt vel prangende og udatalogisk ud det
  der.

  \says{JC}[griber A's plakat og river den i stykke] Det skal I ikke
  tænke på, vi har sørget for et passende antal plakater.

  \says{A}[lidt såret, men nysgerrig] Må vi se dem?

  \says{JC}[fremdraget et frimærke] Ja, her er den!  Og vi skal nok
  sørge for at sætte den op et passende sted.  Det er jo en vigtig sag,
  så det må gøres ordentligt.  Vi sætter den på bestyrerens dør. \act{SS
    løfter næsen lidt}.

  \says{SS} På indersiden.

  \says{JC} Med bunden i vejret.

  \says{SS} Og bagsiden ud.

  \says{B} Jamen, vi vil så gerne have lov til at gøre noget!  Så kan
  vi da i hvert fald skrive til: (Tidsskrifterne).

  \says{JC} Sødt af jer, men det har vi sørget for.  Pil nu af med jer.

  \scene{A og B går slukørede ud.}

  \says{SS} Hø, hø: Anders And.

  \says{JC} Journal of Belingske Tidende.

  \says{SS} Det kan ikke gå galt!

\end{sketch}
\end{document}
