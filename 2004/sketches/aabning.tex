\documentclass[a4paper]{article}
\usepackage[utf8]{inputenc}
\usepackage[T1]{fontenc}
\usepackage[danish]{babel}
\usepackage{charter}
\usepackage{revy}

\revyname{DIKUrevy}
\revyyear{2004}

% HUSK AT AJOURFØRE VERSIONSNUMMER!
\version{1.0}
\eta{3 min.}

% OG STATUS!
\status{Færdig}

\title{Åbningstale}
\author{Jakob Jensen}

\begin{document}
\maketitle

\begin{roles}
\role{R}[Uhd] Revyboss
\role{P}[Marvin] Alexander - Patric-klon (Super bøssen fra telias reklame)
\end{roles}

\begin{props}
\prop{Mobiltelefon}
\prop{Cigaretter}
\prop{Ligther}
\prop{Papir}
\prop{Smoking (R)}
\prop{Tøj (P)}
\end{props}

\begin{sketch}
  
  \scene Tom, mørk, tæppet er trukket for til Intro
  
  \says{R}\act{Kommer ind på scenen} Velkommen til dette års
  DIKUrevy. I år er det jo et helt specielt \ldots{}
  
  \says{P}\act{Kommer ind på scenen ser ikke publikum men går direkte
  han til R og prikker ham på skulderen}
  
  \says{R}\act{Vender sig spørgende om}
  
  \says{P}\act{Kigger på et stykke papir, tøvende} Jeg sku' tale med en
  af revybøsserne.
  
  \says{R}\act{Irriteret} Ja, det mig men hvem er du?
  
  \says{P}\act{Lettet} Det er mig der er Alexander, jeg kommer fra
  Universitetsavisen jeg sku' lave et interview med en af revybøsserne
  angående jubilæet.
  
  \says{R}\act{Stadig irriteret} Det var i går, vi sku' lave det
  interview.
  
  \says{P} Ja, jeg ved det men jeg havde en personlig krise så jeg
  tænkte at vi måske i stedet ku' lave interviewet idag.
  
  \says{R}\act{Måløs} Det har jeg da ikke tid til, revyen er lige
  begyndt. Jeg er igang med at holde åbningstalen, kan du ikke se at
  publikum sidder og venter.\act{peger ud på publikum}
  
  \says{P}\act{Ser for første gang publikum} Kan det ikke vente?
  
  \says{R} Nej det kan det da ik', folk har betalt for at komme her så
  vi kan da ikke bare lade dem vente.
  
  \says{P}\act{Hører igen ting} Hvor mange år fylder revyen?
  
  \says{R}\act{Opgivende} 32 år, vil du så lade mig være i fred?
  
  \says{P}\act{Vender sig om og begynder at tælle på fingrene}
  
  \says{R}\act{Lettet} Når hvor kom jeg fra? Nå ja, I år er det et helt...
  
  \says{P}\act{Vender sig om igen op prikker R på skulderen
  igen, Spørgende} Jamen 32 er da ikke et rundt tal?
  
  \says{R}\act{Overdrevet tålmodigt} Nej, det er en toerpotens
  
  \says{P}\act{Ser undrene ud men lyser pludselig op og kigger
  begejstret op og ned af R} Potens ohh year baby, jeg er altid med på et
  trut i en...
  
  \says{R}\act{afbryder hurtigt} Nej nej nej![Tydeligt] Toerpotens. Det
  er en bestemt mængde af tal (eller hvad det nu er ;-)). 4 er f.eks
  en toerpotens.
  
  \says{P}\act{Endnu mere begejstret} Suuuper! firkant der er endnu bedre,
  så kan vi også lave runde...
  
  \says{R}\act{Afbryder hurtigt igen, tager sig til
  hovedet, Overdrevet tålmodigt} Nej ikke firekant, fire. 4 er en
  toerpotens, altså $2^{2}$.
  
  \says{P}\act{Forarvet og begejstret på en gang} Nej, 2 i enden, der
  sætter jeg altså grænsen.
  
  \scene En mobiltelefon ringer.
  
  \says{P}\act{Roder lidt i sine lommer og hiver en mobiltelefon frem}
  [Siger et eller andet rim ala Harry fra DSB reklamerne]
  
  \says{R}\act{Rømmer sig} Mobiltelefoner skal være slukket under
  hele revyen.

  \says{P}\act{Pakker sin telefon væk, Lidt betyttet} Det må du
  undskylde.
  
  \says{R}[Opgivende] Det er iorden.
  
  \says{P}[Over glad / Lettet] Hvordan er det så et lave en homorevy.
  
  \says{R}[???] Homorevy??? Det er ikke en homorevy, det er en DIKUrevyen.
  
  \says{P}Men er du ikke revybøsse?

  \says{R}Nej jeg er revyboss, det er mig der bestemmer.
  
  \says{P}[Lader som om han forstår] Når Bosh, så i er en slags
  frisører eller hvad? \act{finder en ligther og en pakke cigeretter
  frem}
  
  \says{R}[Irriteret]Der må ikke ryges eller bruges åbent lidt under
  revyen.
  
  \says{P}\act{Står og fumler med cigeretterne}
  
  \says{R}Hvis du absolut skal ryge kan du gøre det i pausen ude
  foran, hvor der også vil blive slogt lidt at drikke. Og for god
  ordens skyld så er vi ikke frisører men universitetsstudrende.
  
  \says{P}[Ser lyset]Ohh, i er fysikere.
  
  \says{R}\act{Går helt bananas, men lyser pludseligt op} Kan du se de
  grønne skilte\act{peger på nødudgangene} det er der man skal gå ud i
  tilfælde af brand.\act{tager Ps ligther} Nu om lidt sætter jeg ild
  til dig, og så løber du alt hvad du kan han mod en af nødudgangene.
  
  \says{P}[Forundret]Jeg troede ikke der måtte være åbent ild under
  revyen. 
  
  \says{R}[Råber MEGET højt] løøøb!
  
  \says{P}\act{Løber mod nærmeste nødudgang}
  
  \says{R}[Lettet] Jeg gi'r jer DIKUrevy 2004.
  
  
\end{sketch}
\end{document}
