\documentclass[a4paper,11pt]{article}

\usepackage{revy}
\usepackage[utf8]{inputenc}
\usepackage[T1]{fontenc}
\usepackage[danish]{babel}

\revyname{DIKUrevy}
\revyyear{2004}
% HUSK AT OPDATERE VERSIONSNUMMER
\version{1.0}
\eta{2 min}
\status{Færdig}

\title{Den danske sangskat}
\author{Uhd}

\begin{document}
\maketitle

\begin{roles}
\role{S1}[Andr\'e] Sanger 1
\role{S2}[JakobJ] Sanger 2
\role{S3}[Madß] Sanger 3
\role{S4}[Uhd] Sanger 4
\role{VO}[HeidiF] Voice over
\end{roles}

\begin{props}
\prop{Stilet tøj}
\end{props}

\scene{De fire tager opstilling på scenen, foran tæppet} 
  
\begin{song} \sings{VO}DIKU er nu stolt af at præsentere et uddrag af DIKUs
  stolte gamle sangskat.

\scene{1. sang. Mel: Blæsten går frisk over Limfjordens vande}

\sings{A}[højtideligt]
Hvis man som mand ser en pige med bryster
da rykker det i hvert et lem
da bli'r man offer for voldsomme lyster
tanken man tænker bli'r tem'lig slem
Hvad gør man? Jo, løsningen den er let
papir og en time på internet

\scene{S bukker. 2. sang. mel: Jylland mellem tvende have}

\sings{S}Fjærten er en farlig vane
- udøv den med nænsomhed
Rundt blandt DIKUs mange skærme
lugtes vederstyg'lighed
Skønt den blot generer nog'n
frygter resten mere
mere en eksplosion

\scene{S bukker. 3. sang. mel: Marken er mejet}

\sings{S}Piger der har nogle ord'ntlige meloner
kan være svær' at finde for en stakkels datalog
nog'n vil betale tre-fir' millioner
andre ta'r til takke med det ingen andre tog
:| Vi ta'r ger'n imod
alt hvad der har blod
bar' det ikke bider eller lugter lidt af fod |:

\end{song}
\end{document}

%%% Local Variables: 
%%% mode: latex
%%% TeX-master: t
%%% End: 
