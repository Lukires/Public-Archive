\documentclass[a4paper,11pt]{article}

\usepackage{revy}
\usepackage[utf8]{inputenc}
\usepackage[T1]{fontenc}
\usepackage[danish]{babel}
\usepackage{alltt}
\usepackage{underscore}


\revyname{DIKUrevy}
\revyyear{1986}
% HUSK AT OPDATERE VERSIONSNUMMER
\version{1.0}
\eta{$n$ minutter}
\status{Færdig}

\title{Kernesketch}
\author{JC, hvem det så end er}

\begin{document}
\maketitle

\begin{roles}
\role{L}[Vilmar] Lærer som leverer al talen
\role{P1}[Mia] En CPU (som i senere versioner er en proces)
\role{P2}[Lotte] En anden proces
\role{SE}[Karen] Syngeenhed
\role{NE}[Lise] Nummerproducerende enhed
\role{Tom}[Karoline] Semafor
\role{Fuld}[Inger] Anden semafor
\end{roles}

\begin{props}
\prop{To store vækkeure}[]
\prop{Et passende antal eksemplarer af ``lystige viser''}[]
\prop{To bakker, hvoraf den ene skal ødelægges}[]
\prop{Overhead eller flip-over}[]
\prop{Sandsynligvis div. skilte og kasser}[]
\end{props}


\begin{sketch}

\scene{En dat1forelæsning om kerner og multiprogrammering.}

\scene{Syngeenhedens funktion: SE får en opslået ``lystige viser'' fra cpu,
  synger et vers og dropper bogen på gulvet.}

\scene{Nummerproducerende enheds funktion: Leverer cpu en lap pair med et nummer
  på.}

\scene{Semaforoperationer:
\begin{description}
\item[Vent] Vil have en ``lystige viser''. Hvis der ikke er nogen sover
  processen, dvs. at semaforen ``grabber'' processen.
\item[Signaler] Aflever en ``lystige viser''. Hvis der er en proces, der sover,
  vækkes den (vha vækkeuret) og får overleveret bogen.
\end{description}
}

\says{L}
Ja, godaften og velkommen, jeg ser at dormitoriet er godt fyldt, men
augusteksamen nærmer sig jo også. Jeg har i samråd med instruktorerne besluttet
at anvende denne ekstraordinære lejlighed til at forelæse lidt om
multiprogrammering. både rapportopgave-- og eksamensbesvarelserne viser også, at
det er tiltrængt.

Se, emnet for aftenens forelæsning er \textbf{samtidig afvikling af parallelle
  processer.}  Hvordan kan det nu lade sig gøre ?  JO.  I moderne datamater kan
de ydre enheder, f.eks. skrivere eller læsere arbejde uafhængigt af og samtidig
med centralenheden -- den såkaldte ``cpu'' eller ``værket'' som det også kaldes
i fagkredse. Hver ydre enhed indeholder en lille styreenhed, som styrer enheden,
deraf navnet. Ser vi på en typisk ydre enhed -- en sangenhed -- udfører
styreenheden en algoritme noget a la:
\begin{alltt}
fig.1:    \underline{Repeat}
               Klar := \underline{TRUE};
               \underline{Repeat} \underline{Until} Bog_Modtages;
               Klar := \underline{FALSE};
               Syng_Sang;
          \underline{Forever};
\end{alltt}

\says{L}
Lad mig demonstrere.

\scene{SE ind, L demonstrerer.}

\says{L}
En anden typisk enhed -- en sidenummerproducerende enhed -- vil f.eks. udføre:
\begin{alltt}
fig.2:    \underline{Repeat}
               Klar := \underline{TRUE};
               \underline{Repeat} \underline{Until} Der_bedes_om_et_nummer;
               Klar := \underline{FALSE};
               Aflever_nummer;
          \underline{Forever};
\end{alltt}

\says{L}
Bemærk at enheden har uendelig mange numre til sin rådighed. Det vil jeg dog
ikke demonstrere.

Hvis centralenheden skal bruge f.eks. syngeenheden, må der finde en vis
synkronisering sted, centralenheden må nemlig vente på, at syngeenheden er
klar. Dette kan f.eks. foregå som:
\begin{alltt}
fig.3:    \underline{Repeat}
               \underline{Repeat} \underline{Until} Klar;
               Aflever_opslået_bog;
          \underline{Forever};
\end{alltt}

\says{L}
Vi ser her, at variablen klar anvendes af begge enheder; Centralenheden aflæser
og venter, og styreenheden sætter klar når den er klar. Klart, ikke ?
Tilsvarende bruges bogen til at indikere, at centralenheden er klar.

Hvis vi kalder centralenhedens lille algoritme for ``syng'' og dens venten på en
sidenummerproducerende enhed for ``hent_nummer'', har vi da følgende lille
algoritme:

\scene{NE og P1 ind}
\begin{alltt}
fig.4:    \underline{Repeat}
               Hent_nummer;             (* fra NE til P1      *)
               slå_op_i_bog;            (* P1 slår op              *)
               Syng;                    (* SE får bog, synger *)
          \underline{Until} ikke_flere_numre;
\end{alltt}

\says{L}
Bemærk slutbetingelsen, vi har jo ikke hele dagen.

\scene{Demonstration.  Først langsomt, så hurtigere}

\says{L}
Det kører jo helt fint og uproblematisk, men vi bemærker at centralenheden
\textbf{venter} det meste af tiden, idet de ydre enheder er mange gange
langsommere. Derfor har man fundet på, at centralenheden jo bare kan lave noget
andet i den tid, den ellers skulle have ventet. Det fører os over til kernen i
stoffet, nemlig parellelle processer. Vi reformulerer vores løsning fra før, men
denne gang som to processer, som vi forestiller os kører samtidigt -- virtuelt
parallelt naturligvis.
\begin{alltt}
fig.5:    \underline{Repeat}  (* P1 *)               \underline{Repeat}   (* P2 *)
               Hent_nummer;                \underline{While} p1_har_bakke \underline{do};
               \underline{While} p2_har_bakke \underline{do};      p2_har_bakke := \underline{TRUE}
               p1_har_bakke := \underline{TRUE}        tag_bakke;
               tag_bakke;                  tag_bogen_i_bakken;
               slå_op_i_bog;               aflever_bakken;
               læg_bog_i_bakken;           p2_har_bakke := \underline{FALSE}
               aflever_bakken;             syng;
               p1_har_bakken := \underline{FALSE}
          \underline{Until} ikke_flere_numre;              \underline{Until} ikke flere numre;
\end{alltt}

\says{L}
Vi ser her, at de to processer kommunikerer ved hjælp af en fælles variabel
kaldet en bakke, som kun een af dem må have adgang til ad gangen.

\act{prøver, det kører og cpu'en udnyttes meget bedre}

\says{L}
Glimrende, strålende.  Det er jo ganske tydeligt, at ikke alene udnyttes
centralenheden bedre, de ydre enheder kører også meget bedre.

Løsningen er imidlertid ikke uproblematisk. Betragter vi -- af
overskuelighedsgrunde -- hver af processerne som selvstænigt kørende, \act{P2
ind} vil vi få følgende situation, hvis de på et tidspunkt ``kommer i trit'':

\act{kører lige hurtigt, slås om bakken, som går midt over}

\says{L}
Problemet er, ar den fælles variabel, bakken, ikke opdateres udeleligt, vi så
lige hvordan bakken blev delt i 2.  For at skire udelelig adgang indfører vi en
særlig type variable, kaldet semaforer \act{TOM og FULD ind} med de tilhørende
operationer ``vent'' og ``signaler''.

\act{demonstrerer i følge beskrivelsen af operationerne}

\says{L}
Kalder vi denne semafor ``tom'' og denne for ``fuld'', kan vi opskrive vores
løsning således, at ``fuld'' har nul ``lystige viser'' og ``tom'' har een
``lystige viser''.  Vi ser her, at semaforerne ikke alene styrer tilgangen til
bakken, de overflødiggør den også.
\begin{alltt}
fig.6:    \underline{Repeat}                        \underline{Repeat}
               Hent_nummer;               vent (fuld);
               vent (tom);                syng;
               slå_op_i_bog;              signaler (tom);
               signaler (fuld);
          \underline{Until} ikke_flere_numre;       Forever;
\end{alltt}

\says{L}
Og mere når vi desværre ikke idag, men næste gang vil jeg komme ind på begrebet
``kerner''. Tak for i dag.

\end{sketch}

\end{document}
