\documentclass[a4paper,11pt]{article}

\usepackage{revy}
\usepackage[utf8]{inputenc}
\usepackage[T1]{fontenc}
\usepackage[danish]{babel}


\revyname{DIKUrevy}
\revyyear{1986}
% HUSK AT OPDATERE VERSIONSNUMMER
\version{1.0}
\eta{$n$ minutter}
\status{Færdig}

\title{1.delsforelæsning}
\author{DIKUrevyen 1986}

\begin{document}
\maketitle

\begin{roles}
\role{S}[Jan] Lærer
\role{T}[Steen] Lærerens tanker
\role{D}[Sanne] Dulle (en kvindelig elev)
\end{roles}

\begin{sketch}

\scene{En forelæsning.  Der skal bruges en overhead eller en tavle el.lign.
  Teksten er opbygget således, at hveranden replik siges af læreren, som er på
  scenen, og de andre siges af en stemme, som ikke er på scenen, og som skal
  illudere lærerens tanker.  Dullen kommer på scenen til aller sidst lige før
  sine replikker.}

\says{S} Goddag

\says{T} Sikke noget fis, den her dag er elendig !

\says{S} Inden vi går i gang med dagens emne -- optimal repræsentation af
balancerede ternære taludtryk -- har jeg nogle praktiske bemærkninger vedrørende
eksamen.

\says{T} Det lød overbevisende, mit imponereniveau bliver bedre med alderen.
Men eksamen -- bvadr !  Alle de håbløst tåbelige studenter, der ikke har fulgt
med.

\says{S} For visse af emnerne er det nok eksamensforberedelse at løse sine
rapportopgavebesvarelserer grundigt

\says{T} Men for visse studerende ville det afgjort være en fordel at lade være.

\says{S} Den resterende del øves bedst ved at regne de opgaver, jeg har stillet
til øvelserne.

\says{T} Hvis de kan regne dem \textbf{må} de kunne bestå: Jeg ku' ikke

\says{S} Med hensyn til de kritikpunkter af programmel og udstyr, nogle af jer
har fremført, kan jeg sige, at vi arbejder med sagen

\says{T} Godt de ikke ved, at arbejdet består i at prøve at kule sagen så godt
som muligt.  De sku' ta' og være taknemmelige for at vi overhovedet gider banke
noget ind i deres tykke hoveder \act{taber nogle papirer på gulvet}

\says{S} Hovsa.

\says{T} Satans ! \act{fumler lidt rundt, samler nogen af dem op, taber sin
  kuglepen el. lign.}

\says{S} Satans !

\says{T} Hovsa. \act{får samlet sine ting og sig selv}

\says{S} Er der nogen spørgsmål ?

\says{T} Åh nej -- det skulle jeg ikke have gjort. Ham kværulanten, der altid
kæfter p, sidder på forreste række -- han bralrer helt sikkert noget ævl af sig.

\says{S} Ja, der på første række

\says{T} Argh !!

\says{S} Hvad I kommer op i til eksamen ?  Det fortæller jeg selvfølgelig ikke.

\says{T} Idiot ! Tror han det er Svaneparken ??

\says{S} Om I må have PC'er med ?

\says{T} Maskinliderlig er han også.

\says{S} Nej, men gerne en mainframe.

\says{T} Ha ! -- der fik jeg ham !

\says{S} Om jeg kan anbefale en bestemt model ?

\says{T} Troede jeg. Hvor dum kan man være ?

\says{S} Du kan prøve at finde en med ordlængde 23.

\says{T} Den hopper han ikke på -- jo fande'me !

\says{S} Og hvis der ikke er flere spørgsmål vil jeg gå over til det ternære
system. \act{tænder overhead , transparent omvendt på}

\says{T} Gid jeg havde brugt tiden i går på forberedelse i stedet for at drikke
rødvin.

\says{S} Talsystemet har tre cifre: 0, + og -. Ved hjælp af disse kan man
repræsentere et vilkårligt reelt tal på utrolig kompakt form, f.eks. skrives 631
som: +0+0-0 -- og det kan reduceres til 0 ! Tilsvarende for 34, som skrives som
0-+0 = 0.

\says{T} Åh, av, hvor mit hoved dog dunker. Det blev vist ikke helt rigtigt det
her. Satans, der røg kridtet -- og mit hoved kan ikke tåle at komme nedad
(bøvs).

\says{S} Tak.

\says{T} Åh, der var en af de to søde piger, som altid kommer med grønne æbler,
der samlede det op. Der er nu ikke noget så oplivende som et kvindeligt islæt i
talsystemerne. \act{en pause}. Hvorfor ser de allesammen så underlige ud ?
\act{kigger på sin transparent, drejer den en halv omgang}. Det hjalp. Nu sidder
de også rigtigt på stolene.

\says{S} Som I ser reducerer dete talsystem den nødvendige ordlængde til
talrepræsentation til een bit, nemlig fortegnet på det nul alle tal
repræsenteres ved.

\says{T} Skide smart egon -- og tiden går. Nu kan jeg snart kome ned og få min
danskvand -- min \underline{gamle} dansk vand -- eller stryg bare ``vand''.

\says{S} Og her komme så yderligere en pointe : Med denne yderst kompakte
talrepræsentation er det muligt at udføre lige så mange operationer i parallel
som datamatens ordlængde.  Dette giver helt nye dimensioner til spørgsmålet om
paralleldatamater.

\says{T} Parallellitet -- et herligt ord. Jeg kan ligefrem mærke madrassen, når
jg indtager min parallelle stilling med en kølig drink ved siden af mig !  Ah !!
\act{bladrer om til nyt blad i sine noter}.  Hvad står der: Myggebalsam,
vandmelon, shampoo -- det kan jeg da ikke stå og sige ! Satans ! Jeg må
impvrovisere.

\says{S} Pladsen, en sådan datamat optager, svarer til ganske få chips for en
datamat med en megabyte mikroprocessor.

\says{T} men min plads svarer til flere poser franske kartofler -- det giver
også en bedre balance. Saltbalance altså.

\says{S} På denne måde regner man med i løbet af få år at få mainframes i
lommeformat.

\says{T} så passer de også bedre til jeres kyllingehjerner, folkens.

\says{S} Et spørgsmål ? Ja ? Hvordan man repræsenterer tegn ??

\says{T} Åh ! En af de skrappe, gid fanden havde ham. Bare han var lige så dum
som de andre.

\says{S} Ja, det er et forskningsproblem, der endnu er uløst og som
konstruktøren af de balancerede ternære systemer arbejder med i øjeblikket.

\says{T} Ideel undskyldning for ikke at vide noget -- de andre ved det heller
ikke.

\says{S} Flere spørgsmål ?

\says{T} Kære Vorerre, lad dem blive slagne med stumhed !  Jeg har væddet en
kasse bajer på, at efter den sidste svada er der ingen, der spørger om noget.

\says{S} Ikke ?

\says{T} Tak, oh tak.

\says{S} Ja, så tror jeg vi slutter for i dag, og husk nu at regne alle
eksamensrelevate opgaver -- instruktorerne har fået besked.

\says{T} Hvordan kommer jeg nu lettest til at følges med hende den rappe lille
dulle på anden række ?  Hun ser ud som om hun har slået rod -- NEJ, hun vil
spørge om noget, møf møf...

\says{S} Hej, er der noget du vil snakke med mig om

\says{T} På divanen ovre på mit kontor

\says{D} Hvor gammel er du?

\says{S} Jeg er 39

\says{T} Åh -- moden god elsker.

\says{D} Har du virkelig ikke lært mere på alle de år ? Vorherre bevares. Så
forstår jeg s'gu bedre DIKU's problemer !
\end{sketch}

\end{document}
