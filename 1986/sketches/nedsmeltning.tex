\documentclass[a4paper,11pt]{article}

\usepackage{revy}
\usepackage[utf8]{inputenc}
\usepackage[T1]{fontenc}
\usepackage[danish]{babel}


\revyname{DIKUrevy}
\revyyear{1986}
% HUSK AT OPDATERE VERSIONSNUMMER
\version{1.0}
\eta{$n$ minutter}
\status{Færdig}

\title{Nedsmeltningssketch}
\author{Flemming, Karen, Vilmar}

\begin{document}
\maketitle

\begin{roles}
\role{I}[Lise] Interviewer
\role{E}[Vilmar] Ekspert
\role{SB}[Ragnar] Studerende/russisk bonde
\role{BI}[Lotte] Bestyreren/russisk kraftværksinspektør
\role{L1}[Mia] Lystig svend/kosak
\role{L2}[Lars] Lystig svend/kosak
\role{L3}[Charlotte] Lystig svend/kosak
\end{roles}

\begin{props}
\prop{5 kosaksæg}[]
\prop{Overheadprojektor}[]
\prop{Vodkaflaske, porcelænssovsekande, 4 kopper}[]
\prop{4 blå nykøbing, 4 glas}[]
\prop{Papkasser, illuderende VAX/kraftværk mm.}[]
\prop{Et spil kort, en tromlerevolver med knaldperler}[]
\prop{Et bord, 4 stole}[]
\end{props}

\begin{sketch}
  \scene{Scene 1: Et interview, hvor en ekspert forklarer noget om nedsmelting,
    udslip og halveringstider.}

\says{I} Dete er DIKU-nyt med en ekstraudsendelse i anledningen af ulykken med
kernekraften på Dat-1.  Hos mig i dormitoriet har jeg Kerne-eksperten K. Hansen,
som vil fortælle nærmere om, hvad der egentlig er gået for sig.

\says{E} Tak. Jeg bryder mig nu ikke om, at du kalder det ``en ulykke'', i
fagkredse omtales den slags som ``hændelser'' i forbindelse med kerner. Ja, der
findes ligefrem hændelsesstyrede kerner.

\says{I} Nåh, jah -- men denne øhh hændelse, kan vi få et bud på, hvad der
egentlig er sket ?

\says{E} Det er meget svært at udtale sig om på nuværende tidspunkt, men alt
tyder på, at der er tale om en såkaldt \textbf{kernenedsmelting}.

\says{I} En \textbf{KERNENEDSMELTING !!!}

\says{E} Ja, ja, ja -- slå nu koldt vand i blodet. Lad mig forklare. Som alle jo
ved, indeholder de fleste datamater -- eller ``værker'', som vi kalder dem -- en
kerne, som styrer processerne.  Normalt reguleres kernens aktivitet af
forskellige operationer og mekanismer, f.eks. afbrydelser, så værket kører
kontrolleret og stabilt. Jeg kunne forestille mig, at der må være sket en
blokering af en af de regulerende mekanismer, så kernener kommet ud af kontrol.

\says{I} ER det det, man kalder ``det værst tænkelige uheld'' ?

\says{E} Nu er De der igen med Deres ``uheld''.  Her er tale om en
\textbf{hændelse}.  Iøvrigt synes jeg, at det er en negativ måde at se det p.  I
forbindelse med kerner plejer vi at tale om ``det bedst tænkelige held'' --
især når den kommer til at virke.  Men for at vænde tilbage til den aktuelle
hændelse, så yder alt på, at der må være sket en blokering af kølemekanismen,
det såkaldte ``overløb''.

\says{I} Overløb ?

\says{E} Ja.  En stor del af kernens aktivitet vil være aritmetiske
operationer.  Disse vil som regel gå som planlagt, men i visse tilfælde vil det
gå galt, og resultatet vil være momentant uden for kontrol, idet det kræer flere
bit, end den pågældende lagerenhed kan rumme.  Normalt klares dette ved at man
leder de overskydende bit væk gennem overløbet, men hvis dette er defekt, sker
der en ophobning af overløbsbit, som i ekstreme tilfælde kan medføre en
kernenedsmeltning.

\says{I} Jamen -- er det ikke farligt. Jeg tænker her på udslip.

\says{E} Der vil naturligvis ske et vist udslip, det kan man ikke udelukke --
men at gå så vidt, som at kalde det farligt, nej det vil jeg ikke. Det kommer an
på halveringstiden, dvs. den tid det tager at halvere en bitmønster-værdi.

\says{I} Bitmønsterværdi ??

\says{E} Ja.  I værker af denne type vil bittene samles i bitmønstre. Normalt
halveres et bitmønster relativt hurtigt, nemlig i løbet af den tid det tager at
udføre en fortegnsbevarende højreskiftsoperation.

\says{I} Åh, javel, ja. Ja det er klart, det forstår jeg fuldstændig.

\says{E} Der er her tale om en såkaldt kontrolleret halvering, som ved
gentagelser hurtigt medfører en neutralisering af bitmønsteret, hvis værdi højst
vil være nul.  Der er imidlertid en særlig isotop -- den såkaldte MAKINT -- som
kan være genstand for ukontrolleret halvering.  Her kan vi historisk beregne
halveringstiden, men vi kan ikke styre den.

\says{I} MAXINT?

\says{E} MAXINT er det største positive repræsenterbare heltal.  Jeg har vist en
planche over den anslåede halvering af MAXINT.

\scene{Planche}

\says{E} Vi ser her, at det først er omkring 1990 at MAXINT bliver nul, dette
gælder dog kun for de såkaldte to-komplements-værker.  For de noget ældre og
mere primitive et-komplement-værker, som vi ser det på f.eks. RECKU eller RECAU,
sker der slet ingen halvering af MAXINT, som konstant holder sig på en meget høj
værdi.  På RECKU er MAXINT-værdien målt til at være $2^{35} - 1$, og den har ike
ændret sig siden midt i 70'erne.  En anden ulempe ved et-komplement-værker er
deres noget mere komplicerede overløbssystem samt at de indeholder den meget
ubehagelige isotop minus-nul, som kan volde frygteligste problemer.

\says{I} Tror du, der er sluppet nogen MAXINT ud ?

\says{E} Det er svært at sige, men jeg tvivler på det.  MAXINT er som sagt et
meget stort tal, så hvis det var sluppet ud, ville man have set det.

\says{I} Ja, mange tak, Hansen, hvis delige vil blive her et øjeblik. Jeg har
netop fået at vide, at vi har en optagelse fra DIKU's maskinstue fra minutterne
omkring ulykken.

\says{E} \underline{Hændelsen!!}

\says{I} Omkring hændelsen, naturligvis. Vi viser nu et klip med Jørn Bo Sørense
og hans lystige svende.

\scene{Scene 2: Intervieweren og eksperten trækker sig tilbage til et diskret
  hjørne af scenen.  INd kommer de tre lystige svend (LS1, LS2, LS3) syngende:}
\end{sketch}

\begin{song}
\sings{LS1-3}
Hej ho, hej hå, er VAX'en gået i stå
det ordner vi en anden dag, hej ho, hej hå, hej hej
hej ho, hej hå, vi ta'r os et par blå
tra la la la, la la la la , hej ho, hej hå
\end{song}

\begin{sketch}
  \scene{LS1 går hen til VAX'en, lukker frontpanelet op og fremdrager 3 blå
    nykøbing og 3 glas.  LS2 tager -- fra en kasse mærket ``bridge-net'' -- et
    spil kort.  LS3 sætter ig ved et bord midt på scenen, smækker fødderne op,
    filer negle.}

\says{LS1} Pladerne ned, vi skal arbejde. Ha, ha. \act{sætter sig}

\says{LS2} \act{sætter sig, deler kort ud}. Ho, ho, ho.

\says{LS3} Skal vi ikke synge en lystig vise, hi hi hi ?
\end{sketch}

\begin{song}
\sings{LS1-3}
Katinka, katinka, luk vinduet op
nu vil jeg spendere en vise
Et forår har kærtegnet pigernes krop
og nu er du lige til at spise
men først skal vi ha' os en blå eller to
og så skal Katinka og Søren til ro
Katinka, katinka, luk vinduet op
og hør min harmonika vise.
Tra la la la ........
\end{song}

\begin{sketch}
\scene{mens der synges, drikker de blå nykøbing, og spiller kort under megen
  latter. ind kommer bestyreren (BI)}

\says{BI} Sig mig lige -- hvad foregår der her ?

\says{LS1} Øh -- vi gennemprøver det nye Bridge-net, specielt i forbindelse med
flydende aritmetik, ha ha.

\scene{LS2 henter en blå og et glas til BI, hælder op, spilder}

\says{BI} Nå ja, jeg kan se at overløb fungerer til overflod.

\says{LS3} Ha, ha, den var god.  Ellers risikerer vi jo en kernenedsmeltning.

\scene{Alle fire er ved at komme om af grin over denne bemærkning}

\says{BI} Jamen så skål da -- og lykke til med arbejdet. \act{bestyreren ud
  igen}

\says{LS1-3} Katinka, katinka ......

\act{En studerende (SB) kommer løbende forpustet ind}

\says{SB} Skynd jer, kom. Den er helt gal oppe i motorolarummet, vi kan ikke
downloade.  Stemningen er på kogepunktet.

\says{LS1} Kan I ikke downloade, ha ha.  Et øjeblik \act{går hen til bridge-net
  boksen, trykker på en knap}. Sådan, ha ha.

\says{SB} Kan vi downloade nu ?

\says{LS1} Nej. Men nu har jeg down'et nettet sa I heller ikke kan køre på
terminalerne til VAX'en, så gør det ikke så meget at I ikke kan downloade, ha
ha.

\says{SB} Altså det går galt det her, folk er enormt ophidsede, der er
aflevering om tre dage -- og -- og -- måske koger dete hele over altså !

\says{LS3} Ha ha, en nedsmeltining måske. Nej, du -- vi klarer det hele med lidt
flydende overløb. \act{hælder lidt øl ned ad SB. Der må næsten ingenting være
  tilbage i flasken}.

\says{LS2} Nå, det var da kedeligt, ha ha, men det betyder nok ikke noget, vi
ser på det onsdag i næste uge, smut du bare igen. \act{SB ud}

\says{LS1-3} Frokost ! Hej ho, hej hå, .... \act{går syngende ud}.

\scene{Lyset slukkes over det meste af scenen undtagen der hvor I og E befinder
  sig. Mens de taler sørger flyttemændene for at ændre scenen, så en passer til
  scene 3}

\says{I} Ja, det bekræfter jo din hypotese om årsagen til u -- øh til
\textbf{hændelsen}, ikke ?

\says{E} I høj grad.  Vi så her klart at det centrale overløbssystem svigtede.
Meget interessant, jeg kunne tænke mig at bruge dette som basis for en
rekonstruktion af en anden hændelse for nylig.

\says{I} En anden hændelse ?

\says{E} En anden hændelse, ja.  Et \textbf{strålende} eksempel, hvis jeg selv
må ha' lov.  Her kan vi med meget stor sikkerhed ...

\says{I} \act{afbryder} Sikkerhed ?  Jeg troede sikkerhed nærmest var en by i
Rusland ?

\says{E} Ja, netop  Se nu selv.

\scene{Scene 3: Som scene 2 bortset fra at
\begin{enumerate}
\item Alle taler russisk (eller noget der lyder som russisk) og er iført kosakskæg.
\item ``VAX'en'' drejes 180 grader og bliver til kontrolpanelet på et atomkraftværk
\item De øvrige rekvisitter er også ændret -- men det fremgår
\end{enumerate}}

\scene{Ind kommer de tre lystige svende syngende, melodien er pramdragerens sang.}
\end{sketch}

\begin{song}
\sings{LS1-3}
Hum hum hum -- hum. HO.
Hum hum hum -- hum. HO.
Atomski kaputski fexi Njetski
Hum hum hum -- hum. HO.
\end{song}

\begin{sketch}
\scene{LS1 går hen til kontrolpanelet, lukker det op, tager vodkaflaske og
  sovsekande over til bordet, tømmer flasken omhyggeligt over i kanden.  Henter
  så tre tekopper samme sted. Ser sig forsigtigt omkring, prøver så at tømme
  flasken ned i sin hals.}

\says{LS2}  Njet ! Glasski forbudtski ! \act{henter pistol i samme kasse, som
  kortspillet var i.}

\scene{LS3 sætter sig med fødderne på bordet.}

\says{LS1} Arbejdski. HO!

\says{LS2} \act{skyder med revolveren} Hik !

\says{LS3} Bum bum. Muskiski !

\says{LS1-LS3} \act{4 første linier af ``katinka'', melodi: ``kalinka''}

\scene{Mens der synges hældes vodka fra kanden op i kopperne, der skåles, grines
  og drikkes. Ind kommer inspektøren (BI)}

\says{BI} Stjepin hva skerski ?

\says{LS3} Donnerwetter ! Inspektørski.

\says{LS1} Afprøvning rouletski, da da. HO !

\scene{LS2 sætter pistolen til tindingen, trykker af.  Det skal være en
  ``klikker''}

\says{BI} Da da, alkoholski medicinski njet !

\says{LS3} Hik ! Njet alkoholski, podlika. Smagski ? \act{henter den fjerde kop,
  hælder en kop ``sovs'' op fra kanden, rækker til inspektøren, som drikker den
  med stort velbehag. (``podlika'' er russisk og betyder faktisk sovs -- har jeg
  ladet mig fortæle)}

\says{BI} Ahhh. Splendida podlika sovsski imot radioaktiviski.

\scene{Alle 4 er ved at komme om af grin over denne bemærkning}

\says{BI} Nastarovja -- bye bye.

\scene{En russisk bonde (SB) kommer forpustet ind}

\says{SB} Kvikski, Tjernobyl bum bumski. Samovar kaput, ikke koke te.

\says{LS1} Ikke koke te ? Momentski. \act{går hen til kontrolpanelet, trykker på
  en knap} So, HO.

\says{SB} Samovar funkski, koke te nu, da ??

\says{LS1} Njet. Tjernobyl bum bumski, samovar bum bumski, tutti bum bumski,
klask i numski \act{klasker SB bagi}. HO HO.

\says{SB} Gospodin pu pu ! Literaturnaja Gaseta, kameratski.

\says{LS1-3} Frokostski. \act{alle ud, syngende : Hum hum hum ....}

\scene{Eksperten er gået ud tidligere, intervieweren bliver tilbage.}

\says{I} Ja, for at kunne holde publikum orienteret om hændelsen, vil jeg bede
de lystige svende om at klargøre lokalnettet, så vi løbende kan følge med.

\scene{Interviewer ud, sketchski slutski}
\end{sketch}
\end{document}
