\documentclass[a4paper,11pt]{article}

\usepackage{revy}
\usepackage[utf8]{inputenc}
\usepackage[T1]{fontenc}
\usepackage[danish]{babel}


\revyname{DIKUrevy}
\revyyear{1986}
\version{1.0}
\eta{$n$ minutter}
\status{Færdig}

\title{Afslutning}
\author{Vilmar}
\melody{``Og det var Danmark'' og ``When the saints, go...''}

\begin{document}
\maketitle

\begin{roles}
\role{S1}[Karen] Sanger
\role{S2}[Lise] Sanger
\end{roles}

\begin{sketch}
\scene{Dette nummer er om fodbold, hvis nogen skulle være i tvivl.}

\says{S1} Hør kan du fortælle mig hvad au lait metyder?

\says{S2} Ja ja det er fransk og betyder med mælk, feks betyder cafe au lait ,
kaffe med mælk

\says{S1} Nå

\says{S2} Men hvorfor spørger du om det?

\scene{Alle kommer op på scenen stille syngende:}
\end{sketch}

\begin{song}
\sings{Alle}
Og det var DIKU, og det var DIKU
mæ mælk
mæ mælk
mæ mælk.
\end{song}

\begin{sketch}
\scene{derefter går vi over til at synge flg. sang:}
\end{sketch}

\begin{song}
\sings{Alle}
I bli'r smidt ud.
I bli'r smidt ud.
I bli'r smidt ud og skal til fest.
I skal danse hele naten.
Nu går vi op til sommerfest.

Nu går vi op.
Nu går vi op.
Nu går vi op til sommerfest.
Og vi skal danse hele natten.
Nu går vi op til sommerfest.
\end{song}

\end{document}

