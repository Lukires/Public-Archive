\documentclass[a4paper,11pt]{article}

\usepackage{revy}
\usepackage[utf8]{inputenc}
\usepackage[T1]{fontenc}
\usepackage[danish]{babel}


\revyname{DIKUrevy}
\revyyear{1986}
\version{1.0}
\eta{$n$ minutter}
\status{Færdig}

\title{Lokalnetsang}
\author{Vilmar}
\melody{Dire Straits: ``Money for nothing''}

\begin{document}
\maketitle

\begin{roles}
\role{G}[] Revyens guitarist
\role{F1}[Jacob] Flyttemand
\role{F2}[Flemming] Flyttemand
\role{M1}[Står ikke] Medarbejder fra EDB-afdelingen
\role{M2}[Står ikke] Medarbejder fra EDB-afdelingen
\end{roles}

\begin{props}
\prop{Snor}[]
\prop{Kasser}[]
\prop{En bærepose med påskriften ``Datanet'', indeholdene eks. af:}[]
\prop{``DATA''}[]
\prop{Gummistøvle}[]
\end{props}

\begin{sketch}
\scene{Flyttemændene bixer rundt med snor, kasser o.lign.  Det skal se
  \textbf{meget} forvirrende ud, men alligevel minde svagt om en maskinstue.}

\says{M1} \act{peger på rodet} Hva' det ??

\says{M2} Nåh, de er bare ved at montere lokalnettet.  Vi skal have et net rundt
i hele huset, så alle kan alting -- helt uden besvær.

\says{M1} Lokalnet ? \act{henter bærepose, peger spørgende}

\says{M2} Well, ikke helt, det du har der er et datanet \act{hiver et eksemplar
  af ``DATA'' op fra nettet}.  Men princippet er det samme.

\says{G} \act{kommer ind med guitar, ledninger, pedaler mm. i armene} Kan jeg få
koblet de her audiounit på nettet ?

\says{M1}[i munden på hinanden] Nej

\says{M2}[i munden på hinanden] Ja

\says{M2} \act{til flyttemænd} Fix lige en audioconnectorbox til den unge mand
her. \act{flyttemænd ud, kommer tilbage med en guitarforsærker, stiller den et
  passende sted i nærheden af resten af musikken.  M1, M2 og G er i mellemtiden
  travlt beskæftiget med at kravle rundt på gulvet og forbinde G's udstyr}

\says{M2} Jeg tror den er der nu, lad os lige prøve. \act{tænder, over
  sanganlægget spilles lancier-musik, vals el.lign}

\says{G} Jeg er vist ikke helt med ...

\says{M1} Nåh, det virker da meget godt, du fik da svar tilbage.

\says{M2} Hm !  er der mon en fejl et sted ? \act{går over til noget af
  flyttemændenes rod, dykker ned i en kasse, dukker op, ryster på hovedet,
  dykker ned i en anden kasse og fremdrager en tavle med nogen kridtstreger
  på. En af stregerne er brudt og ``faldet ned'' over de andre .  M2 visker den
  faldne streg ud og retablerer den. Tavlen ned i kassen}

\says{M2} Liniebrud.  Prøv igen.

\scene{G prøver, intet sker.}

\says{M1} Skal hans audioconnectorbox ikke bootes ?

\says{M2} Selvfølgelig. \act{hiver gummistøvlen op fra datanettet og slasker den
  (forsigtigt !!) mod guitarforstærkeren.} Sådan. Prøv igen.

\scene{G starter på forspillet.}
\end{sketch}

\begin{song}
\sings{M1}
Hvad skal vi gøre, det hele flyder
skal det her være en arbejdsplads ?
de samme skærme skal kunne bruges
til RECKU og VAX, vi må ha' linier en masse
Det er umuligt, det hele flyder

\sings{M2}
Åh, pjat med dig, man kan hvad man vil
nu skal du høre, hva' vi kan gøre
et stort lokalnet er hvad der skal til

\sings{M1 + M2}
Vi installerer bokse og kasser
forbinder skærme med dit og med dat, hey hey
vi trækker kabler, så nettet passer
lokale netværk er sagen, min skat
\end{song}

\begin{sketch}
\says{M1} Jamen -- hvordan virker sådan et lokalnet ? Hvordan kan det vide hvad
der skal hvorhen og hvorfra ?

\says{M2} Jo, altså, der sidder nogen bokse hist og her og de styrer det lissom.

\says{M1} Hvordan styrer det ? Skulle der ikke sættes sådan et net op?

\says{M2} Jo, netop.

\says{M1} Jamen -- hvordan virker så det ?

\says{M2} Joh, nu skal du bare høre: \act{synger}
\end{sketch}

\begin{song}
\sings{M2}
Hvert tegn du trykker, bli'r til en pakke
som pakkes ind og sendes afsted
og pakkes ud i den anden ende
bli'r til det tegn som VAX'en får ned
den tygger på det, sender tegn tilbage

\sings{M1 + M2}
som pakkes ind og sendes afsted
og pakkes ud i den anden ende
hos terminalen, nu ved vi besked

Vi installerer .......
\end{song}

\end{document}
