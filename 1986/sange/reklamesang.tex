\documentclass[a4paper,11pt]{article}

\usepackage{revy}
\usepackage[utf8]{inputenc}
\usepackage[T1]{fontenc}
\usepackage[danish]{babel}


\revyname{DIKUrevy}
\revyyear{1986}
\version{1.0}
\eta{$n$ minutter}
\status{Færdig}

\title{Reklamesang}
\author{Karoline og Vilmar}
\melody{``Warswawianka''}

\begin{document}
\maketitle

\begin{roles}
\role{L0}[Ragnar] Lærer
\role{L1}[Jan] Lærer
\role{L2}[Mia] Lærer
\role{SM}[Lars] Sprechstallmeister
\end{roles}

\begin{sketch}
\scene{Indledning: Biografreklamefanfare.}

\scene{Der står 2 kontorstole på scenen (som flyttemændenne har bragt ind under
  præindledningen).}

\scene{Musikken begynder at spille mens de 3 lærere kommer ind og går rundte om
  stolene. Musikken stopper og 2 af lærerne kaster sig straks ned på
  stolene. Den 3. skal synge når musikken begynder igen:}
\end{sketch}

\begin{song}
\sings{L2}
Til dig, der er træt af dit benhåre arbejde,
hvor du må knokle og får nerver og stress
har vi et tilbud du ik' ka' sige nej te'
hvor vi er skire på du bliver tilfreds.

\scene{Så kommer omkvædet, som de alle 3 skal stå op og synge med på:}

\sings{L0 + L1 + L2}
Søg ind på DIKU, her kan du finde
dig godt til rette i et fredeligt job
frokosten er på en time herinde
du sover længe og bli'r aldrig sagt op

\scene{Derefter er der igen kamp om stolen, den stående synger næste vers osv.}

\sings{Den næste der står op}
Der snakkes så tit om de mange studenter
og folk de hævder, at de udgør et problem,
men det kommer helt an på, hvad du forventer
du må jo tænke på, at du kan bruge dem.

\sings{L0 + L1 + L2}
Så søg ind på DIKU for her kan du få
god, billig arbejdskraft fra massernes kor.
Tænk på den forskning studenteren ku' nå
og som du så ku blive krediteret for.

\sings{Den næste der står op}
Forskningen får tit en ordentlig omgang
men til kritikken kan vi bare sige et
vist er vor forskning papirspild og tomgang,
men gør det noget når vi forsker så lidt.

\sings{L0 + L1 + L2}
Så søg ind på DIKU, for her kan du få
en stilling, hvor der er både roligt og trygt
der snakkes så tit om de højder vi ku' nå,
men det er endnu ikke blevet forsøgt.

\sings{Den næste der står op}
Men der er bare én ting vi må vide,
før du kan lukkes ind i videnskabens hal.
Du må bevise du rigtig kan slide,
at du er flittig og din forskning genial.

\sings{L0 + L1 + L2}
Så har du stress af maskiner, programmer
og kan du nøjse med $\frac{1}{4}$ million
søg ind på DIKU i rolige rammer
læn dig tilbage og vent på pension.
\end{song}

\begin{sketch}

\scene{Ved sidste linie sætter L1 og L2 sig søvnigt ned og bliver kørt ud af
  flyttemændene.  SM går hen til L0 og siger:}

\says{SM}[til L0] Oh -- det er vel nok heldigt, at det netop er dig, der er
tilbage, du skal nemlig holde en Dat-0 forelæsning.

\says{LO} En forelæsning? Jamen det har jeg slet ikke forberedt.

\says{SM} Plejer du da det?

\scene{Hvis Piccolonummeret er en sang, kan LO derefter sige: Jeg kan virkelig
  ikke holde en forelæsning, jeg må finde på noget andet.}
\end{sketch}

\end{document}

