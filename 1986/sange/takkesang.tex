\documentclass[a4paper,11pt]{article}

\usepackage{revy}
\usepackage[utf8]{inputenc}
\usepackage[T1]{fontenc}
\usepackage[danish]{babel}


\revyname{DIKUrevy}
\revyyear{1986}
\version{1.0}
\eta{$n$ minutter}
\status{Færdig}

\title{Takkesang}
\author{Vilmar}
\melody{The Archies: ``Summerprayer for Peace''}

\begin{document}
\maketitle

\begin{roles}
\role{SM}[Lars] Sprechstallmeister (synger for)
\role{T}[Karen] Takkesanger
\role{K0}[Jacob] Kor
\role{K1}[Mia] Kor
\role{K2}[Inger] Kor
\role{K3}[Charlotte] Kor
\role{K4}[Lotte] Kor
\role{K5}[Ragnar] Kor
\role{K6}[Flemming] Kor
\end{roles}

\scene{\textbf{Note fra digitalisator:} I rollebesættelsesoversigten står der en
  takkesanger på rollelisten for denne sang, men i sangen står der en ``stemme''
  som ``læser beløb og kommentarer op fra liste''.  Jeg har antaget at det er
  den samme person.  Der står ikke hvem der synger hvad.}

\scene{I anledning af de mange herlige penge, venlige mennesker og
  organisationer har doneret os.}

\begin{song}

\sings{Nogen}[omkvæd]
Tusinder af kroner
fornærmet, fortæret
tusinder af kroner
til at lave en revy

\sings{Nogen}[vers 1]
\begin{verbatim}
Fagrådet      3000 kr. til at bygge scene for -- under
              forudsætning af, at de bliver venligt
              omtalt. Det være hermed gjort !
Naturfagsbogladen  5000 kr til et lydanlæg
Kantineforeningen  3000 kr i overskud fra julefrokosten
              og 2000 kr af FLB's penge, som de allige-
              vel opbevarede
\end{verbatim}

\sings{Nogen}[mellemstykke og omkvæd]
Åh, tak for Jeres støtte
Tak for alt I gav
tak for det hele, fordi I ku' dele
vi takker Jer i dag

\sings{Nogen}[vers2]
\begin{verbatim}
Sidste 3 års revyer 4500 kr i opsparet overskud
DIKU's bestyrelse   Et sæt nogler  -  til låns  - under
              forudsætning af, at de bliver venligt omtalt.
              Det være hermed gjort.
HCØ's administration for lån af dette prægtige lokale
Publikum i år       kr
       - Tak skal I ha' -
\end{verbatim}
\end{song}

\end{document}

