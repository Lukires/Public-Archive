\documentclass[a4paper,11pt]{article}

\usepackage{revy}
\usepackage[utf8]{inputenc}
\usepackage[T1]{fontenc}
\usepackage[danish]{babel}


\revyname{DIKUrevy}
\revyyear{2015}
% HUSK AT OPDATERE VERSIONSNUMMER
\version{1.0.1}
\eta{$5.5$ minutter}
\status{Nogenlunde færdig}

\title{Universitetet}
\author{Mikkel Storgaard, Simon Linneberg, Niels, Julie, Troels}

\begin{document}
\maketitle

\begin{roles}
  \role{V}[Simon] Vært der har stribet jakkesæt, grim butterfly, og er livstræt
  \role{U}[Sebastian] Ekstremt støvet underviser
  \role{R0}[Nanna] Russen Nanna
  \role{R1}[Ejnar] Russen Ejnar
  \role{R2}[Jonas] Russen Jonas
  \role{R3}[Peter] Russen Peter
  \role{B0}[Spectrum] Bureaukratiet, har en robotagtig og sjæleløs stemme
  \role{B1}[Andreas] Bureaukratiet, har en robotagtig og sjæleløs stemme
  \role{M}[Mia] Minister
  \role{X}[Troels] Instruktør
\end{roles}

\begin{props}
\prop{Airhorn}[]
\prop{Pistol}[]
\prop{Jakkesæt hvori både B0 og B1 kan være}[]
\prop{2 slips, syet sammen de mødes}[]
\prop{Øl}[]
\prop{AV: Jingle-lyd}[]
\prop{Whiteboard med hjul + markers}[]
\prop{AV: Skakspil}[]
\prop{En bunke A4-papir}[]
\end{props}


\begin{sketch}

\scene{V og R0-R3 starter inde.}

\says{V} Velkommen til studiestart.

\says{V}[stoneface] Vi har tre smilende veloplagte bogfriske russer,
som er spændte på at starte her på vores studie.

\scene{R0-3 vinker energisk til publikum.}

\says{R0-3} Heeeej, jubiii, hurra!!

\scene{R0 og R1 highfiver, R3 ser dette, og forsøger at highfive V, som
ignorerer det totalt.}

\scene{V sukker og går træt hen til R0.}

\says{V} Nanna, hvad er så dine drømme med studiet??

\scene{V puffer mikrofonen op i ansigtet på R0.}

\says{R0} Da jeg fyldte fem år, så jeg et dobbelt-stjerneskud, som lyste hele
nattehimlen op.  Siden den aften har jeg været dybt fascineret af universets
gåder, og det blev mig fuldstændig klart, at mit liv skulle vies til at stille
og besvare de store spørgsmål om universets dybeste hemmeligheder. \act{ånder
ind}

\says{R0} Jeg drømmer om,..

\act{V båtter i sit horn.}

\says{V} Nå, det var sørme ærgerligt!  Det var fremdriftshornet der
lød der.  Ud med dig.

\says{R0} Jamen jeg er jo ikke engang startet endnu :( :(

\says{V} Nå, men det er altså ikke mig der laver reglerne.

\scene{V skyder R0.}

\scene{\textbf{Pistolskud}}

\says{V} Hvad med dig?

\says{R1} Jeg drømmer om at revolutionere det paradigme, vi i dag kender som
funktions-oriente-..

\scene{V hæver faretruende sit horn og stirrer træt.}

\says{R1} \act{rømme} JEG MENER jeg vil være hurtigt færdig og kode
Java for Kapitalen!

\scene{V sukker og sænker hornet igen.}

\scene{V giver mikrofonen videre til R2, som kigger bekymret på fremdriftshornet.}

\says{R2} Jeg vil bare gerne læse bøgerne sådan helt overfladisk, og
så gå til multiple-choice-eksamener.

\scene{R2 giver mikrofonen videre til R3.}

\says{R3} Jeg vil bare give generiske svar til generiske opgaver.

\says{V}[læser mekanisk op] Velkommen, nye studerende på Københavns
Universitet.  Giv hånd til de studerende.  \act{Kigger foragteligt på
  R0 og læser så videre.}  Velkommen til de bedste, hårdeste og
sjoveste år af jeres liv.

\scene{\textbf{AV: KU-jingle}}

\says{V} Vi starter med Rusturen!

\scene{R jubler!}

\scene{V rækker R1 en dejlig kold GT. R1 får den lukket op, og skal lige til at
drikke en tår af den.  Da tager V den ud af hånden på R1.}

\says{V} Og det var så rusturen.

\scene{Et whiteboard og U kommer ind på højre side af scenen.}

\says{V} Velkommen til første forelæsning.

\scene{\textbf{AV: KU-jingle}}

\says{V} Hvis I bare følger med i vores moderne, nytænkende,
involverende og udadvendte undervisningsmetoder, så skal I nok bestå!
\act{Sukker}

\scene{Den tøre mand begynder at forelæse yderst monotont om et eller
  andet.  Man kan ikke tyde mandens håndskrift, og han taler meget
  utydeligt, og står som regel med ryggen til publikum.}

\scene{R3 har mistet opmærksomheden, mens R1-2 finder nogle ark frem, og begynder at kradse
tilfældige noter ned.}

\scene{V trækker fremdriftshornet frem og båtter.  R0 og R1 giver
  deres noter til V.}

\says{V}[til R3] Det var fremdriftshornet der lød, du har ikke været til
eksamen, og er smidt ud.

\says{R3} Eksamen?  Jammen det har jeg da ikke hørt noget om!

\says{V} Det stod på Absalon under studiebeskeder.

\says{R3} Hvad er Absa-

\scene{V skyder R3.}

\scene{\textbf{AV: Pistolskud}}

\says{V}[sukker] Det er 2. og sidste uge i jeres studieforløb.  I
denne uge skal I vinde over Bureaukratiet i skak, for at få skrevet
karakterene fra jeres sidste opgave ind.

\scene{\textbf{AV: KU-jingle}}

\scene{Der vises et spil skak og Bureaukratiet træder ind - det er to
  personer der taler samtidigt og i samme jakkesæt}

\says{V}[til R1] Vi starter med dig.

\says{R1} Jeg vil gerne rykke min bonde 2 felter frem.

\scene{På skakspillet rykkes en bonde frem.}

\says{B} Du skal udfylde formular 76a i 3 eksemplarer, med underskrift fra
studieleder, din mor og statsministeren for at få forhåndsgodkendt dit ryk af
denne bonde.

\scene{Bonden flyttes tilbage, B rykker sin dronning gennem alle skakbrikker og
slår R1's konge.  B har vundet.}

\says{R1} Hvad! det kan du da ikke, det er imod reglerne!

\says{B1} Vores SharePoint er desværre ikke opdateret med reglerne fra det 19. århundrede.  

\says{B0} Vi har ganske vist printet dem ud, men vi venter stadig på at scanne dem ind, så vi har dem på digital form. 

\says{B1} Du kan kontakte Green Lighthouse mellem 10:00 og 10:15 

\says{B0} på ulige datoer 

\says{B1} hvor planeterne står på række, 

\says{B0} hvis du ønsker ydderligere information eller rådgivning til dit studieforløb. 

\says{B0+B1} Green Lighthouse er meget miljøvenlig.

\scene{V båtter i hornet.}

\says{V} Det var ærgeligt.  Det lader til at du ikke har bestået.

\scene{g skyder R1.}

\scene{\textbf{AV: Pistolskud}}

\scene{Vigtigt: Ejnar må ikke dø
  rigtigt her, for jeg skal også slå ham ihjel scenere.  V peger på R2
  med pistolen.}

\says{V} Det er din tur, Jonas.  Du skal spille, Jonas, og du skal bestå.

\says{R2} HJÆÆÆÆLP!  Jeg vil ikke! Kan jeg ikke droppe ud!

\says{B0+B1} Nej!  

\says{B0} Har du først tilmeldt dig et kursus, skal du bestå.

\says{B1} Ellers har du intet at leve for. 

\says{B0+B1} Det siger reglerne selv.

\says{R2} Men jeg har ikke tilmeldt mig, det sidste jeg husker er at
jeg gik i børnehaven, og så var der en eller anden reform, så jeg
skulle begynde på universitetet.  Jeg fik 2 dage til at gennemføre
folkeskolen, og min gymnasietid bestod af en multiple choice test med
én svarmulighed.

\says{V}[læser op] Fødes du i Danmark, skal du arbejde.

\says{R2} Men... kan du ikke bare lade mig bestå?  Så kommer jeg endnu
hurtigere igennem.

\says{V}[ser pludselig overrasket ud] Hov!  Det... har du jo ret i.
Jeg ved ikke hvorfor jeg ikke havde tænkt på dette noget før.  Det gør
det jo straks meget nemmere.  Tillykke, du har en Ph.D. i datalogi.

\says{R2} Men jeg studerede østrigsk eskimologi.

\says{V} Ikke mit problem.

\scene{Pinlig pause.}

\says{R2} Sååæh, får jeg et bevis eller sådan noget?

\says{V}[overrasket] For hvad?

\says{R2} Var jeg ikke lige blevet Ph.D.?

\says{V} Det står der ikke noget om i min database.

\scene{V kigger på sit armbåndsur.}

\says{V} Nå, nu har du vist ikke mere tid.

\scene{V båtter i hornet og skyder R2.}

\scene{\textbf{AV: Pistolskud}}

\scene{\textbf{AV: KU-jingle}}

\scene{B0+B1 forlader scenen.}

\scene{Ministeren kommer ind på scenen og ser alle de døde studerende.}

\says{V} Velkommen, undervisnings- og forskningsminister Sofie Carsten
Nielsen.

\says{M} Jeg kan se at den nye reform er en klar succes!  \emph{Ingen}
forsinkede studerende.

\says{V} Hov, var du ikke selv syv år om din uddannelse?

\scene{Båt i fremdriftshornet, V skyder M.}

\scene{\textbf{AV: Pistolskud}}

\scene{LYS NED.}
\end{sketch}
\end{document}

% Hækkeløb hvor hækkene er fritidsaktiviteter

% Telefon-afvisnings-disciplin

% 100 meter specialeskrivning
