\documentclass[a4paper,11pt]{article}

\usepackage{revy}
\usepackage[utf8]{inputenc}
\usepackage[T1]{fontenc}
\usepackage[danish]{babel}


\revyname{DIKUrevy}
\revyyear{2015}
% HUSK AT OPDATERE VERSIONSNUMMER
\version{1.8}
\eta{$1.5$ minutter}
\status{Shhhhhh!}

\title{Datamat-tysseren}
\author{Camilla, Nana, Lasse, Andreas, I kan også bare kalde mig Simon, og Phillip ...ellers ved vi det ikke}

\begin{document}
\maketitle

\begin{roles}
\role{DT}[Niels] Datamat-tysseren
\role{V1}[Maya] Tilsynsførende eksamensvagt
\role{V2}[Brandt] Eksamensvagt
\role{E1}[Nanna] Eksaminand
\role{E2}[Jonas] Eksaminand
\role{E3}[Simon] Eksaminand
\role{X}[Klaes] Instruktør
\end{roles}

\begin{props}
\prop{Datamat}[Person, der skaffer]
\prop{Blæserstøj (AV)}[Pilgård]
\end{props}

\begin{sketch}

\scene{Eksamenslokale. Alle eksaminander sidder og taster ivrigt på deres datamater.}

\scene{E1s datamat (D) larmer enormt meget. Blæserstøjen skal kunne høres i baggrunden gennem hele sketchen, og skrues op, når D har replikker.}

\says{D} *whhhhhhfffff*

\scene{V1 og V2 ser bekymrede til.}

\says{V2} Den blæser larmer da frygteligt meget.

\says{V1} Ja, vi må hellere tilkalde... 

\says{V1+V2} DATAMAT-TYSSEREN!

\scene{V1+V2 springer frem på scenen med armene i vejret, spotlys op på datamat tysseren (til højre på scenen, set fra TeXnikken)} 

\says{DT} \act{DT tysser på lyset} SHHHHHHHHHHHHHHHH!

\scene{Spotlys forsvinder igen}

\says{V1+V2} DATAMAT-TYSSEREN!

\scene{V1+V2 springer længere frem på scenen med armene i vejret, spotlys op på datamat tysseren (til højre på scenen, set fra TeXnikken)} 

\says{DT} \act{DT tysser på lyset} SHHHHHHHHHHHHHHHH!

\scene{Spotlys forbliver tændt, DT undrer sig og opdager V1+V2}

\scene{DT springer frem på scenen med stor fanfare. Han har høreværn og kappe på, tys-logo på trøjen, og datalog-mave (pude under trøjen).}

\says{DT} \act{tysser på bandet/lyden} SHHHHHHH

\scene{Eksaminanderne bliver forstyrrede, og kigger meget irriterede på ham.}

\scene{DT letter på høreværnet og lokaliserer hurtigt problemet. Han haster hen til E1s datamat og læner sig målrettet hen over den.}

\says{DT}[råber] SHHHHHHHHHH!

\scene{Datamaten larmer ufortrødent videre.}

\says{D} *whhhhhhfffff*

\says{DT} Shhhhhhhh!

\says{D} *whhhhhhfffff*

\scene{E1 kigger bekymret på DT. De andre eksaminander arbejder koncentreret på deres opgave.}

\scene{V2 går rundt og deler papirer ud til eksaminanderne, selvom de tydeligvis ikke har brug for det.}

\scene{V1 ser skeptisk på E2s skærm.}

\says{V1}[jysk accent] Den der linje kode der, hvad gør den?

\says{E2} Øeh... Det' en... det' en lukke-parentes.

\says{V1} Ej, det' ikke et godt valg...

\scene{DT løfter på høreværnet}

\says{DT}[henvendt til V1] Shhhhhh!

\says{D} *whhhhhhfffff*

\says{DT}[henvendt til D] Shhhh!

\scene{DT bliver mere og mere desperat. Han kigger sig diskret omkring, for at sikre sig at der ikke er nogen der holder øje med ham.}

\says{D} *whhhhhhfffff*

\scene{DT hiver puden frem fra sin trøje, og presser den nænsomt mod E1s skærm. E1 ser bekymret til.}

\scene{DT trykker på power-knappen, idet han endnu en gang tysser blidt.}

\says{DT} Shhh!

\says{D} *whh....*

\scene{D stopper med at larme. DT slipper powerknappen, løfter høreværnet, nikker selvtilfreds, og tager puden væk igen.}

\says{E1}[forfærdet] Men, men.. Min...

\says{DT} SHHHHHHHHH!

\scene{Lys ud}


\end{sketch}
\end{document}
