\documentclass[a4paper,11pt]{article}

\usepackage{revy}
\usepackage[utf8]{inputenc}
\usepackage[T1]{fontenc}
\usepackage[danish]{babel}


\revyname{DIKUrevy}
\revyyear{2015}
% HUSK AT OPDATERE VERSIONSNUMMER
\version{1.1}
\eta{$3$ minutter}
\status{Næsten færdig}

\title{Jurasketchen}
\author{Amanda, Brainfuck, Phillip, Troels, Mikkel}

\begin{document}
\maketitle

\begin{roles}
\role{D}[Bitre-Mikkel] Datalog
\role{R}[Brandt] Jurarobot
\role{S}[Jonas] Stævnet (Preben)
\role{V0}[Simon] Vagt
\role{V1}[Andreas] Vagt
\role{AP}[Ejnar] Antipiratgruppens advokat
\role{X}[Niels] Instruktør
\end{roles}

\begin{props}
\prop{Tæppe til at klæde R over.}[Person, der skaffer]
\prop{Jyske Lov, Danske Lov og Karnov-bøger som man kan stå på}[Person, der skaffer]
\prop{Datamat til at hælde kaffe på, som der kan komme røg ud af.}[Person, der skaffer]
\prop{Kop med kaffe}[Person, der skaffer]
\prop{Pistol}[Person, der skaffer]
\prop{Pawels ``Kan du bevise det''}[AV]
\end{props}

\begin{sketch}

\scene{D står på scenen, mens R står under et tæppe bag ham.  R står allerede på bøgerne under tæppet.}

\says{D} Til trods for de konstante rygter, har DNA-undersøgelser vist at
jurister faktisk \emph{er} mennesker.  Og dette har haft katastrofale
konsekvenser for deres arbejde.  I op imod 28\% af retssager har en
advokat udvist medmenneskelighed.  For bare en uge siden, klagede en
advokat over at sin kaffe-latte var for varm.

\says{D} Men som med så mange andre fag, har juristerne bedt -- os -- dataloger om
at løse deres problemer.

\says{D} Og det giver god mening.  Datamater har ingen følelser.  Lad mig illustrere.

\scene{D hælder kaffe på en datamat.}

\says{D} Datamaten piver eller klager ikke.

\scene{Der kommer røg ud af datamaten.}

\says{D} Den dør bare.  Det er effektivisering.

\says{D} Datalogien har jo allerede effektiviseret økonomi, ved at gøre
aktiehandel så hurtig, at kun datamater kan være med.

\says{D} Det handler om algoritmer og hardware, ikke om mennesker og HCI.
Det kunne juristerne lære noget af.

\says{D} Projektet har taget 8 år, 5 år over deadline og 10 milliarder
over budget, men nu har vi resultatet.

\scene{D river et klæde af R, som var det en statue.}

\says{D} Må jeg introducere vores jurarobot 1.1!

\says{D} For ikke at gøre denne proces alt for mærkelig for befolkningen, så
har vi iklædt den som en jurist gerne skulle se ud i dag.

\scene{Robotten står på en bunke Karnov-bøger, oven på en bog hvor der stå ``Danske Lov'' og ovenpå en bog hvor der står ``Jyske Lov''.}

\says{D} Og som I kan se, hviler den på et juridisk grundlag.

\says{D} Lovtekst er som programmering.  Når en gerning er sket, fortages en dom.
For eksempel giver forræderi 16 år i fængsel.  Det er jo så simpelt.

\says{D} Robotten er også udstyreret med nogle standard juraspecifikke vendinger,
som man kan frembringe således:

\scene{D drejer på ryggen af R.}

\says{R} Skyldig.  \act{D forsætter.}

\says{R} Det må bero på en konkret vurdering.

\says{D} Og selvfølgelig...

\says{R}[efterligner Pawel] Kan du bevise det?

\says{D} Så tror jeg vist det er blevet tid til lidt QA.

\says{D} Under normale forhold ville en retssag have taget op til flere
uger før en dom var afsagt.  Men med en 64-bit RISC
processor har vi formået at reducere processen til blot en brøkdel.

\scene{Retssagen.}

\scene{AP kommer ind.}

\says{AP} Vi har et søgsmål!  Preben har downloadet Grand Theft Auto fem ulovligt!

\says{S (Preben) bliver slæbt ind af V*.}

\says{AP} Og da vi kan se at du er dårlig til GTA, var det jo sandsynligt at
du ville blive fanget.

\says{R} Givet statistikken for denne forbrydelse er det sandsynligt at den stævnede vil
begå butiksindbrud, stjæle biler og downloade yderligere materiale
uden tilladelse.

\says{R} Det giver 20 år ekstra i fængsel.

\says{S} Hvad!?

\says{R} Jeg kan regne ud, at du kun kommer til at leve 19 år endnu.  Det er derfor ikke
økonomisk forsvarligt at holde dig i live.

\scene{R tager en pistol frem og skyder S.}

\scene{Lys ud.}

\end{sketch}
\end{document}
