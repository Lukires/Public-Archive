\documentclass[a4paper,11pt]{article}

\usepackage{revy}
\usepackage[utf8]{inputenc}
\usepackage[T1]{fontenc}
\usepackage[danish]{babel}


\revyname{DIKUrevy}
\revyyear{1978}
% HUSK AT OPDATERE VERSIONSNUMMER
\version{1.0}
\eta{$n$ minutter}
\status{Færdig}

\title{Hvem vil købe et førstedelskursus?}
\author{EM, JBC}

\begin{document}
\maketitle

\begin{roles}
\role{S}[] Sælger
\role{a}[] Lærer
\role{b}[] Lærer
\role{c}[] Lærer
\role{d}[] Lærer
\role{e}[] Lærer
\role{f}[] Lærer
\role{g}[] Lærer
\role{h}[] Lærer
\role{i}[] Lærer
\role{j}[] Lærer
\role{k}[] Lærer
\role{l}[] Lærer
\role{m}[] Lærer
\role{n}[] Lærer
\role{o}[] Lærer
\role{p}[] Lærer
\end{roles}

\begin{props}
\prop{Rekvisit}[Person, der skaffer]
\end{props}


\begin{sketch}

  \scene{Et bord med vore lærere omkring.  En sælger i passende
    mundering er ved at indlede sin salgstale.}

  \says{S} Før jeg begynder på en detaljeret gennemfang at mit
  compagnys produkter, vil jeg godt sige, at jeg "= og hele mit firma
  "= er meget happy for at have fået lov til at komme her til
  datalogisk institut, og jeg er proud over, at hele lærerforsamlingen
  vil lisne til, hvad jeg har at sige.

  Vores firma lever af at sælge 1. delskurser til vore kunder.  Ikke
  blot kurserne, vi er også leveringsdygtige i studiekoordineringer,
  bifag (de har måske set nogle af vores reklamer) forelæsninger, ja
  alt til faget hørende.  Efter denne lille indledning, vil jeg så
  sige velkommen til vores workshop.

\says{a} Voksok?  Hvad betyder det?

\says{b} Det er nok en butik, hvor man sælger arbejdere.

\says{S} Dagens emne er fornyelse af undervisningsmateriale på Deres
undergraduate kurser.

\says{c} Er vi nu også undergraduerede?  Jeg troede kun vi var underbemandede.

\says{S} Vores organisation har lavet et program specielt sammensat
til deres behov.  Jeg kan belyse det med en case.

\scene{Planche:}

\begin{verbatim}
case kursus of
  dat0: instroduction to computing
  dat1: advanced computing
  dat2: management information systems
esac
\end{verbatim}

\says{S} Som det fremgår, lægger vi op til en radikal sanering af
deres virksomhed.

\says{d} Det kan ikke blive før i '85, for før er mine transparenter
ikke slidt op.

\says{e} Og jeg har allerede lavet eksamensopgaverne til næste år; jeg
tager blot dem fra i år og retter fejlene.

\says{f} På mit kursus har en mikrofilm af en afskrift af en kopi af
original til de reviderede autoforelæsninger.  Så der kan vi jo ikke
lave meget om.

\says{S} Som svar på et udpræget forbrugerønske har vi endvidere
prøvet på at lave det hele noget mere jordnært og dagligdags.  For
eksempel kan man jo meget naturligt komme ind på hunde-problematikken
når man skal behandle pointere, ligesom der er mange lighedspunkter
mellem en førerhund og en oversætter.

\says{g} Jeg er hunderæd for, at disse her planer er lidt for vov'ede.

\says{h} Desuden kan vi da ikke anvende en så løs og udenlandsk terminologi.

\says{Spredt mumlen.  Lærerne synes ikke om oplægget.}

\says{S} Jeg kan forstå på d'personers holdning, at De ville
foretrække en prøve på vores gamle danske terminologi. \act{Trækker en
  flaske Gammel Dansk op og skænker.}

\says{i} Det var måske ikke så tosset at indøve nogle bottom-up varianter.

\says{S} Så vil det måske også interessere Dem at se vores
undervisningsmateriale til illustration af matricer? \act{Haler en
  kasse øl frem.}

\says{j} Den matrix giver en god forståelse i flydende regning.

\says{k} Bare det nu ikke er en tynd matrix?  Kan du se om pladserne
er tomme?

\says{l} Det må gøres til genstand (!) for en algoritmeanalyse.

\says{m} Det skulle ikke være svært at finde flaskehalsen.

\says{S} Ja, ja, men den kan meget mere.  Her kan vi også øve os i at
fjerne etiketter \act{demonstrerer på en ølflaske} og programmøren kan
jo dårligt hoppe nogen steder hen uden etiketter.

\says{n} Glimrende, jeg kan allerede nu se resultater af
undervisningen.  Jeg er for eksempel ikke nu i stand til at hoppe
nogen steder.

\says{S} Ofte vil det være nødvendigt med dobbelt buffer for at få
et tilstrækkeligt hurtigt gennemløb i systemet.  \act{Haler endnu en
  kasse øl frem.}

\says{o} Han er nu slet ikke så tosset, ham den sælger.  Se nu hvordan
jeg kan sortere de røde fra de grønne hutigere end du kan søge
$O(n\log(n))$.

\says{p} Så kan du måske samtidig lære at kende forskel på en venstre
Tuborg og en højre Tuborg?

\says{S} Mens d'personer lader indtrykkene synke til bunds, vil jeg
tillade mig en hurtig og koncis gennemgang af samtlige emner.

\end{sketch}
\end{document}
