\documentclass[a4paper,11pt]{article}

\usepackage{revy}
\usepackage[utf8]{inputenc}
\usepackage[T1]{fontenc}
\usepackage[danish]{babel}


\revyname{DIKUrevy}
\revyyear{1978}
\version{0.1}
\eta{$n$ minutter}
\status{Færdig}

\title{Terminalsangen}
\author{HHK, JBC, EM}
\melody{Shu-Bi-Dua: ``Vuffelivov''}

\begin{document}
\maketitle

\begin{roles}
\role{S}[HHK] Sanger
\end{roles}

\begin{song}
  Jeg har en skærm med mange taster
  én for hvert symbol
  Og bagved sidder lysintensiteten.
  Dens ledning har mange tråde,
  én til hver sin bit
  plus en ekstra en til pariteten.

  Når man har venner og kærester, så er man normal,
  Men de ta'r tiden fra mig og min terminal.

  Jeg er koblet via DIXI,
  når DIXI ellers vil,
  og der plads på centrets multiplekser.
  Når jeg har lyst så kan jeg sidde
  og lege natten lang
  med RECKUs mange programmelkomplekser.

  Når man har venner og kærester, så er man normal,
  Men de ta'r tiden fra mig og min terminal.

  Jeg kør' på en maskine
  der klarer tusind jobs,
  Selvom de ta'r syv lange og syv brede.
  CAU'en har den to af
  og det er vældig smart:
  Én til hvis den anden sku' vær' nede.

  Når man har venner og kærester, så er man normal,
  Men de ta'r tiden fra mig og min terminal.

  Jeg spiller skak og kryds og bolle
  Den hele lange nat,
  Det er nu temlig trist man ingen kender.
  For selv om den er dejlig,
  Så er den datamat,
  Nu kun et surrogat for menn'ske-venner.

  Når man har venner og kærester, så er man normal,
  Og har det bedre end mig med min terminal.
\end{song}

\end{document}

