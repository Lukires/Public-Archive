\documentclass[danish]{article}
\usepackage{revy}
\usepackage[utf8]{inputenc}
\usepackage{babel}
\usepackage{a4wide}

\title{Kvindelogikprogrammering}
\author{Uffe og folk}

\version{4} % HUSK AT AJOURFØRE VERSIONSNUMMER!!
\status{næsten færdig -- skal skrives til 2 personer} % ...OG STATUS!!
\eta{7 min.}

\revyyear{2000}

\begin{document}
\maketitle

\begin{roles}
  \role{F1}[Uffe] Forelæser
  \role{F2}[Christoffer] Forelæser
\end{roles}

\begin{sketch}
  
\says{F} Vi har I mange år forsøgt at tiltrække flere kvinder her på datalogi.
Det har vi gjort fordi\ldots well\ldots \act{fjerner ``savl'' fra hagen} fordi
det synes vi kunne være fedt.

\says{F} Nu ligger DIKUs udspil så endelig klar. Ved at tage udgangspunkt i
logikprogrammering har vi opfundet begrebet ``kvindelogikprogrammering'' -- også
kendt som ulogisk programmering.

\says{F} Vi har konkretiseret dette i en pendant til sproget Prolog. Det har vi
valgt at kalde ``Kontralog''.

\says{F} Kontralog er dog en smule anderledes end de fleste programmeringssprog
vi kender. Tag for eksempel tildelingssætningen ``a lig med 3''. Denne sætning
vil i nogle få tilfælde have effekten at a tildeles værdien 3. I Kontralog vil a
dog i de fleste tilfælde blive tildelt en vilkårlig anden værdi. Enkelte gange
kan a sågar helt forsvinde.

\says{F} Ligeledes findes naturligt nok udviklinger af de velkendte betingede
sætninger. Vores ``if then else'' konstruktion får en tredelt udgave i
konstruktionen ``if then else otherwise''. Dette kan dog udskrives til en ``if
perhaps'', der kan følges vilkårligt senere i programmet af en matchende
``unless otherwise''.

\says{F} Det skal nævnes at der naturligt nok ikke findes variable i Kontralog,
men kun konstanter. Sprogets indeterministiske natur gør dog dette til en mindre
hindring end man ellers skulle tro. Alt er naturligt nok dynamisk og man finder
derfor ingen statiske erklæringer -- enhver værdi er af sin natur volatile.

\says{F} De primitive typer i Kontralog er blevet forenklet i forhold til gængse
programmeringssprog. Alle de ``besværlige typer'' er afskaffet, såsom heltal og
tegn. Tilbage er kun flydende kommatal til at opbevare de mere reelle værdier og
så selvfølgelig strenge, ja til tider ``helt vildt strene'', som til gengæld er
udvidet til at kunne klare vilkårligt lange samtaler.

\says{F} Kontralog indeholder selvfølgelig også en videreudvikling af tidens
trend: objektorienteret programmering. Objekterne bliver til subjekter og de har
hverken metoder eller egenskaber, men fornemmelser og følelser. Disse pakkes
naturligt nok ind, i visse tilfælde, og man skal således ikke regne med at få lov
til at give argumenter til et subjekts private følelser.

\says{F} For at føre parallellen fra objekterne helt over i subjekterne har et
subjekt brug for hvad der svarer til konstruktør og destruktør. Her har vi
indført en kontrollør og en inspektør -- de såkaldte ``uniforme''. Hvis både
kontrollør og inspektør mangler er der mulighed for såkaldte suspekte subjekter.

\says{F} Alt dette tænkes kørt på en virtuel maskine -- hvad ellers? -- hvor
\emph{alt} fortolkes. Ofte bliver det endda kun partielt evalueret. Denne
maskines byggesten -- dvs. bytekoden eller maskinkoden om man vil -- har tre
basisfaktorer: 1) ``fordi jeg siger det'', 2) ``derfor'' og 3) catch-all-casen
``sådan er det bare''.

\says{F} Denne virtuelle maskine er desuden udstyret med en uendelig stor cache.
Ved enhver beslutnings\-sætning sammenlignes den igangværende evaluering med
tidligere evalueringer, og man kan ofte ende med resultatet ``bare ærgerligt''
eller ``der kan du bare se!'' Enhver reference til denne cache fra programmørens
side har i praksis vist sig at få fortolkeren til at gå ned. Cachen kan ikke
flushes eller på anden måde slettes -- alt bliver husket!

\says{F} Det eneste begreb som har overlevet transformationen fuldstændigt
intakt er ``kommando promptet''. \emph{Det} var faktisk noget kvinderne var
meget glade for.

\says{F} Enkelte datastrukturer er også overlevet næsten intakt. F.eks. har
træet ikke ændret sig meget, selvom kvinderne måske synes det er noget rod og
prøver at kante sig uden om. De slår knuder. Der findes stadig tre forskellige
slags gennemløb, de har bare fået andre navne: præ(menstruel)order, out of order
og, sidste, men ikke mindst, postordre\ldots også kendt som TV-shop.

\says{F} Listerne overlever også næsten intakt, selvom vi endnu ikke har
afdækket fuldt hvad konsekvensen bliver af at erstatte alle
sorteringsalgoritmerne med randomiseringsalgoritmer.

\says{F} Et af de endnu uafdækkede problemområder i kvindelogikprogrammering
generelt er at vi oplever periodemæssigt svingende køretidskompleksitet, hvor
der oftest er udelelig adgang til hele subjektet, og udsultning kan forekomme.
Vi ser dog frem til at se hvilke fordele kollektiv garbage collection på
toilettet kan give os, og glæder os til at kaste os over problemet med ``De
Anorætiske Filosoffer''.

\says{F} Til slut skal nævnes at vi naturligvis i hele processen har været
stærkt inspirerede af vores matematiske kolleger, der har forsøgt at definere
kvinden som ``R pånær Q''\ldots altså alt det irationelle. De arbejder netop nu
på en hypotese om ``C pånær Q''\ldots altså alt det komplekse og irrationelle.
Mon ikke de også når frem til at det er ``C pånær R''? Altså alt det komplekse
og imaginære der ikke er reelt.

\says{F} Vores næste projekt er så pendanten til Miranda: ufunktionel
programmering i Amanda. Tak for i aften!

\scene Tæppe

\end{sketch}

\end{document}
% Local Variables: 
% mode: latex
% TeX-master: t
% End: 
