\documentclass[danish]{article}
\usepackage{revy}
\usepackage[utf8]{inputenc}
\usepackage{babel}
\usepackage{a4wide}

\title{MCSE}
\author{Rune og Uffe (og Adam og Mike)}

\version{3} % HUSK AT AJOURFØRE VERSIONSNUMMER!!
\status{færdig} % ...OG STATUS!!
\eta{4+2+3 min.}

\revyyear{2000}

\begin{document}
\maketitle

\begin{roles}
  \role{R}[Sidsel] Robot (kvindelig?)
  \role{M}[Uhd] Mand
  \role{S}[Jan] Slidefører
\end{roles}

\begin{props}
  \prop{Møntindkast} på robotten
  \prop{OK/Cancel knap} til robotten
  \prop{1-3 knapper} til robotten
  \prop{Lyd} af stopur
  \prop{12 slides} nummereret fra -1 til 10; specificeret i teksten
\end{props}

\begin{sketch}
  
\scene Sketchen her er delt op i 3 dele. I mellem hver del skal der være noget
andet. Eks. en sang eller en musical.

\scene Fælles setup:

\scene På scenen står en overhead projektor. Der er en slide med et Microsoft
logo på (slide -1). Ved siden af står en robot (R).

\scene ------------------------------------------------------------------------------

\scene 1. del

\scene En mand (M) kommer ind og går hen til R.

\says{R} Velkommen til Microsoft MCP-certifikation. Indkast kr. 1495 for at
trække en test.

\scene M indkaster beløbet.

\says{R} Microsoft Licensaftale. Når du klikker OK, indvilger du i at sælge din
sjæl til Microsoft de næste 666 år. Klik OK, for at fortsætte. \act{holder OK og
  Cancel knapperne frem}

\scene M klikker OK. R tager knapperne til sig igen.

\says{R} Testen tager netop 10 minutter at udføre. Når du klikker på OK, starter
stopuret. \act{holder knapperne frem}

\scene M klikker OK. R tager knapperne til sig igen. Lyd af stopur.

\says{R} Læn dig venligst tilbage imens Microsoft Selftest software opretter en
spørgsmåls-database. Hvis du består testen, er du klar til at være
IT-administrator i det pulserende erhvervsliv. Advarsel: Nogle af spørgsmålene
kan godt være lidt svære.

\scene Der går lidt tid, mens R laver nogle bevægelser der symboliserer at den
er ved at opbygge en database.

\scene R skifter slide til Blue Screen Of Death (slide 0) og falder sammen (a la
robotterne i SW ep 1)...

\says{M} FUCK! Jeg hader Microsoft \act{genstarter R}

\says{R}[skifter tilbage til slide -1] Velkommen til Microsoft MCP-certifikation.
Indkast kr. 1495 for at trække en test.

\says{M}  Nu igen!? Nå, firmaet betaler... \act{indkaster beløbet, test går igang}

\says{R}[skifter til slide 1] Spørgsmål 1 ud af 1209: Du skal til at starte din
Windowssession. Hvad vil du trykke? \act{holder 1-3 knapperne frem}

\scene Slide 1 læses højt:

\begin{enumerate}
\item   Ctrl
\item   Alt-Delete
\item   Alle ovenstående taster
\end{enumerate}

\says{M}[totalt panisk] FUCK! Hvad fanden er det nu... det har jeg læst om et
eller andet sted... \act{satser på 3}

\says{R}[skifter til slide 2] Korrekt. Spørgsmål 2 ud af 1209: Du har netop
gjort klar til at race ud af den verdensomspændende informationsmortorvej.
Hvilket ikon vil du klikke på for at starte? \act{holder 1-3 knapperne frem}

\scene Slide 2 læses højt:

\begin{enumerate}
\item (som ikon) horsefuck.mpeg
\item (som ikon) ILOVEYOU
\item Ingen af ovenstående -- bilservicen er sat til autorun
\end{enumerate}

\says{M} Hmm, den service har jeg godt nok ikke hørt om før, så mon ikke det er
sådan en Windows bare kører uden at man ved det? \act{trykker lidt forsigtigt på
  3}

\says{R}[skifter til slide 3] Korrekt. Spørgsmål 3 ud af 1209: Bilservicen
starter ikke automatisk. Hvad gør du? \act{holder 1-3 knapperne frem}

\scene Slide 3 læses højt:

\begin{enumerate}
\item Gi'r op \emph{udtales ``gearer op''}
\item Installerer ny driver
\item Trykker Ctrl-Alt-Delete
\end{enumerate}

\says{M} Ahh! Det er jo et trickspørgsmål... Bilen har da ikke brug for en
driver... \act{trykker 3}

\says{R}[skifter til slide 4] Korrekt. Spørgsmål 4 ud af 1209: Du ræser nu ud af
den verdensomspændende informationsmortorvej. Du kommer til en rundkørsel, så
der er flere ruter at vælge imellem. Hvad gør du for ikke at gå i uendelig
løkke? \act{holder 1-3 knapperne frem}

\scene Slide 4 læses højt:

\begin{enumerate}
\item Kører en korteste vej-algoritme og lader derfor være med at køre mere end
  en gang rundt
\item Satser på at routeren vælger den hurtigste vej ud
\item Trykker Ctrl-Alt-Delete og håber, at rundkørselen er forsvundet af sig
  selv efter genstart
\end{enumerate}

\says{M} Puh, nu bliver de svære... \act{trykker 3, forsigtigt}

\scene Slut på 1. del

\scene ------------------------------------------------------------------------------

\scene 2. del

\scene M er allerede på scenen. Han ser en smule træt ud.

\says{R}[skifter til slide 5] Korrekt. Spørgsmål 102 ud af 1209: Pludselig ser
du en storbarmet blondine-blaffer med et skilt, hvorpå der står ``ILOVEYOU''.
Hvad gør du? \act{holder 1-3 knapperne frem}

\scene Slide 5 læses højt:

\begin{enumerate}
\item Kører forespørgselen MenerDuMig.sql
\item Siger til hende: ``jeg er til det lidt avancerede, så hvorfor stopper du
  ikke noget op i mit sikkerhedshul?''
\item Trykker Ctrl-Alt-Delete
\end{enumerate}

\says{M} Der er da ikke nogen sikkerhedshuller i Windows 2000! Pfff!
\act{trykker beslutsomt 3}

\says{R}[skifter til slide 6] Korrekt. Spørgsmål 103 ud af 1209: Du kører
ualmindeligt langsomt og bliver overhalet af fire clockcykler. Hvad gør du?
\act{holder 1-3 knapperne frem}

\scene Slide 6 læses højt:

\begin{enumerate}
\item Bliver sur og overclocker processoren til 9,82 $m \div s^2$a
\item Installerer Linux og giver dem baghjul
\item Trykker Ctrl-Alt-delete
\end{enumerate}

\says{M} Hm. Linux lyder da ret fedt. Det prøver jeg... \act{trykker 2}

\says{R}[skifter til slide 7] Ikke korrekt! Det rette svar var: 3. Trykker
Ctrl-Alt-Delete. Spørgsmål 104 ud af 1209: Pludselig hører du en sagte hvisken
fra forrest i bilen. Hvad er det? \act{holder 1-3 knapperne frem}

\scene Slide 7 læses højt:

\begin{enumerate}
\item En silent-proces, der gør opmærksom på sig selv
\item Windowshviskeren
\item Ligemeget, jeg trykker Ctrl-Alt-delete
\end{enumerate}

\says{M} Jeg synes jeg begynder at ane et mønster... \act{trykker 3}

\scene Slut på 2. del

\scene ------------------------------------------------------------------------------

\scene 3. del

\scene M er allerede på scenen. Han ser meget træt ud. Hvis muligt er han også
blevet sminket gammel, dvs. mel i håret og rynker i ansigtet. Når han bevæger
sig er det langsomt og rystende.

\says{R}[skifter til slide 8] Spørgsmål 1206 ud af 1209: Du kører ind i en zone
med ubalancerede vejtræer. Pludselig vælter et af dem, og du er lige ved at
støde ind det. Du prøver at bremse, men får en fejl-kode 12BC4F-0C8A. Hvad
betyder denne fejl? \act{holder 1-3 knapperne frem}

\scene Slide 8 læses højt:
        
\begin{enumerate}
\item Bremseservicen er ikke installeret
\item Ligemeget, bilservicen crasher sikkert alligevel, før jeg når at støde ind
  i træet
\item Ligemeget, jeg trykker Ctrl-Alt-delete
\end{enumerate}

\says{M} Det her er da ret åndssvagt? \act{trykker 3}

\says{R}[skifter til slide 9] Korrekt. Spørgsmål 1207 ud af 1209: Desværre
kolliderer du, så maskinen slukker. Da du starter op igen, virker Windows ikke.
Hvad gør ud? \act{holder 1-3 knapperne frem}

\scene Slide 9 læses højt:

\begin{enumerate}
\item Kører SCANBIL og håber, at den retter fejlen
\item Geninstallerer windows
\item Trykker Ctrl-Alt-Delete og håber maskinen virker nu
\end{enumerate}

\says{M} Ah, den er jo tricky. Det er jo nok 2... \act{trykker 2}

\says{R}[skifter til slide 10] Ikke korrekt! Det korrekte svar er: 3. Trykker
Ctrl-Alt-Delete og håber maskinen virker nu. Spørgsmål 1208 ud af 1209: Du får
maskinen op at køre igen, og beslutter dig for at bliver MCSE-certificeret for
at undgå den slags problemer i fremtiden. Hvad gør du? \act{holder 1-3 knapperne
  frem}

\scene Slide 10 læses højt:

\scene 3. Tager en MCP-eksamen

\says{M}[overrasket] Øh... \act{trykker tøvende 3}

\scene En kort pause. Hvorpå R skifter til Blue Screen Of Death (slide 0) og går
i stykker.

\says{R}[skifter til Windowslogo (slide -1)] Velkommen til Microsoft MCP-certifikation.
Indkast kr. 1495 for at trække en test.

\scene Tæppe

\end{sketch}

\end{document}
% Local Variables: 
% mode: latex
% TeX-master: t
% End: 
