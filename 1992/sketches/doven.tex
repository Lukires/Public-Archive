\documentclass[a4paper,11pt]{article}
\usepackage{revy}
\usepackage[utf8]{inputenc}
\revyyear{1992}

\title{Doven evaluering}

\version{0.9}

\author{Jacob}

\begin{document}

\maketitle
\begin{roles}

\role{Lærer 1} Dat2-underviser
\role{Lærer 2} Dat2-underviser
\role{Lærer 3} Dat2-underviser
\end{roles}

\begin{sketch}

\says{Lærer 1} Her er et brev fra studienævnet, der står at vi skal
evaluere dat2.

\says{Lærer 2} Hvorfor det? Hvad skal studienævnet bruge det til.
Undervisningen  var god i år, og vi har begge bestået det kursus i
"FORELÆSNINGSMETODIK" som studienævnet rekvirerede fra Pædagoisk
Institut. Der er ingen grund  til at spørge.

\says{Lærer 1} Hvor har du ret, men tror du studienævnet køber den ?

\says{Lærer 2} Næh, vi må finde på noget.

\says{Lærer 1} Nu har jeg det, vi bruger doven evaluering.

\says{Lærer 2} Doven evaluering ?

\says{Lærer 1} Ja, vi skriver til studienævnet at vi vil evaluere
dat2 efter den metodik der står beskrevet i kursusbog 2, side 44 og
følgende sider --- Den  køber de.

\says{Lærer 2} Ja, men hvad så.

\says{Lærer 1} Jo ser du, ved doven evaluering, bliver der intet
gjort før nogen ønsker  et resultat, og hvis vi evaluerer dat2 på den måde, så er vi allerede  færdige nu.

\says{Lærer 2} Nå  --- Men hvad hvis studienævnet ber om at se skemaet, f.eks. til  august.

\says{Lærer 1} Ja, så er vi jo nødt til at lave et spørgeskema til den tid.

\says{Lærer 2} AHA, og hvis de vi ser nogle resultater, så kopierer
vi bare skemaerne og deler dem ud til de dat2-studerende der havde dat2 i år. De er ikke her. De arbejder og læser bifag, så vi får kun 2 svar.

\says{Lærer 1} Ja , og så svarer vi selv, så er halvdelen af svarerne
positive, god ide. 

\says{Lærer 3} kommer ind

\says{Lærer 3}[Højt og overrumplende] HVAD LAVER I?

\says{Lærer 1 + 2}[I kor] Vi har lavet evaluering af dat2, vi
benytter doven evaluering 

\says{Lærer 3} Doven evaluering - AHA - fint.

\says{Lærer 1} Jeg fik en ide, hvad nu hvis man kodede
eksamenssystemet i miranda. 

\says{Lærer 3} Hvorfor det ?

\says{Lærer 1} Jo Miranda, er dovent, så der vil ikke skulle afholdes
nogen eksamener for de studerende der ikke skal have udskrevet noget
eksamenbevis. OG kun de færreste bliver kandidater, såder er en besparelse.

\says{Lærer 2} Forklar 

\says{Lærer 1} Ja, de studerende går først til eksamen når de skal
have udskrevet deres kandidatbevis, for før har man jo ikke brug for
karaktererne. Der kan sparer en masse tid på retning ef
rapportopgaver.  

\says{Lærer 1 + 2 + 3}[I kor] Godt, vi skriver et brev til rektor

\end{sketch}

\end{document}
