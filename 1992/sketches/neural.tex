\documentclass[a4paper,11pt]{article}
\usepackage{revy}
\usepackage[utf8]{inputenc}
\revyyear{1992}

\title{Neurale netværk}

\version{1.0}

\author{Allan og Gnyf, id\'e: NULL}

\begin{document}

\maketitle

{\large \em
"Hellere et neuralt netværk i hånden, end 10 neuroner på taget!"\\
- en praktisk demostration af neurale netværk

\bigskip

"Og nu til noget helt andet....., en seriøs sketch!"
}

\begin{roles}

\role{En lærer/konferenciers}
Ser meget lærd (og ...) ud

\role{Tekniker}
Kittel, hovedtelefoner o. lign.

\role{8 neuroner}
iklædt ørevarmerne, for at antyde at en
neuron ikke kan høre, men kun modtage input ved ydre påvirkninger
(læs: slag i hovedet).

\role{2 bistre neuroner}
Iklædt
stenaldertøj og et par stenalderkøller (lavet af skumgummi) 

\role{Travelling Salesman}
Med en stor kasse med en albatros i.

\end{roles}



\scene
Ind på scenen kommer en lærer/konferenciers

\begin{sketch}



\says{Lærer}
Velkommen til det naturvidenskablige program HVæVL!
   {\em (kigger på sit stik\-ordskort)} I aften skal vi høre om en
af videnskabens nyeste opfindelser: Neutrale net. Disse
forventes at få stor indflydelse på fodboldens fremtid. Ud
over de to sædvanlige net, placeres der to net mere i kanten
af banen. {\em (evt. overhead)} Når bolden så havner der, får begge hold
...

\scene
Tekniker ind, peger på kortet og slår Læreren i hovedet med et
sammenrullet stykke papir.

\says{Lærer} I aften skal vi høre om en af videnskabens nyeste
opfindel\-ser: Neurale Net!  Vi har derfor inviteret en ekspert i
studiet. {\em (går hen i hjørnet og tager en lektor-hat [sort,
firkantet med kvast] på)}

\says{Lærer}
 Datalogien har - ligesom andre fag - sine modefænomener,
tidligere var det fraktaler, men nu er det store emne neurale
netværk. Desværre er det de færreste, som egentlig ved hvad et
neuralt netværk er, og jeg er derfor blevet bedt om at holde et
kollokvie om emnet. 

	Et kunstigt neuralt netværk består af efterligninger af
biologiske celler - neuronerne - og forbindelserne mellem dem -
synapser. Kan vi få en neuron ind på scenen" {\em (vinker en neuron
ind)}

\scene
Ind kommer en neuron. Vedkommende stiller sig op ved siden af
læreren.

\says{Lærer}[peger på neuronen] Dette er selvfølgelig ikke en
rigtig neuron, men en kunstig simulering af en! Kendetegnende ved
en neuron er, at den modtager en række input, og hvis summen af disse
input overstiger en vis grænse, sender neuronen et output videre. Lad
mig demonstrere!

\says{Lærer}[tager en sammenrullet avis, og peger på den]
 Ekstra Bladet {\em (slår neuronen, som ikke reagerer)}

\says{Lærer}[tager nu en stooor sammenrullet avis, og peger på den]
Søndags Berlingeren {\em (slår neuronen, som vakler, men slår ud med
hånden)}
	
\says{Lærer} 
Ved at variere styrken af forbindelsen til neuronen --
kraften af slaget -- kan man ændre netværkets output.

	Men \'en neuron gør ikke et netværk, og man skal både have
input og output neuroner.

\scene
vinker 7 andre neuroner ind på scenen, så
der nu er 8 neuroner på scenen. De stiller sig op med 2 bagerst, 4 i
midten og 2 forrest. De 4 i midten kommer slæbende med et bord (eller
andet, der kan skjule nogle rekvisitter), og placerer det foran dem
selv.

\says{Lærer}[pegende på de respektive grupper]
Dette netværk
indeholder inputneuroner, outputneuroner og endelig neuronerne i det
skjulte lag.

Det interessante aspekt ved et neuralt netværk er selve
oplæringen. Det foregår ved træning: Man præsenterer et input, og
observerer hvilket output, netværket returnerer. Derefter rettes
styrken af forbindelserne, så det passer med det forventede output.
Lad mig straks demonstrere!

{\em (går hen til inputneuronerne)} Hvad kaldes en HD'er, som
kun har bestået første del?

\scene 
Inputneuronerne slår nogle neuroner i det skjulte lag, som
slår videre (på må og få, nogle rammer ved siden af) til
outputneuronerne. Læreren går hen til outputneuronerne, som stikker
mikrofonen hen til dem. 

\says{Outputneuroner} 
En DD'er

\says{lærer}
Man ser at netværket
ikke er optrænet, men det gøres nu

\scene
 Neuronerne i det skjulte lag
skifter hovedbeklædningen, samtidig med "slåinstrument"


Ideen er at "optræne" netværket til korrekt opførsel, ved at
gennemløbe følgende algoritme:

\begin{enumerate}
\item
 Fortæl inputneuronerne en 1/2 vits. De slår nogle skjulte
neuroner.  
\item De skjulte neuroner slår mere eller mindre kaotisk til
outputneuronerne.  
\item Ouputneuronerne svarer med plaaatte svar, som fører til....  
\item Optræning af skjulte neuroner: De skifter hovedbeklædning og
slåinstrument. F.eks. kunne man benytte brandhjelme og vandpistoler,
colakasketer og colaflasker, Pointers Brothers med deres solbriller,
Gregers Koch kokkehue og risengrødske.  man kunne ende med
studenterhuer og bøger "...for nu er de trænede"
\end{enumerate}

Ovenstående giver mulighed for at fyre en lang række meget platte
vitser af. Revygruppen er sikkert i stand til at finde en enkelt eller
to.


\says{Lærer} Nu har vi fået optrænet netværket, og lad os så se,
hvad det dur til!

\scene Læreren går helt hen til netværket og fortæller den MEST
PLATTE vittighed, hvorved de går amok, giver det rette svar, men
kommer til at ramme/strejfe ham. Han vinker (lidt irriteret)
neuronerne ud af scenen.

\says{Lærer}[Holder en kunstpause, ser ud til at glæde sig til noget]
Se, normalt arbejder man kun med \'et eller to skjulte lag. Men {\em jeg}
vil nu gennemføre et storstilet forsøg: {\em (tæller op langs
rækkerne med fingeren)} Et neuralt net med ikke mindre end 19 skjulte lag!

I dette forsøg er neuronerne i det skjulte simuleret af
publikum, outputneuronerne er den forreste række, og vi har et par
kraftige inputneuroner klar. {\em (peger op bagerst, hvor to bistre
neuroner pludselig bliver oplyst af spotlights.)} ...For at illustrere
forbindelserne i det neurale netværk bedes publikum benytte deres
program, som de ruller sammen, og vi er klar til begynde optræningen

\says{Lærer}[henvendt til de to inputneuroner bagerst]
Hvorfor var
posten i DIKU's bestyrelse altid forsinket før i tiden?

\scene
De to neuroner slår
vidt om sig, og rammer et par folk. På dette tidspunkt breder
begynder folk sikkert at slå vidt på de nærmeste. Sikken aktivitet
i netværket. Læreren spørger nogle personer på første række om
svaret, men får det naturligvis ikke:

\says{Lærer}[ikke overrasket]
Næh, netværket er endnu ikke helt optrænet; det rigtige
svar var jo selvfølgelig at det er fordi posten først kom til
Jul!

\says{Lærer}
Nå, men vi prøver igen {\em (henvender sig igen til de to
inputneuroner)} Hvad er det, som the traveling salesman egentlig
sælger?

\scene
Inputneuronerne slår igen vildt om sig, og mens
slagudvekslingerne foregår, smækkes de øverste døre op, og ud
træder en mand med en sælgerkasse med en kæmpe fugl.

\says{Salesman}[Råber af sine lungers fulde kraft]
{\LARGE Albatros!}

\scene 
Han begiver sig ned til midterdørene, alt imens han højlydt
afviser alle, der forsøger at købe fuglen. Han forlader salen via
midterdørene. Læreren har i mellemtiden forladt scenen, og der er
klart til et nyt nummer.


{\footnotesize
Kommentar: Hvis sketchen benyttes i 1. akt, har vi også vores
pausenummer, idet en stemme ved pausens start kundgør, at der
desværre ikke er nogle pausefisk, da de er blevet spist af en stor
uidentificeret fugl. Fuglen kunne så stå på scenen i pausen.}

\end{sketch}

\end{document}
