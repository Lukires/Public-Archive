\documentclass[a4paper,11pt]{article}

\usepackage{revy}
\usepackage[utf8]{inputenc}
\usepackage[T1]{fontenc}
\usepackage[danish]{babel}


\revyname{DIKUrevy}
\revyyear{2017}
\version{0.1}
\eta{$2$ minutter og $10$ sekunder}
\status{Sørgelig :'(}

\title{Frokosten}
\author{Niels}
\melody{The Beatles: ``Yesterday''}
% https://www.youtube.com/watch?v=haWRUpPw_tI
% https://open.spotify.com/track/1e0hllQ23AG0QGFgezgLOq

\begin{document}
\maketitle

\begin{roles}
\role{S}[Mathias] Sanger
\role{St}[Niels] Statist
\role{K1}[Cecilie] Kor
\role{K2}[Vivien] Kor
\role{K3}[Rasmus] Kor
\role{X}[Freja] Instruktør
\role{XX}[Nicklas] Sanginstruktør
\end{roles}

\begin{props}
\prop{AV: I starten af nummeret skal der stå: Baseret på virkelig hændelser}[]
\prop{Et bord}[]
\prop{Fire stole}[]
\prop{Fire tallerkener}[]
\end{props}


\begin{song}
\scene{Lys op.  St sidder alene ved sit bord og smører sig en mad gennem hele sangen.  Der er fire stole og tallerkener ved bordet, men det er kun St der er mødt op.  Til sidst i sangen taber St sit flotte stykke smørrebrød på scenen. :( }

\sings{S}%
Frokosten
spiste jeg i dag uden en ven
Jeg sad i kantinen helt alen'
mens jeg spist' æg
til frokosten

\sings{S}%
Sidste gang
var vi fire folk der spiste sam'n
Der var karrysild og ramasjang
ved frokosten
den sidste gang

\sings{S}%
Hvor var vennerne
De var nok
et sjovere sted
Mad fra stor buffet
Det er dét
de' venner med

\sings{S}%
Næste gang
hvad skal jeg dog gøre med min trang
til at spise rødløg dagen lang
Hvor er min' venner næste gang

\sings{S}%
Hvor er vennerne
De er nok
et sjovere sted
Mad fra stor buffet
Det er dét
de' venner me-e-e-ed

\sings{S}%
Frokosten
bliver aldrig helt den sam' igen
Køleskabet er min en'ste ven
Jeg glemmer aldrig frokosten

\sings{S}%
Mm mm mm mm mm mm mm

\scene{Lys ned.  Alle i publikum græder!}
\end{song}

\end{document}
