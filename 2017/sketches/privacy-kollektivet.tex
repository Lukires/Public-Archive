\documentclass[a4paper,11pt]{article}

\usepackage{revy}
\usepackage[utf8]{inputenc}
\usepackage[T1]{fontenc}
\usepackage[danish]{babel}


\revyname{DIKUrevy}
\revyyear{2017}
\version{1.0}
\eta{$3$ minutter}
\status{Pizzazztastic}

\title{Privacy-kollektivet}
\author{Niels, Simon, Sebbe}

\begin{document}
\maketitle

\begin{roles}
\role{P0}[Brandt] Privacy-interesseret, rimelig fornuftig
\role{P1}[Mikkel Storgaard] Privacy-interesseret, kendis-skør
\role{P2}[Mads] Privacy-interesseret, forenings-skør
\role{P3}[Kasper] Privacy-interesseret, skør-skør
\role{S0}[Romeo] Statist; privacy-interesseret, i baggrunden
\role{S1}[Kim] Statist; privacy-interesseret, i baggrunden
\role{S2}[Mathias] Statist; privacy-interesseret, i baggrunden
\role{S3}[Sebbe] Statist; privacy-interesseret, i baggrunden
\role{X}[Simba] Instruktør
\end{roles}

\begin{props}
\prop{Mappedatamater}[]
\prop{AV: Telefon der ringer}[]
\prop{AV: Facebook-post (se slut på sketchen)}[]
\end{props}


\begin{sketch}

\scene{Lys op.  P0, P1 og P2 er samlet.  P1 og P2 sidder med mappedatamater.
S* er der også.}

\says{P0} Fedt at I kunne komme!  Jeg tænkte at vi kunne arrangere nogle
månedlige café-events hvor vi hjælper hvem der nu har lyst med almennyttig
kryptografi, og--

\scene{P3 kommer ind.}

\says{P3}[afbryder P0] NED MED CENSUREN, LÅLÅLÅLÅ, CEN-SUR GØR MIG
\textbf{SUR}!!!

\says{P1} <P3>!  Længe siden!  Går det godt med din decentraliserede,
distribuerede content distribution platform?

\says{P3} Nej, folk ville hellere bruge FACEBOOK!  NED MED DET AUTORITÆRE
DATAREGIME!

\scene{P3 sætter sig ned og surmuler med sin mappedatamat.}

\says{P0} ... Som sagt, invitere offentligheden til at komme forbi og få
kendskab til open source-programmer til kryptering af--

\says{P2}[afbryder P0] AHEM!  Bør vi ikke \emph{formalisere} det her for at
undgå at personlige problemer overskygger foreningens aktiviteter, og for at
klargøre vores formål?

\says{P0} Nu--

\says{P2}[rejser sig op og tager en stor bunke papirer frem] Jeg har faktisk
forberedt nogle vedtægter.

\scene{P2 tager en lille klokke frem og ringer med den.  P0 er lettere
irriteret.}

%KOMMENTAR: FØLGENDE REPLIK BØR GØRES MERE TIGHT!
\says{P2}[læser op] Velkommen til den første generalforsamling for foreningen
"Befri Internettet Nu!".  Paragraf 1.  "Befri Internettet Nu!" er en autonom,
autoritærsky forening med fokus på censurfri færden på internettet, fri software
og fri kultur.  Paragraf 1, stykke 1.  "Befri Internettet Nu!" udgøres af
parter. \act{AV fader ned; P2 fortsætter med at læse paragraffer op i resten af sketchen.}
Paragraf 1, stykke 2.  En part er en person eller gruppe, der via
tilpas kryptografisk bevis er anerkendt af de andre parter i foreningen.
Paragraf 1, stykke 3.  En part udfører--

\scene{P1 bliver træt af at høre på P2.}

\says{P1} Hey, har I set det her???  Edward Snowden har lige lavet en
Twitter-besked hvor han anbefaler at man ikke bruger Google Drive til at gemme
sine kreditkortoplysninger?  Han har \emph{så} ret!

\says{P3} Ja, NED MED GOOGLE!  GIV SNOWDEN ASYL!

\says{P0} Vi må altså lige beslutte hvad vi gør med hensyn til de sociale
medier. \act{P3 begynder at kede sig} Jeg har lavet en Twitter-bruger, men skal
vi også have en presence på Facebook?  Jeg synes godt det kan give mening, hvis
bare vi passer på ikke at--

\says{P3}[afbryder P0, bliver glad igen] FOLKETINGET ER DUMME!

\says{P0} ... Jeg tænker at første gang er på Laundromat-caféen på Østerbro om
to uger.  Jeg tager--

\says{P1}[afbryder P0] Den café kender jeg godt! Der mødte jeg engang en der
kender Edward Snowden! Altså, via et chatforum -- ja, jeg sad og arbejdede på
en decentraliseret central database over kendte bitcoin-brugere.
\act{Bliver tvivlende} Nej, nej, der kan vi ikke tage tilbage til. Vi kan
umuligt overgå den oplevelse!

\says{P0} Aj, men--

\says{P3}[afbryder P0] WowowoWAUW, JULIAN ASSANGE siger at han snart lækker
dokumenter om at USA er nogle svin!  Det bliver CRAZEY!

\says{P0} FOR FANDEN, I er jo totalt useriøse!

\scene{P0 giver op og går sin vej.  S* følger med.}

\says{P1} Hvad synes I lyder bedst som slogan?  ``Wi-ki-leaks er flot og
po-li-ti-ker er snot'' eller ``be-fri net-tet i en fart -- over-våg-ning er
ik-ke rart''?

\says{P3}[rejser sig og svinger armene i luften mens han går ud]
    LÅ LÅ LÅ POLITIKER ER SNOT
    LÅ LÅ LÅ POLITIKER ER SNOT
    \act{fortsætter indtil han er ude, og også lidt efter}

\says{P1}[begejstret] Jeg tror vi har os et slogan! \act{noterer ned}

\scene{AV: Telefon der ringer}

\scene{P1 tager telefonen}
\says{P1}[mens han går ud] Mor!? Nej, jeg har jo sagt at du ikke må ringe
på dette nummer! NSA lytter jo med!

\scene{Lyden til P2 fader ind igen}

\says{P2} \ldots Paragraf 778, stk 13a: Eksistensen af denne forening må
\emph{under ingen omstændigheder} komme til kendskab af statslige agenturer,
herunder især NSA, CIA, FE og Københavns Universitet.

\scene{P2 bliver færdig med at læse op og lægger papirerne ned.}

\says{P2} Så skal vi bare have fundet en formand. Jeg vil gerne stille op.
Nogen imod?

\says{P2} Fedt! Det skal på Facebook!

\scene{Lys ned}

\scene{AV: Facebook-post hvor P2 har postet, at han er blevet formand af den
totalt hemmelige privatlivs-forening. NSA har liket den.}

\end{sketch}
\end{document}
