\documentclass[a4paper,11pt]{article}

\usepackage{revy}
\usepackage[utf8]{inputenc}
\usepackage[T1]{fontenc}
\usepackage[danish]{babel}


\revyname{DIKUrevy}
\revyyear{2017}
\version{1.1}
\eta{$4.25$ minutter}
\status{Færdig}

\title{e-bogs-lobbyen}
\author{Niels, Simon, Sebastian}

\begin{document}
\maketitle

\begin{roles}
\role{M}[Niels] Ministeren - nydelig mand, dog ikke synderligt klog
\role{L}[Sebbe] E-bogs-lobbyist - con-artist-agtig type
\role{I}[Simon] Intercom - sketchens største rolle
\role{X}[Simba] Instruktør
\end{roles}

\begin{props}
    \prop{Et bord til ministeren}[]
    \prop{En stol til ministeren}[]
    \prop{En buzzer som M kan trykke på for at kommunikere med intercom}[]
    \prop{Folkeskolelærer-agtigt kontorartikler}[]
    \prop{Reengineering a university department}[]
    \prop{Pixiebog -- den gode}[]
    \prop{En tyk bog der kan slås med}[]
    \prop{En usb-pind der kan kastes med}[]
    \prop{AV: Søjlediagram der sammenligner pris på bog og ebog (se sketchen)}
    \prop{AV: Søjlediagram der sammenligner pris på bog og ebog efter afgifter (se sketchen)}
    \prop{AV: Søren skal tegne en penis live ud over bogen}[]
    \prop{Skærm}[]
    \prop{Kindle}[]
    \prop{Kontrakt}[]
    \prop{Fyldepen}[]
\end{props}


\begin{sketch}
    % Uddannelsesminister; mødet handler om indkøb af bøger til uddannelsessektoren.

    \scene{Vi er på M's kontor. Han sidder ved computeren og læser i sin bog.}
    \says{I} Hr.\ uddannelsesminister? Din aftale er kommet.
    \says{M} Åh, send ham ind!
    \scene{M klikker på sin buzzer.}
    \scene{L kommer ind.  M går over for at give L hånden, skal til at vise M en plads.}
    \says{L} Dette er en bog\ldots
    \scene{L hiver en stor bog frem.}
    \scene{L slår M med bogen. M falder komisk om.}
    \says{L} Det vi ser her er et kæmpe problem i hele verden! Bøger er simpelthen
             for voldelige. Derfor arbejder \emph{vi}, fra e-Bog \& Idé, ihærdigt på
             at fremme udbredelsen af e-bøger. Observér!
    \scene{M prøver at rejse sig op.}
    \scene{L kaster en USB-pind på M. M falder om igen.}
    \says{L} En USB-pind gør slet ikke ondt!
    \scene{M prøver at rejse sig op igen.}
    \says{M} Hvad skete der\ldots? \act{Bliver afbrudt efter `Hvad'}
    \says{L} Jo, tænk bare på alle de krige der er opstået siden bogen blev opfundet.
             Og nogle af dem er endda på grund af bøger.
    \says{M}[groggy] Ja, krig er et problem.
    \says{L} Et eksempel på dette er Bibelen, som har forårsaget flere krige igennem
             historien. Hvis nu man havde udgivet den som
             e-bog i stedet for, så kunne folk bare have ladet være med at købe de
             kapitler de ikke kunne lide. Så havde de krige jo været undgået!
    \says{M} Bibelen? Mig og mit parti går stærkt ind for religionsfrihed! Om det så er
             kristendom, jødedom, \ldots
    \scene{Awkward pause.}
    \says{L} Og nu tænker du sikkert, hvad med Islam?
             e-Bøger understøtter for eksempel en episodisk model!
             Til jøderne kan man lave en basispakke\ldots \act{AV: billede af ``Det Gamle Testamente''}
             \ldots som så kan opgraderes med DLC'en ``Det Nye Testamente'' til kristendommen. \act{AV: tilføj ``Det Nye Testamente'' DLC til tidligere billede}
             Til særlige kendere, der anbefaler vi en remastered edition.
    \scene{AV: Billede af Koranen}
    \says{M} Men den slags kan vi jo ikke udsætte vores børn for!
    \says{L}[afstandstagende] Nej, nej, men den samme model kan anvendes til bøger i folkeskolen,
             hvor brugerbetalte DLC-pakker vil gøre differentieret undervisning meget nemmere!
    \says{M} Så bliver det også meget billigere at distribuere bøgerne ud til skolerne!
    \says{L}[udbryder] PIRATKOPIERING!?
    \says{M} Nej, nej, du har misforstået noget.
    \says{L} Faktisk er e-bøger meget mere sikre
             end almindelige bøger. En bog kan du jo bare putte i kopimaskinen. Kan
             du måske det med en e-bog?
    \says{M} \act{klør sig på siden af hovedet}
             Øh\ldots det må jeg vist lige spørge min sekretær om.
    \says{L} \act{skubber M ned i sin stol} Næh, \emph{stol} blot på mig.
             Med always-on DRM har piraterne \emph{ingen} chance.
    \says{M} Men bliver det billigere?
    \says{L} Ja! Og det er jo et stort problem. E-bøger kan simpelthen ikke konkurrere
             rent prismæssigt mod de langt dyrere almindelige bøger.
    \scene{L viser et søjlediagram, hvor for 3-4 bøger, prisen på e-bogen sættes op mod
           prisen på den almindelige bog. Prisen er betydeligt højere på papirudgaven.
           Sidste bog er Jyrki's bog, hvor prisen på begge er negativ, men mindre negativ
           på papirsudgaven.}
    \says{L} Ergo blive vi nødt til at gøre almindelige bøger \emph{endnu} dyrere!
    \scene{L viser et nyt diagram, magen til det fra før, men hvor søjlerne på de almindelige
           bøger er endnu højere. Jyrkis almindelige bog går i 0.}
    \says{M} Nårh ja, ligesom elbiler!
    \scene{M hiver ``Reengineering a university department'' frem fra bag bordet.}
    \says{M} Men hvad med den hér? \act{viser bogen frem}
    \scene{AV: Forsiden på ``Reengineering a university department''}
    \says{M} Hvordan skal jeg få det til at se ud som, at jeg er klog, hvis de ikke kan se forsiden?
             Så kan de jo lige så godt tro at jeg læser en\ldots en pixiebog!
             \act{hiver en pixie-bog frem inde fra Jyrki-bogen}
    \scene{M opdager hvad han har gjort og gemmer den hurtigt væk.}
    \says{L} Bare rolig, vores e-bogs-læsere kan sættes op til automatisk at tweete hver
             gang du læser i bogen.
    \scene{M nikker tilfredst, kigger sin checkliste igennem og folder den væk.}
    \says{M} Ja, det ligner at vi har det formelle på plads, men jeg har egentlig også mine
             egne spørgsmål. \act{Hiver en ny liste frem}
    \says{M} Her den anden dag sad jeg og så ``The Day After Tomorrow''.
             Hvad skal vi dog gøre, hvis der kommer en enorm-kuldebølge, og vi ikke
             har nogle bøger vi kan brænde af for at holde varmen?
    \says{L} Du lever altså i fortiden. Nutildags brænder man DVD'er, og ikke bøger.
             \act{hiver sin Kindle frem}
             Desuden er der plads til 10.000-vis af bøger på min Kindle!
             Tænk på hvor længe sådan en kan brænde!
    \says{M} Nårh ja! Godt tænkt! Men\ldots lige en sidste ting: Vi vil jo gerne have
             folk hurtigere igennem deres uddannelse, og dengang \emph{jeg} gik i skole,
             var det vigtigt for mig at læse de noter andre elever havde skrevet i mine
             biblioteksbøger. Du ved, for sådan at perspektivere og den slags.
             Hvordan kan de studerende nogensinde bestå uden disse noter?
    \says{L} \act{tænker lidt} Øh\ldots jamen, så må vi lave nogle specielle skole-e-bøger.
    \scene{AV: Billede af en skole e-bog}
    \says{M} Okay?
    \says{L} Øh\ldots jaja! Så kan de skrive noter i dem, og de noter kommer så i alle
             de andre studerendes bøger. \ldots I realtid!
    \scene{AV: Bogen dækkes af en masse noter fra studerende, mere og mindre seriøse, mens Søren tegner en penis på den. I realtid!}
    \says{M} Til evig gavn for alle studerende, \emph{før, nu og i fremtiden!}
    \says{L} Med alt dette på plads, kan vi så ikke formulere en kontrakt hvor e-Bog \& Idé bliver enedistributør til skolevæsnet?
    \scene{L tager en kontrakt frem og skal til at give den til M.}
    \says{M} \act{tager sin egen kontrakt frem, og giver den til L} Ja, du skriver bare under hér.
    \scene{L begynder at bladre kontrakten igennem.}
    \says{L} Der står her at du vil have en direktørstilling? Og hvorfor starter projektet først i 2020?
    \says{M} \act{giver L en fyldepen og nikker tilfredst} Fordi der er jeg ikke længere minister.

    \scene{Lys ned, tæppe for.}
\end{sketch}
\end{document}
