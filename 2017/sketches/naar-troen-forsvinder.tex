\documentclass[a4paper,11pt]{article}

\usepackage{revy}
\usepackage[utf8]{inputenc}
\usepackage[T1]{fontenc}
\usepackage[danish]{babel}


\revyname{DIKUrevy}
\revyyear{2017}
\version{0.1}
\eta{$2.5$ minutter}
\status{Jeg tror ikke på det}

\title{Når troen forsvinder}
\author{Simon, Niels, Andreas}

\begin{document}
\maketitle

\begin{roles}
\role{P}[Sebbe] Studenterpræst ved KU
\role{S}[Kasper] Studerende
\role{E}[Ejnar] Eksaminator
\role{TM}[Torben] Torben Mogensen (Gud)
\role{X}[Mikkel Storgaard] Instruktør
\end{roles}

\begin{props}
\prop{Rekvisit}[]
\end{props}


\begin{sketch}

\scene{ P træder højtideligt ind.  Spot på P.}

\says{P} Jeg har modtaget et brev fra en studerende der desværre har mistet
troen...

\scene{P holder en kunstpause.}

\says{P} ... på at kunne bestå eksamen.

\says{P} Den studerende, Preben, som ønsker at være anonym, har bedt mig læse brevet
op her til denne gudstjeneste.

\scene{P stiller sig over i den ene side af scenen.  P tager brevet op og
begynder at læse op.}

\says{P} Når troen forsvinder: En eksaminants bekendelse.

\scene{Lys på hele scenen.  På den anden side af scenen står et bord med E
og nogle papirer.}

\says{P} Det var mundtlig eksamen.  Mellem mine aktiviteter som frivillig i
Folkekirkens Nødhjælp og suppe-opøser for de hjemløse på herberget,
havde jeg selvfølgelig fundet tid til at læse op til fuldt pensum.

\scene{S træder fuld ind på den anden side af scenen med nogle ølflasker i
hænderne.}

\says{S} SKÅÅÅÅL!

\says{E} Træk venligst et emne.

\scene{S trækker et emne og kigger på det.}

\says{P} Til trods for mine grundige studier, kunne jeg mærke at Gud ikke
var med mig i dette øjeblik.

\says{S} SATANS!

\scene{S går amok, slår i bordet, skubber ting ned og er grundlæggende ikke
særlig tilfreds.}

\says{P} Jeg havde tredive minutter til at læse op på max flow, men
teorien syntes ikke at stemme overens med egen imperi.

\says{S} Max flow?! Nu skal I bare se løjer!

\scene{S kaster algoritmebogen væk, stiller sig op og begynder at bunde
sin øl. Samtidig pisser han i bukserne, så man ser hans bukser blive
gennemblødte.}

\says{P} Til at begynde med stillede de mig udfordrende og tvetydige
spørgsmål.

\says{E} Kan du \emph{fortælle} om dit emne?

\says{S}[kigger på flasken, og den er tom] Ææææh...

\says{E} Hvis du starter med at fortælle hvad ``max-flow'' problemet
handler om?

\says{P} Jeg kunne mærke at Gud havde en anden plan for mig. Spørgsmålene
gik da også meget hurtigt helt væk fra pensum.

\says{E} Er du sikker på at du studerer her?  Hvad er dit navn?

\says{S} Det står ikke i bogen!

\says{P} Jeg har jo altid sagt ``Lad den uskyldige kaste den første sten.'',
men eksaminator var uenig og kastede hurtigt den metaforiske sten
efter mig.

\says{E} Du får -3.

\says{P} Til det svarede jeg: ``Guds rige er uendeligt.  Lad kun Gud bedømme
mig!''.

\says{S} FUCK JER! JEG SKRIDER!

\says{P} Da dette var mit tredje eksamensforsøg, forviste KUs bureaukratiske
magter mig. Som Abraham bønfaldte dig, Gud, da du bad ham om at ofre sit barn
Isak, bønfalder jeg dig ligeså, min uddannelse på alteret. Please lad mig bestå!

\scene{P folder brevet sammen, og tænker et øjeblik}

\says{P} Ja, det er i sandhed lige så svært for en kamel at komme igennem et nåleøje,
         som det er for en studerende at komme igennem DIKU.

\says{TM}[voice over] DNUR.

\scene{Lys ned.}


%% Alternativ slutning; E kommer i tanke om at det er løbende evaluering, og S består alligevel
\end{sketch}
\end{document}
