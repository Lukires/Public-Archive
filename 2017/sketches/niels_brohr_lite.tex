\documentclass[a4paper,11pt]{article}

\usepackage{revy}
\usepackage[utf8]{inputenc}
\usepackage[T1]{fontenc}
\usepackage[danish]{babel}

\revyname{DIKUrevy}
\revyyear{2017}
% XXX: HUSK AT OPDATERE VERSIONSNUMMER
\version{3.5}
\eta{$3$ minutter}
\status{Lidt mindre tight end minilogen}

\title{Niels Brohr-monologen}
\author{Sebastian, Spectrum, Søren Pilgård, Johan, Ejnar, Thomas}

\begin{document}
\maketitle

\begin{roles}
  \role{I}[Andreas]  Interviewer; TV-vært
  \role{NB}[Ronni] Niels Brohr
  \role{X}[Caroline] Instruktør
\end{roles}

%\begin{props}
%
%\end{props}

\begin{sketch}

\scene{Nyhedsprogram; interviewer sidder midt for på bordet og kigger ud mod
       publikum, á la et nyhedsprogram, med Niels siddende som gæst ved siden af.}

\says{I} Den nye Niels Bohr-bygning er delt i to af Jagtvej. Som vi lærte til sidste
         års DIKUrevy er disse to dele forbundet med en såkaldt ``Skywalk''. Med i studiet har vi
         lederen af byggeprojektet; Niels Bohrs bror, Niels Brohr.

\says{I}[Til Niels] Niels; hvorfor konstruere en skywalk, når der nu også bygges en tunnel?

%
%
%\says{I} Velkommen til Niels Bohrs bror, Niels Brohr. Du har fået til opgave at
%         lede konstruktionen af en skywalk over Jagtvej. Hvorfor valgte i en todelt
%         bygning forbundet af en skywalk?


\scene{NB træder frem; får spot; I forsvinder i baggrunden.}

\says{NB} %
    Broer ligger dybt i den danske kulturarv.
    Betragt blot Danne\emph{bro}g! Vi ser to hvide broer der skyder over et rødt landskab.
    En der forener venstrefløjen med højrefløjen, og en der bygger bro
    mellem toppen og bunden af klassehierakiet. Forenet i et stort dansk
    \emph{bro}derskab!

    Lad os kigge videre til dengang København blev grundlagt i \emph{bro}nzealderen.
    Vi kan tydeligt se på Københavns stednavne, Nørre\emph{bro},
    Øster\emph{bro}, Vester\emph{bro}, at broer er Danmarks fundament!
    De har vel hørt om kontinentaldriften, har De ikke?
    Havde vi ej broer til at sammenholde Danmarks øer, da ville de drive fra
    hinanden. Se blot Grønland! Og de Vestindiske Øer!

    Og nu må vi jo huske på, at brobygning ligger til grund for hele vores økonomi.
    Se blot på vores pengesedler! % Broer!
    Jeg har i øvrigt undersøgt sammenhængen mellem massen af broer i landet
    og vores \emph{bro}ttonationalprodukt.

\scene{Niels Brohr hiver grafen frem og begynder at pege på den.}

\says{NB} %
    På denne graf ser vi en klar
    korrelation. Bemærk især midten hvor de \emph{næsten} er ens!

    Ja, broer er vejen frem, og man skal jo tænke på fremtiden - og børnene!
    \act{Læg tryk på hvert ord i følgende sætning.}
    \emph{Alle}. \emph{børn}. \emph{\bf elsker}. \emph{broer}!
    Det hedder jo ikke ``Tunnel, Tunnel, Brille'', vel?
    Og ``Jeg gik mig under sø og land'' - det er jo det rene vrøvl!

    Jeg har en drøm! En drøm om en fremtid, hvor det er muligt at gå en Eulertur igennem
    \emph{hele} verden! Broer fra kyst til kyst, fra Holste\emph{bro} til \emph{Bro}nholm!

    Jeg har en drøm!

    Jeg har en drøm!
    Tunneller? Næææh! De er slet ikke \emph{bro}gervenlige nok!
    De skal væk! Vi må være hårde og \emph{bro}tale!

\scene{NB samler fatningen igen, træder tilbage og sætter sig i interview-stolen som om intet er hændt.
       Lys tilbage til normalt scenelys.}

\says{I} Det var desværre alt vi havde tid til i aften. Tak til Niels Bohrs bror, Niels Brohr.
         Følg med i næste uge hvor vi tager et kritisk blik på mordet af Georg Mohrs mor, Georg Mor.

\scene{Lys ned}

\end{sketch}
\end{document}

%%% Local Variables:
%%% mode: latex
%%% TeX-master: t
%%% End:
