\documentclass[a4paper,11pt]{article}

\usepackage{revy}
\usepackage[utf8]{inputenc}
\usepackage[T1]{fontenc}
\usepackage[danish]{babel}


\revyname{DIKUrevy}
\revyyear{2017}
\version{0.1}
\eta{$4.25$ minutter}
\status{Ikke færdig}

\title{Lektiecafé}
\author{Niels, Simon, Sebastian}

\begin{document}
\maketitle

\begin{roles}
\role{I}[Kasper] Instruktor (tjener)
\role{I'}[Mads] Instruktor (tjener), står i køkkenet
\role{DM}[Romeo] Mandelig datalog, fornem person
\role{DK}[Vivien] Kvindelig datalog, fornem person
\role{S1}[Kim] Statist på restauranten
\role{S2}[Brandt] Statist på restauranten
\role{S3}[Caroline] Statist på restauranten
\role{S4}[Arinbjörn] Statist på restauranten
\role{X}[Freja] Instruktør
\end{roles}

\begin{props}
\prop{3 caféborde/borde}[]
\prop{6 stole}
\prop{Menukort}
\prop{Opgaver}
\prop{Desinficeringsmaskinen KEN}[]
\prop{3D-printet figur ("opgave")}[]
\end{props}


\begin{sketch}

\scene{(overtex) Kantinen - DIKUs originale lektiecafë}

\scene{Vi er på en cafe, der sidder gæster ved cafe borde. Instruktor har tjener tøj på}
\scene{Der er et 'køkken', evt cafeen skilt med kantinen logo på. I' står i køkkenet}
\scene{Alle sidder i hipstertøj, drikker latte}

\scene{DM + DK kommer ind. De er fint klædt på, og dybt forelskede. I tager imod dem}

\says{I} Velkommen til lektiecafeen, har i en reservation?
\says{DM} Vi har en forhåndsgodkendelse til 2 personer - Under navnet Preben.
\says{I}[Viser dem til deres bord] Ja, værsgo og sid ned.
\scene{I trække stolen ud til DK, men DM skubber I væk, og trækker stolen endnu
længere ud}
\says{DK} Tak, skat
\says{DM} Vi vil gerne se ugesedlen
\says{I}[giver dem ugesedlen] Ja, selvfølelig. Dagens nød omhandler (m,k)-perfekte tal.

\scene{DM+DK ser i menuen}
\says{DM} Jeg vil gerne have en IP opgave
\says{DK} Hmm.. jeg ved ikke rigtig. Har I sådan nogle raw-opgaver?
\says{I} Jo, men de fylder desværre så meget, at man skal være mindst 4 om at løse dem\ldots
         Eller to specialestuderende. Hvad med en JPEG?
\says{DM} Ej nuller-pus, nu må du altså ikke være besværlig. - Jeg går
selvfølgelig ud fra at opgaverne er laserprintet?
\says{I} Nej, Hr.! De er 3D-printet med fairtrade plastic, printsaldo-neutralt importeret
         fra biblioteket nede ved Panum.
%\says{DM} Nu har det vel aldrig været frosset?
\says{DK} Jeg skal bare have en opgave 2 med ekstra monader, og en opgave 5a til dessert.
\says{I} Vil i have en tag-hjem eksamen, eller løser i her?
\says{DM} Vi havde nok ikke taget en studieplads, hvis vi ikke ville løse her vel?.
\says{I} Nej, selvfølgelig. Opgaven kommer straks.

\scene{I går om til køkkenet}
\says{I'}[Diskret til I] Vi har den her opgave. Den er vist et par år gammel.
Kan vi stadig bruge den?
\says{I}Ja, vi smider den bare igennem printeren.
\scene{I' tager opgaven og gør noget ved den}
\says{DK}[(overhører I's samtale) til DM] Ejjj, skattepus, vil du bare lade dem
servere gamle opgaver til mig?
\says{DM} Instruktor! Sig mig lige, serverer i bare års gamle opgaver?
\says{I}[finder på en undskyldning] ehh.. det er en årgangsopgave.. det er et klassisk..
et klassisk Nils Andersen opgavesæt fra 1983...
\says{DK}Uhygiejnisk! Kom, lad os finde et andet sted
\says{I}Nej, nej! Vi har god hygiejne i Kantinen! Vi har fået en elitesmiley af
eksamenskontoret fordi over 40\% består vores eksaminer. Vi gør også rent en
gang i mellem.

\scene{I' sætter en opgave frem på disken, og ringer på en klokke, I går op og henter
opgaven, og serverer den for DM på et fad}

\scene{I skal til at gå tilbage til køkkenet, men DK bryder ind}
\says{DK} Undskyld, har i wifi?
\says{I} Er Eduroam okay? % Jeg henter et password?
\says{DK} Nej, så er det lige meget

\scene{I skal til at gå tilbage til køkkenet, men DM bryder ind}
\says{DM}[skubber tallerknen væk] Altså, nu vil jeg jo ikke være besværlig, men...
er det muligt at få lavet min opgave 2 om til en 3b? Jeg så hende ovre ved det
andet bord, og den så bare så nem ud!
\says{I} Beklager, SML-opgaver er ikke længere i sæson.
\says{DM}[peger på hende med SML-opgaven] Jamen...
\scene{I' tager hendes SML-tallerken væk}
\says{I}[peger ned på uret] Beklager, akademisk kvarter.
\scene{I går op i køkkenet igen}
\scene{DM finder sin fyldepen frem, og begynder at løse sin opgave} %Han skal vel vente på DK får sin mad
\says{DK} Hvordan er din opgave skatte-nuser-pulder-basse?
\says{DM}[Prikker til opgaven med sin fyldepen] Jeg tror altså ikke det er introduktion til programmering det her
- Det ligner Implementation af Programmeringssprog

\scene{DM rejser sig op, og støder ind i I som skulle til at aflevere opgaven til
DK, men taber den på gulvet i stedet}
\says{I} LØGRUS!
\says{DK} Undskyld hvad sagde du til min Bamse-lamse-lambda-bimse-basse?
\says{I}[underspillet] Ehh\ldots ups! Jeg sagde "nøjjj ups".
\says{DM} Der er altså en bug i min aflevering!
\says{I} Det beklager jeg meget, jeg opdaterer den straks.
\scene{I tager en stor rød tusch frem og sætter en streg på opgaven.
       Han skal til at give den tilbage, men sætter lige en streg mere}

\says{I}[undertrykt sur] Værsgo hr.
\scene{DM begynder kritisk at løse opgaven, men der er noget galt}
\says{DM} Den her opgave er slet ikke nem! Jeg forstår slet ikke hvad jeg skal!
\says{I} Jo, jo, det er ny nordisk opgaveformulering. Det er minimalisme!
\says{DM} I har jo bare fjernet halvdelen af bogstaverne!

\says{DK}[Ser på sit ur] Nu har jeg siddet og ventet på min karakter i 4 blokke,
                         og den er altså stadig ikke kommet endnu!
\says{I}[opgivende] Jeg skal straks snakke med den undervisningsansvarlige!

\scene{I går over i køkkenet. Skriver på sin maskine.}

\scene{AV: Man ser at I har en mailbox fyldt med ubesvarede mails fra undervisningsansvarlige.
           I skriver en ny mail. Får et autosvar om at han ikke er tilgængelig.
           Den skal slutte med et "J".}

\scene{I sukker. Desinficeringsmaskinen KEN vibrerer og I tager det som et svar.}
\scene{I går over til DM igen.}

\says{I} Jeg har hørt at KEN er færdig om lidt, så den kommer snart.
\says{I}[råber til I'] Husk at trække bundproppen på KEN inden du går!
\says{DK} Ej, det kan jeg altså ikke lige overskue, skal vi ikke droppe ud
honey-bonny-blomster-kæreste-skatte-honey?
\says{DM} lad os. Instruktor, må vi bede om eksamen?
\says{I} Selvfølgelig. Det bliver 7.5 ECTS, vil i dele?
\says{DM}[visker til DK] Hvor mange ECTS giver man i drikkepenge?
\scene{DK begynder at rode i sin taske}
\says{DM} \act{signalerer "jeg har den!" til DK}
          Jeg skal nok betale pusse-basse-søde-bløde-skatte-pige-person.
          Instruktor! Skriv den på reeksamen!
%\scene{DM + DK kindkysser}
%\says{DK} Du er bare den bedste, kysse-mysse-trold-skat....
%\says{I}[Afbryder] JA! tak!
%\scene{I tager rabat kortet, og scanner det eller noget}
%\says{I} Nåå, det er dit 5. -3, så får du et gratis 00 næste gang

%% EVT. BEDRE PUNCHLINE, MÅSKE OPSTRAMNING


%%"Instruktor! Jeg sagde jeg ville have opgaven *uden* monader! Men der er altså en option-type hér!" "Ja, men du behøver ikke at bruge do-notation..."
%Der ligger en skål med kridt oppe ved kassen for at friste kunderne.
%Glas-montre med forskellige opgaver man kan vælge imellem.
%En mor der sidder og giver sin baby en opgave. Folk kigger fordømmende på hende. "Nej, det er okay; den er i Scratch."
%To der opdager at de har fået præcis den samme opgave og bliver forargede. "Er det har bare en masseproduceret vare?"

\scene{Sluk bål, mammut for.}

\end{sketch}
\end{document}
