\documentclass[a4paper,11pt]{article}

\usepackage{revy}
\usepackage[utf8]{inputenc}
\usepackage[T1]{fontenc}
\usepackage[danish]{babel}


\revyname{DIKUrevy}
\revyyear{2017}
\version{0.1}
\eta{$3$ minutter}
\status{Færdig}

\title{Analogi}
\author{Troels, Phillip, Nana, med småændringer i 2017 af manuskriptkomiteen}

\begin{document}
\maketitle

\begin{roles}
\role{A}[Niels] Manden der forklarer analogier
\role{X}[Caroline] Instruktør
\end{roles}

\begin{props}
\prop{AV: Et screenshot af en smartphone på Twitter hvor nogen beklager sig}[Søren?]
\end{props}


\begin{sketch}

\says{A} Hej med jer!

\says{A} Efter første akt har DIKUrevyen fået skarp kritik for at vi havde en
analogisketch.

\scene{AV: Et screenshot af en smartphone på Twitter hvor nogen beklager sig.}

\says{A} Det har vi nu gået en hel akt og tygget på, og vi må give jer ret: Man
\emph{kan} simpelthen ikke \emph{forvente} at studerende \emph{forstår} hvad en
analogi \emph{er}!

\says{A} Derfor har vi \emph{nu} på KUA oprettet en ny tværfaglig
bacheloruddannelse i Analogi.  Vi har fået den ære at holde første forelæsning
her i Store UP1 til DIKUrevyen, så I bedre kan forstå de resterende numre.

\says{A} Ser I, en analogi er lidt som en metafor... hmm...

\scene{A står og vugger lidt frem og tilbage.}

\says{A} \act{lidt til sig selv} Det er nok for abstrakt.  \act{nu til publikum}
Lad mig prøve at relatere det til noget \emph{I} kan genkende fra \emph{jeres}
forskellige studier.

\says{A} Så, til de jurister der måtte være i publikum... en analogi er ligesom
når I har en kopimaskine og kopierer fra A3 til A4 -- og tilbage igen.  For hver
gang gennem analogien, altså kopimaskinen, bliver kopi-kvaliteten
dårligere.  Efter 7-8 udledninger, altså kopier, har man bare en sløret grå
masse.  Og så er det hele meget mere forståeligt!

\scene{Visualiseres undervejs med AV, med et analogibillede (f.eks. ``datamat ~
bil'') der bliver mere og mere sløret.}

\says{A} Og til jer matematikere...  Som I ved, i en isomorfi, der har man en
pil fra én mængde til en anden mængde.  Sådán er en analogi sådan set
også... men pilen er lidt krøllet på midten. \act{Visualiseret med med
overhead.}

\says{A} \act{lidt til sig selv} Hm, hvordan får man nu fysikere til at forstå
analogier?  \act{nu til publikm} Altså... en analogi er ligesom da I startede på
fysik, fordi I ikke forstod hvad katten lavede i den kasse, og pludselig bliver
I så sendt til Schweiz for at lede efter en partikel der er blevet væk.  Kan I
se? \act{det kan publikum godt}  Godt.

\says{A} Til datalogerne har vi forsøgt at relatere analogien til noget, I lader
til at have en \emph{virkelig} omfattende \emph{teoretisk} forståelse for:

\says{A} SEX!

\says{A} Så, en analogi er til virkeligheden, lidt ligesom onani er til sex.
Altså, man opnår cirka samme resultat, men man skipper nogle \emph{riiiimeligt}
essentielle punkter.  Til gengæld kræver det mindre tid end den fulde
forklaring.

\says{A}[mere ophidset] Jajaja!

\says{A} En analogi er også lidt som analsex -- det er ikke altid man rigtigt
kommer igennem med det man vil, man føler sig ofte beskidt bagefter, og nu og da
ender det bare med noget værre lort!

\says{A}[til sig selv] Drøngod pædagogik! \act{nu helt oppe at køre}

\says{A} En analogi er lidt ligesom når man af uransagelige årsager har fået
hende den søde pige fra festen til at gå med hjem, men man er nervøs
\act{begynder at vakle}, og så svigter pikken i elskovsøjeblikket \act{nu
trist}, og man siger til hende: ``Det her er lidt ligesom når en bil ikke vil
starte: Når man ikke skifter olie ofte nok, så ruster motoren fast. Min bil
plejede at være større, men det er ret koldt. Hvor skal du hen?''.  På samme
måde kan analogier være svære at få op at stå, \act{lyset begynder at fade ud}
hvis publikum er utålmodige.  Hov, vent, hvor skal I hen?

\end{sketch}
\end{document}
