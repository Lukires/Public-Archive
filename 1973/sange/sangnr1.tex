\documentclass[a4paper,11pt]{article}

\usepackage{revy}
\usepackage[utf8]{inputenc}
\usepackage[T1]{fontenc}
\usepackage[danish]{babel}


\revyname{DIKUrevy}
\revyyear{1973}
\version{0.1}
\eta{$n$ minutter}
\status{Færdig}

\title{Sang nr. 1 (?)}
\author{?}
\melody{``Kom, kom, kom til frelsermøde''}

\begin{document}
\maketitle

\begin{roles}
\role{S}[] Sanger
\end{roles}

\begin{song}
\sings{S} Kom, kom, kom til datalogisk
sommerfest hos Rosenquist,
der er øl og der er mad
spis og drik og vær nu glad
for i morgen fejrer vi jo Sankte Hans.

Se, se, se nu op på scenen
nu vi spille vil for jer,
stykket er måske lidt tyndt
for vi er jo knap' begyndt
men til moro vil det være for enhver.

Nu, nu, nu vil vi fortælle
lidt om vores institut
vi har ikke megen plads
når vi lager en gang gas
må vi bruge bibli'otekets indgangsdør.

Kom, kom, kom til kaffepause
kom og hyg jer her hos os
Vi har kaffe, vi har fløde
Og lidt sukker til at søde
denne bitre drik som vi nu brygger her.

Her, her, her er kaffestue
og skam også bibliotek,
her du kan få lov at læse
hvis du prøver på at blæse
på at vi vil more os, måske lidt højt.
\end{song}

\end{document}

