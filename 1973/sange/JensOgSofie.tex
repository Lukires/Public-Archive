\documentclass[a4paper,11pt]{article}

\usepackage{revy}
\usepackage[utf8]{inputenc}
\usepackage[T1]{fontenc}
\usepackage[danish]{babel}

\revyname{DIKUrevy}
\revyyear{1973}
\version{1.0}
\eta{? min}
\status{Færdig}

\title{Jens og Sofie var kærester}
\author{Peter Johansen}
\melody{``Frankie and Johnny were lovers''}

\begin{document}
\maketitle

\begin{roles}
\role{S}[Lene Weiss] Sanger
\end{roles}

\begin{song}
Jens og Sofie var kærester
Her vil jeg synge om dem
Jens han var programmør
Sofie en IBM
Sin datamat
vil han altid vær' tro.

Hver gang de skulle sortere
båndstationerne spandt
Når hun på sit baggrundslager
Alle hans filer fandt
Sin datamat
vil han altid vær' tro

Efter han traf sin Sofie;
hans liv det blev aldrig som før
Hun lærte han at blive
multiprogrammør
Sin datamat
vil han altid vær' tro

Som vinteren bliver til sommer
det fuldendte ej varer ved
Da reklamen med posten kommer
så glemmes hans kærlighed
Sin datamat
vil han altid vær' tro

Jens han har aldrig hørt magen
brochuren har klart det fortalt
Sofie II er sagen
hun kan næsten alt
Sin datamat
vil han altid vær' tro

Dagen da sælgeren var der
Sofie, hvor var hun jaloux
Hun skrev på skrivemaskinen
"`Programmér mig NU!"'
Sin datamat
vil han altid vær' tro

Med blinkende lys i panelet
den følgende dag ganske nemt
Rivalinden blev installeret
Sofie hun var glemt
Sin datamat
vil han altid vær' tro

Jens ville bryde kontakten
til båndstation nummer et
Men da lyned' det fra stikket
kortslutningen var sket
Sin datamat
vil han altid vær' tro

Jens fald til Jorden og død
de andre de iled' derhen
Ak, udskrev Sofie
jeg følger dig min ven
Sin datamat
vil han altid vær' tro

Om begravelsen blev det berettet
kirken var stuvende fuld
Hans kiste den var kabinettet
til hendes lagermodul
Sin datamat
vil han altid være tro

Moralen af denne historie
er den som jeg nu dig betror
Hver gang du programmerer
så husk på disse ord:
Sin datamat
skal man altid vær' tro
\end{song}

\end{document}
