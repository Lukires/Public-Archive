\documentclass[a4paper,11pt]{article}

\usepackage{revy}
\usepackage[utf8]{inputenc}
\usepackage[T1]{fontenc}
\usepackage[danish]{babel}


\revyname{DIKUrevy}
\revyyear{1973}
% HUSK AT OPDATERE VERSIONSNUMMER
\version{1.0}
\eta{? minutter}
\status{Færdig}

\title{Gud Besøger Edda}
\author{?}

\begin{document}
\maketitle

\begin{roles}
  \role{G}[Mette] Gud
  \role{E}[Mark] Edda
  \role{S1}[] Student 1
  \role{S2}[] Student 1
\end{roles}

\begin{sketch}
  \scene{Edda i sit kontor.  Bank på døren.}

  \says{E} Kom ind.

  \says{G} Goddag.  Er Deres navn Edda Travlt?

  \says{E} Ja, det er det.  Hvem er De?

  \says{G} Man kalder mig ofte vorherre.

  \says{E} Næ, Peter Naur, er du klædt ud til karnival?

  \says{G} Jeg tror ikke De forstår.  Jeg er ham der kan lave alting,
  den almægtige.

  \says{E}[forvirret] Vil De gerne gentage det sidste.

  \says{G} Jeg er den almægtige.

  \says{E} Men de taler ikke med bornholmsk accent.

  \says{G} Hvordan kan Jeg forklare?  Jeg er den højeste af de høje, skaberen af alting, den...

  \says{E} Gu-ud, er det Dem?  \act{Gud nikker ja} De må endelig
  undskylde mig, at jeg forvekslede Dem med et menneske.  Der kommer
  simpelthen så mange her i kontoret at jeg finder det svært at holde
  rede på hvem der er hvem.  Lige i går kom der en, der påstod, han var
  fra ministeriet, og han prøvede at bilde mig ind, at det nu er muligt
  at få en eksamen med datalogi som hovedfag.

  \says{G} Nej, det er Dem jeg vil tale med, Edda.  Jeg er her for at få
  udført en datalogisk praktisk opgave af deres studerende.  Det drejer
  sig om...

  \scene{Bank på døren}

  \says{E} Kom ind.

  \scene{S1 kommer ind.}

  \says{S1} Dav, Edda.  Undskyld jeg forstyrrer dig.  Jeg vidste ikke
  du havde en gæst.

  \says{E} Kom bare ind.  Det ku' være at vores gæst har en opgave du
  kan lave.  Må jeg præsentere for dig: Gud.

  \says{S1}[griner] Og jeg hedder Per Hækkerup.  Kan du ikke være lidt
  mere alvorlig, Edda.

  \says{E} Det {\em er} Gud.

  \says{S1}[forvirret og genert, han trykker Guds hånd] De må meget
  undskylde, Deres højhed.  Det er for mig en fantastisk stor ære at
  træffe Dem.  Jeg er en af deres største fans.  Jeg hedder...

  \scene{Bank på døren.}

  \says{E} Kom ind.

  \scene{S2 kommer ind.}

  \says{S2} Hej, Edda.  Hvordan går deT?

  \says{E} Udmærket.  Jeg vil gerne præsentere for dig vores ærede
  gæst.  Han er ikke nogen anden end gud.

  \says{S2}[nonchalant] Dav, Gud, jeg er Jens.

  \says{G} Goddag Jens.  Det ku' være at De også vil være med til at
  lave et datalogisk praktisk projekt for mig.

  \says{S2} Hvad fanden "= øh "= i himmelens navn ku' det være?` Har
  du tænkt at lave en universmodel.  Vi ku' for eksempel finde ud af
  hvor meget energi der er tilbage i stjernerne.

  \says{G} Hvad jeg har tænkt på er noget meget mere beskedent og
  jordisk.  Jeg har erfaret, at der er nogle få stykker her i Danmark
  der har mistet deres tro på mig.

  \says{S2} Det kan du bande på.

  \says{G} Det er faktisk {\em det}, der er opgaven jeg vil stille, at
  finde ud af hvem det er der har mistet troen, så jeg kan lede dem
  tilbage på den rigtige vej. \act{Til Edda} Er det muligt at lave et
  register over dem der ikke tror, så jeg kan ajourføre mit eget
  kartotek?

  \says{E} Vi plejer normalt ikke at lave hemmelige kartoteker her.
  Men siden det er {\em Dem}, kan det nok lade sig gøre en enkelt
  gang.

  \says{G} Godt, lad os begynde.  Rygtet har nået mine øren at indtil
  flere her i landet har meldt sig ud af folkekirken.  Hvor mange vil
  De anslå tilhører kirken?

  \says{S1} Den seneste statistik viser at 95 procent danskere er
  medlemmer af folkekirken.

  \says{G} Så må opgaven være aldeles let.  Det glæder mig at høre at
  der findes så mange troende her i kongeriget.

  \says{S2} Troende?  Hvem sagde troende?

  \says{G} Har folk ikke meldt sig ind fordi de er troende?

  \says{S2} Meldt sig ind?  Næj, de har snarere ikke giddet at melde
  sig ud.

  \says{G} Det er et underligt system.  Kan vi så ikke registrere hvem
  der kommer i kirkerne?

  \says{S1} Den sidste statistik viser at 2 procent danskere går i
  kirke.

  \says{G} Det er ikke ret mange, men {\em de} må være troende.

  \says{S2} Skal vi ikke slette dem der snorker i kirken, og dem der
  kommer, fordi Knud Børge skal døbes, og dem der kun kommer af gammel
  vane til jul og påske?  Så kan der ikke være ret mange tilbage.

  \says{G} Det betyder vi er nødt til at registrere næsten alle i
  landet.

  \says{E} Det er ikke noget særligt problem, det er allerede gjort.
  Det hedder CPR nummer.

  \says{S1} Men det er ikke {\em så} enkelt.  Sku' vi inkludere dem
  der tror på DEm, men ikke går i kirke.  Og hva' med dem der tvivler.
  Og hva' med dem der...

  \says{S2} Du, Gud, sku' vi ikke hellere gå ud og spille agurk?

\end{sketch}
\end{document}
