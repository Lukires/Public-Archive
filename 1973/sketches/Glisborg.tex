\documentclass[a4paper,11pt]{article}

\usepackage{revy}
\usepackage[utf8]{inputenc}
\usepackage[T1]{fontenc}
\usepackage[danish]{babel}


\revyname{DIKUrevy}
\revyyear{1973}
% HUSK AT OPDATERE VERSIONSNUMMER
\version{1.0}
\eta{? minutter}
\status{Ikke færdig}

\title{Glisborgs spørgeundersøgelse}
\author{?}

\begin{document}
\maketitle

\begin{roles}
\role{A}[Mette] Fru Glisborg
\role{B}[Lissi]
\role{C}[Lene]
\role{D}[Jesper]
\end{roles}

\begin{sketch}
  \says{A} Goddag, mit navn er fru Glisborg.  Jeg kommer fra
  fremgangspartiet.  Vi er ved at lave en undersøgelse om
  statsansatte, og vil gerne vide lidt om hvilke mennesker, der er
  ansat, og hvorledes deres arbejdsvilkår er.  Hvem mon jeg kan tale
  med om det?

  \says{B} Det må være PJo, De du øh skal tale med.

  \says{C} Jamen han er til møder.  Han skulle først snakke om
  forældre + børn og derefter spille bridge.

  \says{D} Hvad med Edda?

  \says{B} Jeg så hun fór forbi ude på gangen lige før, hun er vist
  nede på centralværkstedet og samle en vogn, men måske PLA?

  \says{D} Du ved da godt at det er hans sandkassedag i dag og PJe og
  JSM leger med.

  \says{C} Og TUZ spiller kryds+bolle og Gregers er i bad.

  \says{A} Sig mig, det {\em er} da Datalogisk Institut detteher?

  \says{B+C+D} Ja selvfølgelig, hvad andet kan det være?

  \says{A} Jeg har nogle spørgeskemaer.  På det første skal De bare
  sætte kryds ud for a) udmærket, b) ret godt og d) mindre godt ved de
  forskellige spørgsmål.

  \scene{B, C og D får hver et spørgeskema og læser på skift
    spørgsmålene op og kommenterer dem højlydt.}

\begin{itemize}
\item Hvorledes er De tilfreds med Deres daglige arbejde på Deres
  nuværende arbejdsplads?
\item Hvorledes er De tilfreds med den måde, arbejdet er tilrettelagt
  på Deres nuværende arbejdsplads?
\item Hvorledes er De tilfreds med "`tonen"' på deres nuværende arbejdsplads?
\item Hvorledes er de tilfeds med abejdstempoet på Deres nuværende arbejdsplads?
\item Hvorledes er de tilfreds med muligheden for at holde pauser
  udover de faste spisetider på Deres nuværende arbejdsplads?
\end{itemize}

\scene{Deler skemaer ud.  Alle betragter dem vantro, og begynder at
  grine.  Læser spørgsmålene højt for hinanden, hrtigt, ingen får tid
  til at krydse af.}

\says{D} Har De været arbejdsløs om sommeren i de sidste 3 år?

\says{B} Hvor mange år gik De i skole?

\says{C} Har De gået en klasse om?  Og er de skilt?

\says{D} Har De børn?

\says{B} Har De eget hus?

\says{C} Har De vanskeligheder i hjemmet?

\says{D} Har De været indlagt på hospital på grund af dårlige nerver?

\says{B} Har De fået bøde for færdselsforseelse?

\says{D} Lider De ofte af dårlig appetit?

\says{B} Føler De Dem ofte misforstået af andre?

\says{D} Får De ofte kolde fødder?

\says{B} Har De problemer med afføringen?

\says{D} Har De problemer med Deres sexualliv?

\scene{A samler surt skemaerne sammen og vil nu selv stille nogle
  spørgsmål.  Jesper og Lissi giver følgende svar.}

\says{D} Når kantinen åbner.

\says{B} Efter morgenkaffen.

\says{D} Så er kaffen kold.

\says{B} Ja, for at slukke efter dem, der har været der om natten.

\says{D} Ja, tænk på arbejdsklimaet!

\says{B} Ja, for vi arbejder i frokoststuen.

\says{D} Nej, så må arbejdet hvile.

\end{sketch}

\end{document}
