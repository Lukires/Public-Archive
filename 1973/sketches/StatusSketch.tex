\documentclass[a4paper,11pt]{article}

\usepackage{revy}
\usepackage[utf8]{inputenc}
\usepackage[T1]{fontenc}
\usepackage[danish]{babel}


\revyname{DIKUrevy}
\revyyear{1973}
% HUSK AT OPDATERE VERSIONSNUMMER
\version{1.0}
\eta{? minutter}
\status{Ikke færdig}

\title{StatusSketch}
\author{?}

\begin{document}
\maketitle

\begin{roles}
\role{S}[Ulæselig krusedulle] Speaker
\role{I}[Ulæselig krusedulle] Interviewer
\role{A}[Ulæselig krusedulle] Arne Komet
\role{K}[Jesper] Kameramand
\end{roles}

\begin{sketch}
\scene{TV-avisens kendingsmelodi.}

\says{S} Godaften.  Vi bringer først et eksklusivt interview med Arne
Komet dernæst, hvis tiden tillader det, en samtale med præsident
Nixon, der for 4 timer siden blev afsat som de Forenede Staters
præsident.  Vejrudsigten bliver udsat.  Vejret må vente.  Nu... Arne
Komet, fristaten datalogisk institut.

\scene{K drejer kamera mod interviewer og Arne Komet.}

\says{I} Arne Komet, hvordan gik det til?

\says{A} Jo: Datalogisk Institut har altid indtaget en særstilling ved
Københavns Universitet, og nu da konsistorium var i tvivl om
konfererede med undervisningsministeren og statsministeren, hvad var
da naturligere end at vi meddelte vores beslutning.

\says{I} Arne komet, hvordan gik det til at meddelelsen om oprettelsen
af Fristaten datalogisk institut blev accepteret af regeringen, og at
der i disse knappe tider blev lovet {\em 4} ekstra blyanter
pr. medarbejder PLUS et våbenskjold med datoen for løsrivelsen udført
af et kor af syngende programmører.  Er det i den forbindelse rigtigt
at datalogisk institut har store udgifter til lytteudstyr og til
afspilning af bånd.

\says{A} Det er rigtigt.  vi bruger imidlertid en mere raffineret
elektronisk teknik end den, som det republikanske parti anvender i
USA.  Ved anvendelsen af vor teknik blev alle krav opfyldt.

\says{I} Fortæl, vore seere er spændte.

\says{A} Jeg tror såmænd ikke det kan gøre noget at råbe vor teknik:
De tekniske hjælpemidler kan alle skaffe sig, men den raffinerede
brug!  Først lod Thor Bak os forhandle direkte med ministeriet efter
at vi lod ham høre 5 minutter af Peter Johansens autoforelæsning om
informationsteori.

\says{I} Stakkels mand.

\says{A} Heinesen lovede os de nævnte goder og desuden en
blyantspidser anno 3 efter Glistrup efter at have påhørt Gregers Kocks
epos om fikspunktsteori, og Anker Jørgensen lovede at udveksle
ambassadører med Fristaten kun et kvarter inde i Nils Andersens
kombinatoriske algoritmeanalyse.  Men jeg gentager: Uden disse mestre
på den akustisk elektroniske torturs område var vi aldrig nået så
vidt.

\says{I} Hvem er Fristatens ambassadør til Kongeriget Danmark?

\says{A} Vor mand ved kongeriget Danmark blev en gammel trænet
diplomat Edda Islænder; da hun aldrig er til at træffe, får vi jo
ingen diplomatiske uoverensstemmelser.

\says{I} Det er sagt, at man på instituttet...

\says{A} Ja, jeg må her kort nævnte vort rationaliseringsforslag:
Førhen har datalogisk institut jo haft rigeligt at gøre med at
administrere sine aktiviteter, så der er jo ikke blevet tid tilovers
ti lundervisning og forskning, så derfor har vi besluttet at nu skal
{\em ingen} administrere overhovedet.

\says{I} Har dette forlsag noget med de nyeste...

\says{A} Overhovedet ikke!  Intet med student Glostrups latterlige
kommentarer at gøre.

\says{I} Hvem besætter de øvrige posteR?

\says{A} Stjerne Peter er vores PR mand.  Han udtaler sig om hvad som
helst.  Peter og Peter er maskinmestre.  Et led i vores spareplaner er
bygning af egen datamat med 8 gear og egen tromletaske.  Den vil være
billig i drift.

\says{I} Hvorfor?

\says{A} Den vil aldrig komme til at virke.

\says{I} Hvordan er fristatens internationale image?

\says{A} Helt på toppen.  Torben 5-på-rad varetager vores
internationale forbindelser.  I 1972 blev han europamester i kryds og
bolle.  Han er GO nok.

\says{I} Har du en bemærkning til slut?

\says{A} Ja, vi vil gerne foreslå TV-avisen at rapportere fra
Fristatens område her i aften indtil kl. 10.

\says{I} Det kan desværre ikke lade sig gøre, der skal være
fodboldkamp mod Sverige.  Tak for i aften, Arne Komet.

\scene{Kameraet drejes tilbage mod speakeren.}

\says{S}[ser på klokken] Desværre, Nixon udgår, her er vejret.

\scene{S tavs, kamera mod ham, drejes ikke.  A tager en
  kasettebåndoptager frem, putter en kasette i, nu lder PJo's
  autoforelæsning om semaforer.  Speaker rynker brynene, Kameramand
  hænderne for ørene, styrter bort, speaker tager skilt frem med
  PAUSE, holder det op, hænger det på tv'et, gestikulerer, falder på
  knæ for A.}

\says{S} Stands, stop, ja, jeg lover, vi skal nok.

\scene{A slukker, S sætter sig til rette, kameramand flytter skilt.}

\says{S} I aften har vi en programændring.  Svenskekampen udgår, og vi
bringer en reportage fra Fristaten datalogisk institut.
\end{sketch}
\end{document}
