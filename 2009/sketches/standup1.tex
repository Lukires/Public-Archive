\documentclass[a4paper,11pt]{article}

\usepackage{revy}
\usepackage[utf8]{inputenc}
\usepackage[T1]{fontenc}
\usepackage[danish]{babel}

\revyname{DIKUrevy}
\revyyear{2009}
% HUSK AT OPDATERE VERSIONSNUMMER
\version{1.2}
\eta{1 minut}
\status{Færdig}

\title{Standuptragikeren}
\author{Troels}

\begin{document}
\maketitle

\begin{roles}
\role{T}[Troels] Standuptragiker
\end{roles}

  
\begin{sketch}

\says{T}[Voiceover, med sin egen stemme] Tag godt imod....DIKUs Standuptragiker!

\scene{T kommer ind fra siden}

\says{T} Tak...tak..

Sig mig, hvad er der med det hvor man har læst i 4 år, og så melder man sig til algoritmikeksamen, og så får man en mail om at man har brugt alle sine forsøg og man ikke kan få dispensation?\\

Men ellers sidder jeg ret meget foran computeren. Faktisk kommer jeg ikke så meget ud. Kender I det? Så ringer jeg efter en pizza og da jeg så skal til at sige noget går det op for mig at jeg ikke har talt til nogen i en uge. Kender I det?\\

Kender I så det når man søger job som kodeslave hos kapitalen, og de beder om en punkt.ku-udskrift, men den følger mest af alt samme mønster som Fulbert og Beatrice - som man så forsøger at synge for HR-medarbejderen hos Mærsk, men stemmen knækker over lige efter 3. vers og man får ikke jobbet? \act{afventer publikums reaktion}

\scene{T går ud igen, uden at sige mere}

\end{sketch}
\end{document}

%%% Local Variables: 
%%% mode: latex
%%% TeX-master: t
%%% End: 

