\documentclass[a4paper,11pt]{article}

\usepackage{revy}
\usepackage[utf8]{inputenc}
\usepackage[T1]{fontenc}
\usepackage[danish]{babel}

\revyname{DIKUrevy}
\revyyear{2009}
% HUSK AT OPDATERE VERSIONSNUMMER
\version{1.2}
\eta{$4$ minutter}
\status{Færdig}

\title{Det har været et dejligt år}
\author{Fairchild og Phillip}

\begin{document}
\maketitle

\begin{roles}
\role{HS}[Munter] Helge Sander
\end{roles}

\begin{props}
\prop{Domptørjakke}[Rekvisitgruppen]
\prop{Briller}[Munter]
\prop{Pisk}[Munter]
\end{props}

\begin{sketch}

%\scene{Beskrivelse}

\says{HS} God aften, sidder I godt?

\says{HS} Det havde jeg også regnet med, datalogi er jo kendt for at være det studie hvor man sidder på de største og blødeste bagdele.

\says{HS} Mit navn er Helge Sander, og jeg er blevet inviteret til at sige et par ord på regeringens vegne, inden I 
får lov til at slå jer løs med jeres uproduktive dilletanterier i 3 akter.

\says{HS} Ja, I hørte rigtig. Revyen har bedt mig om at gøre jer opmærksom på at dette års ekstra-curriculære, ikke 
ECTS-producerende, og politisk fejlorienterede revy, er udvidet med én ekstra akt.

\says{HS} \act{Videregiver hemmelighed} ... Ja I har det ikke fra mig, men jeg har hørt at 3. akt udelukkende består 
af sange skrevet på Disneymelodier, for at fejre at dette års Matematikrevy ikke var dræbende kedelig.

\says{HS} \act{slår ud med armene} Det har været et dejligt år!

\says{HS} I, på det naturvidenskabelige fakultet har fået første fase af regeringens nye politik at føle.

\says{HS} Vi har indført studieaktivitetskrav... Ja I troede vel ikke, at det var et tilfælde at fakultetet blev ramt af voldsomme, ``pludselige'' og ``uforudsete'' nedskæringer?

\says{HS} Vi har valgt, helt i Darwins ånd, at skille de svage uddannelsesretninger fra dem vi skal bruge til at skabe 
morgendagens samfund!

\says{HS} Regeringens slogans skal naturligvis også opdateres:

\says{HS} "Uddannelse i verdensklasse" bliver derfor til "Nedskæringer i verdensklasse"

\says{HS} "Fra tanke til faktura" bliver til "Fra tanke til fyreseddel"

\says{HS} Hvilket minder mig om, at hvis I føler jer er for dumme, dovne eller svage til at kunne gennemføre jeres studier, så kan I hvornår det skal være, vælge at droppe ud via en af salens i alt seks nødugange. To oppe bagved, to i siderne og to bag ved scenen. Jo før I gør det, jo bedre for jer selv, og for os allesammen.

\says{HS} Husk altid på, at denne regering går meget op i at sikre det frie valg, også når det drejer sig om at vælge at droppe ud.

\says{HS} Ja... giv jer selv et øjeblik til at overveje om I virkelig lærer noget her på DIKU. Jeg har jo hørt fra min gode ven Poul Henning Kamp, at det ikke er nødvendigt at have en uddannelse for at kunne programmere. Ifølge Poul Henning er fysikere langt bedre til at kode end uddannede dataloger.

\scene{fysikerne hujer, datalogerne buer}

\says{HS} Poul Henning er også et prægtigt eksempel på et stolt individ, der som selvlært programmør ikke er blevet åndeligt forkrøblet af et pylrende uddannelsessystem, der nurser de studerende igennem og giver dem alt hvad de peger på. Og ikke mindst har han vist at man sagtens kan blive til noget, noget der betaler topskat, uden nogensinde at have modtaget SU.

\says{HS} Så I kan takke Poul Henning for at SU'en fra næste studieår bliver halveret.

\scene{publikum buer}

\says{HS} Jeg vil ikke høre den slags klynk på mine universiteter! Der er overhovedet ikke noget at komme efter!

\says{HS} Og I kan lige så godt vænne jer til at der i morgendagens samfund forventes disciplin i verdensklasse. Så hvis jeg ser nogen i salen der har en tændt mobiltelefon eller så meget som tænker på bruge åben ild, så bliver de bortvist på stedet. Ikke kun fra salen, men helt ud áf universitetet. Og dem uden dansk statsborgerskab vil blive smidt helt ud af landet!

\says{HS} Det var alt hvad jeg havde på hjerte, så på regeringens vegne vil jeg gerne afslutte med at sige:

\says{HS} Husk at nyde årets DIKUrevy... Det kunne jo være det bliver den sidste.

\scene{Lys ned}

\end{sketch}
\end{document}

%%% Local Variables:
%%% mode: latex
%%% TeX-master: t
%%% End:
