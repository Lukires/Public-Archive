\documentclass[a4paper,11pt]{article}

\usepackage{revy}
\usepackage[utf8]{inputenc}
\usepackage[T1]{fontenc}
\usepackage[danish]{babel}

\revyname{DIKUrevy}
\revyyear{2009}
% HUSK AT OPDATERE VERSIONSNUMMER
\version{1.1}
\eta{4 minutter}
\status{Opførbar}

\title{phk}
\author{Madß}

\begin{document}
\maketitle

\begin{roles}
\role{R}[Ronni] Revyt (gerne med kort/trimmet hår)
\role{PHK}[Rune] Poul-Henning Kamp
\end{roles}

\begin{sketch}

\scene{R træder frem på scenen}
\says{R} Der er mange datalogistudererne, som mig selv, der går og tænker på
	hvad de skal bruge studiet til, når de engang er færdige. Vi er jo
	trods alt IT-branchens jægersoldater. For at kaste lys over
	mulighederne har jeg derfor inviteret en kendis fra det private
	erhvervsliv.
\scene{PHK træder ind på scenen}
\says{R} Poul-Henning. Du har jo gang i mange projekter, hvor det ville
	være oplagt at sætte en datalog til at kode...
\says{PHK} Kode? Hvad tænker du på? Det er de jo slet ikke rustet til. Det
	ville jo være lige så tåbeligt som at sætte en kok til at lave mad.
\says{R} Ah ... hvis ikke dataloger skulle kode, hvem i alverden skulle...
\says{PHK} Fysikere. Alt andet ville jo slet ikke give mening.
\says{R} Okay. Godt nok får fysikere nogle programmeringskompetencer når de
	skal implementere simulationer af deres fysik...
\says{PHK} Laver du sjov? Du mener da ikke for alvor at du ville sætte en
	fysiker til at lave fysik? Det er de jo slet ikke rustet til. Det
	ville jo være lige så tåbeligt som at sende en sømand til søs.
\says{R} Hvad i alverden mener du med det?
\says{PHK} Det er da oplagt, at når fysikens \emph{love} spiller en så stor en
	rolle som de gør, er det eneste rigtige at sætte jurister til at lave
	arbejdet.
\says{R} Du vil altså hente jurister ind fra det danske retssystem, når der
	er brug for...
\says{PHK} Nej, nej, nej. Jurister i det danske retssystem? Jeg har aldrig
	hørt noget så tåbeligt! Det er de jo slet ikke rustet til. Hvad
	bliver det næste? Læger i sundhedsvæsenet?
\says{R} Jurister skal altså udelukkende beskæftige sig med fysik?
\says{PHK} Så absolut. Det er almen viden at danske retsager bliver afgjort af
	så små faktorer at jura aldrig ville gøre en forskel alligevel. Her
	må man tage nanoscience i brug.
\says{R} Men nanoscience har jo intet med jura at gøre. Det er jo
	højteknologisk forskning. Men det mener du vel heller ikke at
	nanoscience er rustet til?
\says{PHK} Lige præcis. Kun et sted på Københavns Universitet er man i
	stand til at løse den slags højteknologiske problemer.
\says{R} Og det er her dataloger...
\says{PHK} HUMANISTER, naturligvis. Ud over selvlærte, hvem skulle ellers føre
	Danmark ind som et videnssamfund i verdensklasse?
\says{R}   Men alligevel ... humanister? Jeg troede ikke at de lavede noget som
	helst brugbart.
\says{PHK} Ah. Her tænker du naturligvis på matematikerer. Jeg tror i øvrigt
	snart jeg er klar til at præsentere mit udkast til den nye
	studiereform.
\says{R} Hvad? Sætter man en selvlært til at lave studiereformen på
	Københavns Universitet?
\says{PHK} Ja, da. Det er akademikere slet ikke rustet til. Det ville jo være
	lige så tåbeligt som at sætte en frisør til at klippe dit hår.
\scene{R tager hånden op til sit manglene hår}
\says{R} Poul-Henning. Jeg tror altså ikke at publikum køber dine vilde
	påstande.
\says{PHK} Nej, for det er publikum jo slet ikke rustet til. Det svarer jo til
	at lade en mekaniker reparere din bil.
\scene{R følger PHK ud af scenen mens tæppet går for}
\says{R} Nu skal du se Poul-Henning. Nu følger jeg dig ud til bussen, og så
	kan du tage hjem. "Slet ikke rustet" ... vorherre bevares.
\says{PHK} Nej, nej, nej. Det er vorherre slet ikke rustet til...

\scene{Tæppe}

\end{sketch}

\end{document}
