\documentclass[a4paper,11pt]{article}

\usepackage{revy}
\usepackage[utf8]{inputenc}
\usepackage[T1]{fontenc}
\usepackage[danish]{babel}
\usepackage[]{amsmath}

\revyname{DIKUrevy}
\revyyear{2009}
% HUSK AT OPDATERE VERSIONSNUMMER
\version{1.4}
\eta{5 minutter}
\status{Færdig}

\title{110\%  farlig datalogi}
\author{Marvin, Mikkel, Phillip, Munter og andre}

\begin{document}
\maketitle

\begin{roles}
\role{A}[Klaes] Brian Vinter
\role{B}[Mikkel] Professor Hjul
\end{roles}

\begin{props}
\prop{SUN-trøjer, 2 stk.}
\prop{Kitler, 2 stk.}
\prop{Kasse med rekvisitter}
\prop{Kursuskatalog}
\prop{Sammenbundne hønseringe}
\prop{En blå T-klods, a. la. en tetris-brik}
\prop{2 bananer}
\end{props}


\begin{sketch}

\scene{A og B står foran tæppet, iført Sun-t-shirts og hvide kitler. Mellem dem står en tavle, og en kasse med rekvisitter, der skal bruges til at illustrere de videnskabelige principper. Skuespillerne opfordres til at finde på flere genstande i løbet af øveugen!}

\says{A} Goddag og velkommen til denne uges udgave af »Viden.COM«. Programmet hvor vi formidler \ldots Mit navn er Brian Vinter, og dette er min kollega, Professor Hjul. I dag skal vi tale om de seneste gennembrud inden for datalogien, nærmere bestemt den eksperimentelle pakkedatalogi.

\says{B} Om få dage vil verdens største datalogiske eksperiment til dato blive tændt. Man har udnyttet DNS-hullet til at ændre internettets routing-algoritmer, således at alle pakker tager den længste vej, på den korteste tid! Når pakkernes hastighed nærmer sig det tredobbelte af lysets, vil det skabe kollisioner så kraftige, at pakkerne vil blive splittet til deres grundbestanddele.

\scene{A holder en »pakke«, som han trækker fra hinanden, så indmaden ryger ud.}

\says{A} Projektet går under navnet Large Packet Collider, i daglig tale forkortet LPC.

\says{B} Vi ved ikke med sikkerhed hvad pakkerne består af. Ifølge ISO-standardmodellen er de opbygget af »bits«. Disse bits optræder i kvanter af 8 - de såkaldte »bytes«. Bits er indtil videre kun observeret i 2 naturligt forekommende varianter - de såkaldte 0- og 1-bits.

\says{B} Derudover findes muligvis en Clausen-bit, som kan antage vilkårlige værdier mellem 0 og 1. Det er blandt andet denne bit vi håber at finde spor af.

\says{A} Clausen-bitten er opkaldt efter forskeren Hasse Clausen. Man har ledt forgæves efter den, især i kursuskataloger og undervisningslokaler.

\scene{B leder febrilsk efter Hasse i et kursuskatalog.}

\says{A} Mange mener, at den muligvis slet ikke eksisterer, og indtil nu har man ikke kunnet finde en datalogisk mening med den.

\says{B} Men \emph{hvis} den findes, vil det have stor betydning for datalogien!

\says{A} Mange har måske undret sig over hvorfor samfundet skal betale 500 fantasillioner for at give datalogerne en LPC. Vores svar er:

\says{B} Fysikerne har endelig fået deres lille hønsering i Schweiz \act{ryster med nogle hønseringe}. Og at den stadig ikke virker, gør jo ikke ligefrem projektet billigere!

\says{A} Og senest har medicinerne overtalt Københavns Kommune til at financiere deres Large People Collider\ldots

\scene{B vender papiret på tavlen}

\says{B} \ldots også kendt som metrocityringen! 

\scene{På tavlen ses et kort over København, og evt. nogle skitser af mennesker der støder sammen, så man kan se hvad der er inden i.}

\says{A} I modsætning til fysikernes apparat \act{ryster med hønseringene igen}, er vores forsøg ikke kun ren teori - vi vil kunne skabe et stort
\says{B} Kæmpe!
\says{A} Uhyrligt! fremskridt for resten af samfundet, hvis det lykkes os at finde hvad vi i virkeligheden leder efter:

\says{A+B} Higgs-pakken! \act{slår ud med armene}

\says{B} Ved tilstrækkeligt store pakkesammenstød, vil bittenes indbyrdes tiltrækningskræft resultere i at nogle pakker samles igen. I ganske få tilfælde forudser vi dannelsen af den såkaldte

\says{A+B} Higgs-pakke! \act{slår ud med armene}

\says{B} Ifølge teorien vil man med Higgs-pakken kunne finde frem til alle de pakker der nogensinde er forsvundet. Dette giver jo god mening, da mængden af al information i universet er konstant, og pakkerne i givet fald må forsvinde et sted hen.

\says{A} Ja, Higgs-pakken er med andre ord universets NULL-pointer. Og det bliver endnu bedre! Ved hjælp af paritet, kan vi udlede alle de pakker der nogensinde er kommet frem!

\says{B} Jo, ser I, det er i virkeligheden meget simpelt. Higgs-pakken kan fortælle hvor alle tabte pakker befinder sig. Således vil vi have en mængde vi ved noget om, eller T \act{hiver et T frem fra kassen}, og så kan vi jo blot tage den inverse mængde - som vi ikke ved noget om - altså $\bot$ \act{vender Tet på hovedet}. Men vi ved jo, at den inverse mængde må være det data der er kommet frem - så vi må altså vide noget om noget vi ikke ved noget om ... eller noget i den stil.

Altså er vi forvirrede over et paradoks, og vi kan således anvende Gödels ufuldstændigheds-sætning, som siger at et formelt system enten kan være komplet, men inkonsistent - eller ukomplet og konsistent. Det er tydeligt at systemet er komplet, og derfor er det nemt at finde frem til de data der er blevet sendt!

\says{A} Og med denne viden om alle sendte data, vil vi bl.a. være bedre i stand til at afsløre sortseere, skattesnydere og bistandsmodtagere. Dette har fået regeringen til at indse de betydelige samarbejdspotentialer, og har medført at DIKUs nedlæggelse er blevet udsat til \emph{tidligst} år 2012.

\says{B} Derudover vil vi kunne finde kildekoden til Duke Nukem Forever \act{tager solbriller på}

\says{A} Samt den hemmelige ingrediens i Cola \act{tager en Pepsi op af kassen}

\says{B} Og vi kan løse sorteringsproblemet i kvadratisk tid!

\says{A} Alt dette, og meget mere, hvis blot vi finder:

\says{A+B} Higgs-pakken! \act{slår ud med armene}

\says{A} Selvfølgelig er der altid nogen, der er skeptiske overfor fremskridt. Der har 
været dommedagsprofetier fremme om alle mulige ting, der kan ske, når projektet går 
igang. Nogle mener, at for meget tabt data på ét sted, kan skabe et sort sikkerhedshul.

\says{B} Selv den ellers anerkendte danske sandsynlighedsforsker, Torben Bech Mogensen, foreslår, at man spiller 
syvkabale om, hvorvidt man overhovedet skal tænde for apparatet.

\says{A} Men kritikerne glemmer jo, at der hver dag sker pakkekollisioner på netværk 
over hele verden. Det her går bare \emph{lidt} hurtigere\ldots

\says{B} Det var så et lille indblik i hvad vi går og laver på DIKU.

\says{A} Følg med i næste uge, hvor Poul-Henning Kamp lærer en abe at blogge!

\scene{A+B tager begge en banan op ad lommen og nikker selvtilfredse til hinanden}

\scene{TFH!}

\end{sketch}
\end{document}


