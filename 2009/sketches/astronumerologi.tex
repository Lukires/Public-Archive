\documentclass[a4paper,11pt]{article}

\usepackage{revy}
\usepackage[utf8]{inputenc}
\usepackage[T1]{fontenc}
\usepackage[danish]{babel}

\revyname{DIKUrevy}
\revyyear{2009}
% HUSK AT OPDATERE VERSIONSNUMMER
\version{1.1}
\eta{5 minutter}
\status{Færdig}

\title{Astronumerologi}
\author{Phillip og Guldfisk}

\begin{document}
\maketitle

\begin{roles}
\role{A}[Klaes] Fysiker (astrolog)
\role{N}[Dirk] Matematiker (numerolog)
\end{roles}

\begin{props}
\prop{Kittel}
\prop{Fez}
\prop{Fløjlsbukser}
\prop{Ternet 70'er skjorte}
\prop{Seler}
\end{props}
  
\begin{sketch}

\scene{Der står to "forskere" på scenen, en fysiker og en matematiker.}

\says{N} Nu \textit{er} det jo ikke fordi, vi har tænkt os at overtage verdensherredømmet, men vi har observeret, at beståelsesprocenten på DIKU er for lav - og da dette naturligvis ikke kan hænge sammen med inkompetence hos studieledelsen, må svaret søges andetsteds \act{gestikulerer kraftigt mod himlen}. Efter at have konsulteret et hold seriøse forskere fra RUC, er man nået frem til den hypotese, at problemet er forårsaget af dårlig energi mellem vejleder og studerende.

\says{A} Nu \textit{er} det jo ikke fordi, vi har tænkt os at overtage verdensherredømmet, men DIKU har trods alt fået assistance fra fysik og matematik. Mit navn er Emmanuel Dike-skraber Blødsinn Quatsch, professor ved Niels Bohr instituttets afdeling for astrologi.

\says{N} Og jeg er Hilbert van Hamilton Heine Heisenberg, professor i anvendt matematik, ved Matematisk Instituts afdeling for numerologi.

\says{A} Ifølge vores tese, udgør visse personer et problem udelukkende på grund af det stjernetegn, de er født i.

\says{N} Vi vil derfor i de følgende scenarier illustrere hvilke problemer, de enkelte stjernetegn kan udgøre - både for vejledere og studerende. Vi har for eksempel stjernetegnet fisken.

\says{A} Astrologisk står fisken for følelser, og fisk er nærmest overfølsomme. Humøret er meget svingende, en dag jublende glad, den næste dag bundulykkelig. Fisken er en ægte humanist. Han er en meget impulsiv, og springer gerne fra den ene handling til den anden- når han da ikke lige sidder fast i inter*nettet* \act{henvender sig til N, som nu agerer "fisk"}. Kære fisk, du har netop afleveret et datalogiprojekt udformet i hør, bomuld og fedtsten, og det ser meget smukt ud.

\says{N} [som fisk, overstrømmende lykkelig] Synes du virkeligt det? Nu går jeg altså i uendelig lykke!

\says{A} Men det oversætter ikke...

\says{N} [bundulykkelig og meget dramatisk] Oversætter det ikke?

\says{A} [trøstende] Men det er et meget fint program!

\says{N} [glad igen] Virkelig?! :-D

\says{A} Det ser dog ud til, at der er lidt rigeligt med memory-leaks. Tror du ikke at du skulle huske at dispose dine objekter?

\says{N} [forarget] Man kan da ikke bare skille sig af med et objekt... Sådan bruge og smide væk.

\says{A} Men der er jo ingen garbage collector i C++.

\says{N} Garbage? Kalder du MINE objekter for affald? Nu bliver jeg ked af det! \act{begynder at græde}.

\says{A} [henvendt til salen] Det er meget svært, at bruge en fisk til noget fornuftigt, så til de fisk der måtte sidde i salen: 
\says{A+N} I bliver ALDRIG dataloger!

\says{N} Lad os derfor kigge lidt på vædderen i stedet. Væddere tænker kun på sig selv. De handler oftest før de tænker. De er udprægede konkurrencemennesker \act{henvender sig til A} Kære Vædder. Jeg har netop set dit program igennem.

\says{A} Og det er det bedste program, du nogensinde har set! Det slår ALLE konkurrenter! Ship it!

\says{N} Men det oversætter jo ikke...

\says{A} Det program oversætter, hvis jeg siger det! Her er det mig, der bestemmer.

\says{N} Det er vel svært at lave for-løkker i C kun ved brug af konstanter?

\says{A} Jeg bruger aldrig variable. Variable er omskiftelige og vejrer for vinden. Jeg bruger kun konstanter!

\says{N} [henvendt til salen] Væddere kan have svært ved at aflevere et projekt til tiden - og da dataloger \textit{altid} afleverer til tiden, må vi desværre meddele vædderne i salen:

\says{A+N} I bliver ALDRIG dataloger!

\says{N} [indskyder] I kan dog altid få et job i edb-afdelingen!

\says{A} Vi er nu nået til skorpionen. Skorpioner er mystiske og manipulerende. En skorpion skriver ikke slamkode - han skriver DaVinci-kode. De er forfaldne til overtro og til den gren af fysikken vi i daglig tale kalder metafysik. Heldigvis har vi en vaskeægte numerolog her, som naturligvis også er skorpion. Prøv engang at se, hvordan en skorpion er som vejleder: \act{henvender sig til N} Kære skorpion. Hvis jeg følger dine anvisninger, så oversætter mit program slet ikke.

\says{N} [meget mystisk og mistænksom] Man skal ikke altid tro på, hvad en oversætter siger. Variable kan kun tildeles værdier, der ikke slutter på 2, 4 eller 6 og ikke er delelige - eller har en tværsum som delelig med 3. Og så ser jeg, at du har en for-løkke, der itererer fra 0 til 1000, og det går virkelig ikke.

\says{A} Jamen hvordan skal jeg da så lave en løkke, når der er så mange tal, man ikke må bruge?

\says{N} Der er da masser af tal, man gerne må bruge. Du skal bare huske, at deres binære repræsentation ikke må indeholde tallet 1.

\says{A} [henvendt til publikum] Nu \textit{er} det jo ikke fordi, vi har tænkt os at overtage verdensherredømmet, men skorpioner bliver ALDRIG dataloger!

\says{N} Vandmanden har ofte et urealistisk forhold til sandheden og kæmper gerne mod autoriteter som de ikke respekterer. Betragt for eksempel denne vejleder.

\says{A} Hvornår kan du være færdig?

\says{N} Hvis jeg skynder mig - på fredag.

\says{A} På tirsdag? Nej, hvor er det imponerende. Det synes jeg er hurtigt.

\says{N} Nej ikke på tirsdag, det kan jeg ikke nå.

\says{A} Nå så du er allerede færdig. Du kan lægge afprøvningen på mit bord.

\says{N} [til publikum] Vandmænd bliver muligvis vejledere på DIKU \ldots
\says{A+N} men de bliver ALDRIG dataloger!

\says{A} Hvis du er du født i vægtens tegn, er du kunst- og skønhedselskende, du sætter harmoni, fællesskab og kærlighed forud for alt. \act{kunstnerpause} Så du er nok ikke datalog...

\says{N} Men hvad så med jomfruer? Dem er der da masser af på DIKU.

\says{A} [ryster på hovedet] De bliver aldrig dataloger.

\says{N} Hvad med tvillingen?

\says{A} For meget redundans.

\says{N} Tyren?

\says{A} Lugter af GNU.

\says{N} Krebsen? Skytten? Stenbukken? Løven?

\says{A} Glem det.

\says{N} Nu \textit{er} det jo ikke fordi, vi har tænkt os at overtage verdensherredømmet, men så er der jo \textit{ingen} der kan blive dataloger?!

\scene{A og N kigger lumsk på hinanden}

\says{A + N} Muahahahahahahahaha!!!!

\scene{TFH}

\end{sketch}
\end{document}

%%% Local Variables: 
%%% mode: latex
%%% TeX-master: t
%%% End: 

