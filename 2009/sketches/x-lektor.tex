\documentclass[a4paper,11pt]{article}

\usepackage{revy}
\usepackage[utf8]{inputenc}
\usepackage[T1]{fontenc}
\usepackage[danish]{babel}

\revyname{DIKUrevy}
\revyyear{2009}
\version{1.2}
\eta{$4$ minutter}
\status{Færdig}

\title{X-lektor}
\author{Madss}

\begin{document}
\maketitle

\begin{roles}
\role{V}[Amanda] Vært
\role{A}[Kristine] Første lektor der gerne vil fyres
\role{B}[Klaes] Anden lektor der gerne vil fyres
\role{C}[Ejnar] En tilfældig lektor
\role{D}[Brainfuck] Dekanen
\end{roles}

\begin{props}
\prop{En kop kaffe}[Person, der skaffer]
\prop{Nogle papirer}[Person, der skaffer]
\prop{Motorsav}[Person der skaffer]
\prop{Skilte med A, B og C, som man kan hænge rundt om halsen}[Person der skaffer]
\end{props}
  
\begin{sketch}

\scene{V træder ud foran tæppet}
\says{V} Som I sikkert ved, har Niels Bohr institutet fået bevilget en ny
  kageordning som skal finansieres ved at fyre en lektor på DIKU. Jeg
  vil derfor byde velkommen til dette års udgave af \ldots X-Lektor.
\scene{Nogen spiller temaet fra X-Factor}
\says{V} Konkurrencen hvor en heldig lektor vinder en opsigelseskontrakt hos
  sit institut. Lad os hilse på de to finalister\ldots
\scene{Tæppet går fra og der står to personer klar}
\says{V} Sig velkommen til A og B!
\scene{V skal til at præsentere de to deltagere. C går tilfældigvis
forbi i baggrunden med en kop kaffe og nogle papirer. V løber hen og
hiver fat i ham}
\says{V} Undskyld, har du tid et øjeblik?
\says{C} Tjo\ldots
\says{V}[Mod publikum] Og C! Giv en stor hånd til C!
\says{V} Lad os starte med at høre hvorfor de tre deltagere mener at netop de
  er de rette til at vinde.
\says{A} Jeg har fået et velbetalt job i det private
\says{B} Jeg skulle være gået på pension fra 15 år siden
\says{V} Og hvad mig dig, C, hvorfor vil du gerne fyres?
\says{C} Fyres? Jamen, jeg har lige købt hus og fået et barn og \ldots
\says{V}[afbryder] Altsammen fantastisk gode grunde, men lad os nu kaste os ud
  i konkurrencen. For at det hele bliver fair, er det den lektor som har
  bidraget mindst til sit institut som vinder konkurrencen. For at afgøre dette
  kommer der nu to spørgerunder. Første spørgsmål lyder som følger:
  »Hvem har gjort absolut mindst på DIKU for at tiltrække nye studerende?«
\says{B} Ha, denne rundte har A jo klart tabt! Efter at hun afleverede sine 1095
  kopper tilbage til kantinen, kan man jo få kaffe igen. Det kan kun øge
  optaget.
\says{A} Nej, for jeg vaskede dem nemlig ikke op.
\says{B} Men så længe de studerende gør dem rent kan de da ikke gå rundt og
  skræmme nye studerende væk.
\says{A} Måske, men var der så ikke noget om at du holdt åbent kontor i uge 8?
  Der vrimlede det med mennesker.
\says{B} Haha. Så var det ærgeligt at jeg tilbragte hele uge 8 i Val d?Isere
\says{A} Ja, for at hente de 42 gæstestuderende, du har arbejdet det sidste
  halve år på at få optaget på DIKU.
\says{B} Ja, men de blev sat af ved ITU\ldots
\says{V} Hov, hov. Vi skal også lige høre hvad C svarer!
\says{C} Mig? Jeg har i min fritid været rundt på alle danske gymnasier og
  holde oplæg og er fast medlem af PR-udvalget inde på fakultet.
\says{V} Javel, ja. Men da alle dine aktiviteter er foregået udenfor DIKU og
  spørgsmålet lød »Hvem har gjort mindst \emph{på} DIKU« bliver du den klare
  vinder af første runde. Giv en stor hånd til C.
\scene{A og B kigger ondt på C}
\says{V} Men I skal jo andet end at skaffe penge til instituttet.
  Forskning spiller også en mindre rolle, så vinderen af næste runde er
  den som har publiseret det mindste bidrag til forskningsverdenen.
\says{B} Så er det da ærgeligt at A lige har fået sit survey med i Lecture Notes in Computer Science.
\says{A} Ja, omhandlende de artikler jeg mangler at få skrevet i år. Men hvad
  så med din artikel om optimering?
\says{B} Mener du »Bubble-sort: In branch-cut-and-price approach«?
\says{V} Det ser ud til at I har været effektive \ldots hvad med C?
\says{C} Jeg har netop bevist at P = NP
\says{V} Interessant \ldots hvor er det blevet publiseret?
\says{C} Jaa, jeg er først lige blevet færdig, så det er ikke som sådan blevet
  publiseret endnu
\says{V} Jeg ser. Men da spørgsmålet lød\ldots ja, du har vist forstået hvor det
  bærer hen.
\scene{A og B kigger surt på C}
\says{V} Det var slut på anden runde.  Og så må publikum gerne stemme.
\scene{Musik til afstemmingen kører i cirka 15 sekunder}
\says{V} Og publikumsstemmerne er nu talt op og vi
  klar til at præsentere vinderen. \act{trommehvirvel}
\says{V} Og vinderen er\ldots
\scene{D kommer ind på scenen med en motorsav}
\says{D} Hør, hvad er det I laver?
\says{V} Det er jo dekanen. Vi er da ved at finde ud af hvem
  der skal fyres så der kan blive råd til din kageordning
\says{D} Hør, det er da ikke DIKU der skal betale for den slags. Har I slet
  ikke hørt at IT er blevet et satningsområde på KU.
\says{V} Nå, jo.
\scene{X-lektor holdet går ud af scenen. A og B ser skuffede ud}
\says{D}[henvendt til publikum] Ja, nu synes I sikkert at sådan et
  satningsområde lyder frygtelig dyrt \ldots og det er det skam også. Men
  bare rolig. Jeg tror ikke man skal skæres mere end 8 medarbejdere på
  DIKU før det kan løbe rundt.
\scene{D tænder sin motorsav og går ud af scenen.}

\scene{Tæppe}

\end{sketch}
\end{document}
