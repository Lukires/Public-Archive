\documentclass[a4paper,11pt]{article}

\usepackage{revy}
\usepackage[utf8]{inputenc}
\usepackage[T1]{fontenc}
\usepackage[danish]{babel}

\revyname{DIKUrevy}
\revyyear{2008}
% HUSK AT OPDATERE VERSIONSNUMMER
\version{1.0}
\eta{$0.5$ minut}
\status{Færdig}

\title{Vi kan kode}
\author{Phillip, Munter}

\begin{document}
\maketitle

\begin{roles}
\role{S}[Munter] Sanger
\role{A}[Amanda] Datalog 0
\role{B}[Kristine] Datalog 1
\role{PHK}[Rune] Poul-Henning Kamp
\end{roles}

\begin{props}
\prop{En rekvisit}[Person, der skaffer]
\end{props}

  
\begin{sketch}

\scene{Det er mørkt foran tæppet. En speakerstemme tager ordet.} 

\says{S}[Voice-over] Det var alt fra DIKUs pigekor. Nu er det blevet tid til DIKU-sjov!

\scene{S kommer ud gennem fortæppet, og begynder at synge på "Tænk på det bedste du ved -melodien fra Peter Pan.}

\sings{S} \\
Åben et godt IDE \\
Sæt struktur på din idé \\
Tag en cola, knap den op \\
Mærk så kraften i jer's krop \\
Hvad er det så I kan? \\

\sings{A+B} \\
Vi kan kode!\\
Vi kan kode!\\
Vi kan kode!\\
Vi kan kode!\\
Vi kan kode!\\
Vi kan kode!\\

\scene{Poul-Henning Kamp stikker hovedet ud gennem fortæppet.}

\says{PHK}[Meget vred] Nej I kan ej!

\scene{Lys ud}

\end{sketch}
\end{document}

%%% Local Variables: 
%%% mode: latex
%%% TeX-master: t
%%% End: 

