\documentclass[a4paper,11pt]{article}

\usepackage{revy}
\usepackage[utf8]{inputenc}
\usepackage[T1]{fontenc}
\usepackage[danish]{babel}

\revyname{DIKUrevy}
\revyyear{2009}
% HUSK AT OPDATERE VERSIONSNUMMER
\version{1.5}
\eta{5 minutter}
\status{Færdig}

\title{Institutbestyreren}
\author{Fairchild, Phillip, Brainfuck og Munter}

\begin{document}
\maketitle

\begin{roles}
\role{MZ}[Johan] Martin Z (institutbestyrer)
\role{SB}[Dirk] Stein Bagger
\role{TV}[Mikkel] Tilfældig VIP
\end{roles}

\begin{props}
\prop{Bord}
\prop{To stole}
\prop{En potteplante}
\prop{Et diplom}
\end{props}


\begin{sketch}

  \scene{Martin Z sidder på sit kontor. Stein Bagger kommer ind,
    bliver budt velkommen, og sætter sig i gæstestolen.}

  \says{MZ} Okay, Stein \dots Bagger, du har søgt stillingen som
  institutbestyrer, og jeg må indrømme at vi i ledelsen er meget
  imponerede over din ansøgning. Dine uddannelsesmeritter, erfaringer
  fra tidligere jobs, din tid i frømandskorpset og dine anbefalinger
  er helt i top. 

  Men som du også ved står instituttet overfor en række store
  udfordringer som den kommende institubestyrer bliver nødt til at
  håndtere. Hvad vil du kunne tilbyde som institutbestyrer i denne
  svære tid?

  \says{SB} Det kan godt være at det ser sort ud nu, men bare vent!
  Det DIKU først og fremmest har brug for, er at genvinde troen. Jeg
  har mange års erfaring i at flytte fokus væk fra de ting der
  forhindrer at man når ud over sit fulde potentiale.

  \says{MZ} Vil du også være i stand til at genoprette troen på DIKU
  med alle de dårlige historier i medierne?

\says{SB} Jo, ser du Martin... Må jeg kalde dig Martin?

\says{MZ} Øøh...

\says{SB} Martin! Du ser historier om studerende der dropper ud,
fysikere der koder bedre end dataloger, autodiktakte hackere der ikke
kan se vores eksistensberettigelse, og nedlæggelse af
forskningsgrupper. Jeg ser kun upløjede marker der dufter af vår! Det
er et spørgsmål om perspektiv! Og vindermentalitet... For du vil jo
gerne være en vinder, ikke Asger?

\says{MZ} Asger?

\says{SB} Ja, øh.. Det er et andet ord for Martin på.. arabisk!

\says{MZ}[lettere imponeret] Taler du også arabisk?

\says{SB}[falsk smil] Ja ja, aftenskole i Dubai! \act{åbner jakken og
  viser et diplom}

\says{MZ} Det er godt nok imponerende. Nå men DIKU har også problemer
med aktivitetskrav. Dataloger er notorisk længe om at gennemføre deres
studie. Mange af vores kandidater vil give os voldsomt underskud efter
8 til 18 år.

\says{SB} Med lidt hjælp fra San Francisco Technical University vil vi
kunne lave bachelorer på kun ét år, eller mindre. Det tror jeg nok er
i verdensklasse! Og det vil udløse en stor bonus fra regeringen!

\says{MZ} Vi havde egentlig bare tænkt os at udvande uddannelsen,
afskaffe de svære kurser, indskrænke de kandidatstuderendes
valgfrihed, og som kronen på værket indføre tvungen HCI på
bachelordelen. Din smutvej ser dog noget mere farbar ud.

Men hvad med de dårlige historier i pressen? Hvordan får vi vendt den
negative opmærksomhed?

\says{SB} Som vi alle ved kan de fleste problemer løses ved at give
tingene nye navne...

\says{MZ} Men DIKU har jo...

\says{SB} \textit{DIKU}? Nej, det navn skal vi også af med. Det siger
jo ingenting om hvad vi laver her. Næh, Martin: "ECTS-Factory"! Kan du
se det for dig?
\act{vifter melodramatisk med armene}

\says{MZ} Jeg kan godt se at du har en pointe der. Vi havde egentlig
bare tænkt os at lave en overprodceret video med tilfældige mennesker,
der ikke ved hvad vi laver her, og "dataloger" \act{laver
  anførselstegn med fingrene} ude fra KUA. Kan godt se at din idé er
noget bedre. Vil navneændringen så også hjælpe på optaget af nye
studerende?

\says{SB} Jeg garanterer at optaget vil stige minimum 20\% per år!

\says{MZ} 20\% procent per år, kan den slags vækstrater virkelig lade
sig gøre?

\says{SB}[trækker på skuldrene] Who cares? I sidste ende afhænger det
jo af hvem og hvordan man tæller. Det vigtige i denne sammenhæng er at
den slags tal sælger, og gør os til et vinderinstitut. Og alle vil jo
være sammen med en vinder. De nye tal for optaget vil i sig selv
booste det faktiske optag, bare vent og se!

\says{MZ} Jeg forstår, jeg forstår... Men vi har jo også mistet en
forfærdelig mængde penge på det sidste. Hvad kan vi gøre ved det?

\says{SB} Finn, Finn, Finn... \act{opdager sin fejl}Martin!
Startkapital bliver ikke noget problem. Kantinens service
repræsenterer instituttets største asset, vi sælger alle kopperne til
HCØ.

\says{MZ} Fint fint, så har vi til sukker og fløde, men hvad med mere
stabile indtægter? Vi får jo nærmest ingen STÅ fra de studerende?

\says{SB} I har jo alligevel ingen nævneværdige studerende, så jeg
foreslår at vi rømmer nordfløjen og lejer den ud til nogle fiktive
selskaber oprettet efter "Just In Time" princippet.

\says{MZ} Men hvad så når vi flytter til HCØ? Nordfløjen kan da
ikke være en god langsigtet plan?

\says{SB}[søgende] Lang... sigtet? Nå, nu er jeg med!  Flytningen til
HCØ er ligesom flytningen til Amager udelukkende en fiktiv
flytning. Det handler om symbolværdi. Og symbolværdi kan afskrives!
Tænk på alle de virtuelle udgifter en ny "bygning" giver, for ikke at
tale om "flytteomkostninger". Når bare en revisor skriver under på
det, så er den i vinkel.

\says{MZ} Vi har i ledelsen også snakket om at det kunne være rart
med nogle sponsorer fra erhverslivet, er det noget du kunne hjælpe os
med?

\says{SB} Erhversponsorer? Dem kan jeg sagtens finde ... på!

\scene{MZ begynder nu at tage noter, mens han sidder og nikker til
  hvad SB fortæller}

\says{SB} Styring skal komme oppefra Martin, ikke fra de studerende!
De skal holdes i kort snor! Når først Nordfløjen er ryddet, låser vi
Sydfløjen af. Hvis man skal vide hvor man har de studerende, skal man
også vide hvor man ikke har dem - og det er Sydfløjen! Det er min
fløjpolitik! Når de studerende ikke længere har bevægelsesfrihed, har
de ikke brug for information. Så ud med informationen - det er min
informationsstrategi! Og EDB-afdelingen? Væk med de studerende, og op
på fakultetsniveau. "EDB-Faculty"! \act{slår ud med armene, og julelys
  i øjnene}

\says{MZ} Det må jeg sige, du har mange gode idéer Stein. Du vil helt
bestemt høre fra os, når vi er færdige med at interviewe alle
ansøgerne.

\scene{SB forlader scenen. En tilfældig VIP banker på og kommer ind.}

\says{TV} Hej Martin, sad du lige i møde?

\says{MZ} Møde? Mig? Nej du må have hørt forkert! Jeg har i hvert fald
ikke lige pumpet en eller anden godtroende tosse fra erhverslivet for
gode idéer under den falske forudsætning at det drejede sig om et
jobinterview!

\says{TV} \act{trækker på skuldrene} Okay...

\says{MZ} Nej jeg sad skam bare og brainstormede lidt. Og så vidt jeg
kan se er har jeg nu fundet gode solide løsninger til alle vore
problemer!

\scene{TFH!}

\end{sketch}
\end{document}

%%% Local Variables: 
%%% mode: latex
%%% TeX-master: t
%%% End: 

