\documentclass[a4paper,11pt]{article}

\usepackage{revy}
\usepackage[utf8]{inputenc}
\usepackage[T1]{fontenc}
\usepackage[danish]{babel}

\revyname{DIKUrevy}
\revyyear{2009}
% HUSK AT OPDATERE VERSIONSNUMMER
\version{1.2}
\eta{3 minutter}
\status{Færdig}

\title{Besserwisser}
\author{Munter, BoElling, Johan}

\begin{document}
\maketitle

\begin{roles}
\role{D1}[Brainfuck] Datalog 1
\role{D2}[Johan] Datalog 2
\role{G}[Ronni] Gæst
\end{roles}

\begin{props}
\prop{Mappedatamat, helst en EEE}[f.eks. Phillip]
\prop{Gentoo-tshirts}[Johan]
\end{props}

  
\begin{sketch}

\scene{Foran tæppet, to dataloger står med siden til publikum.}
\says{G} \act{Kommer ind og sætter sig på midten af scenen. Åbner sin datamat.}
\scene{Windows startuplyd. D1 og D2 stopper op og vender hovedet langsomt. De springer op og overfalder G}
\says{D1} Det er Windows.
\says{D2} Det er Windows.
\says{D1 + D2} Det kan man ikke!
\says{D2} Du skal bruge Linux!
\scene{D1 og D2 high fiver}
\says{G} Jeg gider ikke skifte til Linux, jeg skal bare nå at skrive min opgave inden deadline i morgen.
\says{D1} I morgen?
\says{D2} Det kan du ikke nå!
\says{D1} I hvert fald ikke med Windows!
\says{D2} Du kommer til at bruge hele dagen på at genstarte!
\says{D1} Du burde installere Gentoo, jeg har en stage1 tarball med på usb-nøgle til dig. \act{Finder USB-nøgle frem i lommen}
\says{G} Hør nu her, jeg skal ikke bruge Gentoo. Jeg skal bare bruge \TeX{}, og det har jeg allerede på Windows.
\says{D1} \TeX{}?
\says{D2} På Windows?
\says{D1 + D2} Det kan man ikke!
\scene{D1 og D2 high fiver}
\says{G} Prøv nu lige at se her. \act{Peger på skærmen}
\says{D1} Mik\TeX{}?
\says{D2} På Windows?
\says{D1} Den har jeg aldrig set inde i Windows Update når jeg har ordnet datamat for mormor.
\says{D2} Heller ikke mig, så jeg installerede Gentoo til hende i stedet.
\scene{D1 og D2 high fiver}
\says{G} I Windows ligger programmerne altså ikke i Windows Update. Man googler bare navnet på det program man vil installere, klikker på det første link og installerer den exe-fil de stiller til rådighed. 
\scene{Publikum råber. D1 og D2 kigger vantro på hinanden.}
\says{G} Kan jeg ikke lige få lidt privatliv mens jeg skriver mail?
\says{D1} Skriver mail?
\says{D2} Det kan man ikke.
\says{D1} I hvert fald ikke i Windows!
\says{G} Nu er det jo også Gmail jeg bruger.
\says{D1} Gmail?
\says{D2} Gmail kan ikke autosignere dine mails med GPG.
\says{D1} Du skal bruge Pine.
\says{D2} Ja. Så kan du også sætte et pre-send-hook op som automatisk \TeX{}'er din klartekstmail og vedhæfter en PDF med indholdet af mailen.
\scene{D1 og D2 high fiver}
\says{D1} Din Firefox er ret langsom hva?
\says{G} Det er Internet Explorer...
\says{D2} Ja. Du skulle have oversat den med -Os flaget så den binære fil blev mindre og hurtigere at indlæse.
\scene{D1 og D2 high fiver}
\says{D1} Hvad er det du har nede i højre hjørne?
\says{G} Det er mit ur.
\says{D2} Dit ur? Vil du ikke meget hellere have en masse muligheder for at sætte andre ting dernede?
\says{D1} Jeg har f.eks. en tæller der viser hvor mange timer jeg har brugt på at oversætte programmer.
\says{D2} Også mig, 42.000 timer indtil videre.
\scene{D1 og D2 high fiver}
\says{G} Jamen jeg skal jo slet ikke oversætte noget!
\says{D1} Det er jo fordi du bruger Windows.
\says{D2} Hvis du nu brugte Gentoo så skulle du oversætte hele tiden.
\scene{D1 og D2 high fiver}
\says{G} Jeg skal ikke oversætte nu, eller hele tiden, jeg skal bare lige finde omega pakken som \TeX{} mangler for at kunne bygge rapporten.
\says{D1} Du kan jo bare google den.
\says{D2} Det kan man godt på Windows.
\says{G} Jeg har prøvet. Jeg er ikke sikker på om den findes.
\says{D1} I Gentoo skriver du bare paludis -i texlive-omega
\says{G} Det kan man ikke i Windows.
\says{D2} Du burde installere Gentoo.
\says{G} Det kunne være jeg skulle prøve det. Har du stadig den USB-nøgle?
\says{D1} Selvfølgelig.
\scene{D1 og D2 går hektisk i gang med at overtage styringen af computeren efter de har skubbet G væk.}
\says{D2} Så skal vi bare lige finde ud af hvad for en arkitektur du har så vi kan sætte build-flags.
\says{D1} Ja, vi skal huske SSE3 og omit-framepointer.
\says{D2} Og min nye yndlings fstack-protecter-all.
\says{D1} Hvor store partitioner?
\says{D2} Bare standard. Og lad os prøve med EXT4
\says{D1} Har du sat distCC op på din maskine?
\says{D2} Jep, det kommer til at gå knaldhurtigt det her.
\scene{D1 og D2 high fiver}
\says{VO} 6 timer senere.
\says{D1} ... og så skal du bare starte X og så er vi klar til at bruge systemet.
\says{G} Fedt. Så skal jeg bare lige ind på min netbank og overføre...
\says{D1} Netbank?
\says{D2} Det kan man ikke
\says{D1} I hvert fald ikke i Gentoo.
\says{D2} Så skal have Virtualbox med Windows.
\scene{D1 og D2 high fiver}
\scene{Lys ud.}

\end{sketch}
\end{document}

%%% Local Variables: 
%%% mode: latex
%%% TeX-master: t
%%% End: 

