\documentclass[danish]{article}
\usepackage{revy}
\usepackage[utf8]{inputenc}
\usepackage{babel}
\usepackage{anysize}

\title{For dum til det her sted...}
\author{Theo Engell}
\melody{For kendt -- Lex \& Klatten}

\version{1.0 -- færdig} % HUSK AT AJOURFØRE VERSIONSNUMMER!!

\revyyear{1998}

\begin{document}

\maketitle

\begin{song}
\sings{Sanger 1} I kender mig for MUD'ning og fra IRC
I hobetal af JPEGs og fra 'skak og mat'
Jeg sidder med min Cola og fastfoodmad, går ik' i bad
Rapportgrupperne udstråler had

\sings{Omkvæd}  Jeg er for dum til det her sted
  Jeg kan sgu ikke følge med
  Men lad mig dog bliv på det her sted
  DIKU er hvor jeg får fred

\sings{Sanger 2} Jeg hygger mig på DIKU - jeg 'i undertal
Der ta'r jeg cola'n med ned til min terminal
og flaskerne de hobes op, sikke't griseri, no'd svineri
Jeg gider ikke at tal' med nogen

  Jeg er for dum til det her sted...

\sings{Sanger 3} Og pludslig får man venner, ja i hobetal
Men vennerne de smutter hvis de får et valg
når de igen kan LaTeX. Kan sgu ej forstå, efter 'ø' og 'å'
venskaberne går (tit) i stå

  Jeg er for dum til det her sted...

\sings{Sanger 4} Jeg gemmer mig på DIKU - der er sommerfest
Selv efter jeg tager briller på, er jeg en hest.
Pigerne fornægter, ja, min eksistens, har sex i mens
jeg logger ind, kom, check min potens

  Jeg er for dum til det her sted...

\sings{Sanger 1} Jeg drømmer, jeg ud' og gå mig en tur
\sings{Sanger 2} Og der møder jeg en pige, der gøre
                 et tegn og så spør'
\sings{Sanger 3} Om vi ikke skal lig' på græsset
\sings{Sanger 4} Og så vågner jeg og skriger

  Jeg er for dum til det her sted...

\vspace{1cm}

\sings{Evt.} Jeg vælter mig i penge - jeg har kort fortalt
             forstået at score kassen, bliver sgu betalt!
             Jeg sørger for at brugerne får brugerspat, med brugerbat
             Jeg' Edb-afdelingens skat!

  Jeg er for dum til det her sted...

\end{song}
\end{document}
% Local Variables: 
% mode: latex
% TeX-master: t
% End: 
