\documentclass[danish]{article}
\usepackage{revy}
\usepackage[utf8]{inputenc}
\usepackage{babel}
\usepackage{a4wide}

\title{Psykologhjælp}
\author{Maren Sødsen}

\version{1 -- færdig} % HUSK AT AJOURFØRE VERSIONSNUMMER!!

\revyyear{1998}

\begin{document}
\maketitle

\begin{roles}
  \role{Psyk} Psykolog
  \role{Data} Datalog - pigens kæreste
  \role{Pige} Pige - Datalogens kæreste
  \role{Toma} Tomatsælger.
\end{roles}

\begin{props}
  \role{Stole og bord. Papirer. Tomater. Chesterfield sofa?}
\end{props}

\begin{sketch}

\scene Psykologens kontor. Psykologen sidder og venter. Ind kommer pigen og
Datalogen.

\says{Pige} Undskyld, men er De Dr. RosenPalle, der står for
samlivs-rådgivning?

\says{Psyk} Ja, det kan du tro. Og du er vel så ... \act{kigger i sine
  papirer}... Jørgen?

\says{Pige} Nej, det er min kæreste! Det er ham, den er gal
med. \act{kommanderer:} JØRGEN!!!

\scene Ind trasker datalogen. Ugidelig. Sætter sig i stolen.

\says{Psyk}Nå ... \act{kigger i sine papirer} ... Jørgen! Hvad er så problemet?

\says{Pige}[før Jørgen når at svare] Jamen altså \ldots Jørgen var jo
engang en rigtig sød og forstående fyr, men han er simpelthen blevet så
nørdet, efter han begyndte at læse datalogi.

\says{Psyk} [interesseret] Nå, datalogi! 

\says{Data} [vågner] Ja, jeg er faktisk lige i gang med \ldots \act{\ldots en
opgave, hvor \ldots}

\says{Pige} JØRGEN! Se, det er det eneste, han kan snakke om! I gamle kunne
han sagtens være sammen med mine venner fra filosofi, men nu sidder han
bare i et hjørne og stirrer ud i luften.

\says{Psyk} Hvad så, når I er sammen med hans venner?

\says{Pige} [forarget] Hvad i alverden skulle vi dog være sammen med hans
nørdede venner for? Det er jo også meget mere end det. I gamle dage drak
han Bordeaux - nu drikker han Gin og Urge. I gamle dage spillede han
forbold - nu spiller han bordfodbold. I gamle dage ringede han tit til mig og sagde, at han elskede mig. Nu ringer han og
siger, at han har fået en 'segmentation fault'.

\says{Psyk} [entusiastisk] De er også bare pisse irriterende \ldots

\says{Data} [vågner] nahrr, normalt skyldes de jo bare off by one fejl.

\says{Psyk} Ja, så er det straks værre med 'bus error'. DEN fatter jeg
intet af.

\says{Data} Jo, det er som oftest en eller anden udefineret pointer, du
kommer til at dereferere. Hvis den så ligger på en ulige adresse vil \ldots

\says{Pige} [afbryder] Og så er han begyndt at gå til det der latterlige
diku revy noget.

\scene Psykologen og datalogen, der før entusiastisk lænede sig frem, læner
  sig nu tilbage igen

\says{Psyk} latterlig \ldots revy ???

\says{Pige} Ja, den der fuldstændigt latterlige revy.

\says{Psyk} og, hvad er det så, der gør revyen latterlig?

\scene Mens pigen forklarer følgende kommer tomatsælgeren ind på scenen

\says{Pige} jo, for det første, er det jo nogle ekstremt kedelige emner.

\says{Toma} [råber] Overmodne tomater. Overmodne tomater. 5 stk. 10 kr.

\says{Pige} Skuespillerne er talentløse.

\says{Toma} [råber] Rådne tomater. 5 stk. 5 kr.

\says{Pige} Sangerne er virkeligt pinlige.

\says{Toma} [råber] Mugne tomater og rådne æg. Billigt billigt. Ja -
faktisk gratis. Gratis gærende æg.

\says{Pige} Og så handler det udelukkende om, at fyre dårlige ordspil af.

\says{Psyk} ordspil? Så som?

\says{Pige} Jo, f.eks: 'Denne vits handler om databaser. Hvad er den?'
\ldots 'søgt!'

\scene Psykologen, datalogen og tomatsælgeren er ved at brække sig af
grin. Tomatsælgeren forlader scenen.

\says{Pige} [prøver at genvinde kontrollen] Øøhhm, sagde jeg, at han gik
til fodbolds-træning to gange om ugen før i tiden? Han går stadig til
træning. Nu træner han bare Quake.

\says{Psyk} [stadig grinende] Quake? Hvor hurtigt kan du så gennemføre
E1M4? Jeg har klaret den på 27 sekunder.

\says{Data} 27? Det er sgu løgn!!!

\says{Psyk} Gu' er det røv!!

\says{Data} Gu' er det da så. Selv \emph{jeg} kan kun klare den på 24.

\says{Pige} Jamen \ldots

\says{Psyk} Hey, nu du ved så meget! Har du spillet Leisure Suit Larry 4?

\says{Data} Om jeg har? Jeg brugte kun 22 timer.

\says{Pige} Kan I ikke \ldots

\says{Psyk} [til pigen] hør kan du ikke hente noget kaffe? Jeg bliver vist
lige nødt til at have en mand til mand snak med \act{kigger i sine papirer}
... Jørgen. \act{til Jørgen:} Ok, hvordan kommer man forbi geden i
varehuset?

\says{Pige} HVAD??

\says{Data} Der skal du jo bruge guleroden!

\says{Pige} Hallo!!!

\says{Psyk} Guleroden? Jeg har sgudda ikke nogen gulerod.

\says{Pige} Hvis I ikke stopper nu, så går jeg!!!

\says{Data} Ok, hils! Jo, den får du fra PlayGirl pigen.

\says{Pige} Hvis jeg går nu, kommer jeg ikke igen.

\says{Psyk} Nåh, ja! Hun er jo i bunny-dragt. Skal man så bruge
cykelpumpen.

\says{Pige} FARVEL! \act{går sin vej}

\says{Data} Cykelpumpen? Hvad i alverden vil du \act{taler
  langsommere. Kigger efter pigen} bruge \ldots cykel \ldots pum \ldots pen
  \ldots til?

\scene{stilhed. Datalog og Psykolog kigger på hinanden.}

\says{Data} Pyhh, tak skal du ha' Bjarne.

\says{Psyk} Det var sku så lidt, Jørgen. Hvad, skal vi ikke gå en tur i
Grand og se Thomas Vinterbergs 'festen'?

\says{Data} Okay \act{de begynder at gå ud af scenen}. Så kan vi gå på Dan
Tyrel bagefter og få en Cappuchino.

\says{Psyk} Du skal iøvrigt låne min 'Søren Kirkegaards samlede værker'


\end{sketch}

\end{document}
% Local Variables: 
% mode: latex
% TeX-master: t
% End: 
