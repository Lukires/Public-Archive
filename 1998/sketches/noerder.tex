\documentclass[danish]{article}
\usepackage{revy}
\usepackage[utf8]{inputenc}
\usepackage[T1]{fontenc}
\usepackage{babel}
\usepackage{a4wide}

\title{Nørdkammerater}
\author{Henning Makholm, Søren Madsen, Theo Engell, m.fl.}

\version{3.6 -- færdig}

\revyyear{1997}

\begin{document}
\maketitle
\begin{sketch}
\begin{roles}
\role{Første og anden nørd} demokodere
\end{roles}

\scene{2 nørder mødes. Anden nørd har en bog under Armen.}

\says{Første nørd} Jamen er det ikke --

\says{Anden nørd} Jeg skulle tage meget fejl hvis det ikke var --

\says{Begge} Davs du gamle!

\says{Anden nørd} \act{klapper nørd 2 på skulderen} Hold kæft! Dig har jeg
sku ikke set siden \ldots \act{ser sig omkring}. hvad-øøhh, hacker du
stadig?

\says{Første nørd} Ja, for satan. \act{ser sig omkring} Jeg er inde i noget
stort. \ldots Jeg har lige hacket DSB's køreplan!

\says{Anden nørd} \ldots \act{Ser tvivlende ud. Kunstpause. Begejstret:}
Hvor \emph{vildt}, mand! Var det ikke svært?

\says{Første nørd} Nej, det er jo det der er det MORsomme, det er det der
er det geniale. Jeg droppede alt det med modemmer og internet. Så tog jeg
ind på Hovedbanegården med min notebook og tastede alle togtiderne ind
\emph{mens} togene kørte forbi.  Ingen logs.  Ingen passwords.  Ingen kontrol.  
Ingenting.  Der er pivåbent.

\says{Anden nørd}[overvældet] Det er genialt. Det er verdensklasse.
\act{holder sin bog op som en imaginær lommecomputer} Så stod du bare og
tastede løs i fuld offentlighted? Jeg har ellers hørt de krypterer deres
køreplan med et random-delay.

\says{Første nørd}[overrasket] Hvad er det du har der?

\says{Anden nørd} Hvilket \act{kigger på bogen} Nåh, den? Det er fordi jeg
er begyndt på universitetet 

\says{Første nørd} Nå for Søren \act{puffer nørd 2 på skulderen} \ldots
Olsen. Kan du lære dem noget?

\says{Anden nørd} Naah, det går lidt trægt. Jeg var nede og snakke med
over-datalogisten der, og han vidste ikke en gang hvad interrupt 42 gør!

\says{Første nørd} Jeg troede de var sådan nogen fessortyper der ved alt.

\says{Anden nørd} Ja, det tror de også selv. Men ikke noget man kan bruge
til noget ude i det virkelige liv. Nåmmen de sagde jeg skulle bruge sådan
en her. Det kaldes en ``bog''.

\says{Første nørd} Bog, bog, bog \ldots jeg har hørt om det et eller andet
sted, men jeg kan ikke liige --

\says{Anden nørd} De handler med dem ovre i kælderen under August Krogh.
Temmelig skummelt.

\says{Første nørd} Hvordan staver du det?

\says{Anden nørd} B-nul-G. De er selvfølgelig vildt dyre som nye, men
jeg fandt en butik inde i byen, hvor de sælger dem brugt.

\says{Første nørd} Brugt?

\scene Første nørd tager bogen

\says{Anden nørd} Ja, det er en lidt ældre model. Men så længe jeg kun lige
skal se om det styrer overhovedet.

\says{Første nørd} Hmm.. \act{bladrer} Det er jo bare en præformateret
textfil \ldots ``Der kom en soldat marcherende hen ad landevejen. Et, to, et,
to.''\ldots Helt ærligt -- hvis du tænder på \emph{det her}, så er du
sgudda mere pervers end de fleste.

\says{Anden nørd} Halli-hallo. Det er en gammel udgave, og det er bare for
at se om det virker. Desuden er det studierelevant.

\says{Første nørd} Studierelevant! Du studerer ikke en skid. Jeg har
gennemskuet dig. Du er bare på det universitet for at score nogen piger.

\says{Anden nørd} Der er ikke nogen piger på universitetet.

\says{Første nørd} Der må sgudda vælte med dem. De er lysår foran os andre
i informationsteknologi.

\says{Anden nørd} Øh?

\says{Første nørd} Jeg opdagede det først med min søster, og da jeg så
spurgte min far sagde han at alle piger i den alder gør sådan.

\says{Anden nørd} Hvordan?

\says{Første nørd} Jo altså: du ved hvor besværligt det er for os normale
mennesker, at koble os på nettet. Man skal have en internetkonto og et modem
og en computer og et tastatur og et program der skal sættes op og gøres
besværgelser for at det alligevel ikke virker.

\says{Anden nørd} \ldots{}ja?

\says{Første nørd} Prøv så at lægge mærke til pigerne, de har en eller
anden analog neurotransmitter eller hvad det hedder som de kobler direkte
på telefonnettet. De skal bare holde den op til siden af hovedet, så er de
plugged! I timevis. Ingen baudrates. Ingen kommunikationsprotokoller. Ren
cyberpunk. De kan chatte i timevis, du.

\says{Anden nørd} Fantastisk. Har du selv prøvet?

\says{Første nørd} En enkelt gang. Men det virker ikke på mig. Der var
bare sådan en irriterende hyletone.

\says{Anden nørd} Det er fordi det ikke findes. Det er din søster og din
far der har taget gas på dig. Du har også altid været nem at narre.

\says{Første nørd} Damn! Dum som en EDB-journalist.

\says{Anden nørd} Men iøvrigt har jeg slet ikke behov for at score, efter
jeg udskiftede den gamle tøs derhjemme.

\says{Første nørd} HVAD? Hvornår er det sket?

\says{Anden nørd} Her sidste weekend. Jeg fik bare et tilbud, jeg ikke
kunne afslå.

\says{Første nørd} Jamen jeg troede du var rimeligt glad for den gamle.

\says{Anden nørd} Der var også en rigtig fed look and feel. Og jeg havde
brugt vildt mange penge på ekstra udstyr og udvidelser. Den var sgu bare
blevet for langsom til mig, nu hvor jeg var startet på universitet. Min nye
er virkelig hurtig. Og vildt fed at spille på.

\says{Første nørd} Fedt! Hvad\ldots kunne jeg ikke komme over på torsdag og
prøve den.

\says{Anden nørd} nahhr, jeg er sgu ikke så glad for at andre piller. Jeg
fik den jo fra Bjarne, og han havde virkelig rodet for meget i opsætningen.
Den var helt balstyrisk de første dage.

\says{Første nørd} Men hvad så? Har du også skiftet computer?

\says{Anden nørd} Nej, jeg kan simpelthen ikke finde ud af, hvad jeg skal
købe. \act{ser på sit ur} Hey, jeg bliver nødt til at smutte, du.

\says{Første nørd} Okay. Men der var lige en ting, jeg ikke fik fat på.
Hvad tid var det på torsdag?


\end{sketch}

\end{document}
