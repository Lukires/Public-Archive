\documentclass[danish]{article}
\usepackage{revy}
\usepackage[utf8]{inputenc}
\usepackage{babel}
\usepackage{a4wide}

\title{M: Musical Klister}
\author{Rune (\texttt{guldfisk})}

\version{2.1 -- færdig} % HUSK AT AJOURFØRE VERSIONSNUMMER!!

\revyyear{1998}

\begin{document}
\maketitle

\begin{roles}
  \role{Else} dum blondine, talerolle
  \role{Edit} Ilse Ensom, endnu dummere, talerolle
  \role{Søren Mukke} Rigtig lækker, narrøv, tale og sang
  \role{MC Einstein} Fysiker, narrøv, talerolle
  \role{Thomas Fjaller} datalog, medlem af boy band, tale og sang
  \role{Karl Koder} datalog, medlem af boy band, tale og sang
  \role{Flemming Flækkedreng} datalog, hård leder af boy band tale og sang
  \role{Core Dumpedreng} datalog
  \role{Doctor  Neill Jones} Verdens potentielle frelser, tale og sang
\end{roles}

\begin{props}
  \prop{Drinkglas}
  \prop{Penne} i fysikerens brystlomme
  \prop{Mobiltelefon}
  \prop{Nørdet boy-band- tøj} til datalogerne, gerne noget løst-hængende og noget strammere for 
neden
  \prop{Smart tøj} til Søren Mukke og pigerne
  \prop{Neil Jones-kostume} Laang næse, lidt kantede briller
  \prop{Fysiktrøje} med påskriften ``Bohr - det scor''
  \prop{Pistol} til Dr. Jones
\end{props}

\begin{sketch}
\scene
Det er nytårsaften år 2000. På bagtæppet hænger med store bogstaver
``2000''
En Fest, Der høres snakken og skålen i baggrunden, Else og Edit ind.)

?fixme: festen skal introduceres som en år 2000 nytårsfest?

\says{Else}[kommer hen til else, skåler med hende] Hej Edit, hvor lang
tid er der til midnat? Det er jo ikke hvert nytårsaften at vi går ind
i et nyt årtusinde.

\says{Edit} Årh ja!

\says{Else} Ej altså, jeg har lige snakket med ham der Søren ikk'oss', og ved du
hvad, han er bare enormt sød.

\says{Edit} Årh ja!

\says{Else} Han er osse bare helt vildt sjov.

\says{Edit} Årh ja!

\says{Else} Og så læser han filosofi som hovedfag og idræt,
litteratur, kunsthistorie og dansens æstetik som bifag.

\says{Edit} Årh ja! 

\says{Søren}[ind] Gud, der er du \act{giver Else et caf\'e-knus}! Ej hvem er det her? \act{kigger på Edit}

\says{Edit}[bliver genert] Årh ja\ldots Mig, øh\ldots Ikk' no'n.

\says{Søren} Gud! sikke et charmerende navn. Jeg hedder Mukke - Søren Mukke, men du 
må godt kalde mig "smukke" \act{han griner som tegn til at de andre skal grine, de griner}



\says{Edit} Årh ja!

\says{Søren} Hold op, hvor er du bare glad for at se mig. \act{gi'r hende hånden}
Nej stop, har jeg ikke set dig nede på Caf\'e SOMMERSKO?

\says{Edit} Årh ja!

\says{Søren} Nåh ikke, der SNØREDE du mig ellers. \act{Latter}

\says{Fysiker}[ind] Hej piger, jeg hedder MC Einstein, og jeg LASER fysik.
\act{griner som tegn til at de andre skal grine\ldots tavshed}

\says{Else} Årh Søren, få ham væk fra mig. Han er så klam.

\says{Søren} Skrid pommes frittes, det er en lukket fest!

\says{Fysiker} Hva' er der galt med mig?

\says{Else} Ud over at du er fysiker\ldots, ingenting.

\says{Søren} Såså, Lad os nu ikke snakke mere om den lille nørd, la' os i stedet
snakke om, hvor fantastisk tiltrækkende, jeg er.

\says{Else} Gud, du er altså bare så romantisk!

\says{Edit}[kigger rædelsslagen ud i kulissen] Årrhh!

\says{Søren} Hvad er der? \act{Kigger ud, bliver rædselslagen} Hjælp
D\ldots{}D\ldots{} Det er datalogmutanterne. Lad os komme væk! \act{Søren og
  pigerne ud}

\scene Sang: Backstreet's back

Søren og Edit kommer diskret cocktail-party-konverserende ind igen

\says{FF} Fed fest, manner. Skal vi ødelægge den?

\says{TF} Ja, Lad os det!

\says{KK} FUCK! Ka' vi ikke bare gå hjem og kode?

\says{FF} Og I ved drenge, jeg forventer, at I gør jeres værste.

\says{TF} Jaja, kaptajn. Ligesom, når vi laver rapport.

\says{KK}[tavshed] Jeg skal pisse.

\says{FF} Så gå ud og pis!

\says{KK} Det kan jeg ikke, der er låst på toilettet.

\says{FF} LÅST\ldots\ de plejer bare at fjerne toiletpapiret.

\says{KK} Det er sgu' da osse et problem.

\says{FF} Gu' det ej. Rigtige mænd bruger håndvasken som bid\'et.

\says{KK}[grædefærdig] Hva' skal jeg gøre, jeg kan ikke holde mig. \act{FF tager
  drinkglasset ud af Edits hånd, giver det til KK}

\says{Edit} Hey!

\says{KK}[tager det, vender ryggen til og forretter det, han nu skal] AAAH.
Undskyld, jeg skulle bare låne det. \act{giver hende det tilbage}

\says{Edit} Årh ja! Hvor langt ude! \act{Står med glasset i hånden uden at
  vide, hvad hun skal stille op med det}

\says{FF} Tager fyrene pis på dig, skat?

\says{TF}[alvorligt] Flemming, der er noget vi må tale om. Det er noget alvorligt. Jeg er forelsket.

\says{FF} Sejt! I hvem?

\says{TF} Hende der \act{peger på Else, som kommer ind sammen med Edit og Søren}

\says{FF} Fedt nok, scor' hende mand!

\says{TF} Jeg ka' sgu' da ikke score, jeg er datalog! Og kan slet ikke hamle op med Søren Mukke.

\says{Søren} Nå Edit. Du er rigtignok en kvinde af få ord. Det kan jeg
lide! Det gi'r mig mere tid til at snakke om mig selv. Hvor kom vi fra. Nåja, vi
kom fra min sommerferie sidste år, hvor jeg tog på white water rafting i den
svenske skærgård for at redde et kuld babysæler fra udryddelse, og der stødte
jeg ind i en fisker\ldots\ du ved der i SKÆRgården, og ved du hvad jeg sagde til
ham? Jeg sagde "Hvordan skær'n". \act{latter}

\says{FF} Smid den idiot væk, og gå hen og tal med hende.

\says{TF} Jeg tør ikke!

\says{FF} Hvorfor ikke?

\says{TF} Jeg ved ikke, hvordan man taler med piger.

\says{FF} Prøv nu at høre her, at tale sammen med en pige er nøjagtig som at
kommunikere med en computer. Man lader dem tale og svarer bare JA, NEJ, OK og
sådan noget. Kom nu. \act{TF går hen til Else}

\says{KK} Skal vi ikke snart hjem, jeg savner min computer.

\says{TF} Hej pige, min e-mail-adresse er \texttt{billg@diku.dk}.
\act{hvisker til FF} Hvor taster man password ind?

\says{FF} Det gør man ikke, sig nu et eller andet sejt.

\says{TF} Kommer du her tit?

\says{Else} Ikke mere, hvis du gør!

\says{TF}[til FF] Er der en fortryd-knap?

\says{FF} Nej, for helvede, du skal jo sige noget smukt, ikke!

\says{TF} Ok, ok! \act{mander sig op} Hej, synes du ikke bare, at Perl er et fedt sprog?

\says{Else} Hvad er Perl?

\says{FF} Gå nu til den, for helvede. Du er SÅ tæt på at score!

\says{TF} Hvad siger du til, at vi to blev gift. Såd'n et rigtigt internetbryllup, du ved?

\says{Else} AAAD!! \act{stikker ham en flad}

\says{TF}[løber væk] ÅH NEJ! CANCEL!! CANCEL!!!!

\says{Søren} Hva' fa'en er den lille computernørd efter dig. Hvis det ikke var
fordi, jeg lige havde fået manicure, så havde jeg\ldots Ja, det havde jeg.

\says{FF} Ska' du spille smart overfor mig? Hvis du ikke holder kæft, så bruger
jeg datalogernes uovervindelige våben!

\says{Søren} HA! Tror, du jeg bliver bange?

\says{FF}[går hen til Søren] Kender du den om bolighajen, der solgte boliger pr
POSTORDER. Han var en rigtig ejendomsMAILER \act{tager fat i Søren ærme}
\ldots{}og hvis du ikke passer på, så fyrer jeg flere vittigheder af.

\says{Søren}[får mavekrampe] Fingrene væk. \act{fjerner hans hånd} Du grisser min
jakke til. \act{kigger ondt på FF} Hvad så med de kosmetikstudenter
der tog på ROUGE-tur og afholdt MASCARAdefester? De var læber
allesammen\ldots

(ELLER: Hvad så med Sønderjyden, der fremførte en sørgelig ak-sang?)

\says{FF}[får kramper] Det er intet imod englænderen, der tabte sin
taske på vejen, og nægtede at samle den op igen, fordi rigtige mænd
aldrig tager ``bag-up''. \textbf{?fixme: er brugt i socketshow!?}

\scene Søren får kramper, fysiker kommer til

\says{MC} Jeg vil være med! Se, her har jeg en pen, og her har jeg en
til pen\ldots\ \act{penne holdes over kors så de danner et +} Nu har i
set PEN-SUM!

\scene Alle vrider sig i krampe

\says{Else}[tager drinken med KKs tis ud af hånden på Edit] Hej MC Einstein,
vil du ikke ha' en drink? \act{gi'r ham drinken med KKs tis}

\says{Edit} Årh ja!

\says{KK} STOP, manner! Nu er klokken altså snart tolv, og jeg vil hjem og kode!

\says{Core} Er klokken tolv. FUCK! \act{Rummet formørkes} Timen er kommet. År 2000!

\scene Sang: The final countdown

\says{Core} Mine damer og herrer. Vi går nu en uoverskuelig fremtid i
møde. År 2000 står for døren. Lad os alle forholde os i stilhed til
rådhusklokkerne slår.

\scene Rådhusklokker spilles bagfra, der høres en crashet udgave af ``I morgen er Verden vor''

\says{Søren}[til Else] Nåmen, sku' vi ikke bare finde et toilet og få det
overstået? Du har jo alligevel drømt om at få min krop længe.

\says{Else}[kigger roligt og vurderende op og ned ad Søren, og siger
derefter tørt] Næh.

\says{Søren} Nåh, du skal ikke være så bange for min over-lækre
udstråling, selvom du er under min score-standard må du godt få min
krop\ldots

\says{Else} Gå hjem og håndkør din orm!

\says{Søren} Jamen.

\says{Else}[til datalogerne] Hej drenge. Lige pludselig synes jeg, I
er overlækre. \act{Går hen imod dem. De bliver næsten bange}

\says{TF} Hun må være gået ned.

\says{KK} Jeg har ikke rørt noget. Det er ikke min skyld, jeg har slet ikke kodet i aften!

\says{TF}[prøver at røre ved pigen] Gud, hun slår ikke igen\ldots\ 
\act{prøver med dyb stemme} Hvad skatter, kunne du tænke dig at høre
om dynamisk allokering af filblokke på eksternt swap-lager?

\says{Else}[kådt] Jaaaa!

\says{FF} Hva' skatter, er du god på et LAN?

\says{Pige}[endnu mere kådt] Ja, \emph{net}op, og især \emph{lokalt} har jeg en rigtig god BASE-T, da jeg er rigtig god til fransk\ldots

\says{TF}[rigtig modigt] Hvad siger du til, at vi interfacer? Vi kunne blive et \emph{twisted pair}!

\says{Pige} JA! \act{overfalder TF med kys}

\says{Søren} NEJ! \act{skiller dem ad} Stop, stop, det er mig, der er den smukke.
Må jeg så bede om ordet.

\scene Sang: Back for good

\says{Else}[til TF] Fortæl mig noget mere om sublineære
optimeringsalgorimer for maximale strømninger i sammenhængende
netværkssystemer. Det er så romantisk.

\says{TF} Selvfølgelig, men først må jeg spørge dig. Vil du GIFte dig med mig,
eller vil du bare gerne se min JPEG (``jåd-pik'')?

\says{Else} Jeg vil gøre alt for dig.

\says{MC} Hvad øh, kender i den om fysikeren der var blind passager på et sørøverskib, og som måtte 
gå PLANCK-en ud? \act{Alle griner}

\says{FF} Verden \emph{er} gået ned\ldots\ Meget kan vi acceptere, men at nogen
skulle grine af en fysikvittighed.  Nej, nogen må tage affære. Hvem kan hjælpe
os? Det må være en, der kan rette i Verdens semantik\ldots\ een med en høj titel.

\says{ALLE} Doctor Neil Jones.

\says{FF} Jeg ringer med det samme. \act{Tager sin mobiltelefon frem} Doctor Jones\ldots\ 
Doctor Jones\ldots\ Calling doctor Jones. Doctor Jones, Doctor Jones. Wake up now\ldots\ 
Yeepeeayaay, yeepeeayoouh, yeepeeayaay youhh\ldots\ Det virker ikke.

\says{Else} Lad mig prøve! \act{tager mobilen} Doctor Jon\ldots

\says{DJ}[med amerikansk accent, og tal nu tydeligt, så det hele ikke bliver til
mudder] Nu skal jeg være der, kære venner. Lad go'e gamle Doctor Jones rette op
på jeres lille år 2000-fejl -- jeg vil gå så langt som til at give jer lov til at
kalde mig for "the Fejl-manager". Det glæder mit gamle hjerte at se, at semantik
endelig kan bruges til noget. I tyve år, er det kun blevet brugt til at dumpe de
studerende, men nu skal det frelse Verden.

\says{KK} FEDT! Har du en .plan?

\says{DJ} No, I'm making this up as I go. Men fortæl mig lige...  Nu, hvor
pigerne slynger sig om halsen på jer.. Er i så virkelig sikre på, at jeg skal
rette op på fejlen?

\says{Dataloger} Naaarh

\says{TF} Hva' så med fysikerne?

\says{DJ}[ser fysikeren, væmmes] Fysikere, why did it have to be fysikere! 
\act{skyder fysikeren} Nå, det var jo et overkommeligt problem.

\says{KK} Betyder det, at vi snart kan gå hjem og kode?

\says{Edit} Åh, Klaus Koder, du er bare så nuttet \act{kysser ham på kinden}

\says{KK} På den anden side\ldots{} Jeg kunne jo lade være med at kode\ldots{} såd'n bare i 
aften\ldots

\scene Sang: We've got it going on
\bigskip

Tæppe

\bigskip

Rune \& Sødsen kommer ind foran tæppet
\says{Søren} Øjeblik, øjeblik, vi har lige en tradition der skal
overholdes!

\scene
Søren vender sig om og skyder Rune, der falder død om. Søren slæber
ham ud bag tæppet.


\end{sketch}

\end{document}
% Local Variables: 
% mode: latex
% TeX-master: t
% End: 
