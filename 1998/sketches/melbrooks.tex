\documentclass[danish]{article}
\usepackage{revy}
\usepackage[utf8]{inputenc}
\usepackage{babel}
\usepackage{a4wide}

\title{Højt at kode}
\author{Jonas (\texttt{duck}), Uffe (\texttt{uffefl}), m.fl.}

\version{2.1 -- færdig} % HUSK AT AJOURFØRE VERSIONSNUMMER!!

\revyyear{1998}

\begin{document}
\maketitle

\begin{roles}
  \role{E} E-lane; fysikstuderende
  \role{K} Hendes kæreste, Ren\'e; datalogistuderende
  \role{J} Jack, hendes eks; hacker
  \role{O1} Chefoperatør
  \role{O2-O5} Operatører
  \role{S} Datalogistudine
  \role{V} Vagt
  \role{VO1} Voice over; mandlig
  \role{VO2} Voice over; kvindelig
\end{roles}

\begin{sketch}

\scene Mens der foretages sceneskift benyttes tiden til følgende

\says{VO1} Velkommen til DIKU. Bemærk at de lige konti er til studierelevant arbejde, mens de ulige konti kun er til leg og spil.

\says{VO2} Nej, det er da omvendt. Det er de \emph{ulige} konti der er til
studierelevant arbejde, mens de \emph{lige} konti er til leg og spil.

\says{VO1} Nu skal du holde op. Jeg har arbejdet her på DIKU i 12 år, og det er
altså de \emph{lige} konti der er til studierelevant arbejde og de \emph{ulige}
konti der er til leg og spil.

\says{VO2}[finder noget papir frem] Jeg har en mail, dateret d. 12/5-1996, fra
Margit, hvor det udspecificeres at det er de \emph{ulige} konti der er til
studierelevant arbejde og de \emph{lige} konti der er til leg og spil. Pånær
selvfølgelig de konti der er fibonaccinummerede. De er instruktorkonti.

\says{VO1} Sig mig, kommer du her og fortæller mig hvordan jeg skal udføre mit
arbejde din lede sæk! Institutsbestyreren \emph{(lydeffekt: hestevrinsk)}
udstedte et dekret d. 15/5-1996 hvor han specifikt sagde at de \emph{lige} konti
skulle bruges til studierelevant arbejde og de \emph{ulige} konti til leg og
spil. Undtaget herfra er alle konti der kan faktoriseres udelukkende med
et-cifrede primtal.

\says{VO2} Din gamle nar! Jeg skal sige dig en ting om DIKU-konti, og det er
ikke to ting. Du skal eddermame ikke fortælle \emph{mig} hvad der er rigtigt.
Jeg har 24 Mb zippet post liggende fra Margit om DIKU-konti nummereringen, og
jeg har læst \emph{det hele}, så du skal sgu ikke tro at du bare kan fortælle
mig hvad der er rigtigt. Jeg er meget sandsynligt den mest vidende her på DIKU
om nummereringen.

\says{VO1}[oven i VO1, fra ``jeg skal''] Dumme kælling! EDB-afdelingen system
fungerer jo efter de retningslinier som instituttet har udstedt, og de er ikke
kommet fra Margit alle sammen. Jeg har et bibliotek fyldt med 1000 sider tykke
bøger om nummereringen af DIKU-konti, og de er allesammen fuldstændigt
opdaterede. Jeg har læst dem allesammen \emph{to} gange og jeg kan stadig huske
det hele. Jeg er den eneste der har det fuldstændige overblik og du kan godt
glemme at fortælle mig noget som helst.

\scene Skænderiet fades ud af lydmanden\ldots Lys på scenen

\scene E, K og S sidder ved en terminal (placeret i en vinkel så\ldots) på et
bord, 3 stole

\says{E} Åh, Ren\'e! Det er så godt at skrive kerne sammen med dig. Det var
\emph{frygteligt} under simulatoropgaven!

\says{K} Mulatoropgaven.

\says{E} Ja\ldots\ I efterårssemesteret.

\says{K} Nåh jah, E-lane, men det kan jo være svært for en fysiker, som dig, at
finde en ordentlig gruppe her på DIKU. \emph{(slesk smil)} Men du fandt jo mig!

\scene E og K fletter fingre, kigger i øjnene og er \emph{så} romantiske.

\says{E} Åh ja, og så er vi jo \emph{mere} end blot rapportmakkere\ldots\
\emph{(han tager kærligt på ham)}

\says{K}[nervøs og lidt flov] Øh\ldots\ ja\ldots

\says{VO2} \ldots{}og husk at man skal have nøgle og gyldigt studiekort for at
opholde sig på DIKU efter klokken fire.

\says{VO1} Nej fjols, det er da studiekort og magnetkort!

\says{VO2} \emph{NU} er det snart nok, jeg skal\ldots

\says{VO1} Ja ja, vi ses i aften skat.

\scene Imens har E og K siddet og været forelsket i hinanden

\says{S}[alvorlig] Så, I to, kan I ikke lige koncentrere Jer om opgaven? Jeg har
nogle problemer med denne her signatur\ldots

\says{K} Natur.

\says{S} Hvor?

\scene E vender sig mod terminalen, K skal til at forklare S hvor der er natur,
men i mellemtiden er J kommet ind og ser hemmelig ud. J holder en hånd op for
munden idet han laver en voice-over forfalskning.

\says{J}[voice-over stil] En cykel, med stelnummer XYZ-4242-42, står parkeret
med lygterne tændt. Ejeren bedes henvende sig i informationen.

\says{K} Åh, det er min cykel!

\scene K forlader scenen i hast. J går hen til E og falder på knæ med blomster i
udstrakt arm.

\says{J} Åh min elskede, jeg har sendt en e-mail i-gen E-lane\ldots\ Men du
svarer ikke! \emph{(snøft)}

\says{E}[forfærdet] Åh Jack! Jamen, jeg er jo sammen med Ren\'e nu. Du og jeg
kunne umuligt finde sammen efter det der skete sidst.

\says{J} Det var et uheld. Jeg laver aldrig mere nudler i dine
eksperimentalgryder. \emph{(får et drømmende blik, uopmærksom på sine
  omgivelser)}

\says{E} Jamen\ldots

\says{S}[udbryder] Hvad laver de grå striber på min skærm og hvor er din analyse
blevet af?

\says{E} Er nettet gået ned?

\says{S} Netop!

\says{E} Er det oppe?!?

\says{S} Nej, nej, nede.

\says{E} Så nettet er nede?

\says{S} Netop!

\says{E} Jamen, sagde du ikke lige\ldots\ \emph{(kigger skeptisk på S)}Jeg må
vist hellere løbe ned på operatørkontoret. \emph{(rejser sig)}

\says{J}[ser E rejse sig] Hvad er det?

\says{E} Det er der hvor de sidder og spiller Quake hele dagen, men det er ikke
vigtigt lige nu. \emph{(går sin vej, mens hun siger)} Jack! Du må forstå det
aldrig kan blive os to igen.

\scene J ser knust ud. S øjner en mulighed for en let fangst.

\says{S}[rykker nærmere på J, meget indladende] Nåh, jeg forstår du er
ledig\ldots

\says{J} Nej, jeg har et job som programmør hos Hewlett-Packard.

\says{S}[smiger] Nå, arbejder du så med deres nye laserprinter?

\says{J} Læserprinter? Nej, kun med dem der kan skrive.

\scene Tæppe for. E træder ud foran tæppet og går hen til åbning og banker på.
Idet hun går ind er hun i EDB-afdelingen. Et skilt er blevet hængt op hvor der
står ``EDB-afdelingen'' og hvor ``D''et hænger skævt og løst på. J og S er væk.
Istedet ses O1-5. O1 er ved bordet, resten rundt omkring.

\says{E} Operatør! Nu skal du bare høre!

\says{O1} Hvad er der i vejen?

\says{E} Nettet er gået ned!

\says{O1} Hvad er det?

\says{E} Det er sådan en samling af datamater, forbundet med kobberledninger,
radio og sattelit, men det er ikke vigtigt lige nu!

\says{O1} HVAD sagde du?!

\says{E} ``Operatør! Nu skal du bare høre!''

\says{O1} Nej, øh\ldots\ -- Det der kom bagefter\ldots

\says{E} Nå, ``nettet er gået ned''

\says{O1} Hvad! Det kan ikke være rigtigt! Du laver grin med mig!

\says{E} Nej, jeg laver \emph{ikke} grin med dig!

\says{O1} Du \emph{må} lave grin med mig!

\says{E} Jeg laver \emph{ikke} grin med dig!

\says{O1} Du \emph{må} lave grin med mig!

\says{E} Hvis jeg lavede grin med dig, ville jeg sige\ldots\ ``Der er gratis
Cola i kantinen!''

\says{O2-O5} Hvad! Er der gratis Cola i kantinen?!?

\scene O2-O5 styrter ud af scenen.

\scene S kommer løbende ind på scenen

\says{S} Operatør, operatør! Embla og Ask er gået ned!

\says{O1} Begge to? På \'en gang?

\says{E+S}[kigger forundret på hinanden, men siger så] ``Operatør, operatør! Embla og Ask er gået ned!''

\scene O2-5 kommer tilbage. O2 har en leverpostejsmad i hånden.

\says{O2} Der var ingen gratis colaer, så vi spiste noget af det der gratis
leverpostej i kantinen. Her, vi har taget noget med til dig. \emph{(rækker O1 en
  mad)}

\says{O1} Åh tak. \emph{(Tager en bid)} Mmm\ldots

\scene O2-5 falder døde om. O1 kigger suspekt på sin leverpostejmad, og holder op med at tygge.

\says{E}[mistroisk] Har I spist af kantinens leverpostej?!?

\scene 01 gisper og falder ned på knæ.

\says{E} I er godt nok ikke for kvikke.

\says{O1}[rallende] Du må\ldots\ du må\ldots\ du må\ldots\ du må\ldots

\scene E skubber til ham så han går ud af hak

\says{O1}[rallende] få nettet\ldots\ op\ldots\ igen\ldots\ \ldots\
passwordet\ldots\ er\ldots\ er\ldots

\says{E}[sekretæragtig, tager noget frem at skrive på] ``R''

\says{O1}[rallende] nej\ldots\ nej\ldots\ passwordet er\ldots\ er\ldots

\says{E} ``R''

\says{O1}[rallende] NEJ! \emph{(falder død om)}

\scene E og S er forvirrede

\says{S}[prøver at tage sig sammen] Vi må få nettet op!

\says{E} Jamen jeg er jo kun en fysiker\ldots

\says{S} Vi må kunne gøre noget med al den hardware!

\says{E} Hardware?

\says{S}[peger] Hard there.

\scene K træder ind

\says{K} Nå, der er du. Jeg har ledt efter dig på hele DIKU.

\says{E} Hvad er der?

\says{K} Der er en frygtelig masse studerende og forskere, og en \emph{kæmpe}
administration, men det er ikke vigtigt lige nu.

\says{E}[besinder sig] Oh Ren\'e, alle operatørerne er døde og nettet er nede.

\says{K} Vorherre på lokum!

\says{E}[pædantisk, peger på herretoilettet] D\'er herre på lokum. \emph{(peger på
  dametoilettet)} D\'er dame på lokum.

\says{E}[desparat igen] Du må få systemet op igen!

\says{K} Jamen det kan jeg ikke uden \texttt{root} passwordet, og DIKU's
EDB-afdeling har den \emph{bedste} sikkerhed.

\says{E}[undrende] Jamen kan du ikke bare spørge hende?

\says{K} Hvem?

\says{E} Ruth!

\says{K} Jamen\ldots\ \emph{(bliver afbrudt af)}

\scene J kommer ind

\says{J} Nå, der er I. Jeg har ledt efter Jer på hele DIKU.

\says{E} Hvad er der?

\says{J} Der er en frygtelig masse vagter og den \emph{grummeste}
institutbestyrer.

\says{K}[uforstående, kigger på E og peger på J] Hvem er det?

\says{J} Søren Olsen! \emph{(lydeffekt: hestevrinsk)}

\says{K}[vantro, skræmt] Søren Oslen? \emph{(lydeffekt: hestevrinsk)}

\says{J} Ja, i egen person.

\says{E}[knuger sig ind til J] Du må gøre noget. Nettet er nede og vi kender
ikke Ruths password.

\scene E skubber ham hen mod en terminal

\says{K} Ja, du må redde systemet ved at logge på som superbruger.

\says{J} Hvor\emph{dan}?

\says{K}[peger på terminalen] D\'er! Og jeg hedder ikke ``Dan''.

\says{J} Jamen maskinerne er så anderledes, allesammen.

\says{S+E+K}[kigger lidt undrende på hinanden, men siger] Jamen maskinerne er så
anderledes.

\scene Stille koncentreret pause mens J sætter sig

\says{J} \ldots{}men \emph{hvad} er passwordet?

\says{K} Nej, ``hvad'' er vores filserver, men det er ikke vigtigt lige nu.

\says{E} Mon ikke passwordet er `42'?

\scene J prøver

\says{J} Nej, det virker ikke.

\says{S} Så prøv med `101010'.

\scene J prøver

\says{J} Nej, det virker heller ikke.

\says{K} Hvad så med `billgates'?

\says{J+E+S}[kigger vantro på K] Arj, helt ærligt!

\scene Sirenen tuder

\says{S} Nå, nu er klokken fire.

\says{J} Hvordan kan du vide det?

\says{S} Hørte du ikke en sirene?

\says{J} Rene. \emph{(udtales lidt som ``rene'' og lidt som ``Ren\'e'')}

\says{K} Hvad er der?

\says{J} Adder. Det er sådan en lille kreds der lægger to tal sammen, men det er
ikke vigtigt lige nu.

\scene V kommer ind

\says{V}[til K] Studie\emph{kort}?

\says{K} Kort? Nej, faktisk ikke, jeg er på 8. år\ldots

\says{V} Intet studiekort? Så må jeg smide dig ud.

\says{K}[bliver smidt ud] Jamen, det der er Søren Olsen \emph{(lydeffekt:
  hestevrinsk)} Han kan sige god for mig!

\says{J} ``God for mig.''

\says{V}[kigger på E, savler] Jah\ldots\ gode former.

\scene K ud. V retter sin opmærksomhed mod S. S viser sit studiekort til V.

\says{V}[til S] Ja tak, og magnetkort.

\says{S} Er det nu også nødvendigt? Jeg mener nord det er \emph{den} vej
\emph{(peger)}, og øst det er\ldots\ \emph{(afbrydes)}

\says{V} Intet magnetkort? Så må jeg smide dig ud.

\scene S ud. V retter sin opmærksomhed mod E. E viser sine kort til V.

\says{V}[til E] Ja tak. \emph{(retter sin opmærksomhed mod J)}

\says{V}[til J] Studiekort?

\says{E}[bryder ind med et dårligt udført bluff] Øhm\ldots\ det er jo chefen for
EDB-afdelingen.

\scene J prøver at spille med, men er heller ikke god til at bluffe

\says{V} Hvordan kan jeg vide det?

\says{E}[mere dårligt bluff] Joh\ldots\ øh\ldots\ han sidder ved\ldots\
superbrugerterminalen.

\says{V}[kigger på terminalen] Men hvem som helst kan jo sidde ved
superbrugerterminalen, uden at være logget ind\ldots

\scene Smilene på E og J's munde stivner

\says{V} \ldots hvis du er chefen for EDB-afdelingen må du jo kunne logge ind.

\says{E}[opgiver] Æh\ldots\ hehe\ldots\ okay, du fik os, han er \emph{ikke}
chefen for EDB-afdelingen.

\says{V}[selvsikker og overlegen] Nårh\ldots\ \emph{(går over til skiltet og
  sætter ``D''et på plads)} Så fik vi sat \emph{det} på plads.

\scene Pause. J kigger på V. J kigger på skilt. J kigger på E. J kigger på
terminal. J kigger på skilt. J kigger på terminal. J kigger på skilt. E kigger
på skilt. J og E kigger på hinanden. J og E kigger på terminal.

\says{J}[taster] E\ldots\ D\ldots\ \emph{(kigger på skiltet)} B\ldots

\scene Pause. J taster enter. Lydeffekt: alt booter op og virker; printerstøj
mm. (``Goodmorning Dave''?)

\says{V} Nå så du \emph{er} chef for EDB-afdelingen. Så må I have en fortsat god
aften.

\scene V forlader scenen. E løber over til J og omfavner ham.

\says{E}[lykkelig] Åh Jack, du gjorde det. Nettet er oppe igen.

\says{J}[stolt] Ja\ldots\ ja\ldots\ øh\ldots\ \emph{(ærgelig)} Nå, men jeg må
hellere gå igen.

\says{E}[lokkende] Åh, bliver du ikke lidt længere?

\says{J} Nej, jeg holdt op med at vokse da jeg var 17.

\says{E} Jack! Jeg vil så gerne være din kæreste igen.

\says{J} Jamen, du kan ikke være min kæreste. Du er jo fysiker -- du tæller
ikke.

\says{E} 1\ldots\ 2\ldots\ 3\ldots\ 4\ldots

\says{J} Nårh, jo\ldots

\scene E og J smiler stift mod publikum.

\scene TÆPPE

\end{sketch}

\end{document}
% Local Variables: 
% mode: latex
% TeX-master: t
% End: 
