\documentclass[a4paper,11pt]{article}

\usepackage{revy}
\usepackage[utf8]{inputenc}
\usepackage[T1]{fontenc}
\usepackage[danish]{babel}
\usepackage{hyperref}

\revyname{DIKUrevy}
\revyyear{2014}
% HUSK AT OPDATERE VERSIONSNUMMER
\version{1.3}
\eta{$5$ minutter}
\status{Færdig}

\title{Human-Centered Computing}
\author{Phillip, Troels, Guldfisk, Simon Shine, Brainfuck}

\begin{document}
\maketitle

\begin{roles}
\role{F1}[Niels] Dr. Preben, HCC-forsker
\role{F2}[Mathias] Dr. Preben, HCC-forsker
\role{F3}[Bitre-Mikkel] Dr. Preben, HCC-forsker
\role{N1}[Jonas] Nazist
\role{N2}[Andreas] Nazist
\role{K0}[Ejnar] Ching, Kinamand
\role{K1}[Peter] Chang, Kinamand
\role{K2}[Arinbjörn] Ching, Kinamand
\role{K3}[Spectrum] Chang, Kinamand
\role{K4}[Sofie] Ching, Kinamand
\role{K5}[Alexandra] Chang, Kinamand
\role{K6}[Caro] Ching, Kinamand
\role{K7}[Maya] Chang, Kinamand
\role{K8}[Markus] Ching, Kinamand
\role{K9}[Kasper] Chang, Kinamand
\role{PS}[Person] Publikumssufflør
\role{N}[Ronni] Clock-neger (Xubuntu), skosværte over HELE kroppen og knogle i næsen
\role{I}[NB] Big-Indian, stereotypisk indianerhøvding
\role{X}[Phillip] Instruktør
\end{roles}

\begin{props}
\prop{Rishatte x10}[Mia]
\prop{Store fortænder x10}[Mia]
\prop{Naziarmbind x 2}[Mia]
\prop{Naziuniform x 2}[Phillip]
\prop{Neger x 1}[Mia]
\prop{Indianer x 1}[Mia]
\end{props}


\begin{sketch}
\scene{F1-F3 går på scenen med stor fanfare.}

\says{F1} I disse dage, hvor kobber-priserne stiger så meget, er der kommet
interesse for andre måder at konstruere digitale kredsløb på.

\says{F2} Her i HCC-gruppen har vi altid snakket om Human-Centered Computing. Men
først for nyligt har vi indset at man sagtens kan benytte et menneske som
digital gate. Datalogi handler jo i virkeligheden mest om mennesker.

\says{F3} Lad os f.eks. sige at jeg er en AND-gate, og mine to kolleger er mit
input. Eftersom deres arme er nede, er jeg også slukket.

\scene{F1 løfter sin ene arm.}

\says{F3} Så, nu blev min ene kollega tændt. Men jeg er en AND-gate, så jeg er
stadig slukket.

\scene{F2 løfter sin ene arm.}

\says{F3} Så, nu blev jeg tændt.

\scene{F3 tager armen op og ned igen.}

\says{F3} Men helt ærligt, hvornår har man sidst set HCC lave noget meningsfyldt?
Så vi besluttede os for at outsource problemet.

\says{F2} Altså, først har vi jo brug for en clock.

\says{F1} Xubuntu!

\scene{N kommer ind på scenen og begynder at tromme rytmisk.}

\says{F3} Vi har også valgt, at lave vores arkitektur som big-endian.

\scene{I kommer ind på scenen, stiller sig med armene over kors, og
  stirrer vredt på publikum.}

\says{F2} Og så outsourcing af selve kredsløbet...

\says{F1} Ja, til at starte med fik vi nogle højtuddannede tyske
ingeniører til at konstruere vores gates.

\says{F2} De krævede dog en XOR-BIT-ant pris.

\says{F3} Ja, man kunne nærmest kalde dem ``large *bill'' gates*

\says{F1} Men vi havde allerede afgivet ORdren.

\scene{Publikum buer.}

\says{F2}  Vi satte tyskerne til at agere serie-forbundne NOT-gates, men desværre
viste det sig at have uheldige implikationer.

\scene{To nazister (N1+N2) marcherer ind på scenen, og stiller sig med
  fronten mod hinanden.}

\scene{N1 hæver armen i en 45 graders vinkel,}

\says{N1} SIEG!

\scene{N1 sænker armen igen. N2 hæver armen i en 45 graders vinkel.}

\says{N2} -NAL!

\scene{N2 sænker armen igen. N1 hæver armen.}

\says{N1} SIEG!

\scene{N1 sænker armen igen. N2 hæver armen.}

\says{N2} -NAL!

\scene{N1 og N2 forlader scenen akkompagneret af marchmusik og lystig
  heilen.}

\says{F1} Tyskerne var effektive, men de menneskelige omkostninger var lidt for
store.

\says{F2} Så vi var nødt til at lægge den idé i massegraven.

\says{F3} Ja, den skulle vist have lidt længere i ovnen.

\scene{Publikum buer.}

\says{F1} Men hvorom alting er, så tog vi et skridt tilbage og spurgte os selv:
"Hvad er billigere end kobber?"

\scene{Publikumssuffløren holder et skilt op med teksten "DIN MOR".}

\says{F2} Kinesere!

\says{F3} Og da de asiatiske folkefærd også udmærker sig ved at være forbløffende
ens, er de særligt velegnede til massekonstruktion af kredsløb.

\says{F1} Lad os demonstrere konceptet ved at konstruere en 2-bit adder.

\scene{Over\TeX: Simpelt kredsløbs-diagram, der viser en 2-bit adder:
\url{http://www.baltissen.org/images/adder2.png}}

\says{F2} Vi får brug for omtrent 10 gates.

\says{F3} Altså, 10 kinesere.

\says{F1} Så vi har bestilt et dusin kinesere på Amazon.

\scene{10 små kinesere marcherer ind på scenen til kinesisk musik. De
  har ris-hatte på, store pap-fortænder og gul makeup. En af dem
  slæber på en kontrabas.}

\scene{Kineserne stiller sig i position. F2 og F3 ligeså. SE TEGNING:
\url{http://harlem.dikurevy.dk/~roschnowski/adder.png}}

\says{F1} Så mangler vi bare Obersturmbannführer Preben.

\scene{N1 og N2 kommer løbende ind på scenen.}

\says{N1+N2} Jawohl, Herr Kursusfører!

\scene{N1 og N2 stiller sig i position.}

\says{F1}[til publikum] Vores første input til adderen er Preben og
Preben. Dem sætter vi til 2.

\scene{F2 vinker til K0 og K2. F3 vinker ikke til nogen.}

\says{F1} Og Preben og Preben er vores andet input. Dem sætter vi til 3.

\says{F1}[vender sig mod N1 og N2] SIEG!

\says{N1+N2} -NAL!

\scene{N1 og N2 heiler lystigt til hhv. K0+K2 og K1+K3.}

\says{F1}[til kineserne] Sæt i værk!

\says{K0}[råber som en Pokémon] AND! AND!

\scene{K0 vinker til K4. K1 og K2 ryster trist på hovedet.}

\says{K3}[hopper på scenen] XOR!

\scene{K3 vinker til K9. K5 og K6 ryster på hovedet. K5 kigger forlegent på
K4.}

\says{K4} OR!

\scene{K9 hopper glad rundt og vinker. K8 ryster på hovedet.}

\says{K7} Carry! Carry!

\says{F1}[til publikum] Som I kan se, fik vi resultatet 1, plus overflow, dvs.
5. Så let kan det faktisk være at lægge to tal sammen.

\scene{Kineserne forlader scenen.}

\says{F2} Efter vores succes med 2-bit adders, er vi ude på Amager gået i gang
med at bygge en 1-gigabyte RAM-kreds af kinesere.

\says{F3}[taster på en lommeregner] Øh.. vil det ikke kræve 8 \textit{milliarder}
kinesere? Vi har jo ikke en kinamands chance.

\says{F2} Måske ikke lige nu. Men du glemmer at antallet af kinesere fordobles
hver 18. måned.

\says{F3} Nå, ja. Maos lov!

\says{F1} Men hvad er \textit{endnu} billigere end kinesere?

\says{F2}[slår ud med armene] Crowdsourcing!

\says{F3} Her i aften har vi jo faktisk et såkaldt "captive audience".

\says{F1} Eller, det har vi i hvert fald lige om lidt. Preben, lås dørene!

\scene{Der låses.}

\says{F2} Vi har et lille kredsløb I skal prøve at simulere. Intel 8080...

\scene{Over\TeX: Et meget avanceret kredsløbsdiagram dukker op:
\url{http://micro.magnet.fsu.edu/chipshots/intel/images/intel8080dielarge.jpg}}

\says{F2} ... sæt i værk!

\scene{Lys ud. ORtæppe for meget hurtigt.}

\end{sketch}
\end{document}
