\documentclass[a4paper,11pt]{article}

\usepackage{revy}
\usepackage[utf8]{inputenc}
\usepackage[T1]{fontenc}
\usepackage[danish]{babel}


\revyname{DIKUrevy}
\revyyear{2014}
% HUSK AT OPDATERE VERSIONSNUMMER
\version{1.0}
\eta{$4.25$ minutter}
\status{I færd med at blive nazificeret}

\title{Datalog: The Debugging}
\author{Troels, Phillip, Ejnar, Mia, Maya}

\begin{document}
\maketitle

\begin{roles}
\role{DM}[Jonas] Dungeon Master
\role{S0}[Kasper] Rollespiller, der bare vil flirte in-game med alle NPC'er
\role{S1}[Nanna] Seriøs, indlevende rollespiller
\role{S2}[Bitre-Mikkel] Munchkin Rollespiller
\role{S3}[Sebastian] Ny rollespiller
\role{X}[Troels] Instruktør
\end{roles}

\begin{props}
\prop{Video med Torben}[Phillip]
\prop{Terningeslag-lydeffekt}[Phillip]
\prop{Pawel-billede}[]
\prop{PC-billeder}[Lea]
\prop{Datalog-regelbog}[]
\prop{Datalog-monsterbog}[]
\prop{Datalog-stor bog}[]
\prop{Fastnettelefon}[Sebastian]
\prop{Bakke til terningeslag}
\end{props}

Tanken er at S1 er en pige (som spiller kliché-agtig datalog). S0 er en fyr som
spiller en kvindelig karakter.

\begin{sketch}

\scene{DM og to spillere. De to andre spillere kommer ind.}

\says{S2} Velkommen til KUA rollespilsklub. Har du nogensinde spillet Datalog før?
\says{S3} Neej, men jeg har kigget lidt på regelbogen.
\says{DM} Nå, men har du fået lavet dig en datalog-karakter?
\says{S3} Ja, her \act{rækker character sheet over}. Han hedder Sigurd.

\scene{De fire andre kigger på S3's character sheet.  Illustration på
  højTeX Pegen og gestikuleren.}

\says{S2} Hmm. Høj Charisma? Hvad vil du bruge det til?
\says{S0} Og Hobby(Fodbold)?
\says{S3} Altså, jeg tænkte at jeg ville lave en velafrundet karakter.
\says{S2} Du har misforstået!  Dataloger er ikke velafrundede, bare runde.
\says{S1} Og hit points? Dataloger har slet ikke så meget liv.

\says{S1}[trækker en bog frem] Ifølge Mogensen så har den
gennemsnitlige datalog hverken venner eller kærester...

\says{DM} Så, lad os nu bare komme igang. Kan i andre ikke lige præsentere jeres
    karakterer?

\scene{HøjTeX viser billeder af karakterne efterhånden som de bliver introduceret.}

\scene{Billede af Malou}
\says{S0} Jeg spiller Malou, en ung pige der bare er vild med funktionsprogrammering
    og fest på Caféen?.
\says{S1}[ruller med øjnene] Totalt urealistisk.
\says{S2} Det er kun for at få de mandlige NPC'er til at lave dine opgaver.
\says{S0} Malou har altså en dyb personlighed!
\says{S1} En udskæring er ikke en personlighed!
\says{DM} Kom nu videre.
\scene{Billede af Preben}
\says{S1} *Min* karakter er tro over for kildematerialet. Jeg spiller Preben, en
    datalogistuderende på 11. år, som bor i sin mors kælder. Preben er rigtig
    glad for at kode i C, og indenterer altid med to mellemrum, og så kan han
    godt lide at spise Pringles med barbecue-smag. I kantinen kan han...
\says{DM} Jaja, næste karakter!
\says{S2} Min karakter hedder Turing \act{de andre sukker}, og er designet som den
    \textit{ultimative} datalog.  På hans Github har han hver dag to tusinde commits og...
\says{S0}[afbryder] Men kan han jo ikke engang lave sin egen toast!
\says{S2} Til gengæld får han +2 til SML!

\says{DM} Godt, lad os nu bare komme igang. Malou og Turing var på vej ind til gruppeeksamen.
    Jeres eksaminator er... \act{ruller terning}... \act{slå op i bog} Pawel!
\scene{Pawel-siden fra Monstrous Manual dukker frem}
\says{S0} Han er sgu da alt for høj challenge rating!
\says{S2} Men tænk på hvor meget ECTXP vi får ud af det!
\says{DM} Pawel har initiativ. Han spørger: Kan I bevise det.
\says{S0} Malou prøver at snakke uden om spørgsmålet.
\says{DM} Slå en D20.
\says{S0} Jeg lægger min kavalergang til.
\says{DM} Pawel er immun. Han sønderlemmer dig for dit usle forsøg på et bevis. Din karakter falder. Hvad gør Turing?
\says{S2} Jeg vil gerne prøve at lave beviset.
\says{DM} Slå for det.
\says{S2} Turing har +31 i quicksort-algoritmer.
\says{DM} Pawel vil \textit{kun} høre om konvekse hylstre.
\scene{S2 slår en terning.}
\says{S0+S1+S2} Piß!
\says{S3} Ja, piß!

\says{DM} I mellemtiden sidder Preben og russen Sigurd i kantinen og
koder.  Det er nu blevet eftermiddag, og Malou og Turing kommer ind i
kantinen. I kan se at de er dumpet.
\says{S1} Preben siger "DNUR."
\says{S2} Turing siger "`bliv nu færdig, gamle røvhul"'.

\says{DM} Videre med plottet!  Jeres karakterer har en OSM-aflevering.
\says{S3} Jamen jeg prøver at kode den.

\scene{Alle de andre (undtagen DM) tager sig til hovedet og jamrer.}

\says{S3}[slår terning] 20!
\says{S0} Nej, se, den ligger skævt.
\says{S3} Så må jeg slå om.
\says{S1}[tager fat i S3s hånd] Nej! Man må ikke røre terningen når den er landet, det bringer uheld!
\says{S0} Men den er skæv!
\says{S2} Din mor er skæv!
\says{S0} Jeg kan ikke se hvordan det er relevant.
\says{S3} Der er et nummer til en terning-hotline bag i bogen.
\says{DM} Jeg prøver at ringe.

\scene{Telefon ringer op, og man ser (på OverTeX) en video af Torben med en
  fastnet-telefon.}

\says{TM} God...
\scene{Torben ruller en terning.}
\says{TM} ... aften, og velkommen til Torbens terninge-hotline.

\says{DM} Ja, hej, vi kan ikke blive enige om hvorvidt denne terning ligger skævt.
\says{TM} Sig mig at i ikke har flyttet på den!
\says{S1} Nej!
\says{TM} Godt. Så er jeg mere rolig. Den slags bringer uheld.
\says{S3} Hvad har jeg så slået?
    \scene{Torben læner sig fremad som om han kigger ud af videoen.}
\says{TM} Øjeblik \act{taster lidt}... Hmm... 1, 2, 4, 8... Pi i mente... Det er en 7'er!
\says{S2} Ej! Den ligger jo...
\says{TM} Torben har talt!

\scene{Torben lægger på, og forbindelsen lukker.  Chokeret tavshed.}

\says{DM} Nå. Skal vi ikke bare sige at det er en 20'er?
\says{S3} Yes!  Jeg består!

\says{DM} Random encounter! \act{DM slår en terning.}  Politikerne har
lavet en reform. I er alle for langt bagud med studiet, og bliver
smidt ud!

\says{S3} Men jeg er rus!
\says{DM} Så må du skrive en dispensationsansøgning. Slå Bureaukrati.
    \scene{S3 slår en terning}
\says{S3} 20!

\says{DM} Det lykkedes! I sender jeres ansøgning til
studieadministrationen.  \act{Pakker sammen og går} Næste session
er om 4-52 uger.

\end{sketch}

\end{document}
