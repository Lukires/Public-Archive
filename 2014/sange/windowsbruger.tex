\documentclass[a4paper,11pt]{article}

\usepackage{revy}
\usepackage[utf8]{inputenc}
\usepackage[T1]{fontenc}
\usepackage[danish]{babel}
\usepackage{hyperref}


\revyname{DIKUrevy}
\revyyear{2014}
\version{1.0}
\eta{$2.25$ minutter}
\status{Windows!}

\title{Windowsbruger}
\author{Troels, Phillip, Brainfuck}
\melody{South Park: ``Uncle Fucker''}

\begin{document}
\maketitle

\begin{roles}
\role{S1}[Mia] Studerende iklædt hoodie
\role{S2}[Maya] Studerende iklædt hoodie
\role{S3}[Daniel] Sanger
\role{S4}[Peter] Sanger
\role{S5}[Vivien] Sanger
\role{S6}[Niels] Sanger
\role{S7}[Alexandra] Sanger
\role{S8}[Nana] Sanger
\role{I}[Troels] Instruktor
\role{K}[Amanda] Kor
\role{K}[Andreas] Kor
\role{K}[Arinbjörn] Kor
\role{K}[Bette-Mikkel] Kor
\role{K}[Bitre-Mikkel] Kor
\role{K}[Brainfuck] Kor
\role{K}[Camilla] Kor
\role{K}[Caro] Kor
\role{K}[Ejnar] Kor
\role{K}[Jenny] Kor
\role{K}[Klaes] Kor
\role{K}[Mark] Kor
\role{K}[Nanna] Kor
\role{K}[NB] Kor
\role{K}[Ronni] Kor
\role{K}[Sebastian] Kor
\role{K}[Simon] Kor
\role{K}[Sofie] Kor
\role{K}[Spectrum] Kor
\role{X}[Phillip] Instruktør
\end{roles}

\begin{props}
\prop{Datamat med Windows på}[]
\prop{4x (Amiga) joystick}[]
\prop{Windowsflag - 1 i hver farve}[Mia]
\prop{Sygt mange blå skjorter}[]
\end{props}


\begin{sketch}

\scene{Lys op. Vi er i et kedeligt øvelseslokale. Der er øvelsestime og kridtstøv over det hele. Instruktoren (I) har en datamat stående. S1 og S2 lytter.}

\says{I} ... og en linje afsluttes med et såkaldt "linebreak", med ASCII-værdien 10.

\says{S1} Gælder det altid?

\says{I} Nej, på Windows, for eksempel, afsluttes en linje først med ASCII-værdien 13, "carriage return" \act{\textbf{Nyt dias}}, og derefter et "linebreak".

\says{S2} Hvordan kan det være?

\says{I} Jo, med gamle dages mekaniske fjernskrivere\act{\textbf{Nyt dias}} betød "linebreak" at man rykkede papiret en linje ned\act{\textbf{Nyt dias}}, og "carriage return" at skrivehovedet vendte retur til venstre side af maskinen\act{\textbf{Nyt dias}}.  Derfor var begge nødvendige for at lave et "linjeskift".

\says{S1} Aha, så det skyldes bagudkompatibilitet?

\says{I} Ja, det kan man godt sige.  Der findes også systemer, der kun bruger "carriage return", eller helt andre tegn, men de er meget usædvanlige. Det minder mig faktisk om en sjov historie angående ordstørrelsen på IBM System/360\act{\textbf{Nyt dias}}! Men den må lige vente til jeg har hentet kaffe...

\scene{I bevæger sig hen mod udgangen af scenen.}

\says{S2} Psst, instruktoren går!

\scene{AV afspiller fanfare:
\url{http://harlem.dikurevy.dk/~roschnowski/wbfanfare.mp3}}

\scene{Instruktoren forlader lokalet.}

\says{S1+S2} Hurra!

\scene{S1 og S2 lister over til instruktorens datamat.}

\says{S2} Hah, se! Han kører Windows!

\scene{S1 og S2 griner lystigt.}

\says{S1} Hey, Maya. Hvordan skifter man vindue i Windows?

\says{S2} Det ved jeg ikke, hvordan?

\says{S1} Man genstarter.

\scene{S1 og S2 griner lystigt.}

\says{S2} Hvordan ved man så at en Windows-maskine er liiige ved at crashe?

\scene{S1 trækker på skuldrene.}

\says{S2} Den er tændt.

\scene{S1 og S2 griner lystigt.}

\says{S1} Vent lidt...

\scene{S1 lyner ned i S2s hoodie, og afslører at S2 har en blå skjorte på nedenunder.}

\scene{META: Heromkring starter videoen, der forklarer hvad der foregår på scenen:
\url{http://harlem.dikurevy.dk/~roschnowski/windowsbruger.mp4}}

\says{S1} ...du bruger jo Windows!

\scene{S2 lyner ned i S1s hoodie, og afslører at S1 har en blå skjorte på nedenunder.}

\says{S2} \emph{Du} bruger Windows!

\scene{Musikken starter. S1 begynder at synge. Imens S2 tager sin hoodie helt af.}
\end{sketch}

\begin{song}
\sings{S1} Tag nu og drop ud, Windowsbruger!
Det' da dig der bruger Windows, Windowsbruger!
(-) Du spiller spil den hele dag
Og du har snart dumpet al' din' fag!

\scene{S2 begynder at synge. Imens tager S1 sin hoodie helt af.}

\sings{S2} (Du sku') selv ta' og drop' ud, Windowsbruger!
Du sku' ta' og køre Gentoo, Windowsbruger!
Du (-) koder aldrig, kan slet ik'
(-) Din gaming gør dig ikke kvik!

\scene{Mellemspil: S1 og S2 tager blå skjorter på og hiver gamle Amiga-joysticks frem som de spiller livligt på. I kommer ind på scenen med en kop kaffe.}
 
\sings{I}[talt] Hør, hvad foregår der her?!

\scene{S1 og S2 danser rundt om I mens de rykker i de joysticks de holder foran deres skridt.}

\scene{S3 og S4 danser ned gennem mellemgangen med joysticks i skridtet. Man kan høre et kor synge ``Uuuuuhh''.}

\scene{S1 og S2 trækker bagtæppet fra. Et kor (K) kommer ind på scenen.}

\sings{K} Windows, Windows
Windowsbruger, Windowsbruger
Windows, Windows

\scene{Alle fra bandet spiller bækkener.}

\sings{S1+S2} Tag nu og drop ud, Windowsbruger!
\sings{K} (Windowsbruger)
\sings{S1+S2} Du bli'r aldrig datalog (-), Windowsbruger!
\sings{S2} (-) Jeg bruger Windows, det er sandt
Min kursusindsats den forsvandt!

\scene{Alle griner. Bandet vifter med Windows-flag.}

\sings{S1+S2} Windowsbruger, det'

\scene{S3-S8 kommer ind af alle døre, og synger hver et bogstav. Instruktoren synger det sidste, mens han falder på knæ.}

\sings{Alle} B-A-L-L-M-E-R
Windowsbruger!!!

\scene{Musikken stopper. Lyset går. Windows-crashlyd. Blå skærme i loftet. Alle heiler. Ingen klapper.}

\end{song}

\end{document}

