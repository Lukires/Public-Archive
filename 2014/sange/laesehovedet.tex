\documentclass[a4paper,11pt]{article}

\usepackage{revy}
\usepackage[utf8]{inputenc}
\usepackage[T1]{fontenc}
\usepackage[danish]{babel}
\usepackage[margin=4cm]{geometry}


\revyname{DIKUrevy}
\revyyear{2014}
\version{0.1}
\eta{$1.5$ minutter}
\status{Ikke faerdig}

\title{Læsehoved't}
\author{Søren}
\melody{?: ``Fætter Mikkel''}

\begin{document}
\maketitle

\begin{roles}
\role{S}[Caro] Sanger
\role{X}[Mark] Instruktør
\role{N1}[Bitre-Mikkel] Ninja
\role{N2}[Markus] Ninja
\role{N3}[Spectrum] Ninja
\role{N4}[Brainfuck] Ninja
\end{roles}

\textit{Sidste linje af omkvædet mangler at blive gjordt færdigt, det indeholder
  rimmene han, spand, mand}
\\
\textit{man kan eventuelt gentage omkvædet tilsidst, og smide noget rocket ind
  over det}

\begin{song}
  I et kabinet
  mellem 0 og 1
  på harddiskens faste plade
  Læsehoved't gik
  med de onde klik
  mens det risede i fladen

  Det var djævelens musik
  \textit{klik, klik}
  der hvor læsehoved't gik
  \textit{klik, klik}
  Og så spilled han på en gammel spand for han var en spillemand. %%Ikke færdig linje
  %datamaten græd, den blev gjordt fortræd,

  Det var klagesang
  der nu nok engang
  til supporten
  ind blev mailet
  Det var fælt ka' i tro
  det var stereo!
  Den var gal i hele raid'et

  Det var djævelens musik
  \textit{klik, klik}
  der hvor læsehoved't gik
  \textit{klik, klik}
  Og så spilled han på en gammel spand for han var en spillemand. %%Ikke færdig linje


% På en skrammelplads Mellem jern og glas
% Og en masse andet affald. Fætter Mikkel gik
% Mellem rustent blik med en sæk og med et syvtal.

% Og der lød en skøn musik der hvor fætter Mikkel gik
% Og så spilled han på en gammel spand for han var en spillemand.

% Der var fuglesang der i luften hang
% Over hele legepladsen. Det var skønt kan i tro,
% det var stereo, fætter Mikkel brummed' bassen.

% Og der lød en skøn musik der hvor fætter Mikkel gik
% Og så spilled han på en gammel spand for han var en spillemand.

\end{song}

\end{document}
