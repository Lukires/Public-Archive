\documentclass[a4paper,11pt]{article}

\usepackage{revy}
\usepackage[utf8]{inputenc}
\usepackage[T1]{fontenc}
\usepackage[danish]{babel}
\usepackage{hyperref}

\hypersetup{
  colorlinks=true,
  linkcolor=black,
  citecolor=black,
  urlcolor=black
}


\revyname{DIKUrevy}
\revyyear{2014}
% HUSK AT OPDATERE VERSIONSNUMMER
\version{1.0}
\eta{$6.25$ minutter}
\status{Går snart i gang med optagelser}

\title{Kjolemand}
\author{Introgruppen}

\begin{document}
\maketitle

\begin{roles}
\role{N1}[Bitre-Mikkel] Ninja
\role{N2}[Markus] Ninja
\role{N3}[Spectrum] Ninja
\role{N4}[Brainfuck] Ninja
\end{roles}

\begin{sketch}
\scene{Se \url{http://intro.dikurevy.dk/Projekter/Kjolemand/Manuskript}
    (brugernavn: introrevyt, løsen: stj3rne) for manuskriptet.}

\scene{Niels synes at filmen er sjov, og desuden varer den nok faktisk ikke
  over 7 minutter.}
\end{sketch}
\end{document}
