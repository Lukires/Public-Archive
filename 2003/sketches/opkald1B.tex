\documentclass[a4paper]{article}
\usepackage[utf8]{inputenc}
\usepackage[T1]{fontenc}
\usepackage[danish]{babel}
\usepackage{charter}
\usepackage{revy}

\revyname{DIKUrevy}
\revyyear{2003}

% HUSK AT AJOURFØRE VERSIONSNUMMER!
\version{1.5}
\eta{x min.}

% OG STATUS!
\status{Færdig}

\title{Opkald under revyen 1 B}
\author{Martin Parm \& Neger-Uffe \& Ulla Tordenskjold \& Heidi}

\begin{document}
\maketitle

\begin{roles}
\role{S}[Søren] virkelig person (Voice over)
\role{L}[Heidi] (Voice-over)
\end{roles}

\begin{sketch}
\act{Lyden af en mobiltelefon afspilles over højtaleren.}

\says{S} Ja, det er Søren.

\says{L} Goddag, det er fra Told \& Skat.

\says{S} Hej skat.

\says{L} Øh\ldots ja\ldots \emph{Told} \& Skat.

\says{S} Hvad siger du?

\says{L} Mit navn er Lise Henriksen\ldots

\says{S} Lise? 

\says{L} Ja\ldots jeg er din sagsbehandler. Jeg har fundet nogle fejl på din
selvangivelse.

\says{S} Det lyder da hyggeligt.

\says{L} Øh\ldots ja\ldots Vi bliver nødt til at gennemgå din selvangivelse.

\says{S} Hva'? Nej, det har jeg altså ikke tid til nu. Jeg sidder lige og ser
DIKUREVY.

\says{L} Der er nogle problemer med den opgjorte trekantshandel.

\says{S} Åh skat, det er jeg sørme ked af, men jeg sidder altså og ser
DIKUREVY. Så det kan altså ikke blive til noget med den trekant lige nu.

\says{L} Det drejer sig om perioden \emph{efter} DIKU revyen.

\says{S} Hva'? Bagefter? Jamen, da skal jeg jo til sommerfest.

\says{L} Det er meget muligt, men du bliver nødt til at sende en rettelse ind.
Hvornår kan vi forvente den?

\says{S} Ja, det ved jeg ikke. Men sent bli'r det i hvert fald.

\says{L} Undskyld, er du klar over hvem du snakker med?

\says{S} Ja.

\says{L} Godt. Hvis du ikke samarbejder må vi sætte dig fra hus og hjem.

\says{S} Ja, skat, det må I gøre. Jeg elsker også dig. I må hygge jer,
ikk'.

\says{L} Hvad? Jamen\ldots

\says{S} Hej.

\end{sketch}
\end{document}
