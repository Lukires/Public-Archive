\documentclass[a4paper,11pt]{article}

\usepackage{revy}
\usepackage[utf8]{inputenc}
\usepackage[T1]{fontenc}
\usepackage[danish]{babel}

\revyname{DIKUrevy}
\revyyear{2003}

% HUSK AT AJOURFØRE VERSIONSNUMMER!
\version{1.2}
\eta{5 min.}

% OG STATUS!
\status{Færdig}

\title{En dårlig sketch, 5 gange}
\author{\tt guldfisk, tischer, marvin og uffe m.fl.}

\begin{document}
\maketitle

\begin{roles}
\role{NA}[Rune] Niels Andersen
\role{S}[Andr\'e] Studerende
\role{SV}[Uffe FL] Studievært
\role{G1}[Carsten] Grøn Mand \#1
\role{G2}[Mads] Grøn Mand \#3
\role{G3}[Uffe C] Grøn Mand \#4
\role{G4}[Uhd] Grøn Mand \#5
\role{ID}[Bo] Islandsk dulle
\end{roles}

\begin{props}
\prop{Læderfrakke}
\prop{Rapport med post-it-notes}
\prop{Jakkesæt}
\prop{Solbriller}
\prop{Tavle}
\prop{Kridtstreg}
\prop{Grønne kostumer}
\end{props}



\begin{sketch}
\scene{NA tørrer tavlen af. S går op til NA}

\says{S} Hremm! Niels Andersen
\says{NA} Ja. \act{Vender sig om mod S}
\says{S} Ja. Jeg har \ldots hrmmm.
\says{NA} Ja min ven \act{Smilende og venlig}
\says{S} Jeg har denne hersens opgave.
\says{NA} JA?
\says{S} Den er ikke blevet godkendt! Og det synes jeg ikke er rigtigt - Prøv at se her. 
\act{S giver NA sin opgave som straks begynder at læse den igennem}
\says{NA} Nå ... hmmmmm \act{bladrer} NÅ \act{bladrer viderer} Ja jeg kan se at i her gennemgår liniens ligning.
\says{S}[Selvglad] Ja er det ikke godt?
\says{NA} Og her \act{stopper med at bladre}. Her har du en vældig fin gennemgang af din arbejdsproces. 
\act{Fortsætter med at bladre}
\says{NA} Nå, her er dit program - fint \act{bladre} - fint \act{bladre} - OK \ldots Det oversætter ikke?
\says{S} Næ \ldots nej men min afprøvning er rigtig god.
\says{NA}Men det ser ikke ud til at afprøvningen er med i rapporten?
\says{S}[Undskyldende] Neej, jeg fik den ikke helt skrevet ren.
\says{NA} Designet! \act{pause} Jeg er ikke sikker på fremstillingen. \act{Viser publikum en side med en masse postit}
\says{S}[Glad] Farven angiver returtypen. Er det ikke bare klasse.
\says{NA} Men er det nu også relevant for din aflevering i Kryptologi?
\scene{Alle fryser. Grønne mænd løber ind på scenen og flytter alt på plads. SV kommer på scenen}

\says{SV} I har netop set en ganske almindelig episode her på DIKU, men hvordan ville denne episode se ud hvis 
Institubestyreren var en tidligere operadirektør? \act{Går ud af scenen} 

\scene{alt går igang igen. Der synges nu så operaagtigt som skuespillerne kan}
\says{S} Niels Andersen
\says{NA} Ja. \act{vender sig}
\says{S} Min opgave er ej blevet godkendt.
\says{NA} Ej blevet godkendt?
\says{S}  Ej blevet godkendt!
\scene{kunstpause. S stikker rapporten til NA. NA tager den.}
\says{NA} \act{bladrer i rapporten. Lukker den} Men rapporten den er helt fantastisk.
\says{S} Den er helt fantastisk
\says{NA+S} Den er helt fantastisk
\says{NA} \act{bladrer i rapporten igen. Kigger pludseligt op chokeret.} Men der mangler jo en afprøvning.
\says{S} \act{Vender sig om. Ser skyldig og skræmt ud.} Den må jeg ha'
tabt her på vejen. 
\says{NA} \act{Peger anklagende på S} Du LYVER! Dø, du usling!
\says{S} Jeg dør!
\says{NA} Han dør!
\says{S} Jeg dør!
\says{NA} Han dør!
\act{S falder død om på stedet}
\scene{Alle fryser. Grønne mænd løber ind på scenen og flytter alt på
plads. SV kommer på scenen}

\says{SV} Men hvad nu, hvis institutbestyreren var Lars von Trier? Sådan noget
ed håndholdte skuespillere\ldots

\scene Grønne mænd holder NA og S i hånden. En anden grøn mand laver en
kridtstreg på gulvet elle de to. S går hen mod NA, men splatter ind i kridtmuren.

\scene ID kommer ind, synger på islandsk og dør.

\scene{Alle fryser. Grønne mænd løber ind på scenen og flytter alt på
plads. SV kommer på scenen}


\says{SV} Men hvordan ville scenen se ud hvis institutbestyreren var instruktøren fra The Matrix?
\scene{S kommer ind iklædt jakkesæt. NA er nu iklædt lang læderfrakke og solbriller.}
\says{S}[Med Mr. Smith accent] Mr. Andersson!
\says{NA} \act{Vender sig hurtigt om} Dig!
\says{S} Ja, det er mig. Jeg er kommet angående min rapport.
\says{NA} Din rapport var for dårlig, en genaflevering er påkrævet.
\says{S} Nej. Min rapport er genial. Det er nytænknig. Men det vil en dinosaur som du ikke indse.
Nu er det på tide at du følger i dinosaurens fodspor og uddør.

\scene{NA og S fryser. De grønne mænd kommer på scenen og tager fat i dem begge. Nu udkæmpes en kamp i bullettime.
Ca. 10 sek. Afsluttes med at S ryger i gulvet}
\says{NA} Du vil aldrig slå mig før du har mestret rapportskrivningens kunst.
\says{S} Rapporten er uvæsentlig. Det er koden der er altafgørende.

\scene{10 sekunders kampscene mere. Afsluttes med S der bliver sparket ud af scenen}
\says{NA} Koden er uvæsentlig. Rapporten er altafgørende.

\says{SV}[``fryser'' scenen] Erhmm... ja. Men hvordan ville denne scene så tage sig ud,
hvis der var ophængt lyssensore i klasselokalerne?

\scene SV spoler tilbage og de starter forfra

\says{S}[I samme stil som første gang] Nils Andersen?
\scene Lys ned, mørk scene, tæppe

\end{sketch}

\end{document}
