\documentclass[a4paper,11pt]{article}

\usepackage{revy}
\usepackage[utf8]{inputenc}
\usepackage[T1]{fontenc}
\usepackage[danish]{babel}

\revyname{DIKUrevy}
\revyyear{2008}
% HUSK AT OPDATERE VERSIONSNUMMER
\version{1.0}
\eta{$5$ minutter}
\status{Mangler en bedre slutning, må gerne strammes op}
\newcommand{\hide}[1]{}
\title{Specialekoppen}
\author{Munter, Kristine, Jakob og Johan}

\begin{document}
\maketitle

\begin{roles}
\role{S1}[Ejnar] Studerende 1
\role{S2}[Klaes] Studerende 2
\role{ST1}[Mikkel] Statist 1
\role{ST2}[Troels] Statist 2
\role{KV} [?] Kantine vagt
\end{roles}

\begin{props}
\prop{En masse kopper}[Rekvisitholdet]
\prop{En masse tallerkener}[Rekvisitholdet]
\prop{Stor skraldespand til papirgenbrug}[Rekvisitholdet]
\prop{Slide med foto af specialekoppen på toiletdøren}[Rekvisitholdet]
\prop{En babykop der ikke kan vælte - Paritetskoppen}[Rekvisitholdet]
\prop{Stor version af DIKU's trivsels regl nr. 1 (I den rigtige
  farve)}[Rekvisitholdet]
\prop{Fejeredskaber, så store og brede som det kan lade sig
  gøre}[Rekvisitholdet]
\end{props}

  
\begin{sketch}

\scene{Der står en masse kaffekopper tilfældigt omkring på gulvet. S1
  står med en kaffekop. Undervejs skal der komme statister og sætte
  kopper og rette på teskeer og vejledere sætter talerkener ved siden af kopperne (det giver mening senere - vi lover)} 
\scene{S1 står på Scenen. S2 kommer gående ud på scenen}

\says{S1} [hviskes] Hey, Peter. Kom lige herover. Er det ikke noget
med at du er i gang med at skrive speciale? Hvordan er det egentlig lige man kommer i gang med det?

\says{S2} Shhh!!! Det er jo ikke officielt. Ingen må vide det.

\says{S2}[Højt] Nej, jeg skriver i hvert fald ikke speciale om Almene
Teoridannelser om Generisk programmering af numeriske løsninger til
sædvanlige og partielle differentialligninger med speciel henblik på
udnyttelse af Standard Fortran 76 og dennes henvisninger til de videre
resultater beskrevet i turings artikel, "über die Wesen des
primtalalgorimus des kvadratisher bubblesort anno 1943" bilag F... -
før nu og i fremtiden

\says{S1} Gør du ik'?

\says{S2} [hviskende] Jo jo du misforstår. Det er hemmeligt pga de nye specialeregler.
Der må jeg kun være et halvt år om mit speciale.

\says{S1} Nå, jeg forstår. Men hvordan kommer jeg så igang med "ikke"
at skrive mit speciale?

\says{S2} Du skal starte med at have en vejleder

\says{S1} Ja, jeg var lige på vej hen for at snakke med Jakob Grue.

\says{S2} Nej for helvede! Det må du aldrig gøre. Hvis du går hen og
snakker med en vejleder bliver det registreret og for hver time der
går inden du afleverer en forside koster det et eksamensforsøg.

\says{S1} Jamen, hvad skal jeg så gøre. Skal jeg skrive en mail?

\says{S2} Nej er du da sindssyg! Regeringen logger det. Helge Sander
møder personligt op og følger dig ned til studieadministrationen.

\says{S1} Jamen hvad så når de altid har lukket

\says{S2} Det er bare ærgerligt. Så dumper du.

\says{S1} Jamen der må da være en måde jeg kan komme i kontakt med en vejleder på.

\says{S2} Ja, men det er super hemmeligt. Har du nogensinde overvejet hvorfor der står kaffekopper overalt på DIKU?

\says{S1} Ja, det er da bare fordi folk ikke ryder op efter sig.

\says{S2} Naive ungdom, alting betyder noget! Kan du se denne kop på
trappen? \act{S2 samler MEGET forsigtigt koppen op.}

\says{S2} [mens han holder koppen meget forsigtigt] Det er Sørens besked
til Martin Z om at han vil lægge et udkast til en arbejdsbeskrivelse i bunden af
papirskaldespandene. Han har sat den på 3. trin. Derfor er det ved
printerne i midgaard.

\scene{S1} \act{hiver teskeen ud af koppen} Men hvad betyder teskeen så?

\says{S2} Du må ikke ændre på noget!

\scene{S1} skynder sig at sætte teskeen i igen

\says{S2} Nej, når teskeen står sådan betyder det at Søren har givet op og
fået et job og er blevet kodeslave for kapitalen.

\scene{S1 ændrer skeens placering og S2 sætter den forsigtigt tilbage på plads}

\says{S1} Jamen behøver kopperne være så ulækre?

\says{S2} Jaja, Kappegrumset danner en 2D stregkode som din vejleder kan
aflæse og mugget signalerer hvor langt du er kommet. Har du slet ikke læst om specialekoppen på toilettet i kælderen?

\says{S1} Specialekoppen ???

\scene{Slide med specialekoppen på toiletdøren dukker op}

\says{S1 : Nå ja, det giver mening. Man starter vel oppefra med at aflevere forside...}

\says{S2} Nej Nej Nej !!! At aflevere forside er det SIDSTE du skal gøre. Tror du nu du har forstået det?

\says{S1} Jah.. næsten. Pånær, hvordan får jeg svar fra min vejleder?

\says{S2} Via vejledertalerkenen, selvfølgelig

\says{S1} Vejledertalerkenen?

\says{S2} Ja, vejlederen sætter talerkener med madrester. Du kan se de
krummer her betyder at der er en grammatisk fejl på side 20 af Mortens
speciale.

\says{S1} Puha. Det lyder meget kompliceret. Er det aldrig gået galt?

\says{S2} Jo. Der var en gang i 90'erne hvor rengøringen tog alt service og satte på plads.

\says{S1} Nåh ja. Det er derfor der er så stor mangel på dataloger nu.

\says{S2} Netop. Så det gør rengøringen aldrig igen.

\says{S1} Det lyder som om man skulle overveje en backupløsning.

\says{S2} Det har vi skam tænkt på. Efter rengøringsfadæsen blev der oprettet en paritetskop, som sikrer at al 
kommunikation på DIKU fungerer.

\says{S1} Hele kommunikationen på DIKU? I én kop?

\says{S2} Ja ja, Lad mig vise dig hvor den gemmer sig. Hvis du går ned i Midgård og kigger i elskabet 
ved nordvægen, løfter øverste hylde og kigger op mod venstre, så står den der. \act{Går ud af scenen mens han 
taler} Jeg henter den lige.

\scene{ST1 og ST2 kommer ind og retter på noget}

\scene{S2 forlader scenen og kommer tilbage igen med en kop omhyggeligt holdt fast i to hænder.}

\says{S2} Dét her er koppen der holder sammen på AL kommunikation på DIKU.

\says{S2} Lad os se om det alt er i orden. \act{Ser på paritetskop, og
derefter ud på servicen} Ja, det ser ud til at være i orden.

\says{S1} \act{Peger på paritetskoppen} Hov! Der er da gåret skår i paritetskoppen.

\says{S2} Ja. Det var da Naur blev filosof.

\says{S1} Nåh, hmm. Men det lyder lidt usikkert.

\says{S2} Bare roligt - det er paritetscheck, så bare der kun er én ting
der går galt, så kan vi altid fikse det.
\hide{
Begin Alt-slutning

\scene{Fuld mand kommer væltende ind og tramper godt rundt i servicen
  - vælter ud igen}

\says{FM} Hey, skal i ikke med på Cafen? 

\says{S1+S2} PISS!

\scene{TFH}

Alternativ slutning 2
}
\scene{Kantine Vagt kommer ind med skildt}

\says{KV} Har i set de \emph{rædselsfulde nye regler} for DIKU?
\act{Vender skiltet med trivelses regler}

\scene{ST kommer ind og begynder at feje servicen væk}

\says{S1+S2} PISS!

\scene{Lys ud, tæppe for hurtigt}
\end{sketch}
\end{document}
