\documentclass[a4paper,11pt]{article}

\usepackage{revy}
\usepackage[utf8]{inputenc}
\usepackage[T1]{fontenc}
\usepackage[danish]{babel}

\revyname{DIKUrevy}
\revyyear{2008}
% HUSK AT OPDATERE VERSIONSNUMMER
\version{1.3}
\eta{$6.5$ minutter}
\status{Færdig}

\title{Rektor Ralf}
\author{Guldfisk, Kristine og Dirk}

\begin{document}
\maketitle

\begin{roles}
\role{RR}[Fairchil] Rektor Ralf
\role{B}[Dirk] Bureaukrat
\end{roles}

\begin{props}
\prop{Jakkesæt}[Fairchil]
\prop{Bord}[Rekvisitgruppen]
\prop{Stol}[Rekvisitgruppen]
\prop{Billedblad eller lignende}[Rekvisitgruppen]
\prop{Kuglepen, gerne med tyl (fx en lyserød fjer)}[Rekvisitgruppen]
\prop{Paryk, platinblond}[Rekvisitgruppen]
\prop{Lyserød top}[Rekvisitgruppen]
\prop{Patter}[Rekvisitgruppen]
\prop{Håndtaske}[Rekvisitgruppen]
\prop{Hvide bukser}[Rekvisitgruppen]
\prop{1 ark hvidt papir, 1 ark laksefarvet papir}[Dirk]
\end{props}

  
\begin{sketch}

\scene{B sidder med et billedblad og filer negle, er overhovedet ikke 
interesseret i hvem hun snakker med. RR kommer stormende ind med sin 
lønseddel i hånden. Det er vigtigt, at replikkerne falder snapt igennem 
hele sketchen - specielt i de korte replikskifter. Ikke noget med at 
smage på dejlige, lange bureaukratord, for så kommer vi aaaldrig 
hjem;-)}

\scene{RR er meget fysisk. Når han siger ``JEG ER REKTOR RALF'' går han ud midt på scenen og råber med knyttede hænder i vejret.}
 
\says{RR}[Henvendt til publikum] Jeg har lige fyret min sekretær. Uduelige tøjte. 
Hun lavede ikke andet end at udfylde formularer og spille syvkabale . 
Nu hvor jeg endelig er sluppet af med hende, så dukker der en fejlbehæftet lønseddel 
op! JEG ER REKTOR RALF. Jeg har magt over jer alle, og jeg 
finder mig ikke i det!

\says{RR}[Henvender sig til B] Du der bureaukrat. Jeg 
har ikke fået den løn, jeg har krav på! På min lønseddel står der 
"Regøringsassistent", og jeg har fået løn som en Rengøringassistent!

\says{B} \act{Vrister sig modvilligt løs fra sin neglefil og billedblad} Og dit problem er?

\says{RR} At jeg ikke er rengøringsassistent.

\says{B} Aha, Lad mig udfylde en løn-rettelses-anmodnings-blanket 27x-B\footnote{det udtales syvogtyve-x-streg-B} \act{tager en blanket frem} Og hvad er du så, 
om jeg må spørge?

\says{RR} Kender du ikke MIG?

\says{B} Jeg kender virkelig ikke alle TAPere.

\says{RR} Jeg er ikke TAPer!

\says{B} Er du da forsker?

\says{RR} Nej.

\says{B} Er du ph.D. studerende?

\says{RR} Nej!

\says{B} Er du studentermedhjælp?

\says{RR} NEJ!!!!

\says{B} Så kan du ikke få løn. Jeg har kun de fire rubrikker.

\says{RR}[Med ond røst] Ved du ikke hvem jeg er? Jeg er din chef.

\says{B} Steen?

\says{RR} Nej, jeg er mere mægtig. Jeg styrer jer alle!

\says{B} Martin?

\says{RR} Nej!

\says{B} Jørgen, er det dig. Jeg troede, du var smuttet!

\says{RR} JEG ER REKTOR RALF. Jeg har magt over jer alle. Jeg er den højeste instans på hele universitetet!

\says{B} \act{Kigger på blanketten, ryster på hovedet} Den rubrik har jeg 
ikke.

\says{RR} Jeg skal have min løn! Ikke én eller anden gemen TAP-løn. Jeg skal 
have penge, MANGE PÆNGE!

\says{B}[Meget pædagogisk] OK. Har du husket at få Steen til at underskrive 
din timeseddel?

\says{RR} Steen skal ikke underskrive nogen timeseddel for mig... Steen er 
kun en ubetydelig TAP'er, en tyende en ugidelige ligegyldig slimklat, som jeg kan 
 FYRE og tilintetgøre ligesom ALLE andre på dette sted!

\says{B} Nå, hvem underskriver så dine timesedler?

\says{RR}[Prøver at agere ``pædagogisk''] Hør her. JEG ER REKTOR RALF. Jeg har 
magt over jer alle, og jeg skal ikke have underskrevet nogen timeseddel!

\says{B}[I samme ``pædagogiske'' tonefald] Så kan du ikke få løn.

\says{RR} Jamen, jeg skal have løn!

\says{B}[Efterligner hans tonefald] Jamen, det kan du ikke få!

\says{RR} Jamen, det skal jeg!

\scene{Hurtig ordveksling her:}

\says{B} Nej.

\says{RR} Jo.

\says{B} Nej.

\says{RR} Jo.

\says{B} Nå! Men så bed din sekretær om at udfylde en ekstraordinær 
løn-Tilrettelses-Anmodnings-formular 56S-7 til 
løn-Tilrettelses-Anmodnings-Administrationen, så vil de svare indenfor tre 
måneder. Du skal bare bede om den lyseblå formular.

\says{RR} Jamen, jeg har fyret min sekretær, og jeg kan da ikke vente tre 
måneder. Er du overhovedet i stand til at forestille dig hvad en mand som jeg har af faste månedlige udgifter?

\says{B} \act{lægger armene over kors og kigger på ham}

\says{RR} Er der slet ikke andre, der kan hjælpe mig?

\says{B} Jo, det er der da. Jonna kan også.

\says{RR} Så sæt mig i kontakt med Jonna.

\says{B} Desværre. Hun er på ferie.

\says{RR} Hva' med Marianne?

\says{B} Til fagligt møde.

\says{RR} Hvad med Gunnbritt?

\says{B} Hun er syg.

\says{RR} Ditte?

\says{B} Sygt Barn.

\says{RR} Inge?

\says{B} Sygt kæledyr.

\says{RR} Britta?

\says{B} Det er din sekretær. Det er hende, har du fyret.

\says{RR} AAAH! Hvad er det her for en service?

\says{B} Det er ikke service, det er Københavns universitet!

\says{RR} Så er det nok. Det her var dråben. Jeg har magt over jer alle. Jeg 
har magt til at ansætte og afskedige hver og én. Jeg fyrer dig på gråt 
papir!

\says{B} \act{Smiler bredt} Desværre, vi har ikke mere gråt papir, men vi har da 
lidt laksefarvet. Du skal udfylde en 75b-c.

\says{RR} Hvad er en 75b-c?

\says{B} Det er den blanket, man anvender, når der ikke er flere 
75b-b-formularer.

\says{RR} Og hvad er en 75B-b-formular så?

\says{B} Det er en fyring-På-Gråt-Papir-Anmodnings-formular, men da vi ikke har 
flere 75B-b-formularer, skal du udfylde en 75B-c-formular, der er en 
fyring-På-Laksefarvet-Papir-I-Tilfælde-Af-Mangel-På-Gråt-Papir-Anmodnings-Formular.

\says{RR} FINT, så giv mig en laksefarvet formular!

\says{B} Ja, så skal du lige hen til Nina og hente det ...

\says{RR} Hvor finder jeg så hende?

\says{B} I stueetagen, men det nytter ikke noget, for hun har fri kl. 3.

\says{RR} Det skulle da ikke være noget problem. Klokken er kun halv 2.

\says{B} Nej, men hun har rygepause først.

\says{RR} \act{Suk} Er der ikke andre, der kan udlevere en sådan formular?

\says{B} Jo, Birgit! Nej vent, hun er stoppet. Netop på grund af stress. At du ikke 
skammer dig.

\says{RR} Hvordan kan nogen have stress i KUs administration. I laver jo ikke 
noget.

\says{B} Hun kunne ikke overskue alle de blanketter, hun skulle udfylde.

\says{RR} Jeg vil da IKKE have hende til at udfylde blanketter!

\says{B} Ville du ikke lige fyre mig på laksefarvet papir?!

\says{RR} Jeg er fuldstændig ligeglad hvilken farve papiret har!

\says{B} Det kunne du bare have sagt. Så kan du bare bruge en gul 
75B-D-formular...

\says{RR} Og, hvor får jeg den henne?

\says{B} Det er da let, den får du hos Rektor Ralf.

\says{RR} JEG ER REKTOR RALF! \act{Begynder at græde} Jeg har magt over jer alle. 
Den højesete instans på fakultetet... \act{bryder grædende sammen}

\says{B} Så er det jo ikke noget problem. Bare spørg din sekretær... Nårh nej...

\scene{I det samme bladres billedbladet om til slutningen, og der ligger en blanket under sidste side. B tager den op, og kigger undrende på den.}

\says{B} Guuud, der er jo netop en laksefarvet 75B-C formular her, som du kan bruge! Nej, vent -- den er jo allerede udfyldt ... af Helge Sander! \act{Kigger på den}
Du skal jo slet ikke have løn, Ralf.

\says{RR} Hvad?!

\says{B} \act{viser ham blanketten} Du er fyret.

\scene{Lys ud, tæppe for hurtigt}

\end{sketch}
\end{document}

%%% Local Variables: 
%%% mode: latex
%%% TeX-maste\says{RR} t
%%% End: 

