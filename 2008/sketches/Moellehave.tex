\documentclass[a4paper,11pt]{article}

\usepackage{revy}
\usepackage[utf8]{inputenc}
\usepackage[T1]{fontenc}
\usepackage[danish]{babel}

\revyname{DIKUrevy}
\revyyear{2008}
% HUSK AT OPDATERE VERSIONSNUMMER
\version{1.8}
\eta{$3.5$ minutter}
\status{Færdig}

\title{Møllehave}
\author{Johan}

\begin{document}
\maketitle

\begin{roles}
\role{D}[Johan] Johannes Møllehave som Datalog
\end{roles}

\begin{props}
\prop{Brun Jakke (Moellehave-/Professor-agtig}[Rekvisitgruppen]
\end{props}

  
\begin{sketch}
\scene{D kommer ind på scenen}

\says{D} Lad mig nu se, hvor kom jeg fra? iPhone, besynderligt
begreb.. Interessant, spændende, facinerende.. iPhone..
  i, P, h, o, n, e .. Et ord, en permutation af bogstaver, en
komposition taget af alfabetet, en cykel, en vej i den komplette graf
af bogstaver, dog ikke en kreds,
  ej heller en tur.

\says{D} Men hvad betyder det? Betyder det noget? Hvor kommer det fra?
Hvor vil det hen? Har det en mening? Hvad er dets historie?

\says{D} Det hele starter med i. Inverteret, integreret,
imaginært. En imaginær del.

\says{D} Efterfulgt af P, der er med stort, tillagt mere betydning. Det må være en
funktion, en relation, en delmængde af krydsproduktet. 
  Så bliver det i gange P anvendt på quadruplen (h,o,n,e).

\says{D} Lad mig se, i enden et de. Euforisk,
Euclidisk, Euler.. Eulers tal, e, den matematiske konstant.
  Grundtallet for den naturlige logaritme. Det begynder at tage
form, skikkelse, rytme.

\says{D} Jeg har nu i*P anvendt på quadruplen (h,o,n,e), hvor i er i, P er P og e er e.

\says{D} Men jeg mangler stadig h, o og n. 

\says{D} H, h, heroisk, hilbert, hermitisk. En hermitisk matrix..
  lig sin egen konjugerede transposition.

\says{D} Og o, oh ja, o, origo, ortogonal, optimal objektfunktion -
jeg har den jo, det må være en optimering af objektfunktion o, under
begrænsningerne angivet i h. Det kan alle da forstå.

\says{D} så er der jo kun én mulighed for n - den nedre grænse
for den optimale løsning der søges, men

\says{D} ..så kan P jo kun være en implementation af den Konjugerede
  Gradient Metode, der skal søge efter den optimale maksimale løsning i det
  kompleske rum, afgrænset af h og med profit retningen afledt af
  o. 

\says{D} Den er jo effektiv med en køretid på O($n^3$), 
  med n korrektioner på n iterationer i n dimensioner på konvekse
  funktioner. 
\says{D} Og Med Golden Section Search i hver iteration til
1-dimensionel optimering. 

\says{D} Meeeen det er da lidt sært, det jo fortsat imaginært. 
\says{D} Der er jo intet reelt ved IPhone!?


%\says{D} Genialt, det skal de have, en lækker sag, der falder i min smag.

%\says{D} Men hvad er det der optimeres? Det må være salget af det hersens objekt,
%genstand. En maksimering af profit, under begrænsninger, knappe resourcer.

%\says{D} Men der mangler noget, i'et gør det imaginært og det er jo
%lidt sært. Resultatet af P anvendt på (h,o,n,e), men herefter
%ganget med i?

%\says{D} Der mangler jo en reel del. 

%\says{D} Jeg står med et halvt tal, en halv eksistens. Har de overset
%noget i navnet? Skulle det i virkeligheden have heddet MyiPhone eller
%WhyiPhone eller måske WhyBuyMyIPhone?

%\says{D} Heurika!? Selvfølgelig, jeg har det.. Hvorfor har jeg ikke bemærket det
%noget før?? Når man sidder fast, må man vende tilbage til begyndelsen,
%udgangspunktet, undfangelsen. Hvor kommer dette objekt, påfund, apparat
%fra? Der finder man den reelle del af udtrykket.

%\says{D} Det komplette komplekse udtryk må naturligvis være a*P(p,l,e)
%+ i*P(h,o,n,e), hvor a og i er enhedsvektorer. Det giver to
%optimeringer, der ender med et komplekst tal. Den fulde analyse over
%salget af iPhone. 

\scene{Lys ud}
\end{sketch}

\end{document}

%%% Local Variables: 
%%% mode: latex
%%% TeX-master: t
%%% End: 

