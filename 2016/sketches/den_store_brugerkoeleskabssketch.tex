\documentclass[a4paper,11pt]{article}

\usepackage{revy}
\usepackage[utf8]{inputenc}
\usepackage[T1]{fontenc}
\usepackage[danish]{babel}


\revyname{DIKUrevy}
\revyyear{2016}
\version{0.1}
\eta{$2.75$ minutter}
\status{Ikke færdig}

\title{Den Store Brugerkøleskabssketch}
\author{Troels, Brainfuck, Niels, Niklas}

\begin{document}
\maketitle

\begin{roles}
\role{F}[Bitre-Mikkel] Fødevareinspektionsnazist
\role{K0}[Niels] Kantinebestyrelsesmedlem
\role{R0}[Andreas] Rus
\role{R1}[Simon] Rus
\role{M}[Peter] Mad
\role{K}[Bette-Mikkel] Keld
\role{X}[Brainfuck] Instruktør
\end{roles}

\begin{props}
\prop{Brugerkøleskab}[]
\end{props}

\begin{sketch}
  \scene{Scenen er kantinen.  Der sidder nogle russer og arbejder ved
    bordene.  Der er komfur/Ken/vask, og (vigtigst!) et
    brugerkøleskab.}
    
  \says{K}[peger på R0] Det har jeg også prøvet!  Har I også prøvet det?!
  
  \scene{TODO: K laver mere biddende komik}

\scene{F kommer ind og går lidt omkring.}

\says{F} Hvem er ansvarlig for dette spisested?!

\says{K}[peger på K0] Det var ham!

\scene{K flygter.}

\says{K0}[kommer hen] Øh, jeg er medlem af kantinebestyrelsen, men...

\says{F} Jeg er fødevareinspektør Preben!

\says{K0}[giver forsigtigt hånd] ...det hedder jeg også.

\says{F} Jeg er her for at inspicere præmisserne!  Vis mig rundt!

\says{K0} Det er jo en selvbetjeningskantine, vi har jo ikke-

\says{F}[giver K0 en lussing] JEG ER OVERFØDEVAREINSPEKTØR PREBEN, OG
JEG VIL INSPICERE PRÆMISSERNE!

\says{K0} Av!  Okay!   Her er komfurene...

\says{F}[skriver på sit ark] Jaaaa....

\says{K0} Og her er et bord... med en rus.

\says{F}[skriver mere og tjekker russen for støv] Hmmm... hvor længe
har den stået ude.

\says{K0} Øhh...

\says{F}[peger på brugerkøleskabet] Og hvad er det?!

\says{K0} Det er bare brugerkøleskabene, de er ikke kantinens ansvar...

\says{F}[går mod dem] Dem vil jeg se nærmere på!

\says{K0}[skærmer for køleskabene med armene] Nejnej, det behøver vi
ikke!  Det er øh... en serverfarm!

\says{F} Ha!  Du vil vel også påstå at Ken er en opvaskemaskine!

\says{K0} Ah, okay, godt ord igen...

\scene{F åbner køleskabet mens K0 står og trisser.  F står og kigger
  lamslået meget længe.}

\says{F}[tager en bæ ud] HVAD ER DET?

\says{K0} Det ligner godt nok en... lort.

\says{F} Hvorfor er der afføring i køleskabet?!

\says{K0} Ja, det er også for galt!  Der er jo hverken navn og dato
på!  Den smider jeg ud med det vuns! \act{Tager bæen og smider den
  ud.}

\says{F}[skribler lidt og kigger mere] Her er en kasse... med seddel på...

\says{K0} Det var da godt!

\says{F} "`Rester fra rustur 2008"'?

\says{K0} Jep, navn og dato, ind med den igen.  \act{Tager kassen og sætter den ind.}

\says{F}[tager en ost ud] Den her ost er jo næsten levende.  \act{Han
  taber osten, som piler af sted over gulvet og ud af scenen.}

\says{K0}[kigger efter osten] Den havde jeg sgu glædet mig til...

\scene{F skriver febrilsk og tager derefter fat i en lang rulle
  medisterpølse som han trækker ud.  Til sidst kommer M med ud, som
  holder fast i den anden ende.}

\says{M} Har du set mit løsen?

\says{F}[chokeret] Hvad i alverden!

\says{R1}[pludseligt] Hov, det er min!  \act{R1 kommer over og tager M med hen over til bordet.}

\says{K0}[til R1] Husk at skrive navn og dato på!

\says{R1} Jajaja...

\scene{F kigger over på hjørneskabet.}

\says{F} Og hvad er det så!?

\says{K0} Det er hjørneskabet.

\scene{Hjørneskabet begynder at synge og F forlader scenen i protest.}

\end{sketch}
\end{document}
