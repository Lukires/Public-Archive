\documentclass[a4paper,11pt]{article}

\usepackage{revy}
\usepackage[utf8]{inputenc}
\usepackage[T1]{fontenc}
\usepackage[danish]{babel}

\revyname{DIKUrevy}
\revyyear{2016}
\version{1.2}
\eta{$2.65$ minutter}
\status{Vi er ikke bombesikre på den}

\title{App til Syrien}
\author{Erik, Brainfuck \& René}

\begin{document}
\maketitle

\begin{roles}
\role{E}[Sebbe] Eksaminator
\role{C}[Ejnar] Censor
\role{R1}[Arinbjörn] Rus
\role{R2}[Mads] Rus
\role{R3}[Kasper] Rus
\role{X}[Brainfuck] Instruktør
\end{roles}

\begin{sketch}

\scene{E og C sidder ved et bord og de tre russer kommer ind}

\says{C} Velkommen til. Her i jeres Softwareudviklingsprojekt, der har i fået
lavet en \ldots

\scene{C løfter rapporten op fra bordet og virker tydeligvis forvirret}

\says{C} \ldots en app til Syrien?

\says{E}[meget entusiastisk] Og det vil vi gerne høre meget mere om.

\scene{R1 bruger fjernbetjening til at starte slides}

\says{R1} Vores projekt startede som et tankeeksperiment.

\says{R2} Ja, vi ville ikke sidde og kode generisk kode, når vi kunne redde
andre menneskers liv.

\says{R3} Så vi tænkte, at vi ville lave en genbrugelig app, der kunne bruges i humanitærekriser, som den vi har nu.

\scene{Slide skiftes og der står med store typer Syrien med et paddehattesky}

\says{R1} Bomber er jo ikke farlige, så længe du ikke er, hvor de sprænger.

\says{R2} Derfor har vi udvilket en cloud-baseret app, så terrorgrupper kan
fortælle, hvor og hvornår de vil sprænge en bombe.

\says{R3} Så ved hjælp af geolokation vil vi sende notifikationer omkring
bombninger i brugerens nærområde.

\says{R1} På den måde kan brugeren finde sin optimale position, så han ikke
bliver ramt, når bomberne sprænger.

\says{R2} For bomber er jo ikke farlige, når du ikke er, hvor de sprænger.

\says{R3} Men hvis nu brugeren ikke når at komme væk. Så opdaterer vi brugerens socialemedier.

\scene{Slide skiftets til et slide, hvor der vises facebook besked med "`Ahmed er død :\'("' med både likes og ked af det reaktioner, Twitter "`Jeg døede \#killedinsyria"', IRC chat viser "`Ahmed: Client disconnected"'.}

\says{R1} Det vil også være muligt at tilkøb custom beskeder via in-app purchases.

\says{E} Jamen, det lyder jo alt sammen meget godt.

\says{C} Har I fået lavet nogle brugertest?

\says{R1} Ja, det har vi!

\says{R2} Via ISIS's twitterkanal fik vi sendt nogle builds til forskellige terrorister.

\says{R3} Men terroristerne havde problemer med at bruge app'en, fordi grænsefladen var på dansk.

\says{R1} Ja, men russerne havde ikke samme problem.

\says{C}[forvirret] Russerne kunne forstå dansk?

\says{R2} Vi havde flere vellykkede brugertests i kantinen.

\says{R3} Nu er det bare en skam at russerne har trukket sig ud.  Både franskmændene og amerikanerne påstår at de ikke kan dansk.

\says{C} Har I så tænkt oversætte app'en?

\says{R1} Vi arbejder allerede på at udvide den til alle de skandinaviske sprog.

\says{R2} Det skal nok udvide målgruppen.

\says{E} Meget fornemt.

\says{C}[lettere fortvivlet] Øh... \act{bladrer i sine papirer, kigger på E og så på R*} kan vi få en demonstration af app'en?

\says{R3} Ja, vi har faktisk fået fysikerne til at hjælpe os med en demonstration.

\scene{Man hører en bombe falde.}

\scene{App'en bipper.  R1 viser telefonen til C.}

\says{R1} Se!  Nu siger app'en at vi skal flytte os.

\scene{Bombe falder og scenen springer i luften}

\scene{IRC chat viser "`DIKUrevy: Client disconnected"'.}

\end{sketch}
\end{document}
