\documentclass[a4paper,11pt]{article}

\usepackage{revy}
\usepackage[utf8]{inputenc}
\usepackage[T1]{fontenc}
\usepackage[danish]{babel}


\revyname{DIKUrevy}
\revyyear{2016}
% HUSK AT OPDATERE VERSIONSNUMMER
\version{1.0}
\eta{$0.3$ minutter}
\status{Ledigt navn søges}

\title{Døbefont}
\author{Bette-Mikkel}

\begin{document}
\maketitle

\begin{roles}
\role{P}[Torben] Præst
\role{M}[Maya] Mor
\role{F}[Rune] Far
\role{X}[Brainfuck] Instruktør
\end{roles}

\begin{props}
    \prop{Døbefont}[]
    \prop{Baby-dukke}[]
\end{props}


\begin{sketch}

\says{P} Velkommen M og F her til kirken.

\says{M} Hej

\says{P} Vil I opdrage jeres barn i den kristne tro?

\says{M} Ja.

\says{P} Og vil i undervise jeres barn i den danske, almindelige folkekirke?

\says{M} Ja.

\scene{F tager spædbarnet i sine arme, og går hen til døbefonten}

\says{P} Er barnet hjemmedøbt?

\says{M} Nej.

\says{P} Hvad er barnets navn?

\says{M} Torben Ægidius Mogensen.

\scene{P kigger forvirret på sig selv.}

\says{P} Beklager, det navn er allerede i brug.

\says{F} Pis.

\scene{Tæppe}

\end{sketch}
\end{document}
