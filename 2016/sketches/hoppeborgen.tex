\documentclass[a4paper,11pt]{article}

\usepackage{revy}
\usepackage[utf8]{inputenc}
\usepackage[T1]{fontenc}
\usepackage[danish]{babel}


\revyname{DIKUrevy}
\revyyear{2016}
\version{0.9}
\eta{$2.75$ minutter}
\status{Ikke færdig}

\title{Hoppeborgen}
\author{Brainfuck, Nana, Bette-Mikkel, Simon, Niels}

\begin{document}
\maketitle

\begin{roles}
\role{MF}[Amanda] Mette Frederiksen
\role{MØ}[Peter] Morten Østergaard
\role{UE}[Arinbjörn] Uffe Elbæk
\role{PS}[Signe] Pernille Skipper
\role{J}[Mia] Johanne Schmidt-Nielsen
\role{L}[Rune] Lars Løkke Rasmuseen
\role{S}[Sebbe] Søren Pape Poulsen
\role{T}[Andreas] Tulle
\role{A}[Bitre-Mikkel] Anders Samuelsen
\role{EK}[Maya] Eva Kjer
\role{EL}[Kasper] Esben Lunde Larsen
\role{UT}[Jenny] Ulla Tørnæs
\role{DR}[Simon] Dronning Margrethe
\role{X}[Brainfuck] Instruktør
\end{roles}

\begin{props}
\prop{Bamse}[]
\prop{Traktorlegetøj}[]
\prop{Puder}[]
\prop{Bog}[]
\end{props}

\begin{sketch}

\scene{TODO: Skal skrives tighter.}

\scene{Frokosten er slut.  L, S, T, A og EK (blå stue) kommer først tilbage fra frokost.  L tager bamsen, mens T tager de fleste puder.  De sidder alle sammen på puder.  E begynder at lege med en traktor.}

\scene{MF, MØ, UE, PD, PS og J (rød stue) kommer ind og sætter sig i den anden side af scenen og surmuler over at de ikke har nogle puder.}

\scene{S gnaver på et bolche.  Han kigger i sin pose, den er tom.}

\says{S} Hvad!  Der er ikke flere bolcher tilbage!

\says{L} Du spiser dem også så hurtigt, Søren.

\says{S} Men Eva sagde at der var 10 bolcher!  Og der var altså kun 8.

\says{L} Det er bare surt for dig så.

\says{S} Så synes jeg ikke at Eva må lege med traktoren længere.

\says{EK} Der var altså 10 bolcher, Søren!  Traktoren er min!

\says{L} Eva har traktoren!  Jeg har bamsen! \act{gestikulerer med bamsen}

\says{S} NEJ!  Der var kun otte bolcher!  Eva må IKKE have traktoren.

\says{L} Det er ikke din traktor, Søren.

\says{S} Jeg vil ikke have traktoren, Lars!  Eva har ikke gjort sig fortjent til den.

\scene{Pludselig ligger de mærke til det over i rød stue.}

\says{MF} Søren har ret, Lars!  Eva lyver om bolcherne!

\says{MØ} Hun burde slet ikke have lov til at lege med den fine traktor!

\says{T} Lige meget med dem, Lars.  \act{rækker tunge}  Det er dig der har bamsen.  Søren og rød stue er bare dumme.

\says{L} Tak, Tulle.

\says{UE} Eva er slet ikke stor nok til at lege med en traktor!  Se hvordan hun bare kører blomsterne over.

\says{PS} Eva bør give traktoren\ldots

\says{J} \ldots til nogen som ikke\ldots

\says{PS+J} \ldots kører blomster over!

\says{A} STILLE!  ALLE VED AT RØD STUE BARE ER DUMME!  TRAKTORER ER BEDST.

\says{S} Eva må ikke lyve, Lars!  Enten giver Eva mig de to bolcher eller også så tager du traktoren fra hende!

\says{L} Nej!

\says{S} Jo!

\says{L} Nejhaj!

\says{S} Jo jo jo!

\says{L} Neeeeeej!

\says{S} Johov!

\says{L} Ellers giver jeg puderne til rød stue!

\says{S} Det tør du ikke!

\says{L} Jo jeg gør!

\scene{DR kommer ind.  Bandet spiller ``Kong Christian stod ved højen mast''.  På OverTeX står der ``Alle bedes rejse sig for Hendes Majestæt Dronning Margrethe II''.}

\says{DR} Børnlille, nu må I til at lege en mere stille leg end politiker, som for eksempel jorden er giftig.

\says{S} Det bliver den først hvis Eva får sin vilje.

\scene{EK giver op, rejser sig op og giver traktoren til S.}

\says{EK} OK, så kan du få din dumme traktor.

\scene{EK surmuler i hjørnet.  DR går ud.}

\says{S} Jeg vil ikke have den. \act{giver den til L} Giv den til en anden.

\says{L} Hvad var alt det påstyr så for?

\says{S} Man kan kun have traktor hvis man opfører sig pænt og ordentligt.  Det har min far selv sagt.

\says{L} Esben! \act{EL kommer ind.} Kom herind og få traktoren.

\says{EL}[har en bog i hånden] Hurra! \act{EL smider bogen fra sig og kaster sig over traktoren} Wroom wroom.  Drøn, ihjevn, løfte løfte.

\says{S} Hvem skal så have bogen?

\says{L} Det skal Ulla.

\scene{UT kommer ind og tager bogen og river bogen i stykker mens hun går ud.}

\says{S} Men hun er jo dum.

\says{L} BAMSEN HAR TALT.

\scene{Lys ned.}

\end{sketch}
\end{document}
