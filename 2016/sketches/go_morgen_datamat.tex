\documentclass[a4paper,11pt]{article}

\usepackage{revy}
\usepackage[utf8]{inputenc}
\usepackage[T1]{fontenc}
\usepackage[danish]{babel}


\revyname{DIKUrevy}
\revyyear{2016}
% HUSK AT OPDATERE VERSIONSNUMMER
\version{1.2}
\eta{$3.5$ minutter}
\status{Færdig}

\title{Go' morgen datamat}
\author{Niels, Simon, Pilgård, Troels, Maya, Phillip}

\begin{document}
\maketitle

\begin{roles}
\role{V1}[Kasper] Vært
\role{V2}[Maya] Vært
\role{K}[Niels] Bumset kok med bare tæer
\role{S}[Kim] Udvikler hos Nets
\role{X}[Brainfuck] Instruktør
\end{roles}

\begin{props}
\prop{Ting til toast}[Klaes?]
\end{props}


\begin{sketch}

\scene{Introvideo på OverTeX. \textbf{Gomorgen-Danmark-agtig jingle}.}

\says{V2} Godmorgen og velkommen til Go' morgen Datamat.

\says{V1} Godmorgen Maya.

\says{V2} Godmorgen Kasper.

\says{V1} Og godmorgen til jer, kære publikum.

\says{V2} I dag har vi flyttet studiet ud til Store UP1 i Universitetsparken.

\scene{V1 og V2 sætter sig ved deres mappedatamater.}

\says{V1} Og er der sket noget nyt på reddit her til morgen, Maya?

\says{V2} Der er et billede af en kat.

\scene{Pause.  De stirrer på katten og på hinanden.}

\says{V1} Det her var altså sjovere med morgenaviser.

\says{V2} Der er nogen på /r/food der klager over hvor dyrt færdiglavet mad er.

\says{V1} Sjovt du siger det!  Vi har nemlig en gæst med i studiet der vil vise os hvordan man laver maden selv!

\scene{V1 og V2 kigger mod sidetæppet, men K kommer op ad bordet de sidder ved.}

\says{V2} Hvordan du så maden selv?

\says{K} Ja, man kan jo købe en færdigpakket toast, men jeg har lært
at lave min egen.

\says{V2} Så har du vel også brugt kvalitetsråvarer?

\says{K} Tjo... Netto havde lukket, så jeg har købt toastbrødet i FøTeX.

\says{V1}[sarkastisk] Nøj.  \act{normal} Men ud over brød, hvad er der så i toasten?

\says{K} Øh...

\scene{K tager ingredienserne op ad lommen.}

\says{K} ... der er...

\scene{K tager osten og læser på pakken.}

\says{K} ... der er... ost i.

\scene{K lægger osten igen.}

\says{V2} Er det roquefort?

\says{K} Det ved jeg ikke, det var bare noget jeg fandt i hjørneskabet.

\scene{V1 tager pakken op og læser højt fra ingrediens-listen.}

\says{V1} "90\% kartoffelmel". Avant garde!

\scene{K river osten ud af hånden på V1.}

\says{K} Jeg så den først! Find din egen!

\says{V2} Er du så blevet inspireret af ny nordisk mad?

\says{K} Jo, altså, jeg boede en uge i nordfløjen da jeg lærte at lave toast.

\says{V2} Hvordan tilbereder man den så?

\says{K} Man bruger en toastmaskine... hvor er toastmaskinen?!

\says{V1} Øøøh...

\says{K} Ligemeget, jeg har snydt lidt hjemmefra, og har allerede lavet den.

\says{V1} Må vi se resultatet?

\scene{Pause.}

\says{K} Jeg har snydt lidt mere hjemmefra og allerede spist den.

\says{V1}[lettere skuffet, prøver at redde situationen] Nå,
men... hvordan smagte den så?

\says{K} Den smagte godt.

\says{V1} Jamen tak.

\scene{V1 skuffer K ud af scenen, som er forvirret, imens fortsætter V2.}

\says{V2} Vi går videre til vejret.

\scene{Spot i den ene side af scenen.}

\says{V1} Et højtryk på Version2's debat har medført pletvis nedetid i skyen,
rapporterer Cloudflare.

\says{V1} Og femdøgnsprognosen for Eduroam ser således ud:

\scene{Man ser 5 flade søjler for de næste dage.}

\says{V1} Men hvis vi ser på syvdøgnsprognosen ser det straks anderledes ud:

\scene{Man ser 2 ekstra søjler. Den 6. søjle går lidt op, og den 7. er
  flad igen.}

\says{V2} Science IT har nemlig varslet oppetid på torsdag, så husk at tage jeres
forholdsregler og pas på derude på webben.

\says{V2} Og det var vejret.

\says{V1} Hæsblæsende.  Sjovt jeg siger det!  Det er ligesom hos Nets.

\says{V1} Vores næste gæst er Sigurd, en programmør der arbejder hos Nets. Sigurd vil demonstrere hvordan den daglige drift og udvikling af NemID forløber.
Og dav, Sigurd.

\says{S} ACK! ...hvis du forstår sådan en lille en.

\scene{V1 og V2 kigger forvirrede på hinanden.}

\says{V2} Du har taget din mappedatamat med.

\says{S} Ja, altså...

\scene{S stiller sin mappedatamat på køkkenbordet. Skærmbillede med
  Eclipse/UML-diagrammer dukker op på OverTeX.}

\scene{S begynder at hamre i tastaturet, mens han fortæller.}

\says{S} Ja, det er jo ikke hvem som helst der kan varetage vort lands
IT-insfrastruktur...

\says{V2} Nej, vi er jo allesammen afhængige af jeres arbejde.

\says{S} ... så det er vigtigt at bruge alle udviklings-ressourcer...

\scene{S hopper med numsen på tastaturet.}

\says{V1} Må jeg også prøve?

\scene{V1 går hen til datamaten og slår moderat på tastaturet.}

\says{S} Ej, man kan godt se at du ikke har Microsoft-certifikat.

\scene{S skubber V1 væk, og begynder at hamre i tastaturet.}

\says{S} Se! Du skal FØLE det!

\scene{S banker hovedet ned i tastaturet 2-3 gange.}

\scene{S klapper datamaten sammen, og smider den ned i bukserne. Han
  stiller sig med armene over kors, mens han stirrer på værterne.}

\scene{V1 og V2 kigger forvirret på hinanden. V1 eskorterer S af
  scenen.}

\says{V2} Ja, tak til Sigurd fra Nets, som har givet os et sjældent indblik i
udviklingen af NemID.

\says{V1} Hvad har vi så på programmet nu?

\says{V2} Nu skal vi snakke med Jyrki om...

\says{V1} Nej.  For vi er desværre løbet tør for tid.  Men vi slutter af med dagens børnetegning. Morten fra Gentofte har kodet fakultetsfunktionen - ikke
specielt godt.

\scene{Tegning bliver vist på Over\TeX. Dårlig kode, tegnet med
  farveblyant.}

\says{V1} Men vi takker for tegningen. Og der er 10 ECTS på vej til dig.

\says{V2} Og Morten har også fået lov at ønske en sang.

\end{sketch}
\end{document}
