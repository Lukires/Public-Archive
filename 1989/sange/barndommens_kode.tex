\documentclass[a4paper,11pt]{article}

\usepackage{revy}
\usepackage[utf8]{inputenc}
\usepackage[T1]{fontenc}
\usepackage[danish]{babel}


\revyname{DIKUrevy}
\revyyear{1989}
\version{1.0}
\eta{$n$ minutter}
\status{Færdig}

\title{Barndommens kode}
\author{KM}
\melody{Anne Linnet: ``Barndommens Gade''}

\begin{document}
\maketitle

\begin{roles}
  \role{S}[] Sanger
  \role{K}[] Koder
\end{roles}

\begin{song}
  Jeg er din barndoms kode
  jeg er dit første sprog
  Jeg er din måleenhed
  for alt, hvad du siden tog.
  Jeg er din lærers omhu
  og din instruktors mod.
  Jeg er de første timers
  spaghettiprogrammers rod.

  Det er mig, der har lært dig om hægter,
  jeg lærte dig labels og hop.
  Jeg gav dig de første problemer,
  som du aldrig fik til at gå op.
  Jeg gik i uend'lig løkke,
  da du glemte en ordre et sted
  og så viderede jeg med nul,
  da du talte indexet ned.

  Jeg er din barndoms kode,
  jeg er dit første sprog.
  Jeg gav dig de rødranded' øjne,
  på dem kan man kende dig nu.
  Møder du en med samme blik
  kan du se, han er hacker som du.

  Møder du nye systemer
  et velstruktureret program
  skal du længes efter de første
  goto-programmers slam.
  Og når du skifter stilling
  tror du vi er færdige, men...
  jeg er din barndoms koe,
  du finder mig altid igen!


  Jeg er din barndoms kode,
  jeg er dit første sprog.
  Jeg gav dig de rødranded' øjne,
  på dem kan man kende dig nu.
  Møder du en med samme blik
  kan du se, han er hacker som du.

\end{song}

\end{document}

