\documentclass[a4paper,11pt]{article}

\usepackage{revy}
\usepackage[utf8]{inputenc}
\usepackage[T1]{fontenc}
\usepackage[danish]{babel}


\revyname{DIKUrevy}
\revyyear{1989}
% HUSK AT OPDATERE VERSIONSNUMMER
\version{1.0}
\eta{$n$ minutter}
\status{Færdig}

\title{Den svenske Koch}
\author{?}

\begin{document}
\maketitle

\begin{roles}
  \role{K}[] Gregers Koch
\end{roles}

\begin{props}
  \prop{1 stor gryde}
  \prop{1 stor grydeske}
  \prop{1 kokkehue}
  \prop{Bøger (Gries og kursusbøger)}
\end{props}

\begin{sketch}

  \scene{Et bord står forrest pås cenen med en stor gryde.  I gryden
    er bøgerne (ude af syne).}

  \scene{Ind kommer Gregers Koch (iført kokkehue og medbringende
    grydeske).  Han går hen til gryden og rører rundt i den med
    fagter og lyde som den svenske kok fra Muppet Show (gerne den
    lille sang).

    Han kigger i gryden og tager Gries bogen op mens han ryster på
    hovedet og udstøder misbilligende lyde, hvoraf kun ordet
    "`Gries"' kan genkendes.  Han kaster den over skulderen med en
    stor armbevægelse.

    Det samme gentages med kurusbøgerne.

    Han rører mere rundt og kigger ned i gryden.  Han ser overrasket
    ud og kigger nærmere efter, hvorefter han vender bunden i vejret
    på den og ryster den, men ingenting kommer ud.

    Han trækker på skuldrene og går.}

\end{sketch}
\end{document}
