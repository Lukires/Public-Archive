\documentclass[a4paper,11pt]{article}

\usepackage{revy}
\usepackage[utf8]{inputenc}
\usepackage[T1]{fontenc}
\usepackage[danish]{babel}


\revyname{DIKUrevy}
\revyyear{1989}
% HUSK AT OPDATERE VERSIONSNUMMER
\version{1.0}
\eta{$n$ minutter}
\status{Færdig}

\title{Sorteringssketch (quicksort)}
\author{?}

\begin{document}
\maketitle

\begin{roles}
\role{A}[]
\role{B}[]
\role{C}[]
\role{D}[]
\role{E}[]
\end{roles}

\begin{props}
\prop{4 sæt ens tøj}
\prop{Et langt bord}
\prop{8 kasser med tal fra 1-8}
\end{props}


\begin{sketch}

\scene{Bordet står midt på scenen.

  En mand kommer ind med de 8 kasser i en usorteret stak og sætter dem
  på bordet.  Han vinker ind i og stiller sig i et hjørne.

  Ind kommer 1 mand i nævnte type tøj.  Han går hen til bordet og
  deler stakken med kasser op i to stakke: En med tallene fra 1-4 og
  en med 5-8, dog uden at sortere yderligere.  Han går ud igen.

  Ind kommer 2 mænd (ens klædt).  De går hen til hver sin stak kasser
  og deler dem op: 1-2/3-4 og 5-6/7-8.  De går ud igen.

  Ind kommer 4 mænd (ens) De deler hver en stak op så kasserne nu
  ligger i rækkefølge på bordet (med lidt mellemrum).  De vender en
  gang på scenen og går tilbage til kasserne som de sætter oven på
  hinanden i par: 1 på 2, 3 på 4, etc.  De går ud.

  Ind kommer de 2 mænd fra før og sætter hver to stakke oven på
  hinanden: 1-4 og 5-8.  De går ud igen.

  Ind kommer mand nr. 1 og sætter de to stakke oven på hinanden.
  Kasserne er nu sorteret.  Han går ud.

  Manden som kom med kasserne går nu frem fra hjørnet og tager
  kasserne med ud.}



\end{sketch}
\end{document}
