\documentclass[a4paper,11pt]{article}

\usepackage{revy}
\usepackage[utf8]{inputenc}
\usepackage[T1]{fontenc}
\usepackage[danish]{babel}


\revyname{DIKUrevy}
\revyyear{1989}
% HUSK AT OPDATERE VERSIONSNUMMER
\version{1.0}
\eta{$n$ minutter}
\status{Færdig}

\title{Rapport-stumfilm 2}
\author{en forfatter}

\begin{document}
\maketitle

\begin{roles}
\role{I}[] Instruktor (stumfilmspianist)
\role{S1}[] Studerende
\role{S2}[] Studerende
\role{Sl1}[] Slave 1
\role{Sl2}[] Slave 2
\end{roles}

\begin{sketch}

  \scene{Der bliver en meget kort pause før de studerende skal på
    scenen igen.  Her rejser pianisten sig og synder: "`Jeg så
    julemanden kysse mor..."'}

\scene{Dias: "`En dobbelt så tilfældig dag under K4."'}

\scene{Ingen musik}

\scene{De studerende kommer ind igen med en 14cm tyk stak udskrifter,
  som lægges på den ene stol.  Ser glade ud.  De råber (lystig musik
  starter): "`Jihaa!"'

  Mod slutningen af rundgangen løber de to studerende bag hinanden.
  Idet musikken stopper, fastfryses den forreste, og den bagerste
  ramler op bag i.

STILHED

De går henimod udskrifterne, kigger på programmet, ser sørgende ud.

Sørgemusik.

Dias: "`En instruktor"'

De studerende henvender sig til den klaverspillende instruktor.
Musikken stopper, pianisten gestikulerer vildt og voldsomt og sætter
til sidste i en meget larmende dim-akkord.  De studerende tager
hinanden over skuldrene og går ud.

Pianisten fyrer et par dumme datalog-jokes af, f.eks. tyrestationen og
SE og Hør-smalltalk.

Dias: "`En specielt udvalgt tilfældig dag under K4"'

De to slaver kommer ind, slæbende på en stor kass emed udskrifter.  De
studerende kommer piskende ind bagefter.  Den ene medbringende en stor
gryde med påskriften KAFFE og to kopper.

Pianisten antyder negerslavestemning.

Musikken stopper.  Den ene studerende hælder kaffe op til den anden.
Der tages en udskrift op fra kassen.  Kigger på papiret.

Blid sørgemusik.  Bliver gradvist mere markeret, tragisk, langtrukken.

De studerende rejser sig op og går i rundkreds.  Der hives hår af i
totter.  De sætter sig ned som grædekoner, gestikulerer gråd.  Efter
et stykke tid rejser de sig langsomt op, griber hver den ene ende af
en endeløs papirbane og bevæger sig snigende hen mod pianisten for at
omklamre (kvæle) ham.  Pianisten stopper midt i en takt, rejser sig
op, bryder monstrativt papiret, affyrer en pistol mod de to
studerende, der falder om.  Flår jakken af, går hen og træder
demonstrativt på den ene idet der gøres front mod publikum.  Råber i
mikrofon: "`Jeg fik dem de små sataner!!!"'

Nummeret er slut.  Orkestret tager over og spiller.

}

\end{sketch}
\end{document}
