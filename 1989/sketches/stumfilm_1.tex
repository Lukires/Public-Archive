\documentclass[a4paper,11pt]{article}

\usepackage{revy}
\usepackage[utf8]{inputenc}
\usepackage[T1]{fontenc}
\usepackage[danish]{babel}


\revyname{DIKUrevy}
\revyyear{1989}
% HUSK AT OPDATERE VERSIONSNUMMER
\version{1.0}
\eta{$n$ minutter}
\status{Færdig}

\title{Rapport-stumfilm 1}
\author{?}

\begin{document}
\maketitle

\begin{roles}
\role{I}[] Instruktor (stumfilmspianist)
\role{S1}[] Studerende
\role{S2}[] Studerende
\role{Sl1}[] Slave 1
\role{Sl2}[] Slave 2
\end{roles}

\begin{sketch}

\scene{Dias: "`En tilfældig dag under K4"'}

\scene{Pianisten spiller sørgemusik}

\scene{To studerende kommer ind med en tynd udskrift.  De sætter sig
  på stolene.  De gestikulerer, at de absolut ikke ved hvad der er
  galt i udskriften.}

\scene{Kort pause i musikken}

\scene{Den ene rejser sig op på stolen, kigger på uret til højre for
  scenen i auditoriet: "`Den er tyve minutter i otte!"'}



\scene{Mere sørgemusik, derefter kort pause.  Den ene rejser sig,
  peger stift på papiret: "`Det er jo en NOP-instruktion det der!"'}

\scene{Den anden rejser sig op, og der danses rundt til lystig
  stumfilmsmusik.  Idet musikken stopper, stopper de studerende
  (freeze), og lunter ud efter hinanden som "`tog"'.}

\end{sketch}
\end{document}
