\documentclass[a4paper,11pt]{article}

\usepackage{revy}
\usepackage[utf8]{inputenc}
\usepackage[T1]{fontenc}
\usepackage[danish]{babel}


\revyname{DIKUrevy}
\revyyear{1989}
% HUSK AT OPDATERE VERSIONSNUMMER
\version{1.0}
\eta{$n$ minutter}
\status{Færdig}

\title{Bachelor}
\author{?}

\begin{document}
\maketitle

\begin{roles}
  \role{JC}[Ludwig] Jens Clausen
  \role{MW}[Torben] McWhopper
  \role{D}[Niels Bo] Dennis
\end{roles}

\begin{sketch}

  \scene{Jens Clausen kommer springende kommer ind på scenen og lander
    i en stor pude (mikrofon foran puden)}

  \says{JC} Jah, heer Jens Clausen og har Datalogisk Institut.  Jah
  har et knald godt tilbu te dej - 1 styk bachelor-grad, kun 3 år
  normeret.

  \scene{JC går snakkende hen mod tavlen.}

  \says{JC} Men hvad består denne bachelor-uddanneles egentlig af?
  Slår man op i ordbogen vil man se at bachelor betyder ungkarl, og vi
  har her på DIKU stykket en passende uddannelse sammen.

  \scene{Planche med uddannelsens hovedpunkter.

    \begin{description}
    \item[1. semester] Introduktion til dåseåbnerteknik
    \item[2. semester] Personlig hygiejne
    \item[3. semester] Ekstern hygiejne
    \item[4. semester] Huset er udgået for...
    \item[5. semester] Reel optræden med ureel hensigt
    \item[6. semester] Udvidet dåseåbnerteknik
    \end{description}}

  \says{JC}[add 1] Vi vil på dette kursus gennemgå dette, for
  en ungkarl, så uundværlige hjælpemiddel.  Historisk set helt
  tilbage, fra da vore kære forfædre - vikingerne - kløvede hinandens
  hjerneskaller, for at skaffe sig til dagen og vejen, og helt frem
  til i dag, hvor flere mikroprocessor-styrede varianter er kommet på
  markedet.

  Elementær ernæringslære er også en vigtig del af dette kursus, vi
  vil her lægge ud med introduktion til DIKU-skiver, og der vil blive
  oprettet øvehold i kamel-taktikken.  Kamel-taktikken er udviklet af
  den berønte ungkarlolog professor McWhopper, fra Kentucky Fried
  University, og går i sin enkelthed ud på ikke at spise i et par dage
  hvorefter man ta'r ud til mooor.  Familie og venner kan med passende
  afveksling også anvendes.

  \says{JC}[add 2] Vi vil på dette semester sætte os grundigt ind i
  først og fremmest underbuksevending og sokke-genbrug.  Vores egen
  internationalt anerkendte underbuksevender - PJO - vil forestå dette
  kursus.  Men der vil også blive plads til gæsteforelæsninger,
  bl.a. ved ungkarlolog professor McWhopper, fra Kentucky Fried
  University.  Han har jo løst en af unkarlogiens gotiske knuder: Det
  kvadrofoniske genbrug af sokker: En normal sok kan kun genbruges én
  gang, ved en simpel vending \act{demo} ...sådan.  Men med en saks
  kan man nu fordoble en soks anvendelighed \act{demo}.  Man kan dog
  også vente et stykke tid, så komme den øgede anvendelighed helt af
  sig selv.  Der kommer dog en dag hvor alt er genbrugt maksimalt.  På
  denne triste dag udvælger man sit mest lugtende og hullede tøj,
  tager en stor sæk over nakken og ta'r ud til mooor.  Jeg håber I kan
  linjen i den pædagogisk sammenhængende første del, idet der her er
  klare paralleller tli første semester kameltaktik.

  \says{JC}[add 3] Dette semester vil omhandle lejlighedens eller
  kollegieværelsets udnyttelse.  Kurset vil have et stærkt matematisk
  præg, idet vi bl.a. skal beskægtige os med vasketøjskomprimering og
  opvaskestabling efter $n\log n$ algoritmen.  Elementær oprydnings-
  og rengørings-teknik vil dog også indgå i kurset.  Her har toilettet
  en central placering.  Og en af dem der har gjort store innovationer
  indenfor sammenhængen mellem undkarle og toiletter har vi den glæde
  at kunne præsentere her i dag - ungkarlologen professor McWhopper
  fra Kentucky Fried University... \act{MW træder ind på scenen}
  Welcome professor, you have made some important work about
  waterclosets, but what is the big difference between a watercloset
  and a bachelor?

  \says{MW} Well, it's very complicated, you see.  But in general a
  watercloset is WC.  And a bachelor is single, you see.

  \says{JC} Thank you professor \act{MW går ud} - og vi går videre til
  næste semester.

  \says{JC}[add 4] 4. semester vil omhandle en af de mest centrale
  problemstillinger indenfor ungkarlogien.  Jeg skal her blot nævne
  hvilke muligheder, der er når man er udgået for kaffe-filtre: Ideelt
  er det naturligvis hvis man har et brugt kaffefilter, enten
  fastgroet i kaffetragten, eller skaffer sig et med
  dyk-i-skraldespanden-metoden.  Men også toilet- og køkken-rulle kan
  anvendes.  Er man også udgået for det, er en sok som har været brugt
  2 eller 4 gange - alt efter hvilken sokke-vendingsteknik der
  anvendes - glimrende som kaffefilter.


  \says{JC}[add 5] Vi kommer her til en meget interessant del af
  studieforløbet... \act{blink med øjet og satanisk smil}.  Vi har dog
  endnu ikke fastlagt pensum, men Emma Gad vil ikke blive anvendt.

  \says{JC}[add 6] Uddannelsen sluttes af med et praktisk orienteret
  fag, hvor den studerende skal demonstrere sine tillærte færdigheder,
  og skaffe sig mad uden mooors hjælp.  Kantine-besøg vil naturligvis
  indgå som et centralt element.  HCØ's salat-bord er sikkert rimeligt
  kendt for de fleste, men desværre er en tallerken salat ikke nok til
  at mætte en rigtig ungkarl.  Men for nylig har ungkarlolog professor
  McWhopper fra Kentucky Fried University udgivet en artikel, som
  giver en løsning på dette problem:

  Ingredienserne i hans geniale forsøg er to tallerknener, to stykker
  A4-papir og en hæftemaskine \act{demo}.  Man tager de 2 stykker
  papir, slår ring om den ene tallerken, og hæfter papirerne sammen.
  Nu fyldes dette rør helt op med salat, og den anden tallerken bruges
  nu som kompressor \act{tegneserie på tavlen}.  Man har nu for
  medelst 15kr fået en salat-mængde svarende til 3 kinakål.  Teknikken
  kan også anvendes til millionbøf, men her anbefales det at bruge
  overheads i stedet for papir.

  Anden del af hans berømte forsøg omhandler frekadelle-trikket.  Hvor
  man under de tre kinakål har placeret et antal HCØ-fritureboller
  \act{tegning med kulsorte frikadeller under salat}.  Vil man også
  have kage, er teknikken ligeledes anvendelig \act{tegning}.

  Til slut har vi her i studiet inviteret en færdiguddannet bachelor
  \act{en student kommer ind}, velkommen til Dennis.

  \says{D} Tak ska' du ha'.

  \says{Jc} Dennis, du har skrevet speciale i "`unkarlens
  morgenrytme"', hvordan starter du selv dagen?

  \says{D} Joh, jeg starter dagen med at fylde mit tandkrus med
  Nescafé, og tager mig derefter et varmt bad.  Jeg er nu både ren,
  vågen og har frisklavet kaffe.  Derefter går jeg ud i køkkenet,
  finder en skål i dyngen, og mejsler den forstenede Bearnaise-sovs
  ud.  Derefter finder jeg mine yndlingsgryn et eller andetn sted på
  gulvet, og hugger et stykke af.  Jeg mangler nu kun at finde en
  liter mælk mellem mine lektiebøger.  Kan mælken ikke hældes uden
  brug af dejskraber, er det tilrådeligt at lugte til den.  Lugter den
  ikek kan den være ubehagelig at drikke, idet bakterierne nu ikke
  længere bryder sig om den.

  \says{JC} Det lyder jo meget interessant, det lyder virkelig som om
  du har fået noget ud af uddannelsen.  Hvad synes du selv?

  \says{D} Jeg har været yderst tilfreds og finder stoffet lige så
  uundværligt som Gries-bogen, og jeg vil helt sikkert fortsætte som
  bachelor resten af mine dage...


\end{sketch}
\end{document}
