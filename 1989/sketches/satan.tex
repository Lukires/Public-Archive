\documentclass[a4paper,11pt]{article}

\usepackage{revy}
\usepackage[utf8]{inputenc}
\usepackage[T1]{fontenc}
\usepackage[danish]{babel}


\revyname{DIKUrevy}
\revyyear{1989}
% HUSK AT OPDATERE VERSIONSNUMMER
\version{1.0}
\eta{$n$ minutter}
\status{Færdig}

\title{SATAN}
\author{?}

\begin{document}
\maketitle

\begin{roles}
  \role{D}[] Djæven
  \role{S}[] Student
\end{roles}


\begin{sketch}

  \scene{På scene: Djævel i djævlekostume (jakkesæt, evt. mappe,
    slips, horn)}

  \says{D} Godaften!  Jeg er kommet for at hjælpe {\em jer} med at få
  flere {\em dygtige} og {\em velmotiverede} studenter.  Jeg kommer
  direkte fra det StatsAutoriserede Teknologi-AnbefalingsNævn,
  forkortet SATAN blandt venner.  Jeg skal hjælpe jer med at
  optimalisere jeres public-image profile, maximalisere
  team-spirit'en, og justere konkurrenceparametrene efter dagens
  management-trend.  I skal udnytte jeres human ressources bedre, hvis
  I vil decrease cost-benefit rate'n.  Det drejer sig simpelthen om at
  supporte klienternes personal development som power-users.  Lad mig
  visualisere konceptet: \act{slår ud med hånden}

    \scene{Student ind}

    \says{S} Øhh...

    \scene{Djæven farer over til studenten og klasker ham på skulderen,
      trykker ham i håret, klapper ham på hovedet og trækker en stol hen
      til ham.  Om muligt alt sammen samtidigt.  I det følgende sidder
      studenten forskræmt og knuger sin rygsæk, mens djævlen smidsker
      rundt omkring ham.}

    \says{D} Nå, hvad kan vi så gøre for dig?  Vil Pascal-pakken passe
    dig?  Eller vil du hellere se C?  Ind og skrive kold kode... Ah, du
    skal måske hacke hex, hvad?  Ned og mimre mikrokode?  Har vi en rigtig
    kernekarl blandt os?  Kan slet ikke vente med at programmere
    processerne?  Måske vil du bakse baserne?  For du vil vel ikke sysle
    systemering?  Gå og granske graferne?  Her giver vi dig mod på at
    turde tyde Turing.  Eller jeg har måske taget fejl?  Sku' det være
    lidt listig Lisp?  Skimme Scheme?  Og så bagefter konstruere compiler.
    Rigtig gramse med grammatikkerne.  Skal du prøve prologen?  Aha, jeg
    kan se det på dig.  Du skal have et par beskidte beviser... din gamle
    Gries!  Nå, hvad vil du have?

    \scene{Studenten hvisker noget i øret på djævlen.}

    \says{D}[råber] Hvad?  En sammenhængende pædagogisk føstedel?  Kan du
    komme ud, dit svin! \act{smider studenten ud}

    Evt slutning:

    \says{D}[til publikum] Ak ja, ungdommen i dag.  Men der er jo altid et
    råddent æble i kurven.  Jeg tror, vi går hurtigt videre.


  \end{sketch}
\end{document}
