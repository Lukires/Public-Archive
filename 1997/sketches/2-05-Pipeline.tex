\documentclass[a4paper,11pt]{article}
\usepackage{revy}
\usepackage[utf8]{inputenc}
\usepackage[danish]{babel}
\textwidth 160mm
\evensidemargin 0pt
\oddsidemargin 0pt
\title{Pipelinesketch}
\author{Søren Trautner Madsen \& Jesper H.\ Olsen}
\version{1.1}
\revyyear{1997}
\parindent0pt
\parskip 1ex minus 1ex
\flushsingsright
\begin{document}
\maketitle
\newcommand{\ansv}{Ansvarlig}
\newcommand{\kvaj}{Kårre Vaj}
\newcommand{\prof}{Professor}
\begin{sketch}
\begin{roles}
\role{\ansv} en revy-ansvarlig
\role{\kvaj} En speciale-studerende
\role{\prof} (en professor)
\role{Risk-, Ko-, Mikroprofessor} 3 andre professorer.
\end{roles}


\says{\ansv} Grundet lokaleproblemer har det desværre ikke været muligt at
booke Auditorium 1 hele aftenen. Vi bliver derfor nødt til nu at afbrydede
revyen, da vi skal give plads til specialeforsvaret: "Optimalt
informationsflow ved forøgelse af data-båndbredde i
undervisningssituationer." Så vær venlige at opføre jer pænt, når vi nu
giver ordet videre til Kårre Vaj!!!

\scene{Tæppe fra. På scenen står en overhead-projektor og et bord med
  slides. K. Vaj kommer ind på scenen.}

\says{\kvaj} Godaften og mange tak fordi I kunne komme allesammen til mit
foredrag om Optimalt informationsflow ved forøgelse af data-båndbredde i
undervisningssituationer. Jeg vil starte med at skitsere en kendt
problemstilling, og derefter formulere og demonstrere en løsning.

I kender alle situationen! I sidder til en forelæsning, professoren kommer
ind og... ja, se her engang:

\scene{En professor kommer ind, tager den øverste slide, går hen til
  overheaden, kigger mystificeret på sliden, klør sig i håret, siger 'aha',
  vender den om, lægger den på overheaden, mumler et eller andet, tager
  sliden af og lægger den på bordet. Stopper.}

\says{\kvaj} Som I ser her, er der en masse spildtid i fremlæggelsen af det
vigtige stof. Vi ved jo alle, at jo flere slides man kan nå at se, jo bedre er
forelæsningen. Lad os derfor prøve, at speede det hele op, ved at nedbryde 
handlingen i logiske trin. Men først må jeg vist lige have 'Sutte' på bordet.

\scene{Op fra sin taske tager K. Vaj en tiger-sutsko, som han aer or sætter
  på bordet}

\says{\kvaj} Se det første trin kan jo betegnes som hentningstrinnet.

\scene{Professoren går hen til slides'ene, tager den øverste slide, og går
  hen til overheaden.}

\says{\kvaj} Efterfulgt af ... ja, lad os kalde det afkodnings-trinnet.

\scene{Professoren kigger mystificeret på sliden, klør sig i håret, siger
  'aha', vender den om og lægger den på overheaden. Det er den samme slide
  som før}

\says{\kvaj} Nu skal pensum så formidles, så professoren adresserer nu
publikum, det såkaldte adresseringstrin.

\scene{Professoren mumler et eller andet, siger så med klar stemme: 'Hov,
  Den har I da set før}

\says{\kvaj} Nå da. Det viser sig, at professoren her brugte noget, der
allerede lå i publikums hukommelse. Det kan man da kalde for lidt af et
hit! Professoren kan nu straks gå videre til sidste trin:
Tilbagelægningstrinnet, eller put-back trinnet.

\scene{Professoren tager sliden af og lægger den på bordet på den anden
  slide.}

\says{\kvaj} Disse fire trin er i teorien uafhængige af hinanden, så man må
konkludere, at man kan opnå en betragtelig hastighedsforøgelse simpelthen
ved at firdoble professorkraften. Men først må jeg vist lige have 'Rulle'
på bordet.

\scene{\kvaj\ tager en vulgær (???) bamse (????) frem og stiller den ved
  siden af 'Sutte'}

\says{\kvaj} Til min demonstration får jeg brug for nogle hjælpere. Lad mig
sige velkommen til: Hr. Risk-professor... Hr. Mikro-professor... og Hr.
Ko-professor!!!

\scene{Tre professorer kommer ind. Den ene har et Risk-spil under armen.
  Den anden er meget lille (Bobobo!!), og den tredie har horn. Hr. Risk
  stiller sig ved overheaden, Hr Mikro stiller sig ved siden af og Hr Ko
  stiller sig ved slides'ne. Den 'originale' professor står nu ved de
  brugte slides}

\says{\kvaj} Godt... Er alle klar? Først HENTNING!!!

\scene{Hr. Ko tager en slide, går over til Hr. Mikro, giver sliden til Hr.
  Mikro og går tilbage igen.}

\says{\kvaj} og nu... HENTNING og AFKODNING!!!

\scene{Hr. Ko tager en slide, går over til Hr. Mikro, giver sliden til Hr.
  Mikro og går tilbage igen.  Samtidig har Hr Mikro kløet sig i håret, sagt
  'Aha' og drejet sliden rundt, og lagt den på overheaden. Der står: ``kr
  200,- på Lucky Devil i tredie løb''}

\says{\kvaj} Bemærk: Der er her tale om en såkaldt Data-Hazard! Og nu....
HENTNING, AFKODNING og ADRESSERING!!!

\scene{Hr. Ko tager en slide, går over til Hr. Mikro, giver sliden til Hr.
  Mikro og går tilbage igen.  Samtidig har Hr Mikro kløet sig i håret, sagt
  'Aha' og drejet sliden rundt, og lagt den på overheaden.  Samtidig har
  Hr. Risk mumlet noget uforståeligt.}
        
\says{\kvaj} Er det ikke smukt??? Og nu... Ja, jeg tror i har fattet
ideen, så LAD SAMLEBÅNDET KØRE!!!!

\scene{Pipelinen kører et kort øjeblik, men 'Professoren' kommer hele tiden
  til at fjerne slides'ne fra overheaden for hurtigt. Hr. Risk kan slet
  ikke nå at sige noget. Samtidig når hr Ko at overdænge hr. Mikro med
  slides}

\says{\kvaj} Nej STOP STOP!!!! Synkroniser dog!!!! \emph{Tager en klokke op
  fra tasken. Professorerne ser på deres ure. } Godt!!! Nu skal i vente med
at udføre jeres trin til i hører klokken!!! ok?  NU!!!! (ring)

\scene{alt går godt!}

\scene{ring}

\scene{alt går godt. \ansv\ kommer ind igen}

\says{\ansv} Hey Kårre!!! Jeg har en overklokke!!! (\emph{Tager en klokke
  frem}) Se Her!

\scene{\ansv ringer uhæmmet på sin overklokke! K. Vaj ser forfærdet på
  ham.  Pipelinen spasser fuldstændigt ud! Proffesorerne går i knæ!!}

\says{\kvaj} Nej!!! Nu gik alle professorerne ned!!!! Nu DUMPER jeg!!!!

\emph{Går grædende ud}

\says{stemme} Revy 97 har opdaget at Kåre er dumpet. Tryk detaljer for at
se opkastet, eller ok for at genstarte revy 97.
\end{sketch}
\end{document}
