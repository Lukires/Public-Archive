\documentclass[11pt]{article}
\usepackage{revy}
\usepackage[utf8]{inputenc}
\usepackage[danish]{babel}
\textwidth 170mm
\title{Fredagssangen}
\author{Theo Engell-Nielsen}
\melody{Fredagssangen med Tæskeholdet}

\version{0.91} % HUSK AT AJOURFØRE VERSIONSNUMMER!!

\revyyear{1997}
\parindent0pt
\parskip 1ex minus 1ex
\flushsingsright

\begin{document}

\maketitle

\begin{sketch}
\scene{Dette er tænkt som afslutningssang til første akt. En forsanger, 
  der ligner Casper Kristensen, og nogle andre personer, koret -- ligeledes i
  formand Mao-outfit, så ligheden med Tæskeholdet er til stede.}
\end{sketch}

\begin{song}

\sings{Sanger} Det var første akt,
        så skal I rigtig op og købe fler' øller
        og måske få tisset en tår

\sings{Kor}(hmmmm...)

\sings{Sanger} \em{igen noget sjovt, og nærmest improviseret}

\sings{Kor}  \em{igen noget andet sjovt}

\sings{Alle}  Kom nu alle publikummer
        tag og hjælp os lidt på vej
        pausen er på ty'v minutter
        og lad det køre som en leg - for

\sings{Sanger}        Så 'der andet akt
        så skal I atter til at grine og græde
        for revyen er sgu' så plat!

\sings{Kor}(opkastfornemmelser)

\sings{Sanger} Vi er ligeglad'
        om I har tøj på eller om I er nøgne
        men på scenen er her ret varmt 

\sings{Kor} (smid tøjet!!!)
        
\sings{Alle} Kom nu alle publikummer...

\scene{Musikken stopper (brat??) og sangeren fortæller, at nu er der
  pause i 20 minutter og det betyder, at man skal være tilbage inden
  klokken ``det og det''.}

\end{song}
\end{document}
% Local Variables: 
% mode: latex
% TeX-master: t
% End: 


