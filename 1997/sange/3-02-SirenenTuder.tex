\documentclass[a4paper,11pt]{article}
\usepackage{revy}
\usepackage[utf8]{inputenc}
\usepackage[danish]{babel}
\textwidth 160mm
\evensidemargin 0pt
\oddsidemargin 0pt
\title{Sirenen Tuder}
\author{Søren T. Martin K.}
\melody{Irene mudder}

\version{1.0} % HUSK AT AJOURFØRE VERSIONSNUMMER!!

\revyyear{1997}
\parindent0pt
\parskip 1ex minus 1ex
\flushsingsright

\begin{document}
\twocolumn[ % alt hvad der står imellem [] bliver skrevet i hele sidebredden
            % Hvis man ikke vil have to spalter, fjernes `\twocolumn[' og `]'
\maketitle
]
\begin{song}
\sings{A}Du står ved hoved-døren
kortet kørt igennem
det burde virke nu, 
hvor klokken den er fem, men

{\scene{omkvæd 1}}
sirenen tuder
systemet er i kludder
siren tuder

Det started' ellers lovende
for DIKUs sirene
Nu får du høretab 
tinnitus og migræne

Hveranden time - 
begynder den at kime
sirenen tuder

{\scene{mellemstykke}}
Og når du logger på 
for at lave lidt kod'
kan du ikke tænke - 
det bli'r noget rod
du har næsten fundet 
helt frem til løsningEN
men det glemmes for
nu hyler sirenen igen

\sings{A}Nu må du bare sid' og lytte
til den hvin
\sings{kor}Vi synes allesammen 
det er helt til grin
\sings{A}at kode på den måde
kræver uhørt disciplin

{\scene{solo}}


Og når du så skal hjem
hyler den på ny
Og lyden kan man høre
i den halve by

{\scene{omkvæd 2}}
sirenen tuder
min kode går i kludder
sirenen tuder

{\scene{mellemstykke}}
Du tog hjemmefra
med en illusion
om at skriv' en masse linier
men skrev slet ikke no'n
nu ligger du og roder
med høretab
på grund a' larmens
djævelskab

{\scene{omkvæd 1}}
siren tuder
systemet er i kludder
siren tuder

\end{song}
\end{document}
% Local Variables: 
% mode: latex
% TeX-master: t
% End: 
