\documentclass[a4paper,11pt]{article}

\usepackage{revy}
\usepackage[utf8]{inputenc}
\usepackage[T1]{fontenc}
\usepackage[danish]{babel}


\revyname{DIKUrevy}
\revyyear{1985}
\version{1.0}
\eta{$n$ minutter}
\status{Færdig}

\title{AI-sketch}
\author{JTP}

\begin{document}
\maketitle

\begin{roles}
\role{L}[JTP] Lærer
\role{AI}[Står ikke] Anne Ingerslev, kendt AI-forsker
\role{S1}[Står ikke] System
\role{S2}[Står ikke] System
\role{S3}[Står ikke] System
\role{S4}[Står ikke] System

% Er med i oversigt over numre men ikke i beskrivelsen af nummeret.  Nogle af
% dem er nok indehavere af de ikke-befolkede roller ovenover.
% \role{A}[CE] Andet
% \role{S}[CH] Sanger
% \role{A}[FSS] Andet
% \role{A}[HVO] Andet
% \role{A}[JT] Andet
% \role{A}[JM] Andet
\end{roles}


\begin{sketch}

\scene{En forelæsning på datalogi 1. del. Publikum = 1. dels-stud.}

\says{L} Ja, så må klokken vist være blevet $\frac{1}{4}$ over, og vi skulle vel
se at komme i gang med forelæsningen. Og i dag har jeg taget en gæst -- Anne
Ingerslev -- med, slipper I for at høre på mig.

Men måske skulle jeg lige sige til jer, der ikke var her sidste gang, at vi er
ved at gennemgå datalogiens forskellige områder, og vi er i dag kommet til det,
der hedder kunstig intelligens, eller blot AI. Og det er jo det du -- Anne --
vil fortælle os om.

Ja, jeg kan da lige fortælle, at Anne Ingerslev har beskæftiget sig med A.I. i
mange år (noget af en navlekigger må man sige), og hun er faktisk en kendt
forsker indenfor sit felt.

Nu har det jo i mange år været et åbent spørgsmål om \act{afbrydes} man har
kunnet lave sådan noget som ...

\says{AI} Ja, og nu er det endelig lykkedes.

\says{L} Ja, Anne. Fortæl os, hvad er det, der er lykkedes.

\says{AI} Det er nu lykkedes at lave en vaskeægte kunstig intelligens. Og
grunden er, at vi har fået disse herlige selvrefererende systemer. Jeg har netop
taget et med her \act{= S1. Giver det et ordentlig slag med hånden}.

\says{S1} Av!

\says{L} Ja, selvreferencer er jo noget, der er kommet på mode igen. Nå, ja,
undskyld jeg afbryder dig Anne, men jeg er ikke sikker på, at alle i auditoriet
er helt med på, hvad et selvreferende system er. Så, hvis du kort, bare helt
kort kunne fortælle os, hvad det er...

\says{AI} Ja, I kender vel alle et dialog-system, ikke? I ved, man giver
systemet et spørgsmål, og det kommer med et svar. Men, hvis systemet derudover
kan komme med sin egen mening, altså fx \act{henvender sig til L} midt under et
af dine meget lange spørgsmål kunne afbryde og sige ``Gab, du keder mig
usigeligt'', ja, så er det et selvrefererende system. Og det er netop sådan et,
jeg har med her. \act{giver S1 endnu et slag}.

\says{S1} Av!

\says{L} Ja, ja.  Godt ord igen Anne. Men \act{til p.} det,vi har her, er altså
et selvreferende system. Det er jo vældig interessant. Jeg får næsten lyst til
at \act{kilder S1} dikke, dikke dikke ...

\says{S1} Av!

\says{L} \act{farer forskrækket tilbage} Nå, gjorde det nu også ondt. \act{tager
  mod til sig og aer forsigtigt S1}

\says{S1} Av!

\says{AI} Ja, jeg må vist heller forklare.  Sandt at sige er det ikke noget
rigtigt selvrefererende system, men kun en selvreferende systemenhed. Men den
kan naturligvis udbygges. \act{S2 ind} Prøv bare at klappe den. \act{L klapper
  S2}

\says{S2} Aahh!

\says{AI} For nu at få den rigtige lyd frem må vi have en speciel
styringsenhed. Den kommer her. \act{S3 = fodboldspiller med bold under den ene
  arm og kasse over hovedet -- ind}

\says{L} Det vil jeg altså sige, at jeg kan.... \act{bokser S3}

\says{S3} Og der er kasse! \act{S3 sparker på S1}

\says{S1} Av!

\says{AI} Prøv så fx at klappe den. \act{L klapper S3}

\says{S3} Og den går lige ind! \act{S3 sparker på S2}

\says{S2} Aahh!

\says{L} Utroligt!

\says{AI} Jeg vil nu gå over til at fortælle lidt om det, jeg specielt har
baskæftiget mig med. Min forskningsindsats har været at give systemet sit
personlige præg. Netop som kvinde er det en ting, der har ligget mig stærkt på
sinde. Jeg har derfor udvidet systemet med endnu en komponent. \act{S4 ind}

\says{S4} \act{som efter en kort pause helt impulsivt og med stor spontaneitet
  udbryder} biip!

\says{AI} Dette er naturligvis kun er første skridt. Siden \act{afbrydes} skal
systemet udvides til ...

\says{L} Jo, men Anne. Ærlig talt. Har sådan et system overhovedet nogen
praktisk eller fornuftig anvendelse?

\says{AI} Ih,ja. Nu hvor systemet har fået sit personlige præg, er det jo
faktisk, hvad man kalder en Personlig Computer, eller en PC.

Hvis du vil have en bestemt PC, fx kan vi lave en IBM-PC, så skal du blot
udskifte denne kompunent \act{S1 ud} med en noget lagsommere komponent
\act{dommer ind med papkasse på hovedet}. Så, nu er det faktisk en IBM-PC'er. Må
jeg ikke give en lille demonstration.

\says{L} Jo, men så tror je jeg vil forlade scenen. \act{L ud}

\end{sketch}
\end{document}
