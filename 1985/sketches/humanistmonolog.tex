\documentclass[a4paper,11pt]{article}

\usepackage{revy}
\usepackage[utf8]{inputenc}
\usepackage[T1]{fontenc}
\usepackage[danish]{babel}


\revyname{DIKUrevy}
\revyyear{1985}
\version{1.0}
\eta{$n$ minutter}
\status{Færdig}

\title{Humanistmonolog}
\author{CE, HVO}

\begin{document}
\maketitle

\begin{roles}
\role{O}[HVO] Oplæser
\end{roles}


\begin{sketch}

\scene{Kurt-nummeret er lige forbi, personerne gået ud. Alt lortet bliver på
  scenen og ind træder en lærer. På bagtæppet står der ``DATALOGI ELLER
  HUMANIORA ??''}

\says{O} Jeg er glad for at jeg er datalog.  Dataloger kan tage på vandretur og
de kan gå til revy!

Det kan humanister ikke.

De sidder hele dagen og læser. Men de bliver ikke klogere alligevel. For
humanister er så dumme. (på nær en sød pige på kinesisk). Jeg er glad for, at
jeg ikke er humanist, for de har sådan en masse bøger. Ja, det har jeg forresten
da også.

Tænk bare at slæbe alle de bøger op på 4. sal, bare fordi man flytter!

Dataloger er så rare. De er altid parat til at advare dig, hvis du skulle tænke
på at trække en lys øle i automaten. Eller en Maarum!! Vi behersker bede sprog
og semantik. Det gør humanister ikke, og de er endda humanister.

Dataloger er også klogere end andre mennesker.

De kan f.eks. lave et PASCAL-program.

Det kan humanister ikke, men de er jo heller ikke dataloger.

Grundlæggende mener jeg, at der findes to slags mennesker: Dataloger, humanister
og de andre, hvilket igen er et eksempel på tilværelsens forunderlige evne til
at lade sig reducere til en ja-nej-ved-ikke logik, som vi dataloger derefter let
kan programmere.

Jeg kendte engang en mand, som døde uden at have lært at programmere.  Det må
være underligt at dø sådan.

Jeg tror også at der er flere dataloger i himlen end humanister. Så dataloger må
jo være bedre mennesker.

Til gengæld ved jeg helt sikkert, at der er flere døde humanister end
dataloger. Så det er osso farligt at være humanist!!

Datalogi er et fag med en masse sjove lærere. Vidste I f.eks. at Brincks PC'er
nu er tilgængelige 3 timer om ugen ??

Det er de heller ikke!!!

Humanister læser Goethe og Freud og på tisk!!

Kunne I tænke Jer noget værre ?? De er jo døde for længe siden. Jeg ælsker logik
-- og dataloger er så logiske.

-- undtagen i Århus.  Ved I hvordan de tæller i Århus ? De tæller: ``dat-1,
dat-2, dat-3'' !  Ha! De er ikke dataloger -- de er matematikere!

Jeg er glad for at jeg er datalog!

\end{sketch}
\end{document}
