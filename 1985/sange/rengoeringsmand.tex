\documentclass[a4paper,11pt]{article}

\usepackage{revy}
\usepackage[utf8]{inputenc}
\usepackage[T1]{fontenc}
\usepackage[danish]{babel}


\revyname{DIKUrevy}
\revyyear{1985}
\version{1.0}
\eta{$n$ minutter}
\status{Færdig}

\title{Rengøringsmand}
\author{HVO}
\melody{Står ikke}

\begin{document}
\maketitle

\begin{roles}
\role{S}[HVO] Sanger

% Er med i oversigt over numre men ikke i beskrivelsen af nummeret.
%\role{A}[PK] Andet
\end{roles}


\begin{song}
\sings{S}%
Når morgnen gryr på DIKU
Og de fleste sover trygt
og selv UNIX-freaks'ne
er som regel gået hjem
så starter jeg min runde
med skrubbe, klud og spand
blandt affaldsdynger
baner jeg mig frem

\sings{S}[omkv]%
Og rydder ud
hælder væk
lukker vinduerne op
laver gennemtræk.
Det er står tilbage
fra dagen før,
hvem tror I ta'r det
gæt hvem der gør.

\sings{S}%
Om lidt, når jeg har sunget
og revyen er forbi
skal vi feste i kantinen
til i morgen
Men når \underline{du} engang står op
med dine grimme tømmermænd
går revygruppen herinde
over gården

\sings{S}[omkv]

\sings{S}%
Nu lakker det mod enden
vi har sunget sidste sang
det her er faktisk
enden på revyen
om lidt så tændes lyset \act{lyset tændes}
og musikkengår i stå \act{musik stopper}
vi må hel're fort'sæt'
et andet sted i byen

\sings{S}[omkv]%
I blir smidt ud
så kom hel're væk
luk vinduerne op
og lav gennemtræk
Det er på DIKU
det nuforegår
vi takker af
slutter for i år

\sings{S}%
Gå over gården
op på første sal
så lidt til venstre
for der er der bal
Det er på DIKU
det nu foregår
vi takker af
slutter for i år.
\end{song}

\end{document}
