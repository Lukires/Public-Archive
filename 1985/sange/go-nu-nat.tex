\documentclass[a4paper,11pt]{article}

\usepackage{revy}
\usepackage[utf8]{inputenc}
\usepackage[T1]{fontenc}
\usepackage[danish]{babel}


\revyname{DIKUrevy}
\revyyear{1985}
\version{1.0}
\eta{$n$ minutter}
\status{Færdig}

\title{Go' nu nat}
\author{DIKUrevy m.fl.}
\melody{Go nu nat ....}

\begin{document}
\maketitle

\begin{roles}
\role{S}[Sangere] Sangere
\end{roles}

\scene{Dette nummer hører ikke til selve revyen, men er et flip som foregår nede
  i gården inden maden. Et antal personer fra revygruppen stiller sig op i
  nærheden af porten til RECKU og synger:}

\begin{song}
\sings{S}%
Go nu nat og gå nu lige hjem
I er snydt og i er bare for nem'
Vi har scoret Jeres penge
så gå hjem i Jeres senge
go nu nat og gå nu lige hjem.

Og I troed' I sku' se revy
og feste vider' til det årle morgengry
I er taget i hoved og røv
I troed ik' vi turde prøv'
go nu nat og gå nu lige hjem

I har troet I sku' ha' det sjovt
og nu er det bare blevet tem'lig flovt
Der er ingen mad og vin
så I er blevet rent til grin
go nu nat og gå nu lige hjem
\end{song}
\scene{Herefter styrter alle de syngende ud igennem RECKU's port, mens resten af
  revygruppen sørger for at spærre/låse døren, så der ikke er nogen der følger
  efter. De syngende revygruppemedlemmer løber hen langs RECKU og ind ad døren i
  den anden ende, op på første sal og hen i aud. 18 og synger ud i gennem
  vinduerne:}
\begin{song}
Ja okay, hvis det absolut skal være
ja, så ku' vi altså heller ik' la' vær
vil I høres vores revy
må I råbe højt i sky \act{gentages indtil der synges med}
go nu nat og gå nu lige hjem
\end{song}
\end{document}
