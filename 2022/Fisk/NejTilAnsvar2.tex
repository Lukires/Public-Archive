\documentclass[a4paper,11pt]{article}

\usepackage{revy}
\usepackage[utf8]{inputenc}
\usepackage[T1]{fontenc}
\usepackage[danish]{babel}

\revyname{DIKUrevy}
\revyyear{2022}
\version{1.0}
\eta{$3$ minutter}
\status{Færdig}

\title{Nej til Ansvar 2}
\author{Lukas}

\begin{document}
\maketitle

\begin{roles}
\role{X}[NAVN] Instruktør
\role{S} [Lukas] Speak
\role{L} [lukas] Lead
\role{V} [Jonatan] ven
\role{T} [Torben] Torben
\end{roles}

\begin{props}

\end{props}

\section*{notes}Fisk med drug PSA vibes. Stage direction er selvdokumenterende



\begin{sketch}
\says{S} Dette er en besked til alle studerende på Københavns Universitet

\says{S} Du er på vej hjem fra studiet med din kammarat.

\says{S} Klokken er mange, og det er mørkt udenfor.
\says{S} Din ven fortæller dig en joke,
\says{v} Har du hørt den om ham der skiftede fra VIM til Emacs
\says{S}  den var sgu meget sjov, men før du når at reagere på den hører du en fnisen fra busken bag jer.
\says{s} Udfra busken springer Torben, formanden for revyen, og spørger om I ikke skal skrive for revyen.
\says{T} Skal i ikke skrive for revyen
\says{S} Du bliver meget fristet, da din ven siger ja
\says{S} Men så husker du
\says{L} Nej
\says{S} “Det er sejt at sige nej.”

\end{sketch}

\end{document}
