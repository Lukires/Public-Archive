\documentclass[a4paper,11pt]{article}

\usepackage{revy}
\usepackage[utf8]{inputenc}
\usepackage[T1]{fontenc}
\usepackage[danish]{babel}

\revyname{DIKUrevy}
\revyyear{2022}
\version{1.0}
\eta{$4$ minutter}
\status{Færdig}

\title{Slik xor ballade}
\author{Bjørn, Eva}

\begin{document}
\maketitle

\begin{roles}
\role{X} [Sean] Instruktør
\role{B}[Schauser] Barn
\role{D}[Sejer] Datalog
\end{roles}

\begin{props}
\prop{Haloween udklædning}
\prop{Klikker}
\prop{Dør}
\prop{2 stole hvor mindst en kan dreje}
\prop{pap-mobil må gerne ligne apple lort}
\prop{noget at have holloween slik i}
\prop{noget at ligge i den overstående beholder}
\end{props}


\begin{sketch}
\scene{Hjemme hos en datalog}

\says{B}“Slik eller ballade!”
\says{D}“Slik eller ballade… Se nu her Rus, det udsagn stemmer logisk set slet ikke overens med den semantiske betydning, du forsøger at tilegne det.
\says{D}“Opskriv følgende logiske tabel og lad P være en bool, som evaluerer til true hvis og kun hvis slik gives, og Q være ækvivalent defineret, men for variablen ballade. Dermed kan vi opskrive følgende sandhedstabel for dit udsagn “Slik eller ballade:”

\scene{D står og skriver ivrigt en sandhedstabel op imens B står og stirrer med tomt blik på D’s vanvid*}
\says{D} “Se, det eneste tidspunkt, hvor “Slik eller Ballade” ikke evaluerer til true, er når både slik OG ballade er falsk. Dermed sætter du mig i en situation, hvor jeg kan risikere at komme i ballade på trods af at have givet dig slik, som indikeret i øverste række hvor både P og Q er sand. “
\says{D} “Næh, hvis du ønsker at skabe en bedre brugeroplevelse, der giver en tryghedsfornemmelse hos brugeren, vil jeg anbefale dig at følge denne sandhedstabel, hvor vi genbruger den semantiske betydning for variablerne P og Q og lader X kendetegne den ukendte, binære logiske operator vil agere på P og Q:”
\says{D} \act{fortsætter sin vanvittige skribling og forklarer sig sine tanker undervejs} Se, vigtigst af alt er naturligvis, at brugeren ikke kan risikere at komme i ballade på trods af at have udleveret slik. Dvs. at når P og Q er sand, da ønsker vi udsagnet P X Q skal evaluere til falsk. Når P = T og Q = F ønsker vi, at P X Q skal evaluere til True idet vi da giver slik og ikke modtager ballade. Naturligvis ønsker vi samme evaluering når Q = T og P = F - hvorfor efterlader vi som en øvelse til læseren.
\says{D} Og til sidst er det op til fortolkning, hvordan vi ønsker P X Q skal evaluere når P = Q = F. Anser vi din sætning “Slik X Ballade” som et ultimatum hvormed du kræver et svar, vil det give mest logisk mening at lade F X F evaluere til False for at indikere, at nøjagtigt en af Slik eller Ballade skal være sand. Hermed får vi altså følgende sandhedstabel:”
\scene{D viser (måske for) ivrigt sandhedstabellen til B, som stadigvæk står og forundret og kigger på D’s langsomme nedstigning til galskab}
\says{D} Nu er du så heldig, at jeg har snydt hjemmefra så…

\says{D}... jeg har selvfølgelig en generisk XOR-sandhedstabel liggende. Vi bemærker, at denne sandhedstabel for udsagnet P XOR Q er logisk ækvivalent med vores udsagn P X Q. Den logiske operatorer, vi har søgt efter, har altså været XOR hele tiden! Ved et variabelskift ser vi nu, at sætningen “Slik XOR Ballade!” er den, du faktisk søger!”
\scene{D smækker døren i hovedet på B, der ser dybt forvirret på den lukkede dør et øjeblik og derefter går tomhændet derfra. Scenen efter ser man så D sætte sig ned foran døren, hvor han afventer sit næste offer}
\scene{Lys ned.}
\end{sketch}

\end{document}
