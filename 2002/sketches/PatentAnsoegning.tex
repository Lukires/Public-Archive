\documentclass[a4paper,11pt]{article}

\usepackage{revy}
\usepackage[utf8]{inputenc}
\usepackage[T1]{fontenc}
\usepackage[danish]{babel}

\revyname{DIKUrevy}
\revyyear{2002}
\version{1.0}
\eta{4 min.}
\status{Færdig}

\title{Patentansøgning}
\author{Theo Engell-Nielsen, Søren Trautner Madsen, Jesper Holm Olsen,
  Andre Tischer, Uffe Friis Lichtenberg og Martin Kock}

\begin{document}
\maketitle

\begin{roles}
\role{A} En ansøger. Bromatolog, kittel og hele moletjavsen.
\role{B1 og B2} Patentansøgningspersonel, mørkt jakkesæt, briller, MIB-stil.
\end{roles}

\begin{props}
\prop{Kittel}
\prop{Sikkerhedsbriller}
\prop{Evt. kolbe}
\prop{2 * MIB outfit}
\prop{Diverse stykker papir}
\end{props}

\scene{På scenen: En (virtuel) skranke, to stole, hvor den ene er
placeret bag den anden for at illudere at den ene er en chefstol, B2s
stol, i baglokalet.}
  
\begin{sketch}
\says{A} Goddag jeg vil gerne indlevere en patentansøgning på
fedtfattig økologisk leverpostej.

\says{B1} Ja så gerne - udfyld venligst denne patentansøgningsblanket.

\says{A} Hmmm, ja. \act{tager blanket, udfylder imens han snakker}
Navn, ja, adresse, tlf, ja, Forudsætninger: "blah blah", øh, er det OK
at man bruger en meget stor foodprocessor i processen som man ikke
selv har lavet?

\says{B1} Ja, det er OK, jeg mener ikke det er patenteret. \act{Får en
  lys idé, men viser det ikke over for A, skriver noget ned. Han
  bliver mere og mere begejstret for at finde flere og flere ting, der
  endnu ikke er patenterede}

\says{B2} [hører snakken og siger højt] Hvis det er en foodprocessor
over ét ton er der ingen problemer.

\says{A} Nåda-da, ok...

\scene B1 laver lidt rettelser på papiret

\says{B2} Hvis der er en Siemens robotarm på æltemaskinen, så er det
for resten allerede patenteret.

\says{A} Århvad, så må jeg rette lidt i ansøgningen, hmmm...
\act{retter til}

\scene B1 laver også rettelser på papiret

\says{B2} Ja, og de andre æltemaskiner med robotarme er der faktisk
også nogle restriktioner på.

\says{A} Hvad er det for nogen?

\scene B1 sukker, lægger papiret fra sig og går over i det virtuelle
patentarkiv, finder et patent frem

\says{B1} Ja, hvis det er en Philips æltearm, så må den kun dreje med
uret - altså set oppefra.

\says{B2} ...altså med mindre det foregår i et æltekar af svensk
fyrretræ formet efter en oskulerende superparaboloide - så skal den
faktisk dreje mod uret og rotere arytmiskt.

\says{A} A-hva?

\says{B1} Robotarmen må desuden maksimalt ha' fem
omdrejningsindstillinger, og...

\says{B2} De må ikke været fortløbende indekseret på maskinens knapper
i forhold til omdrejningshastigheden på ælteagregatet.

\says{B1} Hva?

\says{B2}[kommer frem på scenen og gentager langsomt] De må ikke været
fortløbende indekseret på maskinens knapper i forhold til
omdrejningshastigheden.

\says{A}[ser forvirret ud]

\says{B2}[patroniserende]Det vil sige at knapperne til regulering af
omdrejningshastighed skal være placeret i en tilfældig rækkefølge.
Ellers er det ulovligt!

\says{A} Hvad snakker I om? Det der er da totalt urimeligt!

\says{B1} Nej, nej. Der var nemlig noget med en ansøgning om en
sorteringsalgoritme, mener jeg. Lige to sekunder.

\scene B1 leder i det virtuelle arkiv

\says{B1} Ja, her er en gut der har taget patent på sortering og
sorterede rækker.

\says{B2}[kommer hen til B1] Det kan han da vist ikke, for der er en
her der har patent på brug af variable. \act{viser et B1 et patent}

\says{B1} Men det må han ikke - hans brug registerallokering er ifølge
dette patent ulovlig. \act{viser et B2 et patent}

\says{B2}[roder i arkivet] Men her er der en der har patent på
bits\ldots og binære alrepræsentationer.

\says{B1}[roder med i arkivet] Her er der en der patent på de
naturlige tal, se her, patentansøgning nummer 17.

\says{B2}[udbryder] Hvaffornoget!! Her er der en, der har patent på
patenter! Argh!

\says{B1}[til A] Giv mig den patentansøgning! Det der er jo stinkende
ulovligt! \act{flår ansøgningen ud af hænderne på ham} Idiot!

\scene Alt andet lys end spotten på A skrues ned

\says{A}[kigger ud på publikum] Århvaaaaaaaaad!

\scene Tæppe for/lys ned

\end{sketch}
\end{document}

%%% Local Variables: 
%%% mode: latex
%%% TeX-master: t
%%% End: 
