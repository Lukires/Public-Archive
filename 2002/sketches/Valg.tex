\documentclass[a4paper,11pt]{article}

\usepackage{revy}
\usepackage[utf8]{inputenc}
\usepackage[T1]{fontenc}
\usepackage[danish]{babel}

\revyname{DIKUrevy}
\revyyear{2002}
\version{1.1}
\eta{4,5 min.}
\status{Færdig}

\title{Valg}
\author{Jørgen Elgaard Larsen}

\begin{document}
\maketitle

\begin{roles}

\role{P}[Uffe FL] Poul Nykup
\role{F}[Bo] Anders Fjog
\role{GP}[Uffe C] Generisk Politiker
\role{K}[Peter] Kjeld
\role{R0}[Sidsel] Rektorkandidater 0
\role{R1}[Maja] Rektorkandidater 1
\role{BS}[Christoffer] Batiksmølf
\role{VO}[Jørgen] Oplæser

\end{roles}

\begin{props}
  \prop{Cykelhjelm}
  \prop{Dekan-poncho (batiksmølf)}
  \prop{Dekan-skæg}
  \prop{3 * Jakkesæt}
  \prop{2 * Nederdel}
  \prop{Pæne bukser}
  \prop{lidt papir}
  \prop{Paryk, lang, pink}
  \prop{Rullekrave-sweater, sort}
  \prop{Sandaler}
  \prop{Shorts}
  \prop{Skilt (Generisk Politiker)}
  \prop{2 * Skjorter}
  \prop{Skørt}
  \prop{Slips, rødt}
\end{props}

\scene{ P kommer glad ind.} 
  
\scene{\textbf{teknik}: Hele sketchen foregår foran tæppet. Folk
  kommer ind fra venstre side og forlader scenen i midten. Der skal
  naturligvis hele tiden være lys (stor spot) på dem der står i midten
  og når der kommer en ind fra siden skal der være spot på vedkommende
  indtil vedkommende er kommet ind til midten.}

\begin{sketch}
  
  \says{P} Det er rart at være statsminister. Jeg tager lige "four
  more years". Det virkede sidst.
  
  \says{F} [Kommer løbende ind på scenen] Næ hov, nu er det min tur.
  Det danske folk har længe lidt under Socialdemokratiets lemfældige
  omgang med statens midler.
  
  \says{P} Jamen I er jo sammen med Pia. Det går ikke. Vi i
  Socialdemokratiet er jo \em{tager imaginær mønt frem og slår plat og
    krone} FOR flygninge.
  
  \says{F} Det er vi skam også. Bare de ikke er her.
  
  \says{GP} [Kommer ind på scenen med skilt på] Man kan altså også
  stemme på mig. Jeg er også for eller imod flygtninge.
  
  \says{P} Nej, altså. Det er mig, der er for eller imod flygtninge,

  \says{F} Ja, og mig.
  
  \scene{P+F+GP dasker til hinanden. Lægger mærke til, at publikum
    kigger. Smiler og vender sig mod publikum}
  
  \says{P+F+GP} [i kor mod publikum] Husk: I har et valg!
  
  \says{VO} Og vinderen blev: En af dem!
  
  \scene{F Gør vinder-bevægelser, alle går ud

    \ldots
    
    Efter lidt tid kommer K ind på scenen}
  
  \says{K} Nej, hvor er det dejligt at være rektor! Jeg tror s'gu jeg
  snupper 4 år mere\ldots
  
  \says{R0} [Kommer ind på scenen] Davs du gamle.
  
  \says{K} Davs du. Nå stiller du også op?
  
  \says{R0} Jeps.
  
  \says{VO} Det forlyder netop nu, at den nuværende rektor ikke kan
  genopstille, fordi hans prorektor ikke gider ham mere.
  
  \says{K} [overrasket] Hvad??!? Det er da snyd. \act{Bliver sur} Så
  kan det også være lige meget!!
  
  \scene{K går nedtrykt ud af scenen. R0 rækker næse ad ham}
  
  \says{R1} [Kommer ind på scenen] Davs du gamle.
  
  \says{R0} Davs du. Nå stiller du også op?

  \says{R1} Jeps.

  \says{R0} Det skal nok blive hyggeligt.

  \says{R1} Men har du et valgprogram?

  \says{R0} Ja, jeg nuppede lige Kjelds, inden han gik 
  
  \scene{R0 tager nogle papirer frem. R1 kigger på papirerne. Bagved
    de to kigger K ind og skumler, gør knytnæve mod dem, men får så en
    idé og går ud}

  \says{R1} Jeg ved ikke, det er jo noget tyndt
  
  \says{R0} Ja, men det virkede jo for ham. Men vi må jo alligevel
  hellere aftale at være uenige om noget -- du ved, til valgkampen.
  
  \says{R1} Nåja, ladmig se \act{tænker} Nåjo, der er jo den her debat
  om universitetes ledelsesform.
  
  \says{R0} Du mener om der skal være valgt ledelse eller en rigtig
  bestyrelse, der kan tage fornuftige beslutninger?

  \says{R1} Ja, nemlig.
  
  \says{R0} Jamen, vi vil da have en valgt ledelse. Ellers har vi jo
  ikke noget at lave?

  \says{R1} [Indrømmende] Ja?
  
  \says{R0} \ldots og det er jo slet ikke det man kan stemme om til
  det her valg.
  
  \says{R1} [Gør hjælpende håndbevægelser, vil hjælpe R0 til
  forståelse] Ja?

  \says{R0} \ldots så det er jo egentlig ligemeget, om vi er ueninge

  \says{R1} [Gør hjælpende håndbevægelser, vil hjælpe R0 til
  forståelse] Ja?
  
  \says{R0} [kigger undrende på R1, indtil et lys går op for ham] Nå,
  nå, nå.. OK!

  \says{R1} [anerkendende] Ja!
  
  \says{K} [Kommer ind på scenen med paryk på, laver sin stemme om til
  meget lys] Juhu! Åh Juhu! Jeg vil også stille op!

  \says{R0} Er det ikke lidt sent?
  
  \says{K} Jo, men nu stiller jeg... øh, jeg mener Kjeld.. ikke op, så
  nu vil jeg godt. Jeg ville helst ikke stille op imod ham. Jeg.. øh,
  han er så sød \em{(smiler bedårende)}

  \says{R1} Godt så. Men du har vel et valgprogram?
  
  \says{K} Jaja, jeg har bare brugt mit..øh, Kjelds... gamle, og så
  plastret lidt på om, at vi skal have valgt ledelse.
  
  \says{R0} Næ hov. Det er MIG der mener, at vi skal have valgt
  ledelse.

  \says{R1} Nix, det er mig.

  \says{K} Nej, mig!
  
  \scene{K+R? dasker til hinanden. Lægger mærke til, at publikum
    kigger. Smiler og vender sig mod publikum}
  
  \says{K+R?} [i kor mod publikum] Husk: I har et valg!

  \says{VO} Og vinderen blev: En af dem!

  \scene{K Gør vinder-bevægelser, alle går ud

    \ldots

    Efter lidt tid kommer BS ind på scenen}
  
  \says{BS} Det er godt nok fedt at være dekan! 

  Nå, det er jo snart valg. Gad vide, om der er nogen modkandidater

  \says{BS}[Kigger undersøgende ud over publikum] Nå, det lader til, at der
ikke er nogen modkandidater. Jeg snupper lige
  \ldots en hel masse år mere.
  
  \says{BS} [Mod publikum] Husk: I har ikke noget valg!
  
  \says{VO} Igen iår undrer valgforskere sig over den exceptionelt
  ringe valgdeltagelse blandt studerende.

  \scene{\textbf{teknik}: Lys ned og så går BS ud}

\end{sketch}
\end{document}

%%% Local Variables: 
%%% mode: latex
%%% TeX-master: t
%%% End: 



