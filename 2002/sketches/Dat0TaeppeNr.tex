\documentclass[a4paper,11pt]{article}

\usepackage{revy}
\usepackage[utf8]{inputenc}
\usepackage[T1]{fontenc}
\usepackage[danish]{babel}

\revyname{DIKUrevy}
\revyyear{2002}
\version{1.0}
\eta{1 min.}
\status{Færdig}

\title{Smid tøjet (tæppenummer)}
\author{Uffe Friis Lichtenberg}

\begin{document}
\maketitle

\begin{roles}
\role{B}[Anders] Blotter
\role{VO}[Bo] Voiceover
\end{roles}

\begin{props}
\prop{Trenchcoat}
\prop{Pap med kurve påtegnet}
\end{props}

\begin{sketch}

\scene En mand kommer ind på scenen. Han er iført trenchcoat og intet andet (må man gå ud fra). Han ser sig skummelt omkring og pludselig blotter han sig ud mod publikum. Han har ganske rigtigt intet på, pånær et stort stykke pap der dækker (bla.) de vitale dele. På dette stykke pap er tegnet en stor flot kurve.

\says{VO} Mine damer og herrer, de har netop overværet en fungerende kurveblotter.

\end{sketch}
\end{document}
%%% Local Variables: 
%%% mode: plain-tex
%%% TeX-master: t
%%% End: 
