\documentclass[a4paper,11pt]{article}

\usepackage{revy}
\usepackage[utf8]{inputenc}
\usepackage[T1]{fontenc}
\usepackage[danish]{babel}

\revyname{DIKUrevy}
\revyyear{2002}
\version{2.0}
\eta{4 min.}
\status{Færdig}

\title{Den skidesjove sorteringsvits}
\author{Niels Bosworth og Uhd}

\begin{document}
\maketitle

\begin{roles}
  \role{D0}[Anders] Datalog 0
  \role{D1}[Heidi] Datalog 1
  \role{D2}[Maja] Datalog 2
  \role{D3}[Sidsel] Datalog 3
  \role{D4}[Søren H] Datalog 4
  \role{D5}[Uffe FL] Datalog 5
  \role{D6}[Carsten] Datalog 6
  \role{D7}[Peter] Datalog 7
  \role{D8}[Jørgen] Datalog 8
  \role{D9}[Adam] Datalog 9
  \role{VO}[Bo] Voice over
\end{roles}

\begin{props}
  \prop{10 * T-shirts, sorteringsagtige}
\end{props}

\begin{sketch}

  
  \scene D0 og D1 står på række, med D0 længst mod bandet. D2-D9 står
  klar ude i kulissen til at kommer ind

%  \scene \textbf{teknik}: Hele sketchen foregår foran tæppet.
  
  \scene \textbf{teknik}: Spot på når D0 og D1 er klar på scenen. De
  vil stå ca. midt på scenen. De skal have en stor spot på. Derefter
  vil folk komme ind fra venstre (set fra scenen). Den anden spot skal
  så fange folk når de kommer ind fra venstre. Dem der kommer ind vil
  flytte sig ind midt på scenen så der til sidst står en lang række på
  scenen. Om det kan klares med spots eller mere generel belysning er
  op til teknik at afgøre.
  
  \says{D0} Har du egentlig hørt den skidesjove sorterings-vits?

  \says{D1} Nej.
  
  \says{D0} OK, nu skal du høre. Der sidder tre dataloger på et
  web-bureau.  Så finder den første på en brandgod
  sorterings-algoritme.
  
  \says{D2} [idet kan kommer ind på scene] Hov, vent! \act{D2 stiller
    sig op i rækken ved siden af D1 og tager hans mikrofon til at
    tale} Denne revy er også for Dat-0-erne. Så vi må lige forklare
  ordet ``sortering''. Altså...  \act{Vender sig med pædagogisk
    tonefald ud mod publikum} Hvis man f.eks. har en masse tal i en
  liste, så \ldots
  
  \says{D3} [idet han kommer ind på scenen] Hov, vent! \act{D3 stiller
    sig op i rækken ved siden af D2 og tager hans mikrofon} Vi må nok
  hellere lige forklare for Dat-0-erne, hvad en ``liste'' er. \act{til
    publikum} Altså, en liste er en meget anvendelig datastruktur.
  Det vil sige, det er en simpel type, som \ldots
  
  \says{D4} [idet han kommer ind på scenen] Hov, vent! \act{stiller
    sig op ved siden af D3, tager mokrofon} Vi skal lige forklare
  begrebet ``type'' for Dat-0-erne.  En type bruger man i alle {\em
    fornuftige} programmeringssprog til at begrænse, hvilken slags
  værdier, man kan \ldots
  
  \says{D5} [idet han kommer ind] Hov vent! \act{stiller sig op som de
    andre} Dat-0-erne vil nok gerne have at vide, hvad en ``værdi''
  er. Det lyder jo frygtlig abstrakt, men det er det slet ikke!  Helt
  nede på jorden så er værdier det, som variablene \ldots
  
  \says{D6} [Idet han kommer ind] Hov, vent! \act{stiller sig op som
    de andre} Nu må du lige passe på ikke at forfordele de Dat-0-ere,
  der ikke ved, hvad en "variabel" er.  I de fleste {\em fornuftige}
  programmeringssprog, er det helt ligefremt, hvordan variable er
  implementeret.  For hver variabel bruger man en hægte, der \ldots
  
  \says{D7} [idet han kommer ind] Hov, vent! \act{stiller sig op}
  ``Hægter'' er et meget vigtigt begreb i datalogien, skal I fra Dat-0
  lige vide. Nede i lageret \ldots
  
  \says{D8} [Idet han kommer ind] Hov, vent! \act{stiller sig op i
    rækken} Nu er arkitektur jo ikke pensum på Dat-0, men det retter
  vi hurtigt op på. Altså, maskinen har typisk mange forskellige slags
  "lager". hvor cachen er den hurtigste.
  
  \says{D9} [Idet han kommer ind] Hov, vent! \act{stiller sig op} Nu
  må I Dat-0-ere ikke være bange for et begreb som "cache".  En
  cache...  \act{Går i stå} Det ved jeg sgu' ikke hvordan man
  forklarer, så Dat-0-ere forstår det.  Drenge, jeg rejser lige en
  exception.
  
  \says{Alle de andre} EN EXCEPTION?!
  
  \scene Nu skal det til at køre stramt
  
  \says{D9} [irriteret] Ja, det er en slags afbrydelse. Lad nu være
  med at afbryde. \act{giver hurtigt mikrofonen til D8, vender sig 90
    grader på hælen og går ud af scenen}
  
  \says{D8} [idet D9 går] Altså en cache er et lager \act{giver
    hurtigt mikrofonen til D7, vender sig 90 grader på hælen og går ud
    af scenen efter D9}
  
  \says{D7} \ldots som bruger hægter \act{giver hurtigt mikrofonen til
    D6, vender sig på hælen 90 grader og forlader scenen}
  
  \says{D6} \ldots som implementerer variable \act{giver mikrofon til
    D5, vender sig og går som de andre}
  
  \says{D5} \ldots der inderholder værdier \act{giv mikrofon, ud}
  
  \says{D4} \ldots som har typer \act{giv mikrofon, ud}
  
  \says{D3} \ldots som lister indeholder \act{giv mikrofon, ud}
  
  \says{D2} \ldots der kan sorteres. \act{giv mikrofon, ud}

  \says{D1} [KUNSTPAUSE!] Nåååhh.... den har jeg hørt!

  \scene Lys ned, tæppe for

\end{sketch}
\end{document}

%%% Local Variables: 
%%% mode: latex
%%% TeX-master: t
%%% End: 
