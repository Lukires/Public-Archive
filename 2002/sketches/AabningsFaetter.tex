\documentclass[a4paper,11pt]{article}

\usepackage{revy}
\usepackage[utf8]{inputenc}
\usepackage[T1]{fontenc}
\usepackage[danish]{babel}

\revyname{DIKUrevy}
\revyyear{2002}
\version{2.0}
\eta{5 min.}
\status{Færdig}

\title{Åbningstale}

\author{Jakob Jensen \& Adam \& Søren}

\begin{document}
\maketitle

\begin{roles}
  \role{H}[Bo] Hat (voice over) 
  \role{TM}[Søren H] Troldmand (Revyboss)
  \role{RB}[Adam] Revyboss
  \role{R0}[Jesper] Revyt 0 (musiker)
  \role{R1}[Erling] Revyt 1 (tekniker)
  \role{R2}[Sidsel] Revyt 2 (publikum)
  \role{R3}[Carsten] Revyt 3 (skuespiller)
\end{roles}

\begin{props}
  \prop{Sorteringshat}
  \prop{Skammel}
  \prop{Kappe}
  \prop{Troldmandshat}
  \prop{Kåbe}
  \prop{Skæg}
\end{props}

\begin{sketch}
  
  \scene Tæppet er for. På scenen står en skammel med en gammel hat
  på, og ind fra siden kommer TM og Rx. TM stiller sig midt på scenen
  og Rx stiller sig ved siden af.

  \scene{\textbf{teknik}: Troldmanden kommer først ud foran tæppet og
  der skal spot på. Senere vil der kunne nogle flere op og først når
  de er på scenen skal der lys på dem alle.}

  \says{TM} Velkommen til det første revymøde. For at få en revy op og
  stå, skal vi have besat en række nøglepositioner. \act{Tager hatten
    og læser et navn (R0) op fra en seddel}
  
  \says{R0}[sætter sig på stolen og får hatten på] Hep!
  
  \says{H}[lidt højt efter kort pause] Musiker!
  
  \scene{Musikerne støjer og jubler, og R0 går op på bandscenen og
    tager sit instrument}
  
  \says{TM}[læser op] (RB)! % Navnet på personen
  
  \scene{RB sætter sig på skamlen og får H på hovedet}
  
  \says{H}[lidt højt osv.] Revyboss!
  
  \says{TM}[jubler] Yæy! \act{tager sig selv i det, rømmer sig}
  
  \scene{RB går ud i siden af scenen}

  \says{TM}[læser op] (R1)! % Navnet på personen
  
  \scene{R1 sætter sig på skamlen og får H på hovedet}
  
  \says{H}[lidt højt osv.] Tekniker!
  
  \scene{Teknikerne jubler, og R1 jubler med og går op til teknikken}

  \says{TM}[læser op] (R2)! % Navnet på personen
  
  \says{H}[råber, inden den bliver sat ordentligt på] Publikum!
  
  \scene{Publikum (og R2) går amok, og R2 løber ned og sætter sig på
    sin plads}

  \says{TM}[læser op] (R3)! % Navnet på personen
  
  \says{H}[efter et stykke tid (som i: et par sekunder)] Hmmm...
  hmmm...
  
  \says{R3}[lidt til sig selv] Ikke publikum... ikke publikum!
  
  \says{H} Den er svær! Du ville kunne drive den langt, og få meget
  magt som publikum. Meeeeen, det vil du ikke, siger du. Så bliver du
  \act{kunstpause} SKUESPILLER!
  
  \says{R3} \act{Jubler og der kommer en masse støj/jublen/skramlen
    ude bagfra. R3 går ud bagved}
  
  \scene{På scenen er nu kun TM og RB tilbage. De går tøvende ind mod
    midten}
  
  \says{TM+RB} \act{Kigger sporgende på hinanden, kigger tøvende mod
    publikum, kigger igen på hinanden}
  
  \says{RB} \act{Åbner munden for at sige noget. Fortryder}
  
  \says{RB+TM} \act{Kigger ud på publikum igen. Pause RB læner sig mod
    TM, hvisker, begge fniser, begge retter sig op}
  
  \says{RB}[undrende] Er vi så bosserne?
  
  \says{TM}[samler sig selv] Øhm... \act{hoster beslutsomt}
  
  \says{TM}[med myndig stemme] Det er os en stor ære at være valgt til
  denne \act{tænker} ærefulde post. Vi er stotle over for tredivte
  gange \act{kører sig selv længere og længere op} at kunne præsentere
  \ldots
  
  \says{RB}[afbryder] Hov \ldots \act{hiver i TM's ærme} Var der ikke
  noget, vi skulle sige?
  
  \says{TM}[snaps out of it] Hva'? Som hvad dog?
  
  \says{RB}[entusiastisk] Jojojo! Du ved, det der med at man
  \act{tæller på fingrene} ikke må ryge, eller bruge åben ild under
  revyen. Og at man skal have sin mobiltelefon slukket, og
  nødudgangene \ldots
  
  \says{TM}[afbryder, irriteret] \ldots ja, som er der og der og der
  og der og der \act{peger på alle nødudgange undtagen den ved
    bandscenen}!  Det HAR de sgudda alle sammen fået at vide niogtyve
  gange allerede. Du ville vel osse fortælle dem at der ikke må udøves
  magi i salen, og at medbragte ugler skal placeres i garderoben?
  
  \says{TM}[prøver at genvinde sit ekvilibrium] Nå! \act{lavt til sig
    selv} præsentere \ldots \act{husker pludselig} Nå, ja! \act{hyler
    sig selv op igen} \ldots præsenTEERE \ldots
  
  \says{RB}[hiver i ærme] Jamenjamen! Hvad med den der nye \ldots!
  
  \says{TM} Den der nye?
  
  \says{RB} Ja! Du ved \act{nikker mod bandscenen} den der nye \ldots
  
  \says{TM} Nåååårrrh! \act{til publikum} Som i ser! Har vi fået en ny
  bandscene! som \ldots
  
  \says{RB} Nejnejnejnej! Jeg mener {\it den der} nye \act{peger forbi
    bandscenen}.
  
  \says{TM} \act{Peger spørgende samme sted hen}
  
  \says{RB}[smiler, nikker overdrevet] Jaaah!
  
  \says{TM}[tænkende] Hmm. Ok. \act{til publikum} Som I ser er der
  \act{kigger på RB} I {\it forbindelse} med den nye bandscene
  \act{til publikum} kommet en ny nødudgang under bandscenen.
  \act{Tilbage i det glade emm-cee humør} OG NU! Vil jeg med stor
  fornøjelse præsentere den \ldots
  
  \says{RB}[ærmetræk] Heyheyhey! Hvad med \ldots?
  
  \says{TM}[afbryder, surt] Hvad nu!?
  
  \says{RB}[tænker sig lidt om] Øhhmm... Hmmm. Næh. Ik' noget.
  
  \says{TM} \act{kigger frem og tilbage mellem RB og publikum. Åbner
    og lukker munden et par gange. Giver lidt op --- siger monotont}
  \ldots præsentere den 30. DIKU revy.
  
  \scene{Lys ned, folk ud, tæppe fra, Intro!}

\end{sketch}
\end{document}
