\documentclass[a4paper,11pt]{article}

\usepackage{revy}
\usepackage[utf8]{inputenc}
\usepackage[T1]{fontenc}
\usepackage[danish]{babel}

\revyname{DIKUrevy}
\revyyear{2002}
\version{1.4}
\eta{6 min.}
\status{Færdig - nu med tvilling, korrekte regi-bemærkninger og rekvisitter}

\title{Tidsrejsen år 2002}
\author{marvin efter idé af Anders Sewerin Johansen}

\begin{document}
\maketitle

\begin{roles}
  \role{94}[Uhd] Revyt fra 1994
  \role{94a}[Anders] En anden revyt fra '94, ligner den første. 
  \role{T}[Torben] Torben ÆM
  \role{T2}[Klaus] \ldots og hans tvillingebror
  \role{02}[Jørgen] Revyt fra i år
  \role{48}[Heidi] En revyt fra 2048
  \role{Ø}[Jakob] Manden fra Ørred-nummeret
  \role{MPI}[Sidsel] Interface til mikroprofessor
\end{roles}

\begin{props}
  \prop{Lysblink, røg og bandlarm}
  \prop{En kasse til gamle rekvisitter /m ben}
  \prop{2 * Semantic Park t-shirts}
  \prop{Smart sølv-fremtidstøj}
  \prop{Robotten (til inder-face}
  \prop{Turban (til inder-face)}
  \prop{Et ark sort papir}
  \prop{En rød lampe (evt. en tegning af den)}
  \prop{En sort plastik-sæk}
  \prop{Quake-gøb, drysset med mel}
  \prop{Dekan-skæg}
  \prop{Mørk hudfarve}
\end{props}

\scene{Tæppet er for} 

\begin{sketch}
  

  \scene{\textbf{teknik}: Sketchen starter foran tæppet. Når
  skuespillerne er klar bag tæppet (check med tæppemester) skal der
  spot midt på tæppet hvor 94 skal stikke hovedet ud.}
  
  \says{94}[Stikker hovedet ud gennem tæppet] Undskyld, har nogen set et egern?
  
  \scene{94 forsvinder ind bag tæppet. Efter nogen tid kommer 02 ind
    med en kasse, stadig foran tæppet}
  
  \says{02} [Mod publikum] Kææære publikum. Som I allesammen ved, er
  dette den 30. DIKU-revy. I den anledning har vi gravet denne
  tidskapsel frem med minder fra de tidligere revyer.
  
  \says{02}[Tager en rød lampe frem] Her er for eksempel den røde lampe, som
  vi brugte før vi fik et fortæppe. Når lampen var tændt, var tæppet for, og
  man skulle lukke øjnene. Når lampen blev slukket, kunne man åbne øjnene
  igen.

  \says{02}[Tager en sort plastiksæk frem] Her er den originale
  plastiksæk/kondom-dragt, som Kristian Damm havde på, da han faldt ud over
  scenekanten\ldots
  
  \says{02}[Tager en quake-gøb og dekanskæg frem] Her er en gøb fra Quake for nogle år
  siden -- og det klassiske dekan-skæg.

  \says{02}[Peger om bag bandet] Og derovre er der en terminal, som desværre
  er placeret i en vinkel, som publikum ikke kan se\ldots
   
  
  \scene{Lysblink, bandlarm og røgskyer. Tæppet fra. 48 kommer frem
    fra bagved tæppet, gerne i en sky af røg}

  \scene{\textbf{teknik}: Tæppet går fra}

  \scene{\textbf{band}: Std. eksplosionsnuskerdernoget.bandlarm.net}
  
  \says{48} [Kommer ind på scenen] Æh davs, kan du sige mig, hvad
  klokken er?
  
  \says{02} [kigger på uret] Ja, det er år 2002.
  
  \says{48} Fint nok. Jeg er fra revyen år 2048. Vi manglede også et
  fyld-nummer. Og så fandt vi på at kopiere denne sketch.
  
  \says{02} 2048 er da ikke et jubilæumsår?
  
  \says{48} Nej, men der har været noget slemt overflow i
  beregningerne af jubilæumsår. Anyway: Desværre er kassen med gamle
  rekvisitter blevet væk.
  
  \says{02} Ja, jeg står altså lige og bruger den\ldots
  
  \says{48} Nåja, men min idé var at bruge min tidsmaskine til at
  rejse tilbage og hente gamle rekvisitter. Så er de også friske
  \act{banker på en af 02's rekvisitter, der støver meget}.
  
  \says{02} Hov, hov. Bare fordi vi ikke har en tidsmaskine.
  
  \says{48} Undskyld.
  
  \says{02} Nå, men hvordan er det egentlig på DIKU i år 2048?
  
  \says{48} Fint nok. Jeg har lige taget Dat 2 her til morgen.
  
  \says{02} Du mener vel \em{været} til Dat 2 her til morgen.
  
  \says{48} Nejnej, jeg tog det skam. Du ved, hvordan de skærer i
  pensum hvert år.
  
  \says{02} Bliver I så ikke meget hurtigt færdige?
  
  \says{48} Jo. Gennemsnitstiden for en datalog er nede på 9 år nu.
  
  \says{02} Hvordan kan det tage så lang tid med så lille pensum?
  
  \says{48} Vi har jo mange fag. I år har jeg
  Bio-øko-data-eskimoisk-kemi-optimering. \act{roder i lommen}
  
  \says{02} Hvad leder du efter?

  \says{48} Min mikroprofessor...
  
  \says{02} Du mener vel mikroPROCESSOR
  
  \says{48} Nej da - nå her var han...
  
  \says{02} [Kigger ned i 48s hånd] Den var godt nok lille. Hvordan
  bruger man sådan en svend?
  
  \says{48} Ole. Men det er nemt nok. Man skal bare bruge interfacet...

  \scene{MBI kommer ind i en ENORM kasse med blinkenlights}
  
  \says{94} [Kommer ind på scenen] Undskyld, har nogen set et egern?
  \act{går ledende omkring bag i scenen mens de andre taler videre}
  
  \says{02} [til 48] Hvem er det?
  
  \says{48} Jeg har taget nogle gamle revyttere med fra fortiden. Det
  er da lige sagen til et jubilæum
  
  \says{02} [bestyrtet] Det kan du da ikke? Det giver da allemulige
  tidsparadokser?!?!
  
  \says{48} Nu lyder du som en socio-nano-antropo-pæda-jurist-fysiker!
  
  \says{02} Nej, jeg mener, det er ligesom selvmodificerende kode!
  
  \says{48} Pjat!
  
  \says{T} [Kommer ind på scenen] Davs.
  
  \says{94+02+48} Dav Torben!
  
  \says{02} [Til 48] Hvilket år er han fra?
  
  \says{48} Det ved jeg ikke. Han har været med så mange år. Og han
  prøvede at optimere min tidsmaskine, så min log er gået i stykker.
  
  \says{T} Jah, etellerandet er nok gået galt. I den sidste revy, vi
  besøgte, så jeg foreksempel Mads Tofte på scenen.
  
  \says{02} Det er da meget normalt. Jeg har set ham i en gammel
  revy-video. Hvem skulle han forestille?
  
  \says{T} Ja, det er det, der er det mærkelige. Han prøvede at
  paroidere Arne Glenstrup...
  
  \says{02} Hmm. Der må være gået noget galt med tidsrækkefølgen. Bare
  der ikke er kommet en cykel i tidsgrafen.
  
  \says{94} [Kommer frem på scenen] Undskyld, har nogen set et egern?
  \act{går tilbage igen}
  
  \says{02} Der kan du se, der er allerede gået ged i tiden!
  
  \says{48} Pjat!
  
  \says{T} Jah, etellerandet er nok gået galt. I den sidste revy, vi
  besøgte, så jeg foreksempel Mads Tofte på scenen.
  
  \says{94} [Kommer frem på scenen] Undskyld, har nogen set et egern?
  \act{går tilbage igen}
  
  \says{02} Der kan du se, der er allerede gået ged i tiden!
  
  \says{48} Nej, han sagde egern!
  
  \says{Ø} [Kommer ind, gerne overdrevet langsomt] Ørred! \act{Går igen}
  
  \says{02} [hidsig] Lad nu være med at ødelægge mit nummer!
  \act{tager kassen frem igen, prøver at fortælle om rekvisitter}
  
  \says{48} Jamen, jeg skal også have noget med fra denne revy
  \act{trækker i 02}
  
  \says{02} Lad nu være med at ødelægge mit nummer!
  
  \says{T} Jah, etellerandet er nok gået galt. I den sidste revy, vi
  besøgte, så jeg foreksempel Mads Tofte på scenen.
  
  \says{02} [til 48, peger på T] Se nu, hvad du har gjort!
  
  \says{94} Undskyld, har nogen set et egern?
  
  \says{94a} [Kommer ind på scenen] Undskyld, har nogen set et egern?
  \act{går tilbage og leder sammen med 94}
  
  \says{94} Undskyld, har nogen set et egern? \act{går tilbage igen}
  
  \says{48} [Peger triumferende i retning af 94] Der kan du se, han
  sagde egern!
  
  \says{02} [prøver at skubbe folk væk] Lad nu være med at ødelægge
  mit nummer!
  
  \says{48} Jamen, jeg skal også have noget med fra denne revy
  \act{prøver at få hånden ned i kassen}
  
  \says{94} [Kommer frem på scenen] Undskyld, har nogen set et egern?
  \act{går tilbage igen}

  \says{T2} [Kommer ind på scenen] Davs.
  
  \says{94+02+48} Dav Torben!

  \says{T} Jah, etellerandet er nok gået galt. I den sidste revy, vi
  besøgte, så jeg foreksempel Mads Tofte på scenen.
  
  \says{T2} Jah, etellerandet er nok gået galt. I den sidste revy, vi
  besøgte, så jeg foreksempel Mads Tofte på scenen.

  \says{T} Davs
  \act{går ud igen}

  \says{02} Lad nu være med at ødelægge mit nummer!

  \says{48} [roder i kassen, finder et sort ark papir] Ah, her er det! Et klenodie
  fra alle revyerne: En sort sketch uden punchline!

  \scene{Tæppe, lys ud \em{hurtigt}}
  
\end{sketch}
\end{document}

%%% Local Variables: 
%%% mode: latex
%%% TeX-master: t
%%% End: 
