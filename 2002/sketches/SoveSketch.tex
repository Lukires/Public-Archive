\documentclass[a4paper,11pt]{article}

\usepackage{revy}
\usepackage[utf8]{inputenc}
\usepackage[T1]{fontenc}
\usepackage[danish]{babel}

\revyname{DIKUrevy}
\revyyear{2002}
\version{1.0}
\eta{2 -- 3 min.}
\status{Færdig}

\title{Sovesketch (tæppenummer)}
\author{Uffe Friis Lichtenberg}

\begin{document}
\maketitle

\begin{roles}
\role{P0}[Heidi] Person 0
\role{P1}[Jørgen] Person 1
\role{P2}[Maja] Person 2
\role{P3}[Sidsel] Person 3
\role{VO}[Peter] Voice over
\end{roles}

\begin{sketch}

\scene \textbf{teknik}: Lys på scenen når person har lagt sig. Check
med tæppemester.

\says{P0} \act{ligger pø scenen og sover}

\says{P1} \act{Kommer ind og kigger nysgerrigt og lidt forundret på
  P0. Så kigger han ud på publikum, undres lidt og kigger igen på den
  sovende person.}

\says{P1} \act{Meget forsigtigt føler han op gulvet, ligesom man ville
  føle om en madras var blød og rar. Lidt prøvende lægger han sig ned
  og falder prompte i søvn.}

\says{P2} \act{Kommer kort tid efter ind, fulde af energi og
  rigtigt glade. de ænser ikke P0 og P1, men sætter sig ned på scenen
  og kigger ivrige og engagerede ud på publikum. Uden varsel falder de
  i søvn i hinandens arme.}

\says{P3} \act{Kommer ind på scenen. Han skynder sig og kigger på sit
  ur, som om han er ved at komme for sent, og kommer da til at snuble
  over sine egne fødder --- lander lang som han er og er faldet i søvn
  før han rammer scenegulvet.}

\scene Kunstpause. (Den må godt trækkes lidt: ``det skal gå endnu
langsommere''.)

\scene Synkront vågner alle fem op og forlader salen groggy og forsovne.

\scene Kort pause.

\says{VO} Mine damer og herrer. De har netop overværet en forelæsning set fra forelæserens synspunkt.

\end{sketch}
\end{document}
