\documentclass[a4paper,11pt]{article}

\usepackage{revy}
\usepackage[utf8]{inputenc}
\usepackage[T1]{fontenc}
\usepackage[danish]{babel}

\revyname{DIKUrevy}
\revyyear{2002}
\version{1.0}
\eta{2,5 min.}
\status{Færdig}

\title{Tæppenummer: Brug Colaen}
\author{Carsten Bülow}

\begin{document}
\maketitle

\begin{roles}
\role{J} Jedi ridder 
\role{S} Snusket sælger 
\end{roles}

\begin{props}
\prop{En Pepsi}
\prop{Et lyssværd (ja, et af dem med lys i)}
\prop{Evt. Jedi kostume}
\end{props}

\begin{sketch}

\scene{J står afslappet på scenen og ser ud på publikum. S kommer
  luskende ind og scenen og stiller sig ved siden af J.} 

\says{S} [Kigger udspekuleret på J, og holder Pepsien frem mod ham] Vi
du købe en Pepsi?

\says{J} [Kigger over på S og lader den ene hånd glide forbi S's hoved. Med
overbevisende stemme] Du vil ikke sælge Pepsi.

\says{S} [Tager Pepsien hen til sig igen, med overbevisende stemme]
Jeg vil ikke sælge Pepsi.

\says{S}\act{Tager låget af Pepsien og skal lige til at drikke den}

\says{J} [Kigger over på S og lader den ene hånd glide forbi S's hoved. Med
overbevisende stemme] Du vil få dig et liv og drikke Coca-Cola.

\says{S} [Sætter låget på Pepsien igen. Med
overbevisende stemme] Jeg vil få mig et liv og drikke Coca-Cola.

\says{S}\act{Sætter Pepsien på scenen, og forlader den}

\scene{Lyset slukkes (der skal helst være så mørkt som muligt)}

\says{J}\act{Tager sit lyssværd frem og tænder det (det ser fint ud i mørket). Spiller smart med
  det et øjeblik og hugger så ud efter Pepsien. Slukker sværdet igen
  og forlader scenen (passende lydeffekter ville være perfekt)}

\scene{Slut}

\end{sketch}
\end{document}

%%% Local Variables: 
%%% mode: latex
%%% TeX-master: t
%%% End: 
