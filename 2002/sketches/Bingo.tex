\documentclass[a4paper,11pt]{article}

\usepackage{revy}
\usepackage[utf8]{inputenc}
\usepackage[T1]{fontenc}
\usepackage[danish]{babel}

\revyname{DIKUrevy}
\revyyear{2002}
\version{1.0}
\eta{1 -- 1,5 min.}
\status{Færdig}

\title{Bingo (tæppenummer)}
\author{Søren Horn Nielsen}

\begin{document}
\maketitle

\begin{roles}
\role{BO}[Agnete] Bingo-tal-oplæser
\end{roles}

\begin{sketch}
\scene{Tæppet for. Spot i midten. Bingoagtig taloplæser frem i midten}

\scene{\textbf{teknik}: Der vil kun være \'en person, midt på scenen.
  Spots ned over publikum en gang imellem (kun under taloplæsningen).}

\says{BO} Velkommen til herlig herlig bingo. Jegvil trække tal op af
posen, og i krydser af på jeres bingoplader. Man har først bingo, når
man har en hel række fuld. \act{kigger rundt på publikum.} Nå --- lad
os SÅ! komme igang.

\scene{I programmet er der en side med en bingoplade på. BO læser
  numre op fra brikker fra en pose, og publikum fylder sin bigoplade
  ud.}

\scene{Alle burde gerne få bingo på sammen tid. Her er der mulighed
  for publikum at larme lidt.}

\says{BO} Ihhhh \ldots sikke mange vindere vi har her i aften.
Aftenens præmie er \ldots EN SUPER FED DELTAGELSE I SOMMERFEST 2002
efter Revyen \ldots eller noget.

\scene{Lys ned.}
\end{sketch}
\end{document}
