\documentclass[a4paper,11pt]{article}

\usepackage{revy}
\usepackage[utf8]{inputenc}
\usepackage[T1]{fontenc}
\usepackage[danish]{babel}

\revyname{DIKUrevy}
\revyyear{2002}
\version{1.0}
\eta{3 min.}
\status{Færdig, kor skal placeres}

\title{Jen slide}
\author{Jakob Uhd Jepsen}
\melody{Tørfisk ``VMTJ (togsangen)''} 

\begin{document}
\maketitle

\begin{roles}
  \role{S}[Uhd] Sanger
  \role{K0}[Carsten] Kor 0
  \role{K1}[Søren H] Kor 1
  \role{K2}[Rune] Kor 2
  \role{K3}[Peter] Kor 3
\end{roles}

\begin{props}
  \prop{OHP}
  \prop{OHP-slides (6 * 15-16 stk.)}
  \prop{2 * Mørke bukser}
  \prop{2 * Cowboybukser}
  \prop{2 * Cowboyskjorter}
  \prop{2 * Sun-T-shirts}
\end{props}

\scene{S vil stå lidt til højre for midten af scenen. K* vil stå
  omkring en OHP lidt tilbage og lidt til venstre for midten af
  scenen. K0 + K1 står bagved og tæt på OHP, og K2 + k3 står foran og
  ca. 1,5 m fra OHP. Spot på S, lidt svagere lys på K* indtil omkvæd,
  hvor der begynder at ske nogle ting.}

\scene{Ved alle vers står K* sådan lidt udsmideragtigt (armene over
  kors og let spredte ben).}

\begin{song}
  
  \sings{S} En gang i mellem så sker det at
  æ forelæsning lidt vel hurtig går
  ja når æ mand han ta'r rigtig fat
  så er'd man æ så møj forstår
  ({\bf K*} {\it klør sig i håret med højre hånd})
  så er'd man æ så møj forstår
  ({\bf K*} {\it klør sig i håret med højre hånd})
  af havd der foregår
  
  \sings{S}[stigende tempo] 
  ({\bf K2 + K3} {\it lægger slides på og} {\bf K0 + K1} {\it tager slides af.})
  Jeeeeen slide og to og tre og fir
  der flyver rundt mej det papir
  jen hundred' stykker får man nået på jen jenste tim'
  transparenter er en ting som man ska bruge rigtig mange af før man ka bli'
  en forelæser her hos os
  
  \sings{S}[langsomt igen] At følge med i den rivne fart
  ja det en kunst der slet ik' er let
  æ forelæser han tror vel snart
  at undervise kan man ett
  ({\bf K*} {\it ryster på hovedet})
  at undervise kan man ett
  ({\bf K*} {\it ryster på hovedet})
  helt ud'n en overhead
  ({\bf K*} {\it klapper OHP med den hånd der er denne nærmest})
  
  \sings{S}[stigende tempo]
  ({\bf K2 + K3} {\it lægger slides på og} {\bf K0 + K1} {\it tager slides af.})
  Jeeeeen slide og to og tre og fir
  der flyver rundt mej det papir
  jen hundred' stykker får man nået på jen jenste tim'
  transparenter er en ting som man ska bruge rigtig mange af før man ka bli'
  en forelæser her hos os
  
  \sings{S} En gang imel' går ed rigtig gal
  så er ed lige til man får'n prop
  så sidder stille den hele sal
  og lytter skønt de vil sig stop
  ({\bf K*} {\it laver STOP-tegn med venstre hånd})
  og lytter skønt de vil sig stop
  ({\bf K*} {\it laver STOP-tegn med venstre hånd})
  på jen der læser op
  
  \sings{S}[stigende tempo]
  ({\bf K2 + K3} {\it lægger slides på og} {\bf K0 + K1} {\it tager slides af.})
  Jeeeeen slide og to og tre og fir
  der flyver rundt mej det papir
  jen hundred' stykker får man nået på jen jenste tim'
  

  transparenter er en ting som man ska bruge rigtig mange af før man ka bli'
  en forelæser her hos os

  {\it gentages ad nausium \ldots}
\end{song}
\end{document}


%%% Local Variables: 
%%% mode: latex
%%% TeX-master: t
%%% End: 
