\documentclass[a4paper,11pt]{article}

\usepackage{revy}
\usepackage[utf8]{inputenc}
\usepackage[T1]{fontenc}
\usepackage[danish]{babel}

\revyname{DIKUrevy}
\revyyear{2002}
\version{1.0}
\eta{3,5 min.}
\status{Færdig, sangerne skal fordeles}

\title{Bryg din egen kaffe selv}
\author{Jakob Uhd af Revyen}
\melody{"Dream a little dream of me"}

\begin{document}
\maketitle

\begin{roles}
  \role{S0}[Rune] Sanger 0
  \role{S1}[Jakob] Sanger 1
  \role{F0}[Sky] Fysiker 0
  \role{F1}[Carsten] Fysiker 1
  \role{F2}[Søren H] Fysiker 2
\end{roles}

\scene{Tre personer fra næste sketch (F0, F1, F2) vil være med som et
  slag publikum på scene fra starten.}

\scene{\textbf{teknik}: Personerne der sidder i baggrunden skal der
  ikke så meget lys på. Men når sangen er slut vil de rykke frem og så
  skal der lys på dem. Tæppet skal gå i bag dem når de er rykket langt
  nok frem på scenen.}

\begin{sketch}
  \says{F0}[lidt opgivende] Suk \ldots endnu en tirsdag på Caf\'een?.
  \says{F1} Ja --- der sker sgu' aldrig noget spændende!
  
  \scene{S0 og S1 kommer gående ind på scenen og gør klar til at
    begynde}
  
  \says{F2}[Begjstret] Hov vent \act{kunstpause} --- se nu der. Skal
  de ikke til at synge \act{leder efter ordene} live?
\end{sketch}

\begin{song}
  
  \sings{S}Når DIKU ligger øde
  er der et sted man altid kan møde
  de folk som bli'r, når dag går på hæld
  bryg din egen kaffe selv

  Der mødes dem man kender
  opvasken hilser på dig som venner
  husk blot en ting, der klarer du vel
  bryg din egen kaffe selv

  \sings{S}Lad blot fysiker sid' i rummet
  og drik' sig ihjel
  nej, DIKUs kantine den summer
  af liv og sjæl

  Når studiet er besværligt
  er det det sted der modtar' dig kærligt
  med mad og automatmaskinel
  men bryg din egen kaffe selv
  
\end{song}

\begin{sketch}
  \says{S0 + S1} \act{Bukker osv. går ud.}
  \says{F*} \act{Klapper og hujer sammen med publikum. Gør klar til at
  rejse sig op og gå ud til scenekanten.}
\end{sketch}
\end{document}
