\documentclass[danish]{article}
\usepackage{revy}
\usepackage[utf8]{inputenc}
\usepackage{babel}
\usepackage{a4wide}

\title{Dogmekodning}
\author{beyond, slammer, trauma, fluff, uffefl, larswm, heidi, m.fl.}

\version{2 -- færdig} % HUSK AT AJOURFØRE VERSIONSNUMMER!!

\revyyear{1999}

\begin{document}
\maketitle

\begin{sketch}

\begin{roles}
  \role{1} En formandsnørd
  \role{2} En nørdet nørd
  \role{3} En halvseriøs, men nørdet nørd
  \role{4} En tabernørd
\end{roles}

\says{1} Velkommen til pensumplanlægningsudvalgetsmødet. Som I alle ved har DIKU
jo ikke nogle penge i år. For at fortsætte undervisningen er det derfor nødvendgt
at foretage nogle få nedskæringer, på enkelte områder \act{deler papirerne ud} Hvis
I ser på mit forslag, d\'er midt på siden under overskriften "Dogmerapporter",
kan I se at vi er nået et stykke af vejen, men jeg er bange for at vi må gå lidt
mere drastisk til vers.

Skal vi tage dem punkt for punkt?

Punkt 1: Der må kun kodes i assembler. På denne måde sarer vi udgifter til
licenserne på oversætterværktøjerne.

\says{2} Jamen, hvordan kan vi det? Vi har jo ingen maskiner!

\says{1}[løfter fingeren, vidende] Hvilket bringer os ind på punkt nr.
2: Programmet skal kunne køre på de gamle motorolaer, der er hentet hjem fra
lossepladsen.

\says{2} Ah! Snedigt.

\says{1} Godt. Punkt nr. 3: Der må kun drikkes postevand i opgaveperioden.
\act{kigger rundt på folk}

\says{2} Så sparer vi strømmen til automaterne!

\says{3} Ja, og kantinens evindelig pantsvindproblem forsvinder også\ldots

\says{1} Nemlig. Og vi kan spare vores cola-web-site væk. Punkt 4: Programmet må
ikke postmodificeres.

\says{2}[spørgende] Dvs. debugging er ikke tilladt?

\says{1} Korrekt. Punkt nr. 5: Koden skal skrives med VI.

\says{2} Næh! Det er da for nemt. Koden skal skrives med cat eller echo.

\says{1} God pointe! \act{noterer det} Punkt nr. 6: Rapporten skal skrives i
Word Perfect version 1.3, vi har nemlig nogle gamle studenterlicenser, vi
kunne\ldots{} \act{bliver afbrudt}

\says{3}[afbryder] Nej nej! Postscript dokumenter skrives i hånden, på linieret
kladdepapir.

\says{1} Ja! Ja! Vi kan jo genbruge nogle game eksamensark. \act{noterer det}
Godt!  Punkt nr. 7: Illustrationer skal laves i MS Paint. Dog kan man søge
dispensation om at få lov til at bruge Advanced Art Studio eller Amica Paint.

\says{3} Jamen æh\ldots{} Kunne vi ikke bare bede dem om at udføre
illustrationerne på tavlen. Så kunne vi lave et reservationssystem, med den
gamle kernemasinereservationstavle.

\says{1} Tjoh\ldots{} men\ldots{} æhh\ldots{} \act{bliver afbrudt}

\says{2}[afbryder] Æh, når du siger ``Amica Paint'', mener du vel Deluxe Paint
IV? \act{sniff} Amigaen, se \emph{det} var en maskine der var noget ved\ldots{}

\says{1} Nej da! Advanced Art Studio og Amica Paint var de mest brugte
tegneprogrammer på C64'eren. Vi var jo \emph{nogle} der mindst kodede 4 timer
6510 kode hver dag! Og ja, Amica staves med ``C''.

\says{2} HA! De programmer må være kommet \emph{efter} det gode gamle Koala
Paint.  Den slags high-tech havde vi ikke noget af\ldots{}

\says{3}[afbryder] Skal vi ikke fortsætte med punkt nr. 8? "Der må ikke tages
backup."

\says{1}[får blod på tanden] Arhh\ldots{} High-tech og high-tech. Jeg undlod med
vilje at nævne FLI-graph, som jo laver grafik til en skærmmode der først blev
opfundet i '89. Kan holde 16 farver inden for en char\ldots{} \act{nostalgisk}
Revulotionerende\ldots{} \act{får besindelsen igen} Men det er jo alt for
avanceret til "Dogmerapporter".

\says{3}[ivrig] Ja, præcis! Jeg har en kommentar til punkt nr. 8: "Der må ikke
tages backup". Altså, vi har jo alligevel adrig taget backup i
opagaveperioderne, så det er vel overflødigt?

\says{1}[overvejende] Hmm\ldots{} Det kan vel egentlig godt være, men\ldots{}
[bliver afbrudt]

\says{2}[i en helt anden verden] Og vi skrev vores egne sprite-editrer i
maskinekode. Men ikke i 6510. Vi brugte 6502 kode, for det var nemlig den
allerrigtigste chip. Der var dårlige demokodere, der ikke fattede at man skulle
sætte farver i baggrundsfarve-registrene for skærmen, når man brugte
multi-color-chars\ldots{} Registre 53272 og 53270. Og vi skrev hex-koderne ind i
hånden, for det var før der kom kommercielle assemblere! Final Cartridge? Det
var for bøsser og folk der ikke kunne gennemføre spil uden cheat s!

\says{1}[nostalgisk] $0830 og $ea31\ldots{} De \emph{fede} tal!

\says{3}[pludselig revet med] Skrev i spriteeditorer? Der er sgudda snyd! Man
skal da tegne sin sprites på ternet papir, hvorefter man konverterer dem over
til datalinier i hånden og til sidst lave et lille basicprogram der poker dem
ind i computeren. Alt andet er da en utilladelig overskrivelse af
dogmekonceptet!

\says{4}[rigtig glad og kæk] Jeg havde en Amstrad!!!

\says{1,2,3}[kigger på 4, med afsky]\ldots

\says{1} Du læser \ldots{} fysik, ikke?

\scene eller

\says{4}[sur] Dengang jeg var barn havde vi kun en vissen pind til at tegne i gruset med

\says{1,2,3}[kigger på 4, med afsky]\ldots

\says{1} Du læser \ldots{} fysik, ikke?

\scene eller første scenarie efterfulgt af

\says{5}[kommer ind, megasur] Dengang jeg var barn havde vi kun en vissen pind til at tegne i gruset med

\says{6}[megababe, henter 5] Den har du da stadig, skat!

\end{sketch}

\end{document}
% Local Variables: 
% mode: latex
% TeX-master: t
% End: 
