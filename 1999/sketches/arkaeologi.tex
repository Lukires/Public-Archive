\documentclass[danish]{article}
\usepackage{revy}
\usepackage[utf8]{inputenc}
\usepackage{babel}
\usepackage{a4wide}

\title{Arkæologi}
\author{marvin}

\version{2 -- færdig} % HUSK AT AJOURFØRE VERSIONSNUMMER!!

\revyyear{1999}

\begin{document}
\maketitle

\begin{roles}
  \role{P} Professor Grydeske
  \role{A} Assistent paletkniv
  \role{H1-Hn} Hologrammer
  \role{s} Spøgelse
\end{roles}

\begin{props}
  \prop{lommelygter} 2 stk
  \prop{Skilt ``belgiSk oSte''}
  \prop{H-skilte til hologrammerne, evt H1 og H2.}
\end{props}

Det ville være herligt, hvis lydfolket kunne lægge rumklang på alt, hvad
spøgelset siger. Bemærk, at S i så fald skal træne i at tale tilpas langsomt.

\begin{sketch}

\scene Mørkt på scenen. P \& A ind med lommelygter.

\says{A} Endelig lykkedes det. Vi kom indenfor. 

\says{P} Ja. Det lader faktisk til at være det sagnomspundne DIKU. 

\says{A} Pyh, hvor var det hårdt. Først gravede vi os ned til det, hvor
   legenderne sagde, der var en dør, men den var uigenkaldeligt spærret.

   Så gravede vi på den anden side, og på døren der var der en hellig
   inskription, der advarede mod den almægtige gud, Al Arm!

   Jeg sagde jo, at vi bare skulle have startet med at hakke hul i taget.

\says{P} Ti stille!

\says{A} Dig med din overtro. Bare fordi det stod skrevet, at det almægtige DIKU
   ikke kunne lide hakkere\ldots{}

\says{P} Det var da rimelig nemt
   at finde, da vi først fandt kortet. Se her: Lige
   ved siden af HCØ-krateret og den mystiske monolit-kopi i røde mursten.

\says{A} Ja, det må have været en ordentlig eksplosion, der ramte det der HCØ
   dengang lige før år 2000\ldots{} Knald, mand!

\says{P} \act{går rundt på scenen} Gad vide, hvorfor disse prægtige
   bygninger blev forladt? alle oplysningerne gik jo tabt på grund af den
   defekte backup.

   Og rygterne kommer med mange forskellige forklaringer: Den Skumle Kælder,
   Rester fra den næsten lige så skumle kælder nedenunder Den Skumle Kælder,
   Rester fra den næsten lige så skumle kælder nedenunder Rester fra den næsten lige så skumle
   kælder nedenunder Den Skumle Kælder \ldots{}.

\says{A}[afbryder, lyser på skilt, siger tænksomt] Mystisk. Her står der "belgiSk oSt"?

\says{P} Det er ikke vigtigt nu. \act{trækker af sted med A}

\says{P} Se, her er der et dejligt stort rum. Gad vide, hvad det er, der knurrer
   bag disken derovre?

\says{A} Kakakakaka' det være de berygtede kakakakarklude? \act{Rædselsudbrud}

\says{P} Måske er vi i sikkerhed herovre ved alteret. Det lyser og så står der
   "Coc Acola".

\scene S kommer til syne ved siden af den imaginære colaautomat.

\says{A} Hjælp! Der er nogen! \act{Peger}

\says{S}[Med gravrøst] Hvem forstyrrer min evige grav?

\says{A}[Længerevarende rædselsudbrud, evt. med tigning om nåde] AAAargh!

\says{S} Kommer I for at\ldots{}. UNDERVISE?

\says{P}[Lettere befippet og skælvende stemme] Øøøeeeh\ldots{} Næh, ædle spøjelse.
   \act{Uskyldigt} Vi kom bare lige forbi\ldots{}hehe\ldots{}

\says{S} Fordømt. \act{slår sig på låret} Jeg kunne jo ellers godt bruge en underviser. Jeg er dømt
   til at hjemsøge dette sted, indtil jeg gør min uddannelse færdig. Og det er
   forbandet svært uden undervisere.

\says{P} Hvordan blev du fordømt\act{slår sig på låret} til det?

\says{S}[Lidt flov] Jah, det var jo min egen skyld. Jeg svor, at jeg ville blive
   færdig med mit studium, og at jeg ville være færdig hurtigere end
   gennemsnittet.

   Jeg havde jo bare glemt, at jeg selv tæller med i gennemsnittet. Så jeg
   bliver nødt til at blive færdig med mit studium, før jeg ved hvor længe, jeg
   skal være her. Jeg HADER rekursive definationer!

\says{A}[Friskt] Nå, men det var da heldigt. Kan du fortælle os om, hvad der
   skete med DIKU?

\says{S}[Kigger ondt på A, snakker hurtigt] Hmmm, nåja. Lad gå. Det startede altsammen helt vildt
   godt. DIKU var et herligt sted at være. SÅ tog den ONDE institutbestyrer magten, men det blev det kun
   værre af.

   Men det værste var dengang, politikerne begyndte at blande sig\ldots{}

\says{A} Hvordan det?

\says{S} Jo, politikerne havde jo fundet ud af, er der manglede IT-folk i
   Danmark. Derfor oprettede de et helt nyt center, som kostede en masse penge.

\says{P} Hvordan påvirkede det da DIKU?

\says{S}[Ivrig] Jo, planen var jo dobbeltsidig. Den nye IT-legeplads tiltrak en
   masse forskere fra DIKU. Men det smarteste var, at politikerne skaffede
   pengene ved at skære ned på rigtig datalogi. Så måtte DIKU fyre en masse
   undervisere, og så kunne de studerende ikke blive færdige. De droppede ud, og
   VUPTI, så var der flere tusinde ekstra IT-folk på markedet!

\says{P} Men det var vel ikke alle forskerne, der gik ITU?

\says{S} Nej, resten blev headhuntet til andre steder. DTU, Handelshøjskolen,
   ministerier og især det private. Jeg har vist en holovideo om det
   etellerandet sted \act{roder lidt, trykker på knap}

\says{H1, H2} \act{Kommer til syne, opfører Oracle-sangen. H1 synger, H2 er klædt ud som
   TUZ og går lidt forvirret omkring med en pibe}

\says{S} Se, på den måde forsvandt alle vores undervisere. Der gik lang tid, før
   vi opdagede det, for man kunne vel ikke ligefrem påstå, at de i forvejen var
   \emph{gode} undervisere\ldots{}
   
\says{P} Men hvad skete der så med resten af folkene på DIKU?

\says{S} Tjaaa, EDB-afdelingen flyttede nødstrømsanlægget fra serverne over til
   cola-automaten. Der gik lang tid, før vi opdagede det, for man kunne vel ikke
   ligefrem påstå, at serverne i forvejen havde \emph{høje} oppetider.

   Når maskinerne var nede, var der ikke så meget at lave, så efter kort tid
   stak resten TAP'er af i protest mod, at deres navneskilte hele tiden blev
   pyxlet.

   Så stak EDB-afdelingen af, fordi de ikke kunne få penge til nye maskiner. Man
   kaldte det \emph{hjerne}flugt. Der gik lang tid, før vi opdagede det, for man
   kunne vel ikke ligefrem påstå, at EDB-afdelingen\ldots{} \act{trækker på skuldrene}\ldots{}
   nåja.

\says{A} Hvad så med de studerende?

\says{S} På det tidpunkt blev cola-automaten ikke længere fyldt op, og så var der jo
   ikke andet for de studerende at gøre end at droppe ud af studiet og nedlade
   sig til at få fede stillinger med tårnhøje lønninger i det private.

\says{P+A} Fascinerende\ldots{}

\says{S} Nå, jeg må spøge videre. Det er nok bedst, at I går nu\ldots{}

\says{A} Jaja, tak for snakken. \act{Går målfast hen imod sceneudgangen}

\says{S}[Overrasket, advarende] Hey, ikke den dør. Det er en nødudgang, den kan
  man da ikke gå ud af.

\scene Lyd af alarm. Lys ud. Tæppe.

\end{sketch}

\end{document}
% Local Variables: 
% mode: latex
% TeX-master: t
% End: 



