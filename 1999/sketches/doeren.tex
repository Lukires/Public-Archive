\documentclass[danish]{article}
\usepackage{revy}
\usepackage[utf8]{inputenc}
\usepackage{babel}
\usepackage{a4wide}

\title{Dum som kun en dør}
\author{marvin}

\version{1.4 -- mærkelig, færdig?} % HUSK AT AJOURFØRE VERSIONSNUMMER!!

\revyyear{1999}

\begin{document}
\maketitle

\begin{roles}
  \role{Hv} 1 håndværker, i arbejdstøj, Magnus
  \role{D1,D2} 2 DIKUister D1 = Uffe, D2 = Agnete
  \role{V} 1 vagt, i Falsk/Securitas-uniform, Lars
  \role{T} 1 Tyv, i tyvetøj, Peter
\end{roles}

\begin{props}
  \prop{Dør} 1 Dør med snoretræk til åbning/lukning
  \prop{VVV} Vildt, voldsomt værktøj
  \prop{Smørekande}
  \prop{Magnetkort}
  \prop{Alarmboks} Selvstående, som et parkometer
  \prop{Skilt} Åbent universitet
  \prop{Skilt} ITU
  \prop{Alarmlyd} Evt. kan lyden "spilles" stumt af de optrædende (de holder sig
  for ørene etc.), hvis en rigtig alarmlyd ville forstyrre den øvrige handling
  for meget
\end{props}

\begin{sketch}
  
\scene Der er jo den her dør\ldots{} Denne sketch skal illustrere nogle af de
problemer, der har været med Døren.

Meningen er, at dette er en tillægssketch; den skal ikke opføres alene, men skal
foregå i baggrunden af en anden sketch, der mangler lidt bevægelse. Allerbedst
ville det være, hvis det var en løbende vittighed i løbet af e(t|n) akt. Døren
kunne så blive stående samme sted, enten i den ene side af bagscenen eller måske
ved siden af scenen. Det må nogen finde ud af.

Sketchen opføres i små, korte scener.
                
Døren er en vigtig rekvisit, så den skal laves solidt og nok i god tid, hvis vi
beslutter at gennemføre sketchen. Den skal forsynes med snoretræk, så
scenemesteren kan udvirke, at døren åbner og lukker sig "af sig selv", når dette
kræves i handlingen.

Der er ingen replikker i sketchen; det er en stumsketch.  Dette stiller særligt
store krav til skuespil og kropssprog.

---

\says{D1}Åbner Dør udefra og går igennem. Holder Dør åben for D2, og kigger efter hendes lille sexede bagdel.

\says{D2}Kommer fra den anden side og går igennem. Nikker til D1.

\says{Begge}Lukker Dør og fortsætter.

---

\says{D1}Kommer hen til Dør indefra. Rusker. Rusker *meget*. Opgiver. Går tilbage.

\says{D2}Kommer fra den anden side. Åbner Dør uden problemer og går igennem.

\says{D1}Har set D2 gå igennem. Laver spørgende fagter til D2.

\says{D2}Ser uforstående ud. Ryster på hovedet, trækker på skulderen.

\says{D1}Går forbitret hen til Dør indefra. Rusker. Gør truende gestus til Dør. Går.

\says{Hv}Kommer forbi med en masse ledninger. Skruer alarmboks fast. Går. 

---

\says{D1}Går hvileløst omkring indenfor. Ser ud til at have været spærret inde i lang
tid. Desperat.

\says{Hv}Kommer tilsyneladende tilfældigt forbi udefra. Opdager Dør, som om han aldrig
har set den før. Tager smørekande frem og smører Dør. Går.
        
\says{D1} Styrter glædesstrålende hen mod Dør. Sender fingerkys mod Hv.  Tager i
Dør.

\scene{Alarmen går}

\says{D1} Kaster sig mod gulvet, hænderne i vejret. Kigger lidt rundt. Da ingen
kommer, lister D1 sig ud af Dør.

\says{D2} Kommer indefra, mens alarmen stadigvæk kører. Stopper op, kigger på sit ur og stiller det. Går ud.

\says{V}[Senere] Kommer ugideligt ind på scenen udefra. Kigger sig omkring. Trækker på
skulderen og slukker alarm på alarmboksen, ved at slå på dem med sin MagLite. Han går ind og leder efter ``tyve''

---

\says{D1} Kommer udefra. Tager i Dør, som er låst. Slår med knyttede næver på
Dør.

\says{D2} Kommer også udefra. Tager magnetkort op, kører det igennem. Åbner Dør
og går ind.

\says{D1} Når ikke med ind. Vræler. Går nedtrykt bort.

---

\says{D1} Har fået magnetkort. Kommer udefra, kører kortet igennem. Går ind. Døren
lukker ikke.

\says{V} Ser, at D1 ikke lukker Dør. Udskældende fakter. Går demonstrativt hen
og lukker Dør.

\says{V} Går ud.

\says{D1} Går ind.

\scene{Lille pause}

\says{Dør} Åbner.

---

\says{Dør} Står åben.

\says{V} Trækker Hv hen til Dør. Peger forklarende.

\says{Hv} Nikker forstående. Bruger VVV på Dør. Lukker Dør. Kigger overlegent på
V.

\says{V \& Hv} Går ud igen.
\scene{Lille pause}
\says{Dør} Åbner.

---

\says{Dør} Står åben.

\says{Hv} Kommer udefra. Lukker Dør. Går ud.

\says{Dør} Åbner.

\says{Hv} Kommer. Lukker Dør.

\says{Dør} Åbner.

\says{Hv} Sætter hænderne i siden. Tager VVV frem og arbejder lidt. Tager Dør af
og går. (Alarmboksen står tilbage, på den hænger Hv skilt med ``Åbent Universitet'' )

---

\says{D2} Kommer. Ser forundret ud over at Dør mangler. Går ud uden problemer.

\says{D1} Kommer indefra. Ser mistænksom ud (er dette endnu en fælde?). Går Arne-agtigt
hen mod den manglende Dør.

\scene{I det øjeblik, D1 træder igennem døråbningen, går alarmen}

\says{D1} Slår febrilsk på alarmboksen.

\scene{Alarmen stopper}

\says{D1} Går veltilfreds ud.

---

\says{T} Lister tyve-agtigt ind, stjæler alarmboks, lister væk.

---

\says{Hv} Kommer tilbage med Dør. Sætter Dør i. Åbner og lukker Dør et par
gange. Ser veltilfreds ud. Tager i Dør for selv at komme ud. Rusker \emph{meget}.
Opgiver, går grædende bort.

---

\says{D1} Nærmer sig forsigtigt Dør.

\says{Dør} Går op af sig selv.

\says{D1} Ser meget mistroisk ud. Går alligevel langsomt igennem Dør.

\says{Dør} Idet D1 går igennem, lukker Dør. Åbner og lukker flere gange, som om
den gnasker på D1.

\says{D1} Dør.

\says{Hv} Kommer indefra med et ITU-skilt og hænger på døren. Træder over D1 uden at ænse ham.
---

\end{sketch}

\end{document}
% Local Variables: 
% mode: latex
% TeX-master: t
% End: 
