\documentclass[danish,11pt]{article}
\usepackage{revy}
\usepackage[utf8]{inputenc}
\usepackage{babel}
\usepackage{a4wide}

\title{Nyhederne}
\author{{\tt uffefl} \& {\tt beyond}}

\version{1.05 -- tæppenummer} % HUSK AT AJOURFØRE VERSIONSNUMMER!!

\revyyear{1998}

\begin{document}
\maketitle

\begin{roles}
  \role{O} Oplæser, med en oprigtigt uinteresseret nyhedsstemme
\end{roles}

\begin{sketch}
  \says{O} Du du du da da da da, du du du da da da da! Grafikradio! (men
  først nyhederne) Det forlyder at grafikkurset på Datalogisk Institut er
  blevet en katastrofe. ``Ja, kurset ligger lidt ude for normalen'' udtaler
  den kursusansvar(shavende/-lige), ``man skal heller ikke sætte sit lys
  under en skæppe, men vi søger nu i stedet en bedre forelæser, der ikke
  behøver at være Disjunkt eller Vektor ved DIKU, bare med nogle flottere
  kurver, hvilket jeg har ladet min højre hånd koordinere---så homogent som
  muligt\dots''.  Dekanen udtaler, at man ``efter at have nået dette
  midtpunkt skulle man forsøge at bortskære dele af kurset---ganske objektivt.
  
  Det forlyder at dekanen (\ldots) har skåret rusturene bort. I et
  telefoninterview udtaler dekanen ``Rusture, de skal skæres væk! \ldots''.
  ``Hvad skal der ske med savsmuldet fra alle disse nedskæringer?'' spørger
  vi os selv her på redaktionen. Et passende svar kunne være at der ikke er
  noget problem i den henseende---med det Cirkus dekanen holder sig.

Rapporten kort: ikke bestået, 03.

Førstedelsprognosen: det forventes at der på Dat0 kan forekomme en mild G-opgave
i efterårssemesteret. Der er en chance for at indholdet vil være kryptisk.
Opgaven forventes at vare ca. 3~uger. 

I løbet af semesteret er der også mulighed for en simulatoropgave.
Oversætteropgaven forventes igen at tildrage faglige aktioner. Datorologerne
forventer ikke nogle Zahlige databaseopgaver lige foreløbig.

I forårssemesteret er der risico for en kerneopgave og sprede algoritmiske
obligatorieopgaver. Objektivt set er der ikke noget nyt, da der ikke er
sket spor siden Jul. En lektor udtaler: ``Ehm, møh, \act{ralle, ralle},
Ja-hva' ka' man forvente?''

For interesserede kan vi oplyse at andendelsprognosen kan findes på Tekst-TV
side 102, 107, 193, 203, 204, 312, 409, 413, 415-420, 721, 743 og 809-944. Den
endelige prognose vil blive offentliggjort umiddelbart inden andendelsdagen --
kun få dage efter afleveringsfristen for
kursustilmeldingen. Andendelsprognosen kommer desuden til at ligge et eller
andet sted på nettet, f.eks.\
``www.ku.dk/nat/skat/katalog/fandens/\-til/\-langt/\-ned/i/et/underkatalog/slutprut.asp\%37463\-746374,777,453,\-øv-bøv-detherskal
-skrives\-ned,ellerskanduikkehuskedet\-,sut,sut,sut,djævle\act{pause)},stråhat'',
eller måske et andet sted, hvis det da ikke er blevet skåret bort.

Det var revyhederne, hav en fortsat god aften.

\end{sketch}

\end{document}
% Local Variables: 
% mode: latex
% TeX-master: t
% End: 
