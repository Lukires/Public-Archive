\documentclass[danish]{article}
\usepackage{revy}
\usepackage[utf8]{inputenc}
\usepackage{babel}
\usepackage{a4wide}

\title{Nørder De?}
\author{uffefl, trauma, beyond, agnete, runem, andresen, hynne, panic, fronck}

\version{4 -- FOR færdig} % HUSK AT AJOURFØRE VERSIONSNUMMER!!

\revyyear{1999}

\begin{document}
\maketitle

\begin{roles}
  \role{Søren} Cast-er
  \role{Brian} Kernighan
  \role{Dennis} Ritchie
  \role{Bjarne} Stroustrup
\end{roles}

% uden undtagelse, specielt når man handler ...
% overflov
% Syng kroen i ser = synkroniser
% gæt'char
% metoder = med to der
% pointer                                      container = kun tegner
% null pointer = nul pointer (altså: poænter)
% nedarvning
% argument
% få det i funktion
% udføre (hvad?)


\begin{sketch}
  
\scene Intromelodi til Jeopardy spilles. På scenen står en pult med Brian,
Dennis og Bjarne bag ved.

\says{Søren}[kommer ind] Hello, World!. Mit navn er Søren Cast-er og jeg
er jeres vært i denne, aftenens runde af ``Nørder De?''
Jeg vil som sædvanlig sørge for at krydre showet med nogle lette vittigheder og en hel stak pointer.
Med os iaften har vi Brian, Dennis og Bjarne, som vil udvikle C-ernes
paratviden.  Bjarne er vores stornørder fra sidste omgang, og vil
iaften forsøge at forsvare titlen.

\scene S går over til pulten.

\says{Søren}[til Brian] Brian, du er den første af vores nye udfordrere,
som skal forsøge at nedarve stornørder-titlen fra Bjarne.  Har du nogle
argumenter for hvordan du vil få sat det i funktion?

\says{Brian} Jeg regner med at lægge hårdt ud og vise min videns bredde
først.  Derefter vil jeg søge at [og] beskære de andre deltageres chancer.

\says{Søren}[til Brian] Ja, og er det ikke rigtig at du lige siden du
var en spæd nørd har beskæftiget dig med datalogi?

\says{Brian} Jo, jeg kunne kode før jeg kunne C. Faktisk har
jeg\ldots{}\act{bliver afbrudt}

\says{Søren} Ja, flot! \act{til Dennis} Og Dennis, du har også beskæftiget
dig med datalogien i mange år?

\says{Dennis}Ja, jeg kunne C før jeg kunne kravle.  Det var jo mig der
\ldots{} \act{bliver afbruft}

\says{Søren} Meget imponerende! \act{til Bjarne} Bjarne, du har jo holdt
titlen som overnørd i meget lang tid efterhånden.  Hvad er din hemmelighed?

\says{Bjarne} Tjoh, min hemmelighed er jo nok, at jeg kunne C før jeg ku'
plus'.  Plus min sans for \ldots{} \act{bliver afbrudt}

\says{Søren}[afbryder] Godt!  Skal vi komme igang, mine herrer?  Showet
bliver rodet hvis nørderne ikke får kodet!  Aftenens kategorier er:  
``Grafikfejl'', ``Kvalificering'', ``Array of char'', ``EDB-afdelingen'',
``Tal'', ``Filer'' og ``C64-musik''.

Husk mine herrer, at i ''Nørder De?'' er det \emph{mig} der har svarene og
\emph{jer} der stiller spørgsmålene.

Vi lægger ud med vores stornørd fra sidste uge. Bjarne, hvis du vil vælge en
kategori?

\says{Bjarne} Jeg tager ''filer'' til 128 kr.

\says{Søren}[vender skiltet] Svaret er: ''en nappet elektricitet.''

\says{Bjarne}[trykker med det samme] Hvad er bidtstrøm?

\says{Søren} Fuldstændigt korrekt. Bjarne lægger hårdt ud, kan vi C. Bjarne, du
fortsætter bare\ldots{}

\says{Bjarne}[har fået blod på tanden] ''Grafikfejl'' til 256 kr.

\says(Søren) Og svaret er:  ``Der kommer sådan nogle underlige firkanter''?

\says{Bjarne}[trykker med det samme] Hvad er JPEG

\says{Søren} Helt rigtig.  Forrygende start af vores stornørd.  Meen mon
dog denne lykke er uendelig?  Lad os tage den næste kategori og finde
ud af det!

\says{Bjarne}[ivrigt] ``Grafikfejl'' til 512 kr.

\says{Søren} Grafikfejl til 512 kr: ``Der kommer sådan nogle underlige firkanter''?

\scene Bjarne ser forvirret ud.

\says{Brian}[trykker] Hvad er MPEG?

\says{Søren} Korrekt. \emph{float} gået Brian. Brian, du må vælge en kategori.

\says{Brian} Øh, så vil jeg have ''Tal'' til 128 kr.

\says{Søren}[vender skiltet] Svaret er ''03''

\says{Brian}[trykker med det samme] Hvad er gennemsnittet for
kerneopgaver på DIKU?

\says{Søren} Rigtigt. Og næste kategori?

\says{Brian} ''Tal'' til 256 kr.

\says{Søren}[vender skiltet] Svaret er: ''6''.

\scene Brian, Dennis og Bjarne kigger uforstående på hinanden. Bååååt lyd.

\says{Søren} Hehe. Ja, C det får I nok aldrig. Brian, du må prøve
igen\ldots{}

\says{Brian}[forsigtig] Så må det blive ''EDB-afdelingen'' til 128 kr.

\says{Søren}[vender skiltet] Svaret er: ''NEJ!''

\says{Dennis}[trykker straks] Kommer Embla snart op igen så jeg kan
kode C?

\says{Søren} Nej, det gør den nok ikke før den \emph{får tran} at drikke,
men spørgsmålet er korrekt. Dennis kommer ind lidt C-nt i
kampen her, men du kan jo bare fortsætte?

\says{Dennis} Okay. Så vil jeg gerne have ''Tal'' til 512 kr.

\says{Søren}[vender skiltet] Svaret er: ''65246 2750''.

\says{Dennis}[trykker straks] Hvad er FEDE ABE?

\says{Søren} Det er nemlig rigtigt.  \emph{float} , \emph{float} Dennis!
Dennis, som faktisk også engang har været med i løkkehjulet, ikke sandt?

\says{Dennis} Joda!  Da vandt jeg en dejlig tem-plaid af ren ny uld, som \ldots{} \act{bliver afbrudt}

\says{Søren} Jatak, det er rigeligt. Og næste kategori?

%\hrule
%
%\textbf{Jeg synes egentlig ikke det følgende passer ind, selvom det er 
%  sjovt}
%
%\says{Dennis} Så skal vi høre noget ''C64 musik'' til 100 kr.
%
%\says{Søren} Meget lækkert. \act{vender skiltet}
%
%\scene Musik til Commando spiller.
%
%\says{Søren} Svaret er: ''Komponisten der lavede musikken til Commando''.
%
%\says{Bjarne}[trykker straks] Hvem er Rob Hubbard?
%
%\scene Brian og Dennis hviner af fryd.
%
%\hrule

\says{Dennis} ``kvalificering'' til 256 kr.

\says{Søren}[vender skiltet] Svaret er: ``en bunke tagsten.''

\says{Bjarne}[trykker] Hvad er en \emph{vold af tegl} \act{volatile}

\says{Søren} Det er nemlig rigtigt. Og vores stornørd Bjarne er
tilbage på banen.
Hvad er den næste kategori?

\says{Bjarne} Så skal vi have ``array af char'' til 1024 kr.

\says{Søren}[vender skiltet] OK.

\says{Søren} Og svaret er: ``barsk landarbejdersmykke.''

\says{Brian}[trykker straks] Hvad er \emph{strengkopiering}?

\says{Søren}[jublende] Korrekt, helt korrekt! En streng kopige-ring.

\says{Brian}[overmodig] Så skal vi gå-til C64-musik til 256 kr.

\says{Søren} Meget lækkert. \act{vender skiltet}

\scene Musik til Commando spiller.

\says{Søren} Svaret er: ''Komponisten der lavede musikken til Commando''.

\says{Bjarne}[trykker straks] Hvem er Rob Hubbard?

\scene Brian, Dennis og Søren hviner af fryd.

\says{Bjarne}[føler sig sikker] Lad os tage ''EDB-afdelingen'' til 512 kr.

\says{Søren}[vender skiltet] Svaret er: ''Ja.''

\scene Brian, Dennis og Bjarne kigger meget undrende på hinanden.

\says{Søren}[undskyldende] Uh. Ja. Det er jeg sørme ked af. Der må have 
sneget sig en fejl ind i vores kategoriprogram.

Men nu er det vist også blevet tid til finalerunden.

I finalen er kategorien ''void''. I skal nu bestemme hvor meget i vil satse, og
jeg læser så svaret op.

\scene Lille pause mens deltagerne skriver ned hvor meget de satser.

\says{Søren} Svaret er: ''Der er ingen.''

\scene Lidt Jeopardy musik mens folk skriver.

\scene KUN spotlys. Og kun på den der skal svare. 

\says{Søren} Så er vi vist ved at være klar. Og Brian, hvor meget satsede du?

\says{Brian} Jeg satsede det hele -- 8192 kr.

\says{Søren} Meget flot. Og må vi C dit spørgsmål?

\says{Brian}[vender sit skilt] Hvilke lektorer er der tilbage på DIKU?

\says{Søren} Tæt på. Men det passer jo kun næsten\ldots

Og Dennis, hvor meget har du satset?

\says{Dennis} Jeg har satset 2048 kr.

\says{Søren} Ja, Dennis. Du har jo heller ikke mere. Dit spørgsmål?

\says{Dennis} Hvor er backuppen?

\says{Søren} Nej, ''hvor'' er vores web-server. \act{pause, så publikum
  fatter det} Og iøvrigt skal det formuleres som et spørgsmål.

Så Bjarne, hvor meget satsede du?

\says{Bjarne} Jeg satsede 4 kr.

\says{Søren} \ldots{}øhm, det var vel nok -- øhm -- latterligt\ldots{}
Well, hvad er dit spørsmål til ``der er ingen''?

\says{Bjarne} Hvor i denne sketch er der pointer [pøjnter] \ldots{} øh, jeg mener POINTER [poænter]?

\scene TÆPPE!!!

\end{sketch}

\end{document}
% Local Variables: 
% mode: latex
% TeX-master: t
% End: 
