\documentclass[a4paper,11pt]{article}

\usepackage{revy}
\usepackage[utf8]{inputenc}
\usepackage[T1]{fontenc}
\usepackage[danish]{babel}


\revyname{DIKUrevy}
\revyyear{1981}
% HUSK AT OPDATERE VERSIONSNUMMER
\version{0.1}
\eta{$n$ minutter}
\status{Færdig}

\title{Sygdomssketch}
\author{?}

\begin{document}
\maketitle

\begin{roles}
\role{M}[] Manden
\role{S1}[] Sygeplejerske
\role{S2}[] Sygeplejerske
\role{L1}[] Læge
\role{L2}[] Læge
\role{L3}[] Læge
\end{roles}

\begin{sketch}

  \scene{Scenen er indrettet som et venteværelse, lægekontor eller
    lignende.  Til at begynde med er der ingen mennesker på scenen.
    Lidt efter kommer to sygeplejersker der støtter en mand der går
    meget dårligt, og som hele tiden ømmer sig.}

  \says{M} Avv... åh... det gør sørme ond.  Av!! Auv... av
  av... \act{Han fortsætter med at jamre stille mens den første
    halvdel af sketchen kører.}

  \says{S1} Så hr. Hansen, De kan sidde og sunde Dem lidt her.

  \says{S2} For en sikkerheds skyld må vi hellere lade Dr. Jensen se
  ind til dem.

  \scene{Begge sygeplejersker går ud.  Lidt efter træder en hvidkitlet
    person ind til dem.}

  \says{L1} Goddag, jeg er Dr. Jensen.  \act{Han træder hen til
    patienten.}  Åben munden og sig ah.

  \says{M}[som oven på en bedre middag] Ahh.

  \scene{L1 måler manden med et stetostkop oven på hovedet og
    lign. vanvittige steder.}

  \says{L1} Jatak, De har en slævt hævet Dijkstra.

  \says{M} A... hva...

  \scene{Imens træder en hvidkitlet person ind.}

  \says{L2} Goddag, jeg er Dr. Jensen.

  \scene{L2 kigger på sin lommeterminal, og går hen til patienten.}

  \says{L2} Nå, så det er dem der lider af Hamiltonske kredsløbssygdomme.

  \says{M} Hva... Æh...

  \says{L1} Ja, for ikke at tale om hans hævede Dijkstra.

  \scene{De begynder begge at taste oplysninger ind på deres
    lommeterminaler.  Imens træder en hvidkitlet person ind.  Han
    kigger på sin lommeterminal.}

  \says{L3}[henvendt til de andre læger] Er det her at der er en
  patient, der skal have lagt en heuristik?

  \says{L2} Vi må også hellere fjerne hans semaforer.

  \says{L1} Så må vi først lave en grundig operationsanalyse.

  \says{L2} Jamen det vil kræve en kompliceret fouriertransformation.

  \says{L1} Det kan vi ikke.  Det kræver at patienten har været
  kontekstfri i en uge.

  \says{L3} Kan vi udføre en V-operation i stedet?

  \says{L1} Ikke uden dobbelt præcision.

  \says{L2} Det duer ikke når patienten samtidigt har en malfunktion i
  semantikken.

  \says{L1} Har han da det?

  \says{L2} Ja, men det skyldes også at der er mange slemme databærere
  i luften for tiden.

  \says{L3} Uha, det kan nemt give redundans i hjertefunktionen.

  \says{L1} Men det kan behandles med en kraftig data-reduktion.

  \says{L3} Det er rigtigt, men bagefter kan patienten blive svær at normalisere.

  \says{L2} Han kan begynde at simulere.

  \says{L1}[med gys i stemmen] Tidstro simulering.

  \scene{På dette tidspunkt rejser manden sig op.}

  \says{M} Sig mig, hvad sker her?

  \says{L2} Vi er ved at nå ind til kernen af deres tilstand.

  \says{M} Men de tar helt fejl.  Jeg er bare snublet ude på gangen,
  da jeg kom for at hente Deres computer.  Den er nemlig defekt.

  \scene{Manden forlader scenen.}

  \says{L1} Hente vores computer.

  \says{L2} Som er defekt.

  \scene{Alle tre læger kigger skrækslagne på deres lommeterminaler.}

  \says{L3} HJÆÆLP!!

  \scene{De løber alle tre ud fra scenen.}

\end{sketch}
\end{document}
