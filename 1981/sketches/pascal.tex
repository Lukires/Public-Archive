\documentclass[a4paper,11pt]{article}

\usepackage{revy}
\usepackage[utf8]{inputenc}
\usepackage[T1]{fontenc}
\usepackage[danish]{babel}


\revyname{DIKUrevy}
\revyyear{1981}
% HUSK AT OPDATERE VERSIONSNUMMER
\version{0.1}
\eta{$n$ minutter}
\status{Ikke færdig}

\title{PASCAL-Sketch}
\author{Berit og Satdeva}

\begin{document}
\maketitle

\begin{roles}
\role{F}[] Fanatisk person
\role{P1}[] Plejer
\role{P2}[] Plejer
\end{roles}

\begin{sketch}

  \scene{En meget fanatisk person (evt. i fangedragt eller pyjamas) er
    undsluppet fra en sindssygeanstalt.  Han kommer farende ind på
    scenen og afbryder et nummer.  Han henvender sig til publikum.}

  \says{F} De er efter mig! Jeg er nemlig genial, mine idéer vil kunne
  revolutionere hele EDB-verdenen.  Det er det de er bange for, de
  skiderikker.  EDB som det er i dag, er Deres eneste chance for at
  beholde magten.

  \scene{Han ser sig lidt jaget omkring og henvender sig igen til
    publikum, denne gang lettere konspiratorisk:}

  \says{F} Vil I vide hvad jeg har lavet.  Jeg har lavet nogle geniale
  udvidelser til {\em alle} højere programmeringssprog.  Det vil løse
  alle programmørens problemer.

  \says{F}[nu ivrigt] Se her, alle programmeringssprog har altid
  manglet fleksibilitet.  <Sættes lig med> denne programfacilitet har
  jeg udvidet med udtrykket <sættes omtrent lig med>.  Dette får bugt
  med alle smålige hensyn til små unøjagtigheder.

  \says{F} Vi ved jo alle at de der maskiner er meget klogere end os.
  Det er det vi skal benytte os af.

  \says{F} Tag nu {\tt IF}-sætningen.  Den er jo komplet gammeldags.
  Den mangler jo fuldstændig faciliteter.  Næh, se nu dette eksempel:

\scene{Vises på overhead.

  \begin{verbatim}
    if G <> A * 2 maybe
      A := B
    or perhaps
      C := D
    otherwise try
      E := F
  \end{verbatim}
}

\says{F} Tænk jer nogle skønne programmer der kan skrives med en sådan
sætning.

\says{F} Eller prøv og se på en traditionel tællesætning.  Den er {\em
  håbløs}.  Alle ved jo at det er svært at finde ud af {\em præcis}
hvor mange gange løkken skal gennemløbes.  En tælleløkke burde se
således ud:

\scene{ Vises på overhead.

\begin{verbatim}
for N := 0 to about 100
begin.
  x := x + 1;
  y := y + 1;
end
\end{verbatim}
}

\says{F} Men det er jo for nemt for dem.  Det er derfor de ikke vil lytte til mig.

\says{F} Det er det samme med funktionen {\tt comefrom}, som kan
fortælle programmøren hvor han {\em burde} være kommet fra når han har
hoppet rundt i sit program.  Det er sådan noget der gør, at de ikke
vil lytte på mig.  Det er for farligt for dem.

\says{F} Eller hvad med de programmer der kører for langsomt.  Det
problem kan vi løs emed kommandoen {\tt speed up}.  Det vil fjerne
alle effektivitetsproblemer, og det kan gøres ved, at kun hver anden
sætning udføres.  Tænk jer-- I kan få udført en uendelig løkke på den
halve tid.

\says{F} Eller tag mit forslag til en ny type {\tt DIMENSIONSLESS}.
Den kan bruges til at definere punkter i matematiske programmer.  I
kan så ved hjælp af disse punkter definere linier, planer og ruum.

\says{F} Kan I ikke se hvad det kan føre til?  Jeg arbejder nu på en
ny type {\tt GALAKSE}, hvor vi ikke alene kan erklære rum, men også
planeter og solsystemer.

\says{F} Prøv og tænk jer et system.  Jeg har regnet det hele ud.
Systemet kan blive driftsklar på 7 dage.  De 6 første skal bruges til
initiering, og den 7. dag skal bruge til review.

\says{F} Jeg siger jer... det bliver et system.

\scene{To hvidkitlede (el. lign. påklædning) personer kommer ind på
  scenen, og tager manden under armene.}

\says{P1} Så, så Hr. Madsen, så tar vi tilbage på klinikken.

\says{F}[råb fra kulissen] Det er det jeg siger.  De vil ikke lytte på
mig.

\end{sketch}
\end{document}
