\documentclass[a4paper,11pt]{article}

\usepackage{revy}
\usepackage[utf8]{inputenc}
\usepackage[T1]{fontenc}
\usepackage[danish]{babel}


\revyname{DIKUrevy}
\revyyear{1981}
\version{0.1}
\eta{$n$ minutter}
\status{Færdig}

\title{Ekstranummer}
\author{?}
\melody{The Wall}

\begin{document}
\maketitle

\begin{roles}
\role{S}[] Sanger
\end{roles}

Imens folk i salen råber ekstranummer banker Knud i stortrommen, når
han synes de har skreget længe nok følger endnu en tromme med, på
næste takt singleguitaren og på næste takt bassen.  På den derefter
følgende takt falder vi ind med:

\begin{song}
  Ikke flere ekstranumre
  ikke mere dans og sang
  vi er færdig' med at fjumre
  ikke mere denne gang
  (1-2-3-4, 2-2-3-4, 3-2)
  HEJ, FOLKENS, se så og vågn op
  /: så få fing'ren ud og se og kom i galop :/

  Se så og kom ud af røret
  I skal til at fulde jer
  op på DIKU og bli' pløret
  der er ingen bajer' her
  (1-2-3-4, 2-2-3-4, 3-2)
  HEJ, FOLKENS, se så og vågn op
  /: få så fingren ud og se og kom' i galop :/

  Slut forbi, det er endt, ovre
  finis, færdig, ikke mer
  tror I det kan blive sjov're
  venner, så' det ikke her
  (1-2-3-4, 2-2-3-4, 3-2)
  HEJ, FOLKENS, se så og vågn op
  /: få så fingren ud og se og kom i galop :/
\end{song}

(Og lad så vær med at se forbavset ud når orkestret stopper præcist på
"`galop"')

\end{document}

