\documentclass[a4paper,11pt]{article}

\usepackage{revy}
\usepackage[utf8]{inputenc}
\usepackage[T1]{fontenc}
\usepackage[danish]{babel}


\revyname{DIKUrevy}
\revyyear{1981}
\version{0.1}
\eta{$n$ minutter}
\status{Ikke faerdig}

\title{Borgmestervisen}
\author{IBO, TAH og HVO}
\melody{Skilfinger: ``Borgmestervisen''}

\begin{document}
\maketitle

\begin{roles}
\role{S}[] Sanger
\end{roles}

\begin{song}
Der var en mand på landet
som var EDB-programmør bl.a.
at ha' struktur er programmørens fag
og manden fik en ny hver eneste dag
/: han skrev det hele sammen til et "`smukt"' program
det blev kørt på Recku, jo han kunne sit kram :/

Han prøved' at køre det gang på gang
men de var mange om maskinen så tiden var lang
han regned' igennem og tested af
og så på de ting som programmet ham gav
/: hvor godt hans tal var blevet sammentalt
og det var jo i grunden ikke så galt :/

Javist, således siger han for sig selv
systemprogrammør blir jeg med lidt held
jeg vil sidde solidt på databasen
det blir mig der bestemmer når der åbnes for gassen
/: jeg vil føre mig frem li'som Peter Naur
og alting blir udtrykt i danske termer :/

Jeg lukker en abis, ja tænk jer bare
at fyre typografer det er holdne varer
skal en virksomhed betale for forældet teknik
typografer skal væk det almindelig logik
/: ledelsen blir glad når jeg sparer dem ud
jeg blir så populær, du søde gud :/

Så blir jeg forfremmet og skifter politik
til den berømte socioteknik
extern konsulent blir mit nye job
hvorved min løn sættes klækkeligt op
/: døren står åben til mit kontor
men jeg retter mig sjældent efter brugerens ord :/

Men bedst som det kørte gik maskinen i stå
strukturer og bit i muddergrøften lå
med dem den hele lyksagelighed faldt
som den gamle digter engang har fortalt

{\scene Efterspil:}

Men drømme forgår nu ikke så snart
selvom de blir dyppet i muddergrøftens pløre
snart var han ansat i et lektorat
og nu er han ikke længere bare programmør
\end{song}

\end{document}

