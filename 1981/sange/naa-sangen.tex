\documentclass[a4paper,11pt]{article}

\usepackage{revy}
\usepackage[utf8]{inputenc}
\usepackage[T1]{fontenc}
\usepackage[danish]{babel}


\revyname{DIKUrevy}
\revyyear{1981}
\version{0.1}
\eta{$n$ minutter}
\status{Ikke faerdig}

\title{Nå-sangen}
\author{}
\melody{?}

\begin{document}
\maketitle

\begin{roles}
\role{S}[] Sanger
\end{roles}

\begin{song}
Vores livs sejlads i institut terræn
den er vidt forskellig fra en søndagsroning
skal det skildres i et ganske kort refræk
må der hjælpes med en vekslende betoning
det gør vi så--hvad skal vi ta'
det lille: Nå

Man gled ind på instituttets bænkerad
bare gab.  Så passer læreren kundskabspumpen
med tortur som rekursion af tredie grad
man får hård hud dels på sjælen dels på rumpen
Der stod man så - og gloede dumt
så sa man nå

På dat1 man sad ved store menskers knæ
som et lille dumt og snottet, grimt spektakel
og de sa at noet var mam og noet var bæ
Naurs ord var et mirakuløst orakel
Der sad man så og lytted spændt
så sa man: Nå

Telefonen ringer, men hvor er de henne
lærerne forsker hjemme eller er i gården
hvis de kommer er der noget vi skal sende
og en bunke der skal skrives til i morgen
det må man så.  Et lille smil
et venligt: Nå

Alting kommer først i allersidste stund
sendt til censor, tryk det nu og frankere
og kopimaskinen tar sig tit et blund
ja, især når studenter afleverer
Hvad ska man så helt slappe af
og sige: Nå
\end{song}

\end{document}

