\documentclass[a4paper,11pt]{article}

\usepackage{revy}
\usepackage[utf8]{inputenc}
\usepackage[T1]{fontenc}
\usepackage[danish]{babel}


\revyname{DIKUrevy}
\revyyear{1982}
\version{1.0}
\eta{$n$ minutter}
\status{Færdig}

\title{Slapper-sang}
\author{Satdeva, BO, HVO, MMJ}
\melody{Shu-bi-dua: ``Bageren og servitricen''}

\begin{document}
\maketitle

\begin{roles}
    \role{S}[] Sanger (dreng)
    \role{T}[] Travl (pige)
\end{roles}

\begin{song}
  \scene{2 personer i Kantinen - hver sin side af scenen - hun ved terminal, han i bløde stole med et rundt bord,
         han synger til hende, hun sidder og taster, slutning ? !}

  \sings{S} Syn's du ikke du sku' slukke for maskinen?
            Glem dit sidste kodestykke.
            Hjælp mig nu med at bryde fra rutinen,
            jeg ku' tænke mig at hygge nu.

  \sings{S} Kom hen til mig,
            kom hen til mig.
            Du trænger sgu nok til en rus.
            Sæt dig hos mig,
            sæt dig hos mig,
            så gi'r jeg en pils og et knus.

  \sings{S} Lad nu studierne være, for fanden,
            du har knoklet alt for længe.
            Kom her hen, lad os mærke hinanden,
            der er andet til end penge.

  \sings{S} Kom hen til mig,
            kom hen til mig.
            Du trænger sgu nok til en rus.
            Sæt dig hos mig,
            sæt dig hos mig,
            så gi'r jeg en pils og et knus.

  \sings{S} Kom nu her, lad mig føle du kan kræve,
            lad det komme helt fra sjælen.
            Slip din skærm, lad dog bittene svæve
            og erkend at du bli'r kælen.

  \sings{S} Kom hen til mig,
            kom hen til mig.
            Du trænger sgu nok til en rus.
            Sæt dig hos mig,
            sæt dig hos mig,
            så gi'r jeg en pils og et knus.
\end{song}

\end{document}

