\documentclass[a4paper,11pt]{article}

\usepackage{revy}
\usepackage[utf8]{inputenc}
\usepackage[T1]{fontenc}
\usepackage[danish]{babel}


\revyname{DIKUrevy}
\revyyear{1982}
\version{1.0}
\eta{$n$ minutter}
\status{Færdig}

\title{Berits Ballade}
\author{Ukendt}
\melody{Ukendt}

\begin{document}
\maketitle

\begin{roles}
\role{B}[] Berit 
\end{roles}

\begin{song}
  \sings{B} Dette er historien om en kærlighed til DIKU
            som blev grundlagt i de første studieår.
            Rundt på RECKU hele dagen
            COBOL det var lige sagen
            træthed, raseri, når de sa' stop
            DIKU var vor fælles sag
            studenterne fra alle lag
            stod sammen for at kæmpe for vort fag
            I dagligdagens virkelighed
            med studiet som en mulighed
            i håb om drømmen bli'r til virkelighed.

  \sings{B} Dette er historien om en kærlighed til DIKU
            som slog revner efter et par studieår.
            Eksamensræs tog overhånd
            da censor lagde hårde bånd
            rapportarbejdet blev til tyranni.
            DIKU var vor fælles sag
            ordene gav dårlig smag
            vi kappes om at få en konsulent.
            Fra dagligdagens rodløshed
            til studieplanens modløshed
            i håb om drømmen bli'r til virkelighed.

  \sings{B} Dette er historien om en kærlighed til DIKU
            som blev svigtet, da ballonen endelig sprang.
            Da ideal om sammenhold
            var ødelagt på alle hold
            og eneste motiv var præstation.
            DIKU var vor fælles sag
            men staten den fik overtag
            og studieplanerne blev hulet ud.
            Fra dagligdagens tyranni
            til studietidens hykleri
            i håb om drømmen bli'r til virkelighed.

  \sings{B} Dette er historien om en kærlighed til DIKU
            som blev dræbt, men som skal bygges op påny.
            Eliteræsets snæverhed
            mod fællesskabets rum'lighed
            bli'r kampen som vi alle må ta' op.
            DIKU er vor fælles sag
            studenterne fra alle lag
            forsamles overalt på 1. sal.
            I dagligdagens virkelighed
            med studiet som en mulighed
            i håb om drømmen bli'r til virkelighed.
\end{song}

\end{document}

