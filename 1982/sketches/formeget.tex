\documentclass[a4paper,11pt]{article}

\usepackage{revy}
\usepackage[utf8]{inputenc}
\usepackage[T1]{fontenc}
\usepackage[danish]{babel}

\revyname{DIKUrevy}
\revyyear{1982}
% HUSK AT OPDATERE VERSIONSNUMMER
\version{1.0}
\eta{$n$ minutter}
\status{Færdig}

\title{For meget}
\author{HH}
\begin{document}
\maketitle

\begin{roles}
\role{S}[?] Skuespiller 
\end{roles}


\begin{sketch}
    \scene{Inspireret af Dan Turélls ``Det er ikke let''}

\says{S} Det er ikke let

         De fleste på DIKU synes ofte det hele simpelthen bliver for
                meget, og standser op midt i deres projekter
                i en travl hverdag fordi det ikke er let a
                være dem og have så mange problemer
         og hvis bare de andre foretog sig lidt mere og hvis der bare
                var lidt mere drøn på dem
         så ville det ikke altid være så nødvendigt at man selv kørte
                det hele og skulle nå så meget

         Det er IKKE let
         
         og hver gang man kommer til at tænke sådan
         så skal man gøre som min gamle guru PETER NAUR engang
                sagde til mig efter en autoforelæsning på RECKU
         så skal man prøve at skifte sig selv ud med en anden
         man skal vælge et billede ud af hverdagsdikuhistoriens
                richs album
         og forestille sig hvordan det er at være vedkommende
         tænk på hvordan det er
         at være for eksempel Dat-0'er

         Det er ikke let at være Dat-0'er
         så bor man i et hus med mor i Sorgenfri
                eller på Grønjorden med SU og man kommer på HCØ
                i det store auditorium og i Vandrehallen
         og alle andre omkring en er osse Dat-0'ere, eller matematikere
         og når man endelig kommer på DIKU
                i Kantinen eller i Hullestuen
         så er der ingen man kender
                eller hvis man endelig er i en gruppe så er man bange
         for de andre kan meget mer end en selv -
         og maskinen er der jo altid og kræver
                en fuldstændig
         og på væggen hænger stadigvæk udskriften af det første
                program der virkede
         og man troede at sådan ville det blive ved
                og det kunne altid køre
                anden gang og nu gider programmeringsvagten ikke engang
                se på det selv om man siger
                at man tror man har fundet en fejl i oversætteren
         og man er altid alene med sine problemer
         og man synes det kører for dårligt

         Det er ikke let at være Dat-0'er -

         og hvis det så endelig lykkes en at få en anden Dat-0'er
                til at se på det og man så for en gangs skyld
                forstår hvad han siger og man
                forstår at der ikke må være semicolon foran else
         og når man for en gangs skyld osse kan snakke om hvordan det
                er at være Dat-0'er
         og man finder ud af at man ikke er den eneste der føler sig
                udenfor og set ned på og han siger han også er
                bange for at sige han gerne vil i gruppe
         og man finder ud af at man gerne vil i gruppe sammen
                og man går i gruppe sammen
                og i gruppen giver man hinanden styrke
         og man vil gå ind i de styrende organer med alle
                pamperne så man kan fortælle dem at man
                eksisterer, man er mange, man har et
                behov og man har ret til at være med
         og man går ind i revygruppen og kantineforeningen
                sammen
         og man bliver gode venner med nogen af pamperne
                man drikker øl
                man spiller kort
                og man ta'r med på travetur
         og SÅ er det man ser på listen på opslagstavlen
                at man er dumpet og man
                ser meddelelsen om at man derfor
                beklageligvis ikke kan forvente at blive
                optaget på 2. del før tidligst i efteråret 1994
                Måske
         og når man ser det og man
                indser at der røg ens fremtid
         så er det ikke let at være Dat-0'er

         Men det er heller ikke let at være den tapper der skal sætte
                den liste op på opslagstavlen
         tapperen har en anelse om at de fleste Dat-0'ere
                gerne vil på 2. del at de gerne vil være dataloger
         hun har set dem sidde i Kantinen så mange gange med deres
                fedtede sandkasseprint
         hun har sine anelser og hun ville så
                gerne skåne de stakkels Dat-0'ere
         men hun er nødt til at passe sit job
                og sætte listen op
         ellers går det ud over hendes fremtid
         og tapperen startede ellers i sin tid muntert snakkende
                på jobbet og
                glædede sig til at skrive breve rent
                og sætte beskeder og lister op
         men efterhånden har hun fået dårlig samvittighed over det
         hun synes ikke det hun skriver er andet end skidt
         hun synes det hele er
                dumpelister og betal-for-dit-kursus-beskeder
                og forsinkede noter, dårlige kladder og breve
                om knappe resurser og nedskæringer
         og tapperen synes ikke noget af det
                gør nogen stærkere eller klogere eller gladere
         og når man først har set i øjnene hvad det er man skriver
         og man har indset at det eneste gode
                er kaffetimen om morgenen
                hos Gurli i omstillingen
         OG Gurli så REJSER
         Så er det ikke let at være tapper

         Det er heller ikke let at være 2.-deler
         det ser så let ud udefra:
                kurserne står der på opslagstavlen
         og man skal bare skrive sin ønskeseddel og så håbe
         og læreren siger så hvad man skal læse
         men når man så sidder der på stolen og
                keder sig så man ikke kan lade være med at se på
                det man lærer
         og det ikke er andet end
                bit og løkker og effektivitet og systematisering og styring
         og man synes det smitter en og der lægger sig et sort lag
                over en af bit og løkker og effektivitet og systematisering
                og styring og man går i det altid og man glemmer
                man er menneske
         det er ikke let at være 2.deler

         Men det er heller ikke let at være lærer
                og skulle forske og tænke og skrive noter og forelæse
                og se flittig ud hver dag
         og det er ikke engang det værste
         for så er der alle de flittige 2.delere der vil være færdige
         og som derfor kræver skriftlige projekter og som
                bare går og
                leder efter en villig lærer og de har
                al den tid læreren ikke har
         og en lærer kan komme ud for 10 til 12 af den slags
         på en ganske almindelig arbejdsdag
         og det er de færreste lærere der kan klare det pres
         det er ikke let at være lærer

         Det er ikke engang let at være instruktor
                noterne kommer ikke til tiden
         og når de endelig kommer er de fyldt med fejl og noget
                helt andet end man troede man havde aftalt
         og læreren kommer fuld til instruktormødet
                hvis han overhovedet kommer
         og man er altid bange for de aggressive studenter og for
                at regne galt og for ikke at kunne klare presset og
                ansvaret og man gør sit bedste
         men det er ikke let
                Og når man så lige har stllet repetitionsopgaver
                indenfor pensum
                og man så ser eksamensopgaverne
                udenfor pensum
         og man på den måde finder ud af
                at man har et ansvar men ikke rigtig nogen indflydelse
         Så er det ikke let at være instruktor

         Det er ikke let for NOGEN
         og fordi det ikke er let for nogen sked det så tit at det bli'r
                FOR MEGET
         pludselig bli'r det for meget for en eller anden
         en 2.-deler eller en lærer eller en Dat-1'er der ellers altid
                har passet sit men som pluselig ikke kan klare det mere
                kan klare det mere og det hele slår klik
         på en gang for ham eller hende og bli'r
                FOR MEGET
         og man kan pludselig ikke holde ud at være der mere
         og man får lyst til bare at rejse, til RUC
                eller til Århus eller til USA
                og komme ind der -
                eller holde helt op og rejse til Frankrig
                og bage wienerbrød
         og bare slippe for al den ufølsomme umenneskelige teknologi
                som gør mennesker til idioter
                og maskiner til magt
         og det hele kokser og man kunne løbe skrigende gennem
                gennem hullestuen og lige hen til den nærmeste operatør
         og fortælle ham alt om sine sorger og bekymringer
                siden man startede her
         og man ville gøre det hvis det ikke var fordi man vidste at så
                ville det bare bive for meget for
                Operatøren

         For somme tider BLIVER det bare for meget
         somme tider bliver det bare for meget
         man får paranoia-anfald
         man ved man aldrig vil kunne forlade instituttet
         man ved det er det eneste sted med fremtid i
         man ved man ikke har andre venner end dem på DIKU
         og man synes at hele verden er helt
                firkantet
         og der er et svagt grønligt skær over det hele
         og PLUDSELIG kunne man gå aMOK i blind AGRESSION gå amok
                på den første den bedste man møder i Kantinen
                GÅ AMOK med en leverpostej fra STRYHN
                og bare TVÆRE den ud i hovedet på vedkommende

         For somme tider bli'r det bare for MEGET
         somme tider bli'r det bare FOR meget
         somme tider bli'r DET bare ALT ALT ALT for MEGET.
\end{sketch}
\end{document}
