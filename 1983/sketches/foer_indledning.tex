\documentclass[a4paper,11pt]{article}

\usepackage{revy}
\usepackage[utf8]{inputenc}
\usepackage[T1]{fontenc}
\usepackage[danish]{babel}


\revyname{DIKUrevy}
\revyyear{1983}
% HUSK AT OPDATERE VERSIONSNUMMER
\version{1}
\eta{$n$ minutter}
\status{Færdig}

\title{Før indledning}
\author{Hanne Hansen, Bodil Nielsen, Knud Henriksen}

\begin{document}
\maketitle

\begin{roles}
\role{P}[] Person i pænt tøj
\end{roles}

%\begin{props}
%\prop{Rekvisit}[Person, der skaffer]
%\end{props}


\begin{sketch}

\scene{
    Tæppet er for, folk griner og larmer, råber ekstranummer o.s.v. En person i pænt tøj kommer ind og går frem og tilbage, idet han/hun højtideligt fremsiger:
}

\says{P} Godaften mine damer og herrer. Æredede publikum. Det er i år blevet mig pålagt at byde Dem velkommen til den første optimalt modulstrukturerede revy nogensinde.

\says{P} Den optimale modulstruktur skyldes ikke alene hensynet til instituttets renomme; men også det faktum, at dansklærerforeningen igen i år har købt alle vores tekster.

\says{P} For at sikre, at omtalte optimale modulstruktur fortolkes korrekt, har vi, i god overensstemmelse med dansklærerforeningens sædvanlige praksis, udgivet et sæt noter i form af et program. For de bonderøve, der ikke er bekendt med dansklærerforeningens terminologi, skal jeg her ganske kort gøre rede for den korrekte fortolkning af omtalte noter:

\says{P} I venstre margin findes en numerisk litteral, i det følgende kaldet et tal. Dette tal angiver det relative tidspunkt for den pågældende hændelse målt i forhold til revyens start. Derefter følger en identifikation af den kunstneriske præstation, i det følgende kaldet nummeret. Denne identifikation er entydig, og tjener som henvisning til en egentlig note, som findes på siderne 3 og 4 i programmet.

\says{P} Selve revyen er i år fremkommest som det direkte resultat af en dybtgående analyse af de forgangne års revyer. Disse har vist sig at bestå af følgende elementer:

\says{P}

\begin{itemize}
    \item Mellem 42 og 48 minutters latterbrøl.
    \item Mellem 2 og 15 minutters BUH, idet revyen 75 også er medregnet.
    \item I gennemsnit 12.5 nummer med gennemsnitlig 10.25 pointe.
    \item 20 minutters sceneskift.
    \item 0.333 kort pause.
    \item Mellem 2 og 5 timers dans kombineret med ølafdrikning i DIKUs lokaler.
\end{itemize}

\says{P} Med udgangspunkt i dette resultat vil revyen i år komme til at bestå af følgende moduler:

\says{P}

\begin{itemize}
    \item Først 45 minutters latterbrøl.
    \item Derefter en kort pause, hvor der IKKE vil være mulighed for at købe øl i gården.
    \item Efter pausen 4 minutters BUH.
    \item Så følger 20 minutter, hvor vi så vel foran som bagved tæppet gentager nogle af de vigtigste; men samtidig de mest repræsentative sceneskift fra de sidste års revyer, idet sidste år også medregnes.
    \item Herefter følger aftenens store clou: NUMRENE. Desværre er der, på grund af kravet om korrekt fortolkning af den optimale modulstruktur, opstået et lille problem af tidsmæssig karakter (tegn). Det bliver derfor nødvendigt at opføre alle numrene samtidig, naturligvis virtuelt parallelt. Vi håber imidlertid, at De betragter dette som en detaljeagtig petitesse, der ikke skulle kunne gribe forstyrrende ind i den kulturelle værdi af Deres oplevelse af de 10.5 pointe, der er indlagt i de i alt 12.25 numre.
    \item Når selve revyen er slut vil De komme til at opleve en sand folkevandring, når vi alle sammen går over på DIKU.
    \item Efter folkevandringen vil vi i fællesskab udføre 4 timers organiseret rå-druk og velstruktureret abedans under musikledsagelse.
    \item Til sidst kommer aftenens andet højdepunkt, hvor vi allesammen brækker os i glassene, skider på stolene og forlader etablissementet i god ro og orden.
\end{itemize}

\says{P} Om et øjeblik vil der lyde et BIP, hvorpå revyen 83 starter. Med disse ord, mine damer og herrer, folk, fæ og bolighajer, vil jeg byde Dem velkommen til revyen 83.

\scene{Tæppe}

\end{sketch}
\end{document}
