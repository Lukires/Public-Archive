\documentclass[a4paper,11pt]{article}

\usepackage{revy}
\usepackage[utf8]{inputenc}
\usepackage[T1]{fontenc}
\usepackage[danish]{babel}


\revyname{DIKUrevy}
\revyyear{1983}
\version{1}
\eta{$n$ minutter}
\status{Færdig}

\title{Titel}
\author{Forfatter}
\melody{Shu-Bi-Dua: ``Jeg er en bille''}

\begin{document}
\maketitle

\begin{roles}
\role{S0-2}[] Sangere (piger)
\role{K0-2}[] Drengekor
\end{roles}

\begin{song}
\scene{I løbet af første vers:}

\scene{De tre fyre sidder ved hver sin skrivemaskine og småsover.}

\scene{Ved slutningen af første vers begynder de at røre på sig.}

\sings{S} Vi sidder ved et skrivebord,
          maskinerne, de brummer,
          vi venter på at nogen si'r,
          at i os er der krummer
          Men arbejdsløse er vi ble't,
          på grund af fessorens griller,
          med listerne han har sig téet,
          dat-1'erne han driller.

\sings{S} Kun de færreste ved,
          at de største arbejdsbier,
          kan man finde her,
          blandt førstesalens piger.

\scene{I løbet af andet vers:}

\scene{Der tages tøj af og vises rumper.}

\scene{Evt. i stil med de tre ballondansere fra ``nøjesmaskinen''.}

\scene{Derefter mavetræning og mavedans.}

\sings{S} Og mens vi alle venter på,
          at no'et fra himlen dumper,
          vi bytter tøj på må og få,
          og viser bare rumper,
          Når vi er blevet trætte af det
          så er det tid til en pause,
          til stuens ribbe går vi ned
          og træner vores mavse

\sings{S} Kun de færreste ved,
          at de flotteste maver
          kan man finde her
          hos førstesalens laver

\scene{I løbet af 3. vers:}

\scene{Sætter sig på stolene i ``fuglestilling''.}

\scene{Skilt med psykiatrisk hjælp 25 øre på bordene.}

\scene{En kunde kommer krybende ind og går glad ud igen.}

\sings{S} Inden vi går op igen
          skal den lige ``kaffes''
          Således at vi atter kan 
          på vores pinde træffes
          Vi er dermed atter klar
          vor visdom ud vi øser,
          problemerne på livets vej
          for mange nemt vi løser.

\sings{S} Kun de færreste ved
          at den værste sjælekuller
          får man klaret her
          hos førstesalens duller.

\scene{I løbet af 4. vers:}

\scene{Kæmpe-strikketøj og enorme kogebøger og slankekure farer over bordet.}

\sings{S} Hvis vi har fået nok af mænd
          fra første og fra stuen,
          så låser Pia vores dør,
          vi lukker hjertelugen.
          Og endelig kan vi tale om,
          hvad livet egentlig gælder
          det nyeste i strikketøj
          og soyafrikadeller

\sings{S} Kun de færreste ved
          at kvinders bedste vaner
          bliver dyrket her
          hos førstesalens damer

\scene{I løbet af 5. vers:}

\scene{Børn farer rundt, alle bliver trætte.}

\scene{``Nederste'' skuffe trækkes ud, men er som sædvanlig tom.}



\end{song}

\end{document}

