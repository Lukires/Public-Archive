\documentclass[a4paper,11pt]{article}

\usepackage{revy}
\usepackage[utf8]{inputenc}
\usepackage[T1]{fontenc}
\usepackage[danish]{babel}


\revyname{DIKUrevy}
\revyyear{1983}
\version{0.1}
\eta{$n$ minutter}
\status{Færdig}

\title{Fødselsdagssang}
\author{?}

\begin{document}
\maketitle

\begin{roles}
\role{S}[] Sanger
\role{P}[] Person
\end{roles}

\begin{sketch}
    \says{S} Ja, som i måske hørte i indledningen her først i revyen, så har vi prøvet at normalisere og harmonisere lidt i forhold til tidligere revyer -- sådan lissom, ikk. Og det betyder, at nu er der ca. 5 minutter til pausen -- og vi har ikke noget nummer. Så derfor har de andre i gruppen mobbet \emph{mig} til at sige noget ævl og fylder de sidste 5 \act{kigger på uret} 4 minutter ud. Og så var det, jeg tænkte, at i dag er der 2 kendte personer, der har fødselsdag. Så jeg har skrevet en fødselsdagssang til dem. I kender garanteret melodien, så bare syng med.

\begin{song}
    \sings{S} I dag er det <biiiiiip>s fødselsdag
              hurra, hurra, hurra
              han sikkert dårlig mave får
              af alt hvad han har spist i år
              og dejlig chokolade med 25-øre i

    \sings{S} Og når han gennem parken går
              hurra, hurra, hurra
              så kan han se Margrethes <biiiiip>
              der inde på Amalienborg
              og dejlig chokolade med 25-øre i

    \sings{S} Til sludt vi råber højt i kor:
              Buuuuuuuuuuuhh!
\end{song}

\scene{P kommer ind på scenen}
\says{P} Ja tak, så er det godt nok, nu har du vist sagt nok. Så er der 20 minutters pause.
\end{sketch}

\end{document}

