
\documentclass[a4paper,12pt]{article}

\usepackage{revy}
\usepackage[utf8]{inputenc}
\usepackage[T1]{fontenc}
\usepackage[danish]{babel}

\revyname{DIKU-revy}
\revyyear{2001}
\version{4}
\status{Færdig}
\title{Mand eller Mus}
\eta{8 min}
\author{Revyens Muser}

\begin{document}
\maketitle

\begin{sketch}

\begin{roles}
  \role{K1}[Ccilie] MI kvinde
  \role{K2}[Mille] MI kvinde
  \role{KD1}[Heidi] DKD1
  \role{KD2}[Sidsel] DKD2
  \role{KD3}[Katrine] DKD3
  \role{MD1}[Uffe] datalog
  \role{MD2}[Uhd] datalog
  \role{MD3}[Andre] datalog
  \role{MD4}[Fe] datalog
  \role{MD5}[Jørgen] datalog
\end{roles}

\begin{props}
  \prop{Bord med masser af papir} haves
  \prop{Humanist uniform} Ccilie 
  \prop{Et tæppe pr. tøs} Uhd, Hall, Andresen, Andre, Niels 
  \prop{3 store nørdede T-shirts} 
  \prop{/. hat} Bo
  \prop{laset T-shirt} find selv
  \prop{Joystick} Uffe
\end{props}

\scene{To kvinder iført bondeskjorter, Hallgren sko, Islændersweaters,
  fløjlsbukser og evt. palæstinensertørklæder. kommer frem foran
  tæppet. Siger mod publikum:}

\says{K1} Vi er meget bekymrede!

\says{K2} Vi har ikke hørt noget fra vores udsendte medarbejdere i lang tid!
Det hele startede ellers så lovende. Vi fra toppen af Feminiora studiet, nogen
fra Kvindeligt Univers Amager...også kendt som KUA...

\says{K1} ...ja, og andre fra Musikalsk Inferno...også kendt som
MI...er forsamlet for at glæde os over opnåede forsknings resultater
og opstille nye forskningsmål.

\scene{Tæppet fra. På scenen er der en masse piger (KD1-3 og evt. også
  K3-4) i et Musikvidenskabeligt Inferno - Univers, hvor man synger
  meget. Alle er iklædt humanistisk og/eller videnskabelig uniform:
  Bondeskjorter, Hallgren sko, Islændersweaters, fløjlsbukser,
  palæstinensertørklæder. Der er et bord med masser af bøger og
  papirer.  De to piger går hen til de andre, så alle pigerne er
  fuldtallige.}

\scene{Alle klapper hinanden veltilfredst på ryggen, smiler og ser
  lettede ud.}

\says{KD1} Vi har svaret på alle de store spørgsmål. Vi har forklaret:
\act{KD2 tager en rapport frem} Fornuft og følelse: ``Om systemdesign og
forholdet mellem medie, designer og bruger'' Første-persons
perspektivet.

\says{KD2}[lægger den første rapport væk] ``En dialog mellem
fænomenologi, analytisk bevidsthedsfilosofi og kognitionsforskning''
\act{KD1 tager en rapport frem} Ja, og ikke at forglemme ``Leg og
læring og tværorganisatorisk videns- og erfaringsudveksling i et
selvkonstitueret projektledernetværk''.

\says{K1} Og skål for det!

\says{Alle} Ja skål! \act{Alle skåler og klapper igen hinanden
  tilfredst på ryggen.}

\says{KD1}[Ser lidt utilfreds og tænksom ud] ...men vi mangler stadigt
svaret på det største af alle spørsgsmål, spørgsmålet over alle
spørgsmål. Det ultimative spørgsmål:.......Hvad er.....7 gange 6?

\scene{Sang: Den sindssyge videnskabskvinde}

\says{KD2} Men nu har vi jo udarbejdet et forskningsprojekt, der
skulle kunne give svaret. Det er en forsøgsopstilling, der består af
en masse maskinel, komplicerede og anderledes halvt humane mekanismer.
Og øh hvad hedder det\ldots jo mænd, faktisk.

\says{KD1} Forsøgsopstillingen hedder: Den Interessante Konkluderende
Undersøgelse...bedre kendt som DIKU...Vores unge medarbejder her har
stået for udarbejdelsen af talbetingelserne for forsøget: Mille,
hvilke tal har du givet forsøget at operere med?

\says{K2}[tæller på fingrene] Altså, 0, ikk'? og øhh altså 0, og..og
1, og 0, og 1,\act{fortaber sig - skal forekomme optimistisk og dum
  som en dør...og meget glad, hver gang hun får talt til 1, men kan så
  ikke huske mere og må starte forfra.}

\says{KD1} Ja, øh, det etniske ligestillingsråd forlangte at vi tog
blondiner med i forskningsarbejdet.  Til gengæld har vi sørget for at
der er masser af mus i forsøgsopstillingen. Mus har gode erfaringer
med omfattende forsøgsopstillinger, og jeg har hørt at mænd ikke er
bange for mus!

\says{K2} Åhh du sagde da ``Vi skal også inkludere mus, ikk'?'' Så jeg
har altså indarbejdet en masse musik i opstillingen!

\says{K1} Shit!... Nå... Vi må sørge for at DIKU også bliver fyldt med
mus, inden vi sætter forsøget i gang.  Det klarer jeg!  \act{til
  KD'erne} I må tage af sted og holde øje med eksperimentets fremgang.
Måske skal i smide det der tøj for at glide ind i miljøet uden at
vække mistænksomhed.

\says{KD1 og KD2}[tager langsomt KUA overtøjet af. Publikum: SMID
TØJET!]  Vi tager af sted.

\scene{K'erne går foran tæppet, mens der gøres klar til ``Vi maler byen rød''}

\says{K1} Ja det var sådan det startede, men nu er der gået 10
millioner år og vi på Musikalsk Inferno og Kvindeligt Underlivs
Ambulatorium har ikke hørt fra de udsendte videnskabskvinder siden de
tog afsted!

\says{K2} Derfor har vi med angst og bævere besluttet at tage afsted
for at tilse eksperimentet.

\says{K1} Vi vil rejse inkognito, og under påskud af at ville hjælpe
med at lave en såkaldt SOMMERREVY.

\scene{Tæppe fra til DIKU. Der sidder nogen fyre og koder. Der er
  pykslinger etc.  Gerne også noget med lokale 42 el. lign.}

\says{MD1} Det er mærkeligt der ikke er flere piger på DIKU?

\says{MD2} Hvor? \act{kigger disorienteret rundt}

\says{MD1} Her...på DIKU...

\says{MD2} Bryster!!

\says{MD1}[fumler med sine briller, mens han spørger MD1] Sidder mine
briller lige?

\says{MD2}[vurderende] Ehh..ja.

\says{MD1}[sætter sine briller på skrå] Er det bedre nu?

\says{MD2} Ja ja.

\scene{De to piger kommer valsende hen foran fyrene}

\says{Alle fyre} Blinker, vinker til pigerne.

\says{KD1} Det altså uhyggeligt, - siden jeg ankom her har jeg haft
sådan en underlig fornemmelse i maven.

\says{KD2} Du også? Det - det' egentlig ikke ubehageligt. Men jeg får
sådan en trang til.

\says{KD1} Også mig. Jeg tror det har noget med dem der at gøre
\act{peger på fyrene}

\says{KD2} Hrm hm jeg har analyseret luften på stedet her den er fyldt
med testosteron

\says{KD1}[Skræmt] TESTOSTERON!!!

\says{KD2} Og Feromoner

\says{KD1} FEROMONER!!! 

\says{KD2} Jeg har en fornemmelse af at det lige som har noget med det
centrale i det store spørgsmål at gøre, men jeg kan ikke helt sætte
fingeren på HVAD \act{roder rundt med fingeren alle mulige forkerte
  steder, og i munden, begynder at gispe efter vejret...} Og jeg har i
det hele taget så mange fornemmelser, så det kan være svært at
koncentrere sig.

\scene{Fyrene rejser sig op og er med i en eller anden slags
  koreografi Eksperterne giver efter for deres drifter og tager i
  hvert fald noget af tøjet af}

\scene{Sang: Vi maler byen rød}

\scene{MI eksperterne ankommer til DIKU}

\says{K'erne} HVAD foregår der her!!?

\says{KD1+2} Kvinderne fra Musikalsk Inferno! FLYGT FØR DET ER FOR
SENT!  Stedet er som en slags narkotika!!!  Svaret vi leder efter har
måske noget med disse fænomener at gøre, men flygt, eller I vil heller
ikke kunne slippe væk fra dette sted!

\says{K1+2} Hvad er det for noget vrøvl?

\says{K1} Men, jeg synes pludselig det er lidt mærkeligt at gå i denne
her sweater?

\says{K2} Jeg får også lyst..til at....skifte..

\says{K1} Jeg må synge!

\scene{Alle kvinder på scenen smider tøjet!!!! Indenunder er der
  festkjoler og de skifter til højhælede sko (Hvis det kan la sig gi
  sig på scenen) og vil pludselig til at synge!!!!}

\scene{Sang: My heart belongs to dada \& Du er mit svar}

\scene{Sangen slutter med at alle forsvinder med fyren under tæpper,
hvorunder der fnises voldsomt og endelig dukker vi forpustede og forpjuskede op}

\scene{Evt. kromatisk opadgående akkorder plus VoiceOver der tæller
  til 7 for at understrege pointen med ``7 gange 6''}

\says{K1} Og således fik vi alligevel til sidst svar på det store
spørgsmål:

\says{Alle} 7 gange sex er Rigtig, Rigtig sjovt

\scene{TÆPPE}

\scene{Noter: alternativer til Musikals Inferno --- Moderligt
  Imperium, Malplaceret Interesse, Manglende Intelligens, Mystisk
  Interferens.}

\end{sketch}
\end{document}
