\documentclass[a4paper,12pt]{article}

\usepackage{revy}
\usepackage[utf8]{inputenc}
\usepackage[T1]{fontenc}
\usepackage[danish]{babel}

\revyname{DIKU-revy}
\revyyear{2001}
\version{2.1} 
\status{Færdig}
\title{Stjæl Sunskærmene}
\eta{5 min}

\author{Uffe Friis Lichtenberg}

\begin{document}
\maketitle

\begin{sketch}

\begin{roles}
  \role{R}[Uffe] Rusvejleder
  \role{KC1}[Katrine] Kor Chick
  \role{KC2}[Ccilie] Kor Chick
\end{roles}

\begin{props}
  \prop{Rusvejleder tøj} Uffe har selv
\end{props}

\begin{noindent}

\says{R}[Intro, tale] Mine herrer og den dame på rustur 2001:
        Stjæl sun-skærmene.
        
\says{R} Hvis jeg kun kunne tilbyde dig ét enkelt råd om din tid her på DIKU, så ville det være om sun-skærmene.
        
\says{R} Skærmene på sun-terminalerne er det eneste stykke hardware her
        på DIKU du får adgang til, som vil kunne benyttes sammen med
        din egen PC - hvis du da har én.
        
\says{R} Så stjæl dem og spar flere tusinde kroner - dét er da til at mærke, når man er på SU.
        Resten af mine råd har ikke så åbenlys mærkbar effekt, men er blot baseret på min egen rodede erfaring.
        Disse råd vil jeg give nu.


\says{R}[1. vers, tale] Nyd de næste mange år på DIKU.

\says{R} Vær ikke bekymret for eksamen.

\says{R} Spild ikke tiden med at bekymre dig om penge:
        nogle gange har du overskud,
        nogle gange har du underskud.
        Din SU rækker alligevel ikke til hele studiet,
        men til slut kan du tjene fedt
        - hvis det er dét du vil.

\says{R} Gør én ting hver dag som du ikke kan finde ud af.

\says{R} Spil.

\says{R} Nyd din hjerne.
        Brug den på alle mulige måder.
        Vær ikke bange for den, eller for hvad andre måtte tænke om den.
        Det er den vildeste computer du nogensinde for lov til at eje.

\says{R} Kod.
        
\says{R} Føl ikke skyld, hvis du ikke når igennem på normeret tid.
        De mest interessante mennesker jeg kender gik væsentligt over normeret.
        Nogle af de mest interessante er ikke engang gået i gang med deres bifag endnu.

\says{R} Drik masser af cola.

\says{R} Pas godt på dine hænder,
        du vil savne dem når du får museskade.

\says{R} Brug energi på at lære mange forskellige ting,
        jo mere interessant dit job er, jo flere forskellige ting skal du kunne.

\says{R} Acceptér enkelte absolutte sandheder:
        computere bliver billigere og hurtigere,
        software bliver større og tungere.
        Du vil også blive gammel,
        og når du gør
        vil du fable om din ungdom
        hvor en computer var dyr og langsom,
        software var småt og elegant
        og kode blev skrevet med stolthed.
        Kod med stolthed!

\says{R} Opløs din rapportgruppe!
        Men skrev den virkeligt gode rapporter, så gendan den!

\says{R} Måske består du.
        Måske gør du ikke.
        Måske får du 13.
        Måske gør du ikke.
        Måske skulle du hellere læse fysik.
        Og så igen... måske skulle du ikke.
        Hvad end du gør:
        sæt dine ambitioner højere end du kan nå,
        men bliv ikke skuffet når du ikke når dem.

\says{R} Skriv.

\says{R} Brug UNIX et stykke tid,
        men stop før du mister fornemmelsen for den virkelige verden.
        Brug MacOS et stykke tid,
        men stop før du mister fornemmelsen for maskinen.

\says{R} Byg.
        
\says{R} Selv hvis du kun har gamle komponenter fra 80'erne.

\says{R} Læs pensum,
        selv hvis du allerede kan det.

\says{R} Læs ikke licensaftaler,
        man får ondt i hovedet af jura.

\says{KC}[Omkvæd, sang]
        Brødre og søster I ved det vi klarer det nu.
        En dag vil ånden tag fat i dig, gribe din sjæl.
        Jeg ved du har haft det hårdt, men jeg har ventet på dig på DIKU.
        Og jeg er her når du har brug for hjælp,
        når jeg kan.
        
\says{Kor}[Outro, sang]
        Her er alle nørder.
        Allesammen.
        Og det ' godt!

\scene{TÆPPE} 
\end{noindent}

\end{sketch}
\end{document}
