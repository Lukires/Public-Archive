\documentclass[a4paper,12pt]{article}

\usepackage{revy} 
\usepackage[utf8]{inputenc}
\usepackage[T1]{fontenc}
\usepackage[danish]{babel}

\revyname{DIKU-revy}
\revyyear{2001}
\version{2.1}
\eta{3 min.}
\status{Færdig.}
\title{Besat!}
\author{Anders Sewerin Johanson, Niels H.\ Christensen}

\begin{document}
\maketitle

\begin{roles}
  \role{GS}[Niels] Guitarsmølfen (Rektor) 
  \role{BS}[Anders] Batiksmølfen
  \role{SEK}[Sidsel] Sekretær
\end{roles}

\begin{props}
  \prop{Akustisk guitar} Dunkel
  \prop{Papkasse med papirer i} haves
  \prop{En lang, fed J} laves
  \prop{Islandsk sweater} Haves
  \prop{Briller til SEK} Jaydee
  \prop{Sek.-agtig nederdel} Sidsel
  \prop{Clipboard} haves (fra anden sketch)
  \prop{Vandpibe(r)} Jonas, Katrine
\end{props}

\begin{sketch}

\scene{Rektors kontor. To hippier (Guitar-smølfen
og Batik-smølfen), begge
knokkelskæve og fastgroet i '68}

\says{GS}[Spiller guitar og synger] 
Ooooog itsi-bitsi, tag med mig til Nepal.
Jah, itsi-bitsi, tag med mig til Nepaaaal.
Vejen...

\says{BS} Hey, Keld!

\says{GS} er lavet af... Jaaaaaah?

\says{BS} Nu vi har fået besat rektors kontor \act{fniser}

\says{GS}[fniser]

\says{BS} ...skulle vi ikke drikke hans Cognac, og ryge hans cigarer?

\says{GS}[ærligt bekymret] Jamen... Tror du ikke han bliver sur så?

\scene{SEK kommer ind med nogle papirer}

\says{SEK} Hvis de vil skrive under på referatet fra sidste
konsistorie-møde, Rektor Møllgård?

\scene{GS skriver under}

\says{SEK} Og De, hr.\ Dekan Jeppesen.

\scene{BS underskriver. SEK går igen. Pinlig tavshed}

\says{BS} Øh, Keld? Hva' faen har vi egentlig lavet siden
den ø-lejr i '68?

\says{GS}[fniser] Jah...vi har jo ihvertfald været skæve!
Der står et arkivskab derovre.

\scene{De roder i arkivskabet}

\says{BS} Shit Keld -- du er blevet rektor! Hvor vildt!

\scene{Begge fniser mens de roder videre. GS fryser}

\says{GS} Henrik for helvede. Du har fyret halvdelen af
Naturvidenskab.

\says{BS}[gisper] Nej.

\says{GS} Jo, og det er ikke det hele. \act{Viser ham et papir.}
Rusturene.

\scene{Kæber rammer gulvet}

\says{BS} Dååårligt trip, mand. Hvad mon du har lavet?

\scene{De roder i arkivskabet. De roder mere i arkivskabet.  De sidste
  skuffer rodes igennem og tømmes}

\says{GS} Fandt du noget?

\says{BS}[trøstende] Arj men, du {\bf må} have lavet noget, ikk'?  De
har nok bare glemt at skrive det ned, ikk'?

\says{BS} Kom Keld. Vi må gøre det hele godt igen.  Vi sætter os ned
med de studerende og taler os frem til en god løsning på det hele.

\scene{GS ryster på hovedet og viser ham et papir}

\says{BS} Sty-rel-ses-lov af 1992? De bastards!  Nå nej, vent.

\says{GS} Nå, men hvis vi har udelukket de studerende fra indflydelse,
så må vi jo ordne det med professorerne.  Men vi må finde nogen, der
er progressive og åbensindede.

\says{BS} Årh hvad, se her! Nu er der noget, der hedder ``datalogi''.
Det må være på sådan et moderne og friskt fag, man finder nytænkning.

\says{GS} Når vi møder en datalogi-professor, så kan han hjælpe med at
lave det hele om.  \act{Lægger armen om BS i en krammer.}  Allerførst,
Henrik, så genindfører vi sgu' filosofikum.

\scene{SEK entrer igen}

\says{SEK} Undskyld jeg forstyrrer igen.  Peter Naur fra datalogi er
her til sit klokken 10-møde.

\scene{BS og GS forlader ivrigt scenen arm i arm. TÆPPE}

\end{sketch}
\end{document}
