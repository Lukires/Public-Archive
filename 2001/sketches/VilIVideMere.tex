\documentclass[danish]{article}
\usepackage{revy}
\usepackage[utf8]{inputenc}
\usepackage{babel}
\usepackage{a4wide}

\title{Vil I vide mere??}
\author{Theo Engell-Nielsen}

\version{0.3} % HUSK AT AJOURFØRE VERSIONSNUMMER!!
\revyyear{2001}
\status{Tæppenummer}
\eta{4 min}
\begin{document}
 \maketitle
\vspace{-0.3cm}

\begin{roles}
    \role{Forsker} {\small Evolutionsforsker, kittel, læspende, grynter,
    slikhår, store tykke briller, håbløst {\em u}tjekket, {\tt
      \#include <stdmuldvarp.h>}}
\end{roles}

\begin{props}
  \prop{Flip-over}
  \prop{Tykke briller}
  \prop{Kittel}
\end{props}

\begin{sketch}
  \scene{Forskeren står på scenen ved en flipover}

  \says{Forsker} Ja, vi kender alle sammen Moore's lov, men fordi der
  er lidt mange påvirkede folk, øh, jeg mener naturligvis
  udefrakommende mennesker her i aften, ja, så tager vi lige en hurtig
  repetition: I 1965, da Gordon Moore skulle forberede en tale, gjorde
  han en spændende observation: Da han tegnede et diagram over
  datamat-lagerkredses størrelser, indså han at der var tale om en
  udtalt sammenhæng. Hver ny datamat-lagerkreds havde sådan ca.\ 
  dobbelt kapacitet i forhold til dens forgænger, og hver
  datamat-lagerkreds var lanceret med $18$--$24$ måneders mellemrum.
  Hvis denne sammenhæng fortsatte, ville datamaternes styrke stige
  eksponentielt over relativt korte perioder. Og, det er jo meget
  spændende. Vil I vide mere? ``Do you want to know {\em
    moore}?''\act{tandpastasmil, venter ikke på svar}
%
  \says{Forsker} Det afbildes nemt ved en ret linie på logaritmiske
  papir. \act{Vi ser en eksponentiel kurve på logaritmisk papir} - Ja,
  det hjælper hvis man er knap så påvirket, eller, øh!, har {\em
    bestået bifag} i matematik. Læg mærke til at andenaksen er
  logaritmisk, derfor den rette linie.
%
  \newcommand{\flip}{}
%
  \says{Forsker} Ja, det har faktisk vist sig at sådan er det indenfor
  alt med ``datalogi'':
\begin{itemize}
\item \flip \act{bladrer hver gang} Datamat-lagerkredse, det er dem vi
  lige har set på. \flip Hastigheder på elektronhjernerne. \flip
  Størrelser på eksternt pladelager.
\item {\em Her skal/kan være mange, der bladres hurtigt hen over, man
    skal lige kunne nå at se titlen : grafikkort, problemstørrelser,
    e-mails, datatransmissioner, internetforbindelser, etc.}
\end{itemize}
%
\says{Forsker} Ja, det er jo meget godt, og vi kender dem jo
allesammen\ldots Vil I vide mere? \act{venter ikke på svar} Jamen, så
lad os kaste blikket lidt over på {\tt DIKU} - hold da op det bliver
vel nok spændende, ikke sandt?
%
\begin{itemize}
\item \flip {\em Solgte cola'er, stjålne kaffekopper, øl på "caféen?",
    spildte TAP-timer, Dylans omsætning, alder på forelæsere, gratis
    mundtlige punkter på andendelen\ldots}
\item \flip ``Piger på datalogifaget'' Der er snart $2.3$ piger på
  datalogi, og det\ldots er jo godt\ldots Selvom kurven kunne være,
  ja, lidt mere, skal vi sige, interessant\ldots ja.
\item \flip Øhm, ja, det er så lærerflugten fra {\tt DIKU}, den tror
  jeg vi springer over\ldots \act{griner nervøst}
\item \flip UPS! \act{Springer ind foran flip-overen og dækker den med
    kroppen} Det var så antallet af DIKU-Ph.D-studerende på IT-C, den
  springer vi {\em også} over. Øhm, ja.
\item \flip \act{Lever overdrevent meget op} Datalogernes indkomst er
  ganske spændende, se nu her\ldots ``Lønninger for dataloger'' En
  datalog tjener eddermame kassen\ldots Er I klar over hvorfor det er
  fedt? \act{vildt eksalteret} Er I klar over hvorfor det er fedt?
\item \flip \act{Sjofelt}Det er sgudda fordi det har kvinderne nemlig
  fundet ud af!!! \act{Grafen viser nu en parabel} Hold da kæft I får
  meget sex! På denne måde kommer datalogerne til at overtage verden
  indenfor meget kort tid. Prøv lige at tjekke dennehersen graf, mand!
  ``I morgen er verden vor!'' \act{Laver dobbelt thumbs-up til
    publikum} Dét må jeg nok sige!
\end{itemize}
\says{Forsker} Ja, her slutter mit fore\ldots Hov, jeg mangler vist
lige en sidste graf\ldots \act{bladrer, et afrevet papir dækker toppen
  af den næste planche der har en graf et linært svagt dalende forløb}
Hvad? En svagt fluktuerende linær funktion\ldots Hvad hulen laver den
her? \act{Flår det dækkende papir af, ``Lønninger for fysikere''} Ah,
nu ved jeg det! Det er jo til mit foredrag på HCØ i morgen. Tak for i
aften!
\end{sketch}
\end{document}
% Local Variables: 
% mode: latex
% TeX-master: t
% End: 
