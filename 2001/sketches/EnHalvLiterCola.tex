\documentclass[a4paper,12pt]{article}

\usepackage{revy}
\usepackage[utf8]{inputenc}
\usepackage[T1]{fontenc}
\usepackage[danish]{babel}

\revyname{DIKU-revy}
\revyyear{2001}
\version{2.0}
\status{Tæppenummer}
\title{En Halv Liter Cola}
\eta{1 min}

\author{Anders Sewering Johansen}

\begin{document}
\maketitle

\begin{sketch}

\begin{roles}
  \role{Ældre datalog ÆD}
  \role{Ung kerneskrivende datalog UD}
  \role{Voiceover VO}
\end{roles}

\begin{props}
  \prop{2 stole} haves (kantine)
  \prop{et bord} haves (kantine)
  \prop{en halv liter cola (du'h)} købes
\end{props}

\scene{ÆD og UD sidder og snakker}

\says{ÆD} Nå, hvad laver I så for tiden inde på instituttet?
\says{UD} Ja... Vi skal jo til at skrive kerne. Det bliver vist en ORDENTLIG
OMGANG!
\says{ÆD} Som jeg husker det, handler det om at KOMME I GANG I GOD TID

\says{VO} En halvanden liter cola kommer på bordet...

\says{ÆD} Min gruppe fik jo tretten i kerneopgaven.
\says{UD} Jamen dog! Bare det var os!

\says{VO} ...og pludselig bliver samtalen MEGET hyggeligere

\says{ÆD} Nej ved du nu hvad! I kan da bare få vores gamle rapport og
kildekode. Det finder de da aldrig ud af. Så nemt er det!
\says{UD} Yay!

\says{VO} Og husk: Nyd det så længe det varer

\says{ÆD} Det var godtnok tider! De glade dage med 68000 assembler og de
gamle Motorola maskiner på europakort... Det varmer mit hjerte at man
stadig kan bruge en god gang maskinkode til noget på DIKU!

\says{UD}[hoster] Ja, det er jo det.

\end{sketch}
\end{document}
