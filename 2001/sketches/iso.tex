\documentclass[danish]{article}
\usepackage{revy}
\usepackage[utf8]{inputenc}
\usepackage{babel}
\usepackage{a4wide}

\title{Den Store ISO Certificering(tm)}
\author{Jesper Holm Olsen}
\eta{6 min}
\version{2.11} % HUSK AT AJOURFØRE VERSIONSNUMMER!!
\status{Færdig} % ...OG STATUS!!

\revyyear{2001}

\begin{document}
\maketitle

\begin{roles}
  \role{H}[Jørgen] Hasse Clausen
  \role{MIB1}[Christian] ISO auditør (MIB)
  \role{MIB2}[Adam] ISO auditør (MIB)
\end{roles}

\begin{props}
  \prop{telefon} haves
  \prop{Jakkesæt (MIB)} haves
  \prop{hassetøj} haves
  \prop{CD-ROM} haves 
  \prop{2 clipboards} Haves
  \prop{Mikrofoner} 3 stk
\end{props}



\begin{sketch}
\scene{Indledes med et Grandprixrim}

\scene{Tæppe fra}

\scene{Hasse står alene på scenen med en telefon og
grubler}

\says{H} [bander] For syv sytten djævles fortællinger i
fortælleteknik \act{banker i bordet}! DIKU skal have STÅ'er, MASSER af
STÅ'er, muhahahaha...... Og der skal være flere drenge!!  Masser af
STÅ-drenge!!

\scene{kammer over, men kommer til hægterne}

\says{H} Uden STÅ'er er der jo ikke råd til en stor, velfungerende
og produktiv systemarbejdsgruppe, der må gøres noget aktivt for at
forbedre gennemførselsprocenten - andet selvfølgelig end faktisk at
forbedre kurserne. Hvad gør man monstro ude i det pulserende
erhvervsliv ?  \act{tænker} For syv sytten djævles iterationer i
spiralmodellen, jeg har det: ISO-certificering!  \act{ser snedig ud}
På den måde kunne DIKU skille sig ud fra de andre edb-uddannelser,
markere sig som noget reelt og kvalitetsorienteret --- som en
kravspecifikation i et offentligt edb-projekt! Der må handles!

\scene{Griber telefonen, ringer}

\says{H} Ja, hallo? Er det hos ISO? Ja, jeg ønsker større kvalitet
i min hverdag.. \act{personen i den anden ende snakker}, ja, ok...
hvad siger du? Økologiske varer bestilt på nettet leveret lige til
døren?!? Hvad er det for
noget vås! Det er umuligt, jeg er lektor i datalogi, hvis nogen skulle vide
det var det mig! I må have fået forkert telefonnummer.

\scene{Ringer igen}

\says{H} Er det hos ISO? Ja, jeg vil gerne kvalitetskontrollere en
offentlig institution

\scene{holder røret fra øret, da modtageren tydeligvis griner højlydt}

\says{H} nej, jeg mener det faktisk seriøst...ok, men så sender I
nogle auditører?

\act{Auditørerne kommer ind med det samme som MIB med clipboards og
  ser meget \TeX ede ud}

\says{H} Det var søreme på tide \act{ligger røret på}

\says{H} Så I er fra ISO?

\says{MIB1} Netop. Vi opstiller procedurer for alle arbejdsgange i
virksomheden for at sikre en ens behandling af alle sager. [Her kan afbrydes] Ved at have
nedfældet alle processer skriftligt har man en 100\% vandtæt garanti
for ens udførelse af ...

\says{H} [afbryder] Ja ja, men I udsteder altså
\emph{certificeringer}?

\scene{plaprer de næste replikker} 

\says{MIB2} Ja, lige netop. Og for jeres vedkommende her på DIKU kan vi
benytte os af denne standard-checkliste for højere uddannelsesinstitutioner.

\says{H} [kigger ud mod publikum] Så må vi jo håbe, at DIKU
opfylder kravene --- vi vil jo nødig lave noget om!

\says{MIB1} Godt, vi har her en checkliste for hvilke processer der
skal gennemføres i et kursusforløb for at sikre en anstændig og
ensartet proces i henhold til ISO-010101 standarden.


\says{MIB2} Checklisten er delt op i tre dele, der afspejler et
kursusforløb. Først forelæsninger: Underviseren skal mumle eller på
anden vis gøre sig uforståelig for tilhørerne...

\scene{Hasse ser bekymret ud}

\says{MIB1} ..undtaget heraf er forelæsere der taler tydeligt\ldots finsk.

\says{H} Ja, det kan vi godt klare, check!

\says{MIB2} Den såkalde "spf" (slides pr. forelæsning) skal enten være
skarpt større end 100 eller skarpt mindre end 10.

\says{H} Jeg tager bare halvdelen af mine slides og giver dem til
Knud Henriksen --- check!

\says{MIB2} Mindst 30\% af alle studerende skal komme for sent til
forelæsningerne.

\says{H} Check!

\says{MIB2} Stolene skal derfor larme mest muligt.

\says{H} Check!

\says{MIB1} Anden del af checklisten. Rapportopgaver; Mindst een klasse
i en objektorienteret opgave \emph{skal} hedde ``BankAccount''.

\says{H} Check

\says{MIB1} Vinduerne i terminalrummet må ikke åbnes!

\says{H} Det bliver de jo alligevel aldrig --- Check!

\says{MIB1} Den udleverede kode skal være fyldt med fejl.

\says{H} Check!

\says{MIB1} Den samme opgave skal stilles \emph{hvert} år.

\says{H} Check!

\says{MIB2} ... og fejl må ikke rettes 

\says{H} Enig -- så lærer de også noget om fejlfinding. Check!

\says{MIB1} Mindst to rapportopgaver skal ligge oveni hinanden og have
samme afleveringsdato.

\says{H} Kun to? Er det ikke lavt sat? Vi kan sagtens klare mere
end det. Check!

\says{MIB1} Terminalerne skal eksplodere spontant i løbet af
rapportperioderne...

\says{H}[bekymret] Men det er jo fuldstændigt urealistisk!

\says{MIB2} ...og i den forbindelse er det så kun terminalerne der må
ryge i terminalrummene.

\says{H} Hmmmmmm... edbafdelingen må kunne programmere et eller andet --- Check!

\says{MIB2} Så kommer vi til den sidste del, eksamen: Mindst 15\% af
opgaveteksten stilles uden for pensum. \act{Denne del skal gå hurtigt
  med oplæsning af krav og ``check''}

\says{H} Check!

\says{MIB2} Opgaveteksterne skal være fyldt med fejl.

\says{H} Check!

\says{MIB2} Mindst een delopgave skal være ikke-beregnelig.

\says{H} Check!

\says{MIB2} Mindst 50\% af de studerende skal dumpe.

\says{H} Check!

\says{MIB2}Forelæserne skal undres over at så mange studerende dumper.

\says{H} Check!

\says{MIB2} Og til sidst skal der udformes en rapport der dokumenterer
at grunden til at så mange studerende dumper er at de faktisk
\emph{er} dumme og dovne.

\says{H} ...og at de ikke har lært noget i gymnasiet, Check!  Godt
så, men hvor dyrt bliver det egentlig for sådan en ISO certificering?

\scene{MIB'er hvisker lidt sammen, MIB1 går hen og hvisker noget i øret på
Hasse}

\says{H} For syv sytten djævles håndtering af
menneske-maskinsamspillet i en cyklistmodelrepræsentation af en
middelstor edb-projektering for handicappede! Det var pebret! Hmmm...
Er der virkelig ikke en billigere certificering man kan få?

\says{MIB1} [Forarget] EN BILLIGERE?!?!? Nej, hør nu... Hvis vi bare
sådan gik rundt og udstedte certificeringer uden hold i virkeligheden
ville hele ISO-certificeringsbegrebet jo fuldstændig devaluere! Hvem
ville tage os seriøst?

\scene{Akavet pause. MIB2 tager fat i MIB1 og hvisker ham noget i øret. De
  diskuterer lidt. MIB1 skifter stemmeleje til snu gadehæler, skuler
  omkring}

\says{MIB1} OK, siden det er dig kan vi nok finde en speciel
ordning.

\scene Han åbner den ene side af jakken ligesom en hæler der har ure
inde i foret -- istedet har han en CD-ROM

\says{MIB1} Du kan få denne her.

\scene{tager CD-ROM'en ud, Hasse tager den}

\says{H} Ok?!? Og hvad er så det?

\says{MIB2} En økonomiklasse ISO 9660 certificering!

\says{H} Den tager vi! 

\scene{lys ned, tæppe for}

\end{sketch}

\end{document}

