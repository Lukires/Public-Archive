\documentclass[a4paper,12pt]{article}

\usepackage{revy}
\usepackage[utf8]{inputenc}
\usepackage[T1]{fontenc}
\usepackage[danish]{babel}

\revyname{DIKU-revy}
\revyyear{2001}
\version{1.0}
\eta{3.5 min}
\status{Færdig}
\title{DIKUREVY}
\melody{Gnags: ``Danmark''}

\author{Jesper Holm Olsen}

\begin{document}
\maketitle

\begin{roles}
  \role{S}[Andre] Sanger
  \role{K1}[Uhd] Kor
  \role{K2}[Fe] Kor
\end{roles}

\begin{song}
\sings{S} DIKU, mit yndlings institut
Jeg smelter sammen med din atmosfære
ganske absolut

VIP'er, DIKUs lærerstab
Skal paroderes i DIKUrevyen
Uden for mange gab

Omkvæd:
Ligemeget hvilket ry
der (er) om vores revy
så' vi her påny
Jeg står på scenen i det rum
der for nylig husede modløsheden
og råber "Længe leve opfindsomheden"

Revy, årets sommerfest
forsøger atter at joke med DIKU
det er helt til hest

Ordspil! Pas på du ikke dør
har aldrig hørt den slags dårlige vitser
nogensinde før 

Omkvæd:
Ligemeget hvilket ry
der (er) om vores revy
så' vi her påny
I vores jagt på ideer
bli'r vi sære og får tunnelsyn
Målet blir': "Latter-liggør fysikrevyen"

Intro, lad nørder få lov
at lave 3D på deres computer,
multimedie-sjov

Showet, vi starter og gir' gas
Vi byder velkommen til jer derude
Og håber det blir' spas

Omkvæd:
Ligemeget hvilket ry
der (er) om vores revy
så' vi her påny
Vi lægger ud med at skabe
en stemning der er en smule lummer
så lad os gå til, det første tæppenummer 
 

\scene{TÆPPE}

\end{song}
\end{document}
