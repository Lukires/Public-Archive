\documentclass[a4paper,12pt]{article}

\usepackage{revy}
\usepackage[utf8]{inputenc}
\usepackage[T1]{fontenc}
\usepackage[danish]{babel}

\revyname{DIKU-revy}
\revyyear{2001}
\version{1.3}
\eta{5 min}
\status{Tæppenummer}
\title{Dougals Adams Sang}
\melody{Don MacLean: ``Miss American Pie''}

\author{Niels Christensen}

\begin{document}
\maketitle

\begin{roles}
  \role{S}[Niels] Sanger
\end{roles}

\begin{props}
 \prop{Håndklæde}
\end{props}

\begin{song}

\sings{S}[Intro. Stille] For lang lang lang tid siden
biblioteket
stødte jeg ind i en trilogi
Allerede fra den første færd
bar jeg altid håndklædet her
for måske kom vogonerne forbi

Men maj i år var håndklædet tingen
Douglas' hjerte smed i ringen
faldt om pludseligt livløs
på Jorden - næsten harmløs 

Skal jeg sørge eller ej
når verdens største blaffer-haj
og en lille del af mig
er død den 11. maj?

\sings{S}[Omkvæd] 
Vi græd, Douglas Adams han skred
nu serverer han en burger sam'n med Elvis et sted
Vi beundrede ham for de ting han skrev ned
farvel og tak for fisk, du var fed
farvel tak for fisk, du var fed


\scene{TÆPPE}

\end{song}
\end{document}
