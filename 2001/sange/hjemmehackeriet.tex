\documentclass[danish]{article}
\usepackage{revy}
\usepackage[danish]{babel}
\usepackage[T1]{fontenc}
\usepackage[utf8]{inputenc}
\usepackage{anysize}

\title{Hjemmehackeriet}
\author{Torben Mogensen}
\melody{Hjemmebrænderiet -- De Gyldne Løver}
\eta{2.5 min}
\version{2.1}
\status{Færdig} 

\revyyear{2001}

\begin{document}
\maketitle

\begin{roles}
  \role{S}[Uffe] Hjemmehakker
  \role{K1}[Katrine] Kor
  \role{K2}[Rune] Kor
\end{roles}

\begin{song}
  \sings{S}[1. vers]Jeg bor her i Ishøj på syvende sal
  i en lejligh\'ed, der er stort set normal.
  En stue, et køkken, et bad med WC
  og et kammer, hvor jeg har min hjemme-PC.

  \sings{S}[Omkvæd] Jeg hacker, jeg cracker, jeg downloader spil,
  og jeg logger ind, lig' præcis hvor jeg vil.
  Jeg kender dit password, jeg læser din post.
  For en hacker som jeg er den slags hverdagskost.

  \sings{S}[2. vers] Min fætter har hacket i Pentagons net.
  De tro'ed det var svært, men han syn's det var let.
  De fandt ham dog efter en lengere jagt,
  så nu er han ansat som sikkerhedsvagt.

  Omkv.
  
  \sings{S}[3. vers] Jeg laved en virus, som hed "I Love You".
  Jeg indrømmer dog, jeg fortryder det nu.
  Da jeg gik i banken, min løn for at få
  havde virusen sat der's computer i stå

  Omkv.

  \sings{S}[4. vers] Hvis du sku' få lyst til at hacke lidt selv,
  jeg ønsker dig al mulig lykke og held
  Det giver dig magt som om du var en gud,
  og du kan endda få din pizza bragt ud.
  
  Omkv. ad helveg til

\end{song}
\end{document}









