\documentclass[a4paper,12pt]{article}

\usepackage{revy}
\usepackage[utf8]{inputenc}
\usepackage[T1]{fontenc}
\usepackage[danish]{babel}

\revyname{DIKU-revy}
\revyyear{2001}
\version{1}
\status{Færdig}
\title{Den sindssyge videnskabskvinde}
\melody{Kate Bush "Wuthering Heights"}
\eta{3.5 min}

\author{Rune, Heidi, Martin}

\begin{document}
\maketitle

\begin{roles}
  \role{Q}[Heidi] Qvinde
\end{roles}

\begin{song}

\sings{S}[1. vers]
Her på mit diku, kan jeg finde fred, til mit projekt
Jeg' godt nok sindssyg, min hjerne er defekt
Den er perfekt, ja
Kald mig kun Jeckyll
Kald mig misses Hyde, det ' min leg, ja det er mig, jeg ' no'et for sig.

\sings{S}[1. vers]
Mit felt er komplex(t)
For hvad giver dog syv gange tallet seks
Det ' mit felt på Di-ih, Di-ih, Di-ih-kuu

\sings{S}[Omkvæd]
Fordi, jeg kan bli' ganske fri til min forskning
Det er det sted, jeg kan si', jeg kan forske uå-å-å-ah.
Det er her, fordi jeg er fri til min forskning
Uåh-åh-åh-åh det er her, jeg kan forske uå-å-å-åh


\sings{S}[2. vers]
Oh mit forsøg. Det ' fantastisk - mon det lykke's for mig nu
Jeg er saa snu, og jeg vil brug' mænd til forsøg nu.
Jeg hungrer vildt, det ik' spildt
Jeg styrer det, for jeg er mester.

\sings{S}[2. bro]
Til mit forsøg der vil jeg
bruge mænd for at spare hist og her
Det ' mit felt på Di-ih, Di-ih, Di-ih-kuu

\sings{S}[Omkvæd]

\sings{S}[3. vers]
Ooh jeg bli'r sindssyg, lad mig få-åh-åh, et svar fra jer
Ooh jeg bli'r sindssyg, lad mig få-åh-åh, et svar fra jer
ellers bli'r, jeg ik-ke tilfreds/ de't er mit felt på diku

\sings{S}[Omkvæd ad lib]

\end{song}
\end{document}
