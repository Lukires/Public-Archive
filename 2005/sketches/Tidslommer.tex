\documentclass[a4paper,11pt]{article}

\usepackage{revy}
\usepackage[utf8]{inputenc}
\usepackage[T1]{fontenc}
\usepackage[danish]{babel}

\revyname{DIKUrevy}
\revyyear{2005}
% HUSK AT OPDATERE VERSIONSNUMMER
\version{0.9}
\eta{4 minutter}
\status{Færdig}

\title{Tidslommer}
\author{Ulla, Uffe, Madss, Jakob og Uhd}

\begin{document}
\maketitle

\begin{roles}
\role{D}[Uffe] Dr. Rubbish Quatsch
\role{G}[Uhd] Dr. Bullocks Bl\"odsinn
\end{roles}

\begin{props}
\prop{Militærhjelme (2)}
\prop{Solbriller (2)}
\prop{To hvide kitler}
\prop{To navneskilte}
\prop{To beskyttelsesbriller}
\prop{To kasser}
\prop{To sssSUUUPerdrinks}
\prop{Bord}
\prop{Dukke af studerende}
\prop{Dukke af forelæser}
\prop{Stripperdukke}
\prop{Kæleabe}
\prop{Mandolin/Ukulele}
\prop{Pornoorgel}
\prop{Rumskib}
\prop{Urskive med bevægelige visere}
\prop{Badebold}
\prop{Bordtennisbold}
\prop{Osteklokke/Glas}
\prop{Forside med ITTF}
\end{props}
  
\begin{sketch}

\scene{D og B står på scenen, evt. foran tæppet} 

\says{G} Godaften. Jeg er Dr. Bullocks Bl\"odsinn

\says{D} Og jeg er Dr. Rubbish Quatsch

\says{G} Sidst vi var her, fortalte vi som bekendt om forskning indenfor feltet
kvindelogikprogrammering.
 
\says{D} Men i aften vil vi præsentere et spændende nyt forskningområde:
Tutelationsinduceret Kronologiforskydning. Eller med et andet ord,
forelæserudløste tidslommer.

\says{G} Her er tale om et fænomen som opstår i samspillet mellem ikke fuldt
kompatible forelæsere og studerende. En slags destruktiv

\says{D}[afbryder] eller potentielt konstruktiv

\says{G} interferens mellem hjernebølger på det subjektive plan. 

\says{D}Effekten udarter sig i påvirkning af den kronologiske strøm i systemet i
forhold til omgivelserne. Med andre ord, tiden går langsommere under kedelige
forelæsninger!

\says{G} De fleste har selv oplevet dette: Man sidder til forelæsning og kigger
på uret. Klokken er 10.45 og der er kun et kvarter tilbage. 10 minutter senere
kigger man igen på uret og tænker ``Nå, nu er der nok kun 5 minutter igen''. 

\says{D} Men AK! Klokken er kun 10.46. Dette er et eksempel på de tidslommer der
opstår under ekstremt kedelige forelæsninger.  

\says{G} Denne forskning har adskillige interessante anvendelser indenfor meget
videnskabeligt arbejde. En slags meta-videnskab, om man vil \ldots tøhø.
Eksempelvis kunne mange projektopgaver drage nytte af fænomenet
Tutelationsinduceret Kronologiforskydning. 

\says{D} Sidder man kort før afleverinsfristen
og intet virker, men opdager at i de næste to timer forelæser en tilfældig stiv
finne på et uforståeligt sprog, ja så kunne man med fordel fortsætte arbejdet i
det auditorium, hvorved man vinder tid proportionelt med forskellen i
interessefeltet mellem forelæser og studerende.

\says{G} I dette tilfælde adskillige dage. Denne proces er dog ikke uden sine
egne problemer. Det har vist sig at være en kraftig stressfaktor, idet folk ved
regelmæssig påvirkning føler at tiden smutter fra dem uden at de rigtig når
noget. 

\says{D} Da fænomenet er samspilsafhængigt, kræver det også en vis
deltagelse fra ``modtageren'' for at udløse. Det bliver med andre ord
problematisk at arbejde koncentreret med projektet, da man jo skal
følge \emph{lidt} med i forelæsningen. 

\says{G} En pudsig detalje ved dette fænomen er, at forelæserne selv synes at
udstråle et såkaldt tidslommecontainmentfield, der bevirker at de selv er
ganske immune overfor effekten. Jeg mener, ellers ville Gregers Koch jo være
ældre end Yoda. 

\says{D} Der er også en del spændende fremtidsudsigter. Selvom der endnu ikke er
fundet et konkret eksempel på det, indikerer den seneste forskning fra
ITTF, Institut for Teoretisk Tidslommeforskning, at
finder man en forelæser der er kedelig \emph{nok}, bør man kunne bruge ham til at
rejse tilbage i tiden.

\says{G} Vi har dog endnu ikke løst problemet med at udnytte denne egenskab, uden
at risikere at dø af kedsomhed. Men, den omvendte effekt kunne også tænkes at
udnyttes. Kunne man f.eks. finde en forelæser der var \emph{helt vildt}
interessant, ville tiden jo gå hurtigere. Det ville man kunne udnytte ved
rumrejser til Alpha Centauri og den slags\ldots

\says{D} Ja, men du kan jo godt høre hvor urealistisk det er. Indtil videre har
vi kun fundet en 20-årig kvindlig stripper der forelæser om jokes, mens hun
spiller på pornoorgel og mandolin, samtidig med at hendes kæleabe kravler rundt
på loftet. 

\says{G} Ja, dette gav os en marginal forbedring på 1\% hurtigere tidspassage
\ldots men det kunne lige så vel være måleusikkerhed, idet hendes abe var henne
og pille ved uret.
 
\says{D} Nå, jeg kan se at vores 30 sekunder er gået. Tak for i aften.

\scene{Tæppe}

\end{sketch}
\end{document}

%%% Local Variables: 
%%% mode: latex
%%% TeX-master: t
%%% End: 

