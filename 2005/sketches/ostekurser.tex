\documentclass[a4paper,11pt]{article}

\usepackage{revy}
\usepackage[utf8]{inputenc}
\usepackage[T1]{fontenc}
\usepackage[danish]{babel}

\revyname{DIKUrevy}
\revyyear{2005}
% HUSK AT OPDATERE VERSIONSNUMMER
\version{0.9$\beta$ }
\eta{6 minutter}
\status{Ganske færdig}

\title{Ostekurser}
\author{Ulla, Kofoed, NP, Mad\ss , m.fl.}

\begin{document}
\maketitle

\begin{roles}
\role{K}[Sandfeld] Kunde
\role{D}[Tischer] Dulle fra Delen
\role{KS}[Marvin] Kazoospiller

\end{roles}

\begin{props}
\prop{En kazoo}
\end{props}

  
\begin{sketch}

  \scene{Scenen er delen på DIKU. Det er kursustilmeldingstid. Kazoospiller står
    i baggrunden og spiller på kazoo. Ind kommer en distingveret herre. Bag
    disken står en af delens tøser.}
 
  \says{K}[Overmåde høfligt] God morgen.

  \says{D} God morgen, Hr. Velkommen til kursustilmeldingen.

  \says{K} Ah, tak, gode frøken.

  \says{D} Hvad kan jeg hjælpe dig med?

  \says{K} Nuvel, jeg sad, øh, på bibliotek for litteraturvidenskab ude på KUA og
  skimmede en særudgave af Anders And fra 1934, du ved jubilæumsudgaven, søgende
  efter samtidsreferencer, der kunne kaste lys over amerikanernes selvopfattelse
  under depressionen i 20'erne, og tænkte pludselig ved mig selv, at jeg trængte
  til noget illuminarius naturiæ.

  \says{D} Illu\ldots hvad?

  \says{K} Hungrende efter scientum digitalis.

  \says{D} Huh?

  \says{K} Jeg ville lære noget om computere!

  \says{D} Nårh, computere!

  \says{K} Ja præcis. Så jeg tænkte ``en lille udflugt til de nordlige afdelinger
  er nok hvad der skal til'', så jeg hæmmede min lyst efter Anders og de tre små
  nder, skyndte mig afsted, og infiltrerede jeres udbudsdomicil for at forhandle
  tilmelding til vidensuddeling på plads.

  \says{D} En gang til?

  \says{K} Jeg vil gerne melde mig til nogle kurser.

  \says{D} Nåh, jeg troede du brokkede dig over kazoospilleren.

  \says{K} Åh, Gud forbyde det: man sætter vel pris på alle blæseinstrumenternes
  muse?

  \says{D} Undskyld?

  \says{K} Jeg kan godt lide en god melodi, det bliver man vel nødt til.

  \says{D} Så det er okay at han spiller videre?

  \says{K} Selvfølgelig! Nå, men, nogle kurser, hvis de vil være så venlig.

  \says{D}[ivrig] Naturligvis, Hr. Hvad kunne de tænke dem?

  \says{K} Nuvel, hvad med en lille smule funktionsprogrammering?

  \says{D} Jeg er bange for at kursustilmeldingen til funktionsprogrammering
  liiiige er blevet lukket, Hr.

  \says{K} Nåh, pyt med det. Hvad med matematik og beregninger, så?

  \says{D} Jeg er bange for at vi aldrig har plads på matematik og beregninger i
  første blok, Hr.

  \says{K} Tsk tsk. Ligemeget. Nuvel, min gode frøken, så vil jeg gerne bede om
  to pladser på 1F, hvis de vil være så venlig. Jeg mangler nemlig stadig min
  kerneopgave.

  \says{D} Ah! Dat1F er -- desværre -- udgået. \act{pause} Og kerneopgaven er
  blevet frivillig og ikke en del af styresystemer og multiprogrammering. Men man
  kan tage et andet kursus for at skrive kerneopgaven.

  \says{K} Nå, men så skal jeg jo bare meldes til dét kursus.

  \says{D} \ldots Ja, men, desværre er det kursus jo så ikke blevet oprettet
  endnu.

  \says{K} Det' ikke min heldige dag, hvad? Aah, grafik?

  \says{D} Beklager, Hr.

  \says{K} Datanet?

  \says{D} Under normale omstændigheder, ja. Men i denne blok er det kun for folk
  på overgangsordningen.

  \says{K} Ah. Objektorienteret programmering?

  \says{D} Nej.

  \says{K} Algoritmer? Datastrukturer?

  \says{D} Nej.

  \says{K} Hvad med formel semantik, eventuelt?

  \says{D} Nej.

  \says{K} Robotter?

  \says{D} Nej.

  \says{K} Musik?

  \says{D} Nej.

  \says{K} Kode?

  \says{D} Nej, det bruger vi ikke.

  \says{K} Scientific computing?

  \says{D} \act{pause} Nej.

  \says{K} Billedbehandling og mønstergenkendelse?

  \says{D} Nej.

  \says{K} Multibody Dynamics Animation, Datamatarkitektur, Kryptologi,
  Algoritmer for evolutionelle træer, Læsegruppe i algoritmik, Datalogisk
  praktik, SOA, ASP, PHP?

  \says{D} Erhhh... Nej.

  \says{K} Visual Basic, måske.

  \says{D} Ah! Vi har Visual Basic, ja nemlig ja.

  \says{K}[overrasket] Har I? Udstående.

  \says{D} Ja, Hr. Det er\ldots lidt\ldots æh sejlende.

  \says{K} Oh, jeg kan godt lide når Visual Basic sejler.

  \says{D} Det sejler faktisk ret meget, Hr.

  \says{K} Ligegyldigt. Tilmeld mig flux dette kursus udi Microsofts ynder.
  Mmmwah!

  \says{D} Jeg tror\ldots faktisk det sejler mere end de vil bryde dem om, Hr.

  \says{K} Jeg er ligeglad med hvor dødssejlende det er. Skriv mig så på i al
  hast, så jeg kan få mine sidste 1.5 ECTS.

  \says{D} Oooooooooh\ldots!

  \scene{pause}

  \says{K} Hvad nu?

  \says{D} Det er lige blevet flyttet ud på institut for humanistisk informatik.

  \says{K} IHI?

  \says{D} Ja.

  \says{K} \ldots Aha.

  \scene{pause}

  \says{K} Cg programmering?

  \says{D} Nej.

  \says{K} Dat 2A?

  \says{D} Nej.

  \says{K} Dat 2P?

  \says{D} Nej.

  \says{K} Dat 1P?

  \says{D} Nej.

  \says{K} Database tuning?

  \says{D} Nej.

  \says{K} I\ldots har nogle kurser, har I ikke?

  \says{D}[kvik] Selvfølgelig, Hr. Det er trods alt et institut, det her. Vi
  har\ldots

  \says{K} Nej nej, sig det ikke\ldots Jeg skal have lov til at gætte det.

  \says{D} Fair nok.

  \says{K} Æhhhhhh\ldots Ada?

  \says{D} Ja?

  \says{K} Alletiders, så tager jeg det kursus.

  \says{D} Oh! Jeg troede du snakkede til mig, Hr. Det er mit navn. Ada.

  \scene{pause}

  \says{K} Probabilistiske net?

  \says{D} Formentlig ikke.

  \says{K} Geometriske algoritmer, datastrukturer og paradigmer?

  \says{D} Nej.

  \says{K} Introduktion til omvendt deklarativ programmering?

  \says{D} Nej.

  \says{K} Menneske-datamaskine interaktion i IT-støttet undervisning?

  \says{D} Ikke i dag, Hr.

  \scene{pause}

  \says{K} Nå, hvad med C++?

  \says{D} Tja, det bruger vi ikke til så meget her, Hr.

  \says{K} Ikke så meget!??! Det er et af de mest udbredte sprog i verden!

  \says{D} Ikke her, Hr.

  \says{K} \act{kort pause} \ldots og hvad \emph{er} så det mest udbredte sprog
  her?

  \says{D} SML, Hr.

  \says{K} \emph{Er} det?

  \says{D} Oh, jo, det er frygteligt populært her på DIKU, Hr.

  \says{K} Er det.

  \says{D} Det er det sprog alle lærer, Hr.

  \says{K} Aha. Æh\ldots SML, hvad?

  \says{D} Nemlig, Hr.

  \says{K} All right. Okay. 'Har i et kursus?' spurgte han, med forventning om
  svaret 'nej'.

  \says{D} Lad mig lige tjekke\ldots Nnnnnnnnnej.

  \says{K} Det er et rimeligt ringe institut, det her.

  \says{D} Det bedste på disse egne.

  \says{K}[irriteret] Forklar venligst logikken bag den konklusion.

  \says{D} Det er ikke fysik, vel?

  \says{K} Nej okay, det kan jeg godt se, men\ldots

  \says{D}[kvik] Du har ikke spurgt mig om et HCI-kursus med Erik Frøkjær med
  forelæsninger hver anden søndag kl. 7-23, hvor eksamen bliver afholdt på
  Thulebasen på spansk, men med grønlandsk censor.

  \says{K} Ville det være dét værd.

  \says{D} Muligvis\ldots

  \says{K} Har I --- SÅ STOP DOG MED DEN BELASTENDE KAZOO!!!

  \says{D} Det var det jeg sagde\ldots

  \scene{KS stopper med at spille. Går ud}

  \says{K}[langsomt] Har I nogle HCI-kurser med Erik Frøkjær\ldots blah blah
  blah?

  \says{D} Nej.

  \says{K} Ikke overraskende. Decideret forudsigeligt faktisk. Det ren og skær
  ubegrundet optimisme der fik mig til at formulere spørgsmålet under alle
  omstændigheder. Sig mig:

  \says{D} Ja?

  \says{K}[med overlæg] Har I overhovedet nogle kurser her?

  \says{D} Ja da.

  \says{K} Virkelig?

  \scene{pause}

  \says{D} Nej. Egentlig ikke, Hr.

  \says{K} I har ingen.

  \says{D} Nej, Hr. Ikke et eneste. Det var med vilje jeg spildte din tid på
  denne måde.

  \says{K} Hvad synes du så jeg skal gøre?

  \says{D} Du kom fra KUA, ikke?

  \says{K} Æh, jo\ldots

  \says{D} Drop nu ud!

  \scene{Tæppe}

\end{sketch}
\end{document}

%%% Local Variables: 
%%% mode: latex
%%% TeX-master: t
%%% End: 

