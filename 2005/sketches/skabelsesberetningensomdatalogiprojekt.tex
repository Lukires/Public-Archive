\documentclass[danish]{article}
\usepackage{revy}
\usepackage[utf8]{inputenc}
\usepackage{babel}
\usepackage{a4wide}

\title{Skabelsesberetningen som datalogiprojekt}
\author{Theo, Uffe, nokando og Jakob Uhd af revyen}

\version{5.0} % HUSK AT AJOURFØRE VERSIONSNUMMER!!
\status{færdig} % ...OG STATUS!!
\eta{9 min.}
\revyname{DIKURevy}
\revyyear{2005}

\begin{document}
\maketitle

\begin{roles}
  
  \role{F}[Sandfeld] Forelæser   
  \role{VO}[Uhd] Voice-over m. stor stemme med echo-reverb om muligt
  \role{S1}[Ulla] Studerende 1
  \role{S2}[PeterJo] Studerende 2
  \role{S3}[Eva] Studerende 3
  \role{S4}[Madss] Studerende 4
  
\end{roles}

\begin{props}

  \prop{et bord}

  \prop{4 stole}
  
  \prop{Gammel computer} Evt. overhead som skærm. 

  \prop{4 øl} 

  \prop{4 lagner til togaer}
  \prop{4 sæt englevinger}
\end{props}

\begin{sketch}

\scene En forelæser (F) kommer ind på scenen med en talerstol. Den stilles op
umiddelbart foran tæppekanten. Tæppet er trukket for.

\says{F} På grund af den aktuelle trend med at faseinddele alle elementer af et
menneskes liv, har DIKU-revyen forberedt en gennemgang af en rapportgruppes 7
standardfaser.

\says{F} Som eksempel har DIKU-revyen valgt at bruge Den Første
Rapportopgave\ldots ``Projekt: Himmel og Jord'' skrevet og udført af Gud (og
Himlens Kor af Engle). Gennemgangen vil være krydret med uddrag fra selve
rapporten.

\scene Tæppet trækkes fra og man kan se de fire engle (S1-4). Lyssætningen
bruges til at skifte mellem F, VO og de 4 engle. Dæmp lyset når der er VO så det
virker som om virkeligheden er "på pause" mens der tales. Ligeledes fokuseres
lyset på F når han forelæser.

%%%%%%%%%%%%%%%%%%%%%%%%%%%%%%%%%%%%%%%%%%%%%%%%%%%%%%%%%%%%%%%%%%%%%%%%%%%%%%%

\scene Første dag

\says{F} Første fase: Opgaven stilles. Vi bringer her et forkortet og
illustreret uddrag af besvarelsen.

\says{VO} I begyndelsen skabte Gud himlen og jorden. Jorden var dengang tomhed
og øde, der var mørke over urdybet, og Guds ånd svævede over vandene.

\says{S1} Nåh, skal vi se at komme igang, der er en uge til, vi skal aflevere.

\says{S2} Der skal altså implementeres en del.

\says{S3} Nogle mirakler og vidundere?

\says{S2} Ikke mange -- 8 mirakler og 7 vidundere, men der er noget med nogle
dyr og \ldots mennesker.

\says{S3} Ja, og der var mere, hvor var det nu det var, ja her! Fysikere!

\says{S4} Der skal altså også skrives noget rapport!

\says{S1,S2,S3}[Opgivende] Ja, ja, det kommer vi til. Først skal vi kode!

\says{VO} Gud sagde: ``Der skal være lys!'' Og der blev lys.

\scene Der tændes for computeren

\says{VO} Gud så, at lyset var godt. Så blev det aften, og det blev morgen,
første dag.

%%%%%%%%%%%%%%%%%%%%%%%%%%%%%%%%%%%%%%%%%%%%%%%%%%%%%%%%%%%%%%%%%%%%%%%%%%%%%%%

\scene Anden dag

\says{F} Anden fase: Opgaveteksten analyseres. Gruppen er stadig ved godt humør
og prøver ivrigt at afdække hvad opgaven går ud på.

\says{VO} Gud sagde: ``Der skal være en hvælving.'' Gud kaldte hvælvingen
himmel. Så blev det aften, og det blev morgen, anden dag.

\says{S2}[usikker] Æh... altså, ``Himmelen'', er det den der er placeret
umiddelbart over ``Hel''?

\says{S1} Ja ja! Lissom "Penthouse", bare i sydfløjen!

\says{S3}[peger på sine papirer] Hm, jeg forstår ikke helt den her
reference til ``jorden og vandene''. Skal de så ligge under Penthouse? 

\says{S4}[afbryder] Der skal altså også skrives noget rapport!

\says{S1,S2,S3}[Opgivende] Ja, ja, det kommer vi til. Først skal vi kode!

\scene Der spilles et spil på computeren

%%%%%%%%%%%%%%%%%%%%%%%%%%%%%%%%%%%%%%%%%%%%%%%%%%%%%%%%%%%%%%%%%%%%%%%%%%%%%%%

\scene{Tredie dag}

\says{F} Tredje fase: Forvirring. Analysen af opgaveteksten kan slå mange
grupper ud. Der optræder ofte sorte vendinger, ufærdige sætninger og ikke
gennemtænkte krav.

\says{VO} Gud sagde: ``Jorden skal grønnes: Planter, der slår rod, og alle slags
frugttræer, med blade, der bærer frugt med kerne, skal være på jorden.'' Gud så,
at det var godt. Så blev det aften, og det blev morgen, tredje dag.

\says{S2} Træer, rødder og blade -- jeg forstår det altså ikke.
Kravspecifikationen hænger altså ikke sammen. Er der nogen, der har tænkt på
målgruppen? Hvem har egentlig skrevet opgaven, Laszlo?

\says{S3}[peger på sine papirer] "\ldots{}der bærer frugt med kerne\ldots{}"
Kerne?

\says{S1} Ja, på DIKU koder de kerne!

\says{S4} Der skal altså også skrives noget rapport!

\says{S1,S2,S3}[Opgivende] Ja, ja, det kommer vi til. Først skal vi kode!

\scene Der spilles et andet spil på computeren

%%%%%%%%%%%%%%%%%%%%%%%%%%%%%%%%%%%%%%%%%%%%%%%%%%%%%%%%%%%%%%%%%%%%%%%%%%%%%%%

\scene{Fjerde dag}

\says{F} Fjerde fase: Druk. Som typisk modreaktion på forvirringen, tyer mange
grupper til alkohol.

\says{VO} Gud sagde: ``Der skal være lys på himmelhvælvingen til at skille dag
fra nat. De skal tjene som tegn til at fastsætte festtider, dage og år.'' Gud så,
at det var godt. Så blev det aften, og det blev morgen, fjerde dag.

\says{S1} Hmm\ldots festtider\ldots

\says{S2}[rip] Der skal

\says{S3}[rap] også være

\says{S4}[rup] plads til

\says{S1,S2,S3,S4}[storsmilende] en enkelt øl!

\scene Der skåles med publikum

%%%%%%%%%%%%%%%%%%%%%%%%%%%%%%%%%%%%%%%%%%%%%%%%%%%%%%%%%%%%%%%%%%%%%%%%%%%%%%%

\scene{Femte dag}

\says{F} Femte fase: Tømmermænd. Det er allerede her at nogle grupper falder
fra. 

\says{VO} Gud sagde: ``Vandet skal vrimle med levende væsener, og fugle skal
flyve over jorden.'' Gud så, at det var godt. Så blev det aften, og det blev
morgen, femte dag.

\says{S4}[med ondt i hovedet] Der skal altså også skrives noget rapport!

\says{S1,S2,S3}[Opgivende, med ondt i hovedet] Ja, ja, det kommer vi til. Først
skal vi kode!

\scene Der spilles et tredje spil på computeren

%%%%%%%%%%%%%%%%%%%%%%%%%%%%%%%%%%%%%%%%%%%%%%%%%%%%%%%%%%%%%%%%%%%%%%%%%%%%%%%

\scene{Sjette dag}

\says{F} Sjette fase: Det første hack. Gruppen indser hvor tæt de er ved at være
ved deadline og begynder at udfærdige løsningsforslag. Der er dog ikke nogen
forsikring for at løsningerne kan bruges til noget.

\says{VO} Gud sagde: ``Jorden skal frembringe alle slags levende væsener, kvæg,
krybdyr og alle slags vilde dyr!'' Gud så, at det var godt. Gud skabte mennesket
i sit billede. Og Gud så alt, hvad han havde skabt, og han så, hvor godt det
var. Så blev det aften, og det blev morgen, den sjette dag.

\says{S1} Hvordan pokker skal vi kunne implementere fisk, fugle, kvæg og alt det
andet -- det er jo slet ikke defineret i opgaven?

\says{S3} Så må vi tænke os til det.

\says{S2} En studerende! Hvad med det her? \act{Holder en tegning op af en
  fysiker}

\says{S1, S3, S4} Arj, nu må du tage dig sammen!

\says{S2} Nåhja, de studerende skal udstyres med intelligens\ldots

\says{S3}[roder under bordet] Fuck, fuck, der er ikke mere intelligensserum

\says{S1} Nåh, så får de et 'Caféen?'-gen i stedet, det er der sgu
ikke nogen, der lægger mærke til.

\says{S4} Der skal altså også skrives noget rapport!

\says{S1,S2,S3}[Opgivende] Ja, ja, det kommer vi til. Først skal vi kode!

\scene Der spilles et fjerde spil på computeren
%%%%%%%%%%%%%%%%%%%%%%%%%%%%%%%%%%%%%%%%%%%%%%%%%%%%%%%%%%%%%%%%%%%%%%%%%%%%%%%

\scene{Syvende dag}

\says{F} Syvende fase: Panik, slamkode og kaffe. Mange rapportgruppers arbejde
kulminerer i en sløringsfase; løgn, latin og mavesår.

\says{VO} Således blev himlen og jorden og hele himlens hær
fuldendt. På den syvende dag var Gud færdig med det arbejde, han havde
udført, og på den syvende dag hvilede han efter alt det arbejde, han havde
udført.

\says{S4}[knapper løs på C64eren] ``...og på den syvende dag hvilede han efter
alt det arbejde, han havde udført.''

\says{S3} Så mangler vi bare rapporten.

\says{S4} Det er jo den jeg har siddet og skrevet! Jeg har taget notater og
skrevet dem ind, nu mangler jeg bare konklusionen, men den synes jeg, vi skal
skrive sammen.

\says{S1} Ok, hvad med\ldots ``Afprøvningen afslørede ingen fejl, så livet
forventes at være perfekt.''

\says{S2} Har vi lavet afprøvning? Nåh\ldots \act{smiler nervøst, ryster
  undskyldende med hænderne)} nevermind!

\says{S3} Ja, og så afslutter vi med \act{rømme, rømme} ``Projektet forløb efter
planen. Faktisk blev vi færdig en dag før tid! Vi mener himmelen, jorden og
livet på den faktisk kan bruges til noget, og håber de vil blive brugt. Meningen
med livet har vi vedlagt som bilag 42.''

\says{VO} Gud velsignede den syvende dag og helligede den, for på den dag
hvilede han efter alt det arbejde, han havde udført, da han skabte. Det var
himlens og jordens skabelseshistorie.

\scene{S1,S2,S3,S4 er totalt udbombede...}

\says{S1}[tænkepause] Det var sgu rart at blive færdige.

\says{S2}[tænkepause] Vi får jo ikke 10 for det her.

\says{S3}[tænkepause] Egentlig er alt over 6 jo også spildt arbejde.

\says{S4} Men næste gang, da skal vi i gang i ordentlig tid. Planlægge det hele
fra starten. Så kan vi blive færdige i ordentlig tid.

\says{S1} Ja, det er godt nok blevet noget rod. Næste gang skal vi nok
heller ikke sidde hele natten. Hvad er egenteligt det næste projekt?

\says{S2} Noget med en ark og en masse vand\ldots Sikkert en slags
spildopsamling\ldots{} Det bliver en ordentlig omgang.

\says{S3} Ja, vi for travlt ad himlen til.

\scene Tæppe for.

\says{F} Dette afrunder vores indblik i en typisk rapportperiodes 7
standardfaser. Det skal nævnes at der til tider optræder andre faser i et
forløb; f.eks. ``genstart'' fasen, hvor et ledende element i gruppen indser at
man må begynde forfra, ofte med et nyt styresystem; ``den sorte fase'', hvor
mangler i koden sløres ved kraftig overdokumentation, gerne ved hyppig brug af
fremmedord, og den sjældne ``afprøvningsfase'', hvor medlemmerne bekræfter at de
fejl og mangler der mistænkes at være, rent faktisk også er der. Tak skal I have.

\scene Væk!

\end{sketch}

\end{document}
% Local Variables: 
% mode: latex
% TeX-master: t
% End: 
