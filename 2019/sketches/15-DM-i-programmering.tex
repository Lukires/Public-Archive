\documentclass[a4paper,11pt]{article}

\usepackage{revy}
\usepackage[utf8]{inputenc}
\usepackage[T1]{fontenc}
\usepackage[danish]{babel}


\revyname{DIKUrevy}
\revyyear{2019}
\version{1.0}
\eta{$5$ minutter}
\status{Færdig}

\title{DM i programmering}
\author{Niels, Simon, Søren P., Markus og Caroline}

\begin{document}
\maketitle

\begin{roles}
  \role{P}[Arinbjørn] Preben
  \role{D1}[Magnus] Floppy
  \role{D2}[Jonas] n00bslayer37
  \role{D3}[Michael] Polymorpheus
  \role{V}[Ejnar] Voiceover
  \role{M}[Buchter] Matematikeren
  \role{MF}[Torben] Matematikerens følgesvend
  \role{BM}[Matilde] Ballonmand
  \role{S}[Amalie] Shirley
  \role{P1}[Eva] Frida
  \role{P2}[Lotte] Ghita
  \role{P3}[Mette] Rita
  \role{X}[Simon] Instruktør
\end{roles}

\begin{sketch}
  \scene{D* + P sidder klar på scenen i kantinen}
  \scene{Banner: DM i Programmering, sponsoreret af JobCompany og NetIndex (ingen skjult reklame) - lån evt noget}
  \scene{Skilt: Vind en drone}

  \scene{AV: Ding Dong}
  \says{V} Velkommen til DM i Programmering
  \says{V} Konkurrencen går i gang senere
  \says{V} Husk at checke ind

  \says{P} Hvad så drenge, er I klar til at vinde DM?
  \scene{AV: Check ind lyd}

  \scene{M kommer ind på scenen}
  \says{M} Hva' så drenge - er I klar til at tabe DM?
  \says{M} Min Maple har en konfiguration, med makroer til alle mulige løsninger til alle mulige problemer til alle mulige konkurrencer til alle mulige vejrforhold!
  \says{D2} Øh, det er ret meget
  \says{M} Ja, det er mere end hvad en datalog som dig forstår!
  \says{M} (grynt) Jeg elsker Maple.
  \says{D3} Hallo, vi vinder, for vi har overclocket vores datamat!
  \says{M} (grynt) haha, mine algoritmer kører O(1)
  \scene{M tager kridt frem fra øret}
  \says{M} Og så klarer jeg i øvrigt det hele i hånden
  \scene{M knipser, MF kommer ind med et whiteboard (som Igor)}
  \says{M} Haha, fjollede datalog
  \scene{ M begynder at skrive på whiteboardet og opdager at han ikke kan skrive på det}
  \says{M} (til MF) Hvad er det her? Sig mig, prøver du at sabotere mig?... idiot!
  \says{MF} Undskyld, undskyld, der var nogen der havde skrevet på alle tavlerne.
  \says{M} Ååårhhh, tåbelige rus ... Fjern det lort
  \says{D1} Haha, nu kan du ikke skrive noget kode
  \says{M} Matematikkere behøver ikke at lave koden, men bare bevise at den \emph{kan} laves
  \says{M} Og jeg har dem alle sammen i Maple...

  \scene{AV: Ding dong lyd}
  \says{V} Konkurrencen skal til at begynde
  \says{V} Vi gennemgår nu reglerne
  \says{V} Når du har løst en opgave får du en ballon
  \says{V} Og det var reglerne
  \says{V} Konkurrencen er nu startet

  \scene{M sidder med en finger og meget store bevægelser, trykker Q E D gentagende}
  \scene{Ballonmand giver løbende balloner til M. De sættes fast bag på stolen}
  \scene{AV: to skærme på overtex, ala O(L)}

  \says{P} Hov vent, vi skal lige sætte editoren op
  \scene{Skærmdat sætter terminalen til sort på grøn}
  \says{D2} Nu er vi ægte hackere
  \says{D1} Nej, ægte hackere bruger grøn på sort!

  \says{M} Denne opgave kan løses - QED
  \says{M} Denne opgave kan også løses - QED
  \says{M} Også denne opgave kan løses - QED

  \says{P} Okay drenge, Lad os lave noget opgavefordeling
  \says{D3} Ja, Floppy du tager musen, så er vi andre på tastaturet
  \scene{De begynder at skrive kode (3 linjer kode)}
  \says{D1} Er det forresten ikke meget hurtigere at bruge Dvorak?
  \says{D3} Jo, lad os bruge det
  \says{P} Hvordan gør man?
  \scene{Der kommer random bogstaver ind efter kode, masser af \$\@\#\%}
  \says{D2} Hov se, vi koder Perl

  \says{M} Det er muligt at løse opgaven - QED
  \says{M} Løsningen til denne opgave eksisterer - QED

  \says{D3} Hov, vent, har I set at man må bruge Prolog?
  \says{P} Fedt, lad os gøre det!
  \scene{Holder backspace inde}

  \says{M} Denne her opgave ... har en løsning - QED
  \says{M} Denne opgave er svær QE... (venter efter Q E), men kan løses -(D)

  \says{D1} Har I set, der er kun gået 10 min og der er en der allerede har løst 40 opgaver
  \says{D3} Men der er jo kun 20 i sættet!
  \says{M} Dumme dataloger. Jeg har bare bevist at der var flere opgaver end dem i sættet - et simpelt induktionsbevis ... QED haha grynt \scene{Han får en ballon mere}

  \scene{M begynder at lette på sin stol (fiks det med ninjaer!)}
  \says{M} Hvad sker der? Åh nej, jeg er for god!
  \says{M} Skide dataloger. I har ikke set det sidste fra mig.
  \scene{M flyver du af scenen}
  \scene{En cutout af M flyver op bag tæppet og hænger under loftet}

  \scene{AV: Ding dong}
  \says{V} Og konkurrencen er nu slut
  \scene{Dommeren kommer ind på scenen med en mikrofon. Det er ham der også er voiceover.}

  \says{V} Og vinderen er...
  \scene{Dommeren kigger forvirret rundt}
  \says{V} Hov, hvor blev ham den gode af? Nå, så er det vel jer der har vundet.

  \scene{S står på trappen i auditoriet ved tæppeindgangen og klapper ihærdigt. P1 står ved siden af}
  \says{S} Wuuuuhuuuuu, Preben, Preben, Preben, du er for nice
  \says{S} Hov Frida, hvorfor lægger han slet ikke mærke til mig
  \says{S} Nu har jeg jo lige besluttet at det \emph{skal} være ham jeg vil have. Hvorfor forstår han det ikke?
  \says{P1} Shirley, har du overvejet at være lidt...mindre dig selv?
  \says{S} Du forstår dig på nørder ikke?
  \says{P1} (afbryder) Ej, det gør jeg ihvertfald ikke...
  \says{S} Aj, vi ved jo godt at du er en nørd, men vi holder af dig alligevel. Det handler jo ikke om at være nørdet eller ikke nørdet, men om at være et menneske.
  \scene{P2 og P3 kommer gående ud fra tæppet}
  \says{P1} Ad, sagde du lige det der? Shit hvor er du russet! Jeg er ihvertfald ikke nørd!
  \says{P3} Aj, Frida, vi hørte dig synge. Du er jo helt sikkert en nørd!
  \says{P1} [protesterer] Nej, nej, nej...
  \says{P3} [tysser med hånden] Men Frida, vi er kommet frem til at du måske er et okay menneske alligevel - og så havde du måske egentlig en pointe - så nu er vi også nørder.
  \says{P2} Jaja, det er fordi vi fandt ud af at Prebens venner egentlig er meget søde \act{sender fingerkys ned mod scenen - D2 ser det og besvimer}
  \says{S} Aj vil I så ikke godt hjælpe mig med at få Preben tilbage?
  \says{S} Please, please, please
  \says{P1} Arhh, okay så
  \scene{S og P* går op ad trappen, og ud af midterdøren}

  \scene{Drengene står og jubler}
  \scene{Preben ser Shirley forlade salen}
  \says{P} Shirley, hey Shirley
  \scene{Preben kalder på Shirley, der ikke ser tilbage}
  \says{P} Shirley
  \scene{Spot på P mens han sidder på knæ og hulker og kalder på S}
  \says{P} Hvorfor kan du ikke lide mig mere? Var det fordi jeg var ond og totalt nederen overfor dig?
  \says{P} Hva' Shirley. Var det \emph{det}?
  \says{P} \emph{snøft} \emph{snøft}
  \says{P} Shirley
  \scene{Mørk scene uden om ham (der må gerne være lidt spot på ham) (bandet kan evt spille noget trist)- sceneskift}

\end{sketch}



\end{document}
