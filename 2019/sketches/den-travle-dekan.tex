\documentclass[a4paper,11pt]{article}

\usepackage{revy}
\usepackage[utf8]{inputenc}
\usepackage[T1]{fontenc}
\usepackage[danish]{babel}


\revyname{DIKUrevy}
\revyyear{2019}
\version{0.1}
\eta{$4$ minutter}
\status{Færdig}

\title{Den travle dekan}
\author{Niels}

\begin{document}
\maketitle

\begin{roles}
\role{D}[Sean] SCIENCE-fakultetets dekan.  Har travlt og mange vigtige ærinder.
\role{A}[Marc] SCIENCE-fakultetets administrationschef.  Har været på kurser.
\role{S}[Pauline] Dekanens sekretær.  Bærer rundt på en blok fuld af noter.
\role{M}[Æmilie] Maskinen.  Har en god robotstemme.
\role{V0}[Henrik] Universitetsvagt.
\role{V1}[Brandt] Universitetsvagt.
\role{S0}[Jonas] Finn.  Datalog der aldrig er kommet videre fra 90'erne.
\role{S1}[Magnus] Sigurd.  Politisk og social aktiv fysikstuderende.
\role{S2}[Lotte] Ulla.  Meget klog studerende.
\role{N}[Arinbjørn] Sceneninja
\role{X}[Niels] Instruktør
\end{roles}

\begin{props}
\prop{Fin væg med krog på}[]
\prop{Maskinekostume til M, arme indeni, så M kan rulle papir ud}[]
\prop{Papirsstrimmel der kommer ud af M}[]
\end{props}

\begin{sketch}

\scene{Lys op.  På scenen er dekanens kontor.  På en væg hænger et stort maleri
af dekanen.  D kommer spadserende ind i fin overfrakke.}

\says{D} Ej-ej-ej, at være dekan! \act{til publikum, hurtigt, overraskende} Man
har travlt!

\scene{D tager en bøjle frem fra inderlommen og tager stille og roligt sin overfrakke
af og hænger den på bøjlen.  D kigger rundt for at se om der et sted at hænge
den, men kan ikke se noget.}

\says{D}[mod sceneindgangen] Sekretær!

\scene{S kommer hurtigt ind.}

\says{S}[hilser] Dekan!

\scene{S ser overfrakken på bøjlen og tager den straks.  S kigger rundt for at
finde et sted at hænge den.}

\says{D}[går frem på scenen, kigger mod publikum] Hvad står dagens program på?

\scene{S går rundt på scenen under næste replik for at finde et sted at hænge
bøjlen.}

\says{S} Dagens program er:

Kl. 10-12: Møde med universitetsbestyrelsen om Niels Bohr Bygningen.

Kl. 12-13: Frokost i slotshaven.

\scene{S har fundet maleriet.  S tager maleriet ned og hænger bøjlen med
overfrakken op i stedet.  S står nu med maleriet og kigger rundt.  D bliver lidt
suspekt over at S ikke har sagt noget i et par sekunder og begynder at vende sig
om, men S begynder at tale hurtigt nok igen til at D bliver stående og kigger
frem.}

\says{S}[fortsætter]

Kl. 13-17: Møde med statsministeren\ldots om Niels Bohr Bygningen.

Kl. 17-17:15: Underskrivning af kandidatbeviser for nyligt uddannede studerende.

\scene{D nikker anerkendende.  S giver opgivende maleriet til D.  D er forvirret
og kigger på det.  D ser sig selv.}

\says{D}[pludseligt, uforklarligt, skræmt] AAAAAA!!!

\scene{D smider maleriet væk.}

\says{D} \act{får besindigheden tilbage} \ldots Hooov, vent!  Det lyder da ikke
helt rigtigt\ldots \textbf{et kvarter} til at underskrive kandidatbeviser?  Så
kommer jeg jo for sent til squash med dekanen for jura! \act{laver slag med
luftketcher}

\says{S} Jamen, dekan, det \emph{skal} gøres -- ellers får vi ikke penge fra
staten.

\says{D}[indser alvoren]: \textbf{Gisp} -- frugtordningen!  \ldots Jamen, det må
da kunne klares hurtigere.

\says{S}[undrende] ``Hurtigere''?  \act{bladrer igennem sin notesblok for at
finde noget om hurtighed} ``huuurtigere''?  ``húrtigëre''? \ldots Nej, det ved
jeg ikke noget om, dekan.

\says{D} Det var lige godt sata--\act{afbrydes}

\scene{A kommer ind.}

\says{A} Var der nogen der sagde ``hurtigere''?

\says{D+S} Administrationschef Andersen!

\says{A}[slick] Netop hjemvendt fra kursus.  Det I ser foran jer nu er\ldots en
SCUM-mester!

\scene{S tager noter, men stopper.}

\says{S} Hedder det ikke ``SCRUM''?

\scene{A skal til at sige noget, men vælger ikke at svare.}

\says{A}[mod sceneindgangen] Vagter!

\scene{V0 og V1 ruller en gammel bærbar ind.}

\says{A} Lad mig præsentere: \act{tænker} SCUM-maskinen!

\scene{V0 og V1 går ud igen.}

\says{D} Men Andersen, kan\ldots SCUM-maskinen hjælpe \emph{mig}?

\scene{A tager præsentations-clicker frem fra lommen og klikker.  Der toner et
slideshow frem på OverTeX.  S tager hektisk noter under hele slideshowet.}

\scene{Slide 0: Et stort fotografi af Ghita Nørby der ser træt ud.}

\says{A}[til D] Ved du hvor gammel Ghita Nørby er?

\says{D} 100 år?

\says{A} Sikkert! \act{klikker}

\scene{Slide 1: En gravsten hvor der står ``Ghita Nørby - kendt fra 'Han, hun,
Dirch og Dario''' på.}

\says{A} Hvad tror du dagbladene gør når Ghita stiller træskoene?

\says{D}[tænker] \ldots De laver en nekrolog!

\scene{A klikker.}

\scene{Slide 2: Et foto af en fabrikeret Politiken-artikel med overskriften
``Vores allesammens 'Marianne Borg' er gået bort'' og underoverskriften ``Var
også med i Matador''.}

\says{A} Men har du aldrig tænkt over hvordan de kan have en nekrolog ude så
hurtigt?

\says{D} Jo, det er fordi de skriver dem i forvejen.

\says{A}[forventede ikke at D vidste svaret] Øh, ja \act{klikker igennem 20
slides med ligegyldige komplekse diagrammer der ellers forklarede det -- skal
køres forbi meget hurtigt, og S skal have meget svært ved at følge med i at tage
noter}.

\scene{Slide 23: Teksten ``NEKROLOG = NY STUDERENDE'' i stor fontstørrelse.}

\scene{D kigger skiftevis på slidet og den indrullede maskine.}

\says{D} Ej, vi skal vel ikke\ldots dræbe studerende\ldots vel?

\says{A} Neeej\ldots neej? \act{tænker over det} Neej, haha, jeg siger bare at
\act{slår armen rundt om skulderen på D} ved studiestart, når alle de studerende
starter, så bruger vi\ldots SCUM-MASKINEN til \emph{på forhånd} at udregne hvem
af dem der vil blive færdige -- og så kan du underskrive alle \emph{deres}
beviser på én gang!

\scene{OverTeX slukkes.  S er forpustet af at have taget så mange noter.}

\says{D} På \emph{under et kvarter}?

\says{A} På \emph{SCUM-mester-ære}.

\says{D} Jamen så er det i ord--\act{afbrydes af S}

\says{S} Men, dekan!  Kan vi nu være \emph{sikker} på at maskinen er fuldstændig
præcis?  Ellers får vi jo skældud af\ldots ministeren!

\scene{D reagerer og kigger spørgende på A.}

\says{A} Ja, den er kodet af vores bedste datalogistud--\act{afbryder sig selv}

\scene{A kigger på D, der pludseligt er blevet meget bidsk i sit blik.  S ser
det og skynder sig hen og hvisker noget i ørene på A.  D bliver mere og mere
bidsk.}

\says{A}[fortsætter] --øh, \emph{fysik}studerende! \act{kigger på S, som giver
thumbs up -- det var det rigtige at sige, for dekanen er fysiker}

\scene{D er glad igen.}

\says{D} Jamen \emph{så} er det i orden.  Sekretær, skriv det i kalenderen til
september!

\scene{D begynder at gå sin vej.}

\scene{S bladrer hen til kalendernoterne for september og noterer et møde der.}

\says{A}[råber efter D] Men vil du ikke teste maskinen nu for at være
\emph{helt} sikker?

\says{D}[stopper op, taler til sig selv, gyser] Ministeren\ldots

\scene{D vender sig om.}

\says{D} \ldots Jo, selvfølgelig.  Kva--\act{tænker sig virkelig
om}--\textbf{kvalitet} fremfor alt!  Jeg skulle bare hente en\ldots kaffe!

\scene{D stikker sin hånd ud bag bagtæppet, men får en banan ind.  D prøver at
skjule det.}

\says{A} Vagter!

\scene{V0 og V1 kommer ind med S0 i deres greb.}

\says{A} Dekan, her har vi så for eksempel Finn.  Maskine?

\scene{S noterer stadig helt vildt hektisk i sin notesblok, og gør det især når
maskinen taler.  S skal også strege ting over nogle gange.  Publikum skal kunne
se på S's noteskrivning om noget går godt eller skidt.}

\scene{S0 har taget nogle colaer med og giver dem til V0 og V1.  S0 har også en
til sig selv.  V0 og V1 lægger colaerne fra sig, men S0 tilbyder dem bare nogle
flere.}

\scene{Der kommer en strimmel papir med beregninger ud af M.  S tjekker jævnligt
om strimlerne passer overens med hvad hun noterer.}

\says{M} FINN DRIKKER DAGLIGT 2 LITER COLA OG VIL DØ OM 467,5 DAGE, HVILKET VIL
FORSINKE HANS STUDIETID OG FÅ HAM SMIDT UD EFTER YDERLIGERE 627,5
dage. \act{pause} UNDERSKRIFT IKKE NØDVENDIG.

\says{D} Godt, godt.  \act{til S} Nu husker du at tage noter, ikke?

\says{S}[kigger op, træt] \ldots Ja, dekan.

\scene{D begynder bare at spise af sin banan.}

\says{A} Vagter!

\scene{V0 og V1 går ud med S0 og kommer ind med S1.}

\says{A} Maaaaaaa-skine!

\scene{S1 har et banner med ind og beder V0 og V1 om at holde hver deres
bannerpind.  V0 og V1 ved ikke helt hvad de går ind til, men når banneret bliver
spændt ud, står der ``UDDANNELSE -- IKKE UDANNELSE!'' (eller noget lignende der
heller ikke er særligt godt, men som har gode tanker bag).  S1 jubler.}

\scene{Der kommer en strimmel papir med beregninger ud af M.}

\says{M} SIGURD DELTAGER FLITTIGT I SCIENCERÅDET FOR AT FINDE STUDENTERPOLITISKE
LØSNINGER TIL UNIVERSITETETS PROBLEMER, BYGGER REKVISITTER TIL FYSIKREVYEN OG
UDARBEJDER SJOVE FÆLLESØVELSER FOR DE NYE STUDERENDE.  I SIN FRITID KAN SIGURD
GODT LIDE AT PRØVE AT FØLGE MED I PENSUM. \act{pause} UNDERSKRIFT IKKE
NØDVENDIG.

\scene{S har prøvet at følge med i at notere det hele, men har ikke nået det.
Udmattet kigger hun på strimlen, tjekker at ingen ser hende, og river så
strimlen over og maser den ned i sin notesbog.}

\says{D} \act{taler til selv og kigger på sin nu spiste banan, er trist}
Frugtordning\ldots

\says{D}[til A] Men\ldots der må vel også være \emph{nogen} der kan gennemføre
et af vores fine studier?

\says{A}[selvsikker] Bare vent.  Vagter!

\scene{V0 og V1 går ud med S1 og kommer ind med S2.}

\says{A} Ma-skiiiiiiiiineeee!

\scene{S2 har taget nogle videnskabelige artikler med og deler dem ud til V0 og
V1.  V0 og V1 prøver at læse dem, men forstår dem ikke.  V0 og V1 roterer dem
180 grader, men det hjælper ikke.}

\scene{Der kommer en strimmel papir med beregninger ud af M.}

\says{M} ULLA HAR UDGIVET FEM VIDENSKABELIGE ARTIKLER I MATEMATIKKENS
TOPKONFERENCER OG KAN LØSE EN LIGNING MED TAL UDEN AT BRUGE LOMMEREGNER.
SANDSYNLIGHED FOR AT BLIVE KANDIDAT: 100\%.  UNDERSKRIFT--\act{bliver afbrudt af
S2}

\says{S2}[kigger på sin telefon] Ligegyldigt med kandidaten!  Jeg skal være
ph.d.-studerende i \textbf{USA}!

\scene{S2 løber ud af scenen.  V0 og V1 står stadig og prøver at tyde
artiklerne, men ser at hun slipper væk og løber efter hende.}

\says{M} UNDERSKRIFT IKKE--\act{bliver afbrudt af A}

\says{A}[ordentlig sur, råber] Hold kæft, maskine!

\scene{D, S og A kigger akavet på M og hinanden.}

\says{A} \ldots Dekan, hvorfor sender du mig egentlig på dyre kurser og lader
mig bruge store dele af fakultetets budget på uigennemtænkte it-projekter der
bare gør alting mere besværligt?

\says{D}[glad igen] Fordi jeg jo ikke har \emph{lyst} til at nå til squash med
dekanen for jura!  \act{laver slag med luftketcher}

\scene{Lys ned!}

\end{sketch}
\end{document}
