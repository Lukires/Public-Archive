\documentclass[a4paper,11pt]{article}

\usepackage{revy}
\usepackage[utf8]{inputenc}
\usepackage[T1]{fontenc}
\usepackage[danish]{babel}


\revyname{DIKUrevy}
\revyyear{2019}
\version{1.0}
\eta{$?$ minutter}
\status{Mangler nazificering}

\title{Nørdet - en musical i et akt}
\author{Niels, Simon, Søren P., Markus og Caroline}


\begin{document}
\maketitle

\begin{roles}
  \role{P}[Arinbjørn] Preben 'The\_k1ll3rn4t0r' Nielsen, nørden, logisk, lidt naiv, leder af drengegruppen, Danny
  \role{S}[Amalie] Shirley, pigen, starter som sej og rus, den nye pige, Sandy
  \role{P1}[Eva] Frida, Shirleys veninde, del af pigegruppen, en sød pige, som Frenchy
  \role{P2}[Lotte] Ghita, del af pigegruppen, mellem 'ond', prøver bare at score, mellemting mellem Jan og Martie
  \role{P3}[Mette] Rita, 'leder' af pigegruppen, ond, som Rizzo
  \role{D1}[Magnus] Floppy, del af drengegruppen, sød, som Putzie
  \role{D2}[Jonas] n00bslayer37, del af drengegruppen, lidt dum, som Sonny
  \role{D3}[Michael] Polymorpheus, del af drengegruppen, næsten lige så klog som Danny, som Kenickie
  \role{B}[Theis] Bartender på Caféen, og gameshow-leder (i video), smart, 'sej', *hjælpsom - host*
  \role{R}[Marc] Rusvejlederkæreste, mega fuld dreng, findes på toilettet, mest med i video
  \role{M}[Buchter] Matematiker, ond og nederen
  \role{MF}[Torben] Matematikeren følgesvend, tænk BitDreng
  \role{V}[Ejnar] Voiceover, dommer
  \role{T} Den ene del af Hold Vest, Gameshow video
  \role{F} Den anden del af Hold Vest, Gameshow video
  \role{J}[Lise] Just-eat bud, skal synge, pige
  \role{Da1}[Bjørn] Danser
  \role{Da3}[Rasmus] Danser
  \role{Da4}[Sean] Danser
  \role{F1}[Brandt] Statist
  \role{F2}[Romeo] Statist
  \role{F3}[Vivien] Statist
  \role{F4}[Æmilie] Statist
  \role{NF1}[Flora] Statist
  \role{NF2}[Henrik] Statist
  \role{NF3}[LeeAnn] Statist
  \role{NF4}[Michelle] Statist
  \role{BM}[Matilde] Ballonmand
  \role{E}[Pauline] Eva - statue
  \role{A}[Virt] Adam - statue
  \role{X}[Caro] Hovedinstruktør
  \role{Y}[Simon] Ko-Instruktør
\end{roles}

\begin{sketch}
\scene{TODO: Hele aktet skal strammes op, og nazificeres}
  
\scene{Der er tale om en musical i 1 akt. Den er baseret på Grease, men lavet om til en datalogisk version. Preben og Shirley er de to personer, der mødes i sommerferien. Hvad der ellers sker, kan I læse jer til undervejs.}

  
\scene{Video - simpel animation - Nørdet formet som en colaflaske}
\scene{https://q108.com/events/movies-in-the-park/grease-logo-1240/}

\scene{1: Video+Band: Intro - Grease is the word - bandet spiller ind til pausen, og der laves en 'video' til overtex, som intro til Grease. Med Nørdet stående i en cola-flaske}
\scene{2: Video: Før DIKU}
\scene{3+4+5: Sketch: Studiestart - Adam og Eva}
\scene{3+4+5: Sang: Hvad de laved i sommer - Summer Nights - Kantinen}
\scene{3+4+5: Sketch: De møder hinanden - Kantinen}
\scene{6+7: Sketch: Pigerne gør grin med Preben - Toiletterne i kælderen}
\scene{6+7: Sang: Gør grin med Preben - Look at me - Toiletterne i kælderen}
\scene{8: Sketch: Shirley tager på Caféen? - Caféen?}
\scene{9: Video: Gameshow - Caféen?}
\scene{10: Sang: Preben er den rette - Hopelessly devoted - Caféen?}
\scene{11+12: Sketch: Matematikere er nederen - Kantinen}
\scene{11+12: Sang: Fixer datamat - Grease Lightning - Kantinen}
\scene{13+14: Sketch: Kode for sjov - Kantinen}
\scene{13+14: Sang: Frida vil gerne kode - There are worse things I could do - Kantinen}
\scene{15: Sketch: DM i programmering - Kantinen}
\scene{16: Sketch: Preben tager på Caféen? - Caféen?}
\scene{17: Sketch: Tilbage i kantinen - Kantinen}
\scene{18: Sang: Det er dig jeg vil ha' - You're the one that I want - Kantinen}
\scene{19: Sang: Vi følges sammen - We go together - Adam og Eva}
\scene{20: Video: Afslutning - de flyver væk}

\end{sketch}




\end{document}
