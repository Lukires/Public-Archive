\documentclass[a4paper,11pt]{article}

\usepackage{revy}
\usepackage[utf8]{inputenc}
\usepackage[T1]{fontenc}
\usepackage[danish]{babel}


\revyname{DIKUrevy}
\revyyear{2019}
\version{1.0}
\eta{$2.5$ minutter}
\status{Færdig}

\title{Video: Gameshow}
\author{Niels, Simon, Søren P., Markus og Caroline}


\begin{document}
\maketitle

\begin{roles}
  \role{S}[Amalie] Shirley
  \role{B}[Theis] Bartender
  \role{TM} Den ene af hold vest
  \role{T} Anden del af hold vest
  \role{R}[Marc] Rusvejleder, den nye kæreste
  \role{X}[Caro] Instruktør, generelt
  \role{Y}[Niels] Instruktør, videofokuseret
\end{roles}



\begin{sketch}
  \scene{Gameshow: 3 døre (nummereret 1,2,3) ala Monty Hall Problemet}
  \scene{Bartenderen er i showagtigt kostume, og står sammen med Shirley foran dørene}
  \says{B} Okay, hvilken dør vælger du?
  \says{S} Øhh det ved jeg ikke
  \says{B} Vil du bruge en livline og spørge publikum?
  \says{S} Publikum?
  \scene{B peger ind i kameraet}
  \scene{Falsk publikumafstemning}
  \scene{Svarmuligheder: 0,1,2,NaN}
  \scene{Man ser søjlerne}
  \scene{Publikum vælger 2}
  \says{B} Du har valgt 2, så lad os se på en af de ting, du ikke har vundet!
  \scene{B åbner dør 3}
  \says{B} Du vandt ikke det her (en eller anden præmie)
  \says{B} Nu er valget mellem dør 1 eller 2?
  \says{B} Vil du skifte dør? Lad os slå plat eller krone!
  \scene{Klip til Lykkehjulet}
  \says{B} Hvad vælger du?
  \scene{S drejer hjulet (som kun består af plat og krone). Den ender på plat}
  \says{B} Du har tydeligvis valgt plat, lad os se om mønten er enig
  \scene{B kaster en mønt}
  \says{B} Og det her er plat!
  \says{B} Ej, det var heldigt, så kan vi åbne dør nummer 1
  \scene{Lydeffekter, engle?}
  \scene{Døren ind til pornotoilettet åbnes. Der sidder en og skider}
  \says{B} Hold vest, hvad siger I til det her?
  \scene{Torben og Troels kommer ind, skubber S til siden - de står med en stor lommeregner, har en blyant bag øret}
  \says{TM} Placeringen er jo ikke den bedste
  \says{T} Er lugten nyinstalleret?
  \says{TM} Men udseendet uha (uha godt, eller uha dårligt)
  \scene{T blinker}
  \says{T} Jeg får det til 37
  \says{TM} Jeg får det til 100 millioner
  \says{T} Skal vi gå på kompromis og sige 18?
  \says{B} I har helt ret, han har en promille på præcist 18
  \says{B} Tillykke Shirley, du har nu vundet et helt års forbrug af ... kæreste!
  \scene{B hiver personen ud til Shirley. R krammer S akavet med bukserne nede om anklerne. Han ser mega sej ud. Videoen slutter med at de står og krammer.}


\end{sketch}





\end{document}
