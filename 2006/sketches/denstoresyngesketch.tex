\documentclass[a4paper,11pt]{article}

\usepackage{revy}
\usepackage[utf8]{inputenc}
\usepackage[T1]{fontenc}
\usepackage[danish]{babel}

\revyname{DIKUrevy}
\revyyear{2006}
% HUSK AT OPDATERE VERSIONSNUMMER
\version{0.1}
\eta{$n$ minutter}
\status{Temmelig færdig}

\title{Den store syngesketch}
% (Den store "Synge-med-kode-C-næsten-musical-agtige-sketch")
\author{Rune Perstrup}

\begin{document}
\maketitle

\begin{roles}
\role{A}[Madss] Anklager
\role{R}[Allan] Revyboss
\role{S}[Uhd] Sanger
\end{roles}

\begin{props}
\prop{Balletudstyr til Jedi}
\prop{Talerstol}
\end{props}

\begin{sketch}

\scene{Retslokale} 

Sangeren kan med fordel akkompagneres af en pianist, der kan spille lidt
becifring. Der er ikke umiddelbart nogen presserende grund til at trække
hele bandet igennem alle sangene... medmindre de gerne vil, s'føli...

Situationen er en retssag med en anklager, en anklaget (revybossen) og
bevismaterialet (sanger med evt. pianist). Sketchen kan givetvis kortes ned
og gøres mere tight og skæggere osv., men her kommer den:

\says{A} Hr. Revyboss! Som representant for DIKU-revyen, står du
         anklaget som hovedansvarlig for uopfindsom
         sangskrivning i DIKU-revyen! Erklærer de dem skyldig i
         anklagen?
\says{R} Nej, DIKU-revyens sange er ikke ensformige men altid
         variedere, samfundsdebatterende, knivskarpe,
         dybsindige og interessante. Det er et løst rygte
         udspredt af de løgnagtige lyveløgnere fra fysikrevyen!
\says{A} Er det ikke rigtigt, at jeres sange kun handler om at
         kode i C. De er næsten lige så ensformige som
         fysikrevyen, der er fra et så kedeligt studium, at de
         bliver nødt til at lave en revy, der kun handler om
         dataloger.
\says{R} Nej, nej, nej. Jeg nødt til at pointere, at vi ikke kun er
         en kode-C-revy. Vi er ikke engang bare en kode-revy.
         Vi er \em{meget} mere end det. For eksempel har vi engang
\says{A} Nåhvad. Og hvornår er den så fra, om man må spørge?
\says{R} ... 1996... men den er virkelig god. Fremfør bevismaterialet.                                                                
\says{S}[begynder at lave slam-slam-klap-pause a'la "We will Rock you"]

         Er du træt af semantik\\
         Er det lidt  for svært\\
         Ta'r det lidt for lang tid at få det lært\\
         Så ta' til mig og kig'\\
         Ta' musik\\
         Så slipper i helt for al den hjernegymnastik\\
         Vi skal vi skal kode!\\
         Vi skal vi skal kode!
\says{R}[afbryder] Ja, ok. Den handler måske en \em{lille} smule om
         at kode. Men det er vigtigt for mig at pointere, at vi ikke
         kun er en kode-revy. Vi er meget mere end det. Vi er også
         meget humanistiske. Vi er osse såd'n noget med sprog. Næste
         bevismateriale er fra 2004:
\says{S} (melodi: Michael Jackson: Heal the world)\\
         Der' et sprog,\\
         som vi ved\\
         hver gang vi os sætter til\\
         en maskine for at vi skal lave kode\\
         Dette sprog har vi set\\
         Har en funktionalitet\\
         Og vi ved, at det aldrig går af mode.\\
         Sproget, der har pointere\\
         Der ' ingen der ved hvor de ender\\
         Kom og prøv og C\\
         Kom og kod i C\\
         Kod I C\\
         Hvis det skal brug's til noget\\
         For bruger' og programmør\\
         Er ej i samme båd\\
         Nogen koder Java\\
         De si'r det ikk', men de ved det.\\
         Skal det brug's til noget\\
         Skal det kodes i C.
\says{A}[råber] Stop! Jeg afbryder denne sang. Der blev sunget
         "kode C" indtil flere gange!
\says{R} Ja, oook. Den handlede måske \em{lidt} om at kode, men  det er    
         vigtigt for at pointere, at vi ikke bare er en kode C-revy.
         Vi er meget mere end det. Vi har også en sang fra 200?, og
         den handler selvfølgelig slet ikke om at kode. Fremfør
         bevismaterialet
\says{S} (melodi: Rawhide)\\
         Kode, kode kode\\
         Nu det er på mode\\
         Vil jeg ud og kode ML\\
         DIKUs datalære\\
         Lød "andre sprog er sære"\\
         Vil i lære dem, klar' i det selv\\
         (osv.)\\
\says{A}[afbryder] Neeej, jeg synes altså jeg hørte ordet "kode"
         indtil flere gange, og vi ved alle sammen, at man skal passe
         på sine *kodeord* (kadishhh), så den går ikke min fine ven!
\says{R} Øøøhmm, ok. Vi har også haft en sang om designskolen. Den
         går sådan her:
\says{S} (samme melodi og akk.)\\
         Mode, mode, mode
\says{A} Nej, det går ikke!
\says{R} Nnååårh ok, men vi har en sang som er sådn lidt Geografi-agtig:
\says{S} (samme melodi)\\
         Klode, klode klode
\says{A} NEJ!
\says{R} Jamen hvad så med en musik-sang:
\says{S} Node, node, node.
\says{A} STOP!
\says{R} OK! Vi har faktisk kun haft kodesange, og vi er STOLTE af
         det! Men i år skal det være anderledes. Vi har en helt ny og
         banebrydende sang som aldrig er hørt før. Den kommer her:

         (tre korpulente dalaloger iført balletskørt svanser sig
         igennem Cecilies og min kode ML-sang. Den findes i et
         revy-katalog et sted). Denne sang må meget gerne fremføres
         musical-agtigt med koreografi a'la Søren Horn og Uhd i et
         åbningsnummer 2004... Det var nemlig ubetaleligt skæg!

\scene{Tæppe}
\end{sketch}
\end{document}
