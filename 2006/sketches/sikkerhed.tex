\documentclass[a4paper,11pt]{article}

\usepackage{revy}
\usepackage[utf8]{inputenc}
\usepackage[T1]{fontenc}
\usepackage[danish]{babel}

\revyname{DIKUrevy}
\revyyear{2006}
% HUSK AT OPDATERE VERSIONSNUMMER
\version{1.0}
\eta{$n$ minutter}
\status{Færdig}

\title{Den nye, sikre lektor}
\author{marvin}

\begin{document}
\maketitle

\begin{roles}
\role{SB}[Madss] Institutbestyrer
\role{NL}[Allan] Ny lektor
\end{roles}

\begin{props}
\prop{DIKU-nøgle}
\end{props}

\scene{ SB står på scenenog ser ud, som om han venter på nogen. } 
  
\begin{sketch}

\says{NL} \act{Kommer let forvirret ind på scenen -- han er ikke sikker på, om
  han er kommet detrigtige sted hen}
Undskyld, er det dig, der er institutbestyrer? Dørskiltet er lidt forvirrende,
  der står noget om Belgiske\ldots

\says{SB}[Afbryder gnavent] Jaja. Der er nogen, der altid piller ved skiltene, men
det er ikke vigtigt lige nu. Du må være ansøgeren til den ledige
lektorstilling.

\says{NL} Ja, jeg tænkte det kunne være interessant at\ldots

\says{SB}[Afbryder som om han ikke hørte, at NL ville sige mere] Fint. Et par
af vores \act{rømmer sig} yngste forskere er netop\ldots \act{søger efter en passende
  formulering} gået på pension -- så vi har brug for nye, friske kræfter til
at undervise i datasikkerhed.

\says{NL}[Lyser op] Udmærket! Jeg er meget interesseret i sikkerhed, for\ldots

\says{SB}[Afbryder henført] DIKU er jo en af de store autoriteter indenfor
datasikkerhed, og de studerende er vant til et bredt udbud af kurser på højt
plan.

\says{NL}[Overrasket] Det var satans! Arbejder I så også med andre former for
sikkerhed\ldots

\says{SB}[Afbryder] Men der er jo ikke kun \emph{data}sikkerhed, vi går op
i. Hele instituttet er gennemsyret af en grundholdning om, at sikkerhed går
forud for alt andet. For eksempel har vi netop afskaffet kontanter på hele
instituttet, fordi de hele tiden blev stjålet!

\says{NL}[Begyndende skeptisk] Hmmm. Måske kunne du vise mig lidt\ldots

\says{SB}[Afbryder] Men nu ikke så meget snak. Lad mig vise dig lidt rundt på
instituttet.

\scene{ SB og NL går lidt}

\says{SB}[Peger stolt op mod midtergangen] Se! Her i Store Auditorium har vi for
eksempel et stort metalbur til at spærre projektoren inde i. 

\says{NL} Jamen, der er jo ingen projektor i buret?

\says{SB} Nej, er du gal! Den ville jo blive stjålet med det samme! Tænk dig
dog om, menneske! Kun en tåbe ville lade projektorer stå her.

\says{NL} Jamen, hvor gemmer I så projek\ldots

\says{SB} Nej, du. Da de første syv projektorer var blevet stjålet, fandt vi
på at gemme dem i Informationen. Buret deroppe er bare en honeypot, som lokker
tyvene til -- så behøver vi ikke at tænke mere på dem.

\says{NL} Nåja, jeg syntes nok, jeg så et pengeskab, da jeg var inde i
Informationen for at spørge om vej. Det må være dér, i gemmer projektorerne.

\says{SB}[Bestyrtet] Nejnejnejnej. Det ville hvemsomhelst jo gætte på. Næ,
projektorerne ligger i et \emph{ulåst} skab ved siden af
pengeskabet. Pengeskabet indeholder kun småmønter fra kaffeordningen. Se, dét
er sikkerhed! \act{slår pegefingeren od næsen på snedig, fortrolig vis}
Security by obscurity!

\says{NL} Hmmm. Men I har vel i det mindste installeret alarm i Informationen?

\says{SB} Det ville vi jo gerne. Men det er jo dér panelet til alarmen
sidder. Hvis vi satte alarm på døren, ville man jo udløse alarmen, når man
skulle ind og slå alarmen fra.

\says{NL} Går alarmen da ofte?

\says{SB}[Stolt] Ja! Hver dag kl. 16. Det holder folk på dupperne. Vi fandt ud
af, at folk ikke reagerede på alarmen. Men nu er de vænnet til den, så hvis
den \emph{ikke} går igang, ved de, at der er noget galt. Og ellers kan en af
  de mange studerende fra kantineforeningen og den slags gå ned og slå den
  fra. Koden står på en seddel bag døren.

\says{NL} Har de studerende da nøgle til Informationen?

\says{SB}[Overbærende] DIKU er jo et fremadskuende institut. Andre steder
bruger de public-private keys. Vi er gået skridtet videre og bruger
public-public keys!

\says{NL} Hvordan virker\ldots

\says{SB}[Afbryder, tager nøgle frem] Kan du se den her nøgle? Det er en
A-nøgle!

\says{NL}[Tøvende] A for ansatte?

\says{SB}[Triumferende] Nej! A for Alle og enhver! I gamle dage gav vi
forskellige nøgler til forskelige folk, men er du klar over, hvor besværligt
det er at administrere rettigheder? Og folk kunne aldrig få fat i den nøgle,
de skulle bruge, fordi nøglekontoret kun har åbent mellem 10.30 og halv
elleve. Med den nye ordning sparer vi penge, og der er
meget færre muligheder for at lave fejl.

\says{NL}[Bestyrtet] Hvad! Vil det sige, at alle har adgang til alle lokaler?

\says{SB}Nejnej, rolig nu. De \emph{vigtige} lokaler -- såsom rengøringsrum og
varmecentral -- er det skam kun særligt betroede medarbejdere, der har adgang
til. Vi kan jo ikke lukke hvemsomhelst ind dér.

\says{NL} Hvad så, hvis du skal derind?

\says{SB} Så går jeg ned i Informationen. Der hænger der nøgler til alle de
steder, man ikke kan kommeind med en A-nøgle. Men man skal selvfølgelig have en
A-nøgle for at komme derind!


\says{NL}[Nedslået] Jeg forstår. Men hvordan sikrer I så Jeres netværk? Jeg
har set netværksstik flere steder. Kan hvemsomhelst bare koble sig på nettet\ldots

\says{SB}[Afbryder smilende overlegent] Nej da. De fleste netværksstik er bare til
pynt. Og dem, der virker, bytter vi løbende om på, så folk hele tiden skal
gætte sig til, hvor man kan få adgang. Vi har også sikret det trådløse
netværk: Vores access points er så elendige, at man kun kan få kontakt til dem
udenfor DIKU. Og selv da skal man være heldig hvis DHCP-serveren virker, så
man kan få en IP-adresse.

\says{NL} Jo, men der er vel stadig en chance for at\ldots

\says{SB}[Afbryder] Den er ikke stor. Og desuden indfører vi snart en obskur
adgangskontrol, som kræver at folk installerer særlig software på deres
computere og sætter sig ind i alle mulige nye opsætninger. Det skal nok
sortere folk fra. 

\says{NL} Ja, men kan jo argumentere for, at det lukker The Layer Below -- i
dette tilfælde OSI-modellens lag 0.

\says{SB} Nu er du ved at forstå det. Og når folk ikke længere kan bruge deres
bærbare på DIKU, er de nødt til at sidde i terminalrummene og bruge vores
maskiner. Og de bliver da ikke stjålet så længe, der sidder folk ved dem\ldots

\says{NL}[Afbryder] Jeg tror efterhånden, jeg har fået en god idé om DIKUs
holdning til sikkerhed.

\says{SB}[Glædestrålende] Glimrende! Hvornår kan du begynde i stillingen?

\says{NL}[Koldt] Ring ikke til mig. Jeg ringer til Jer!


\scene{Tæppe}

\end{sketch}
\end{document}

%%% Local Variables: 
%%% mode: latex
%%% TeX-master: t
%%% End: 

