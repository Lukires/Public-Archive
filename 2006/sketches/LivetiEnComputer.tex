\documentclass[a4paper,11pt]{article}

\usepackage{revy}
\usepackage[utf8]{inputenc}
\usepackage[T1]{fontenc}
\usepackage[danish]{babel}

\revyname{DIKUrevy}
\revyyear{2006}
% HUSK AT OPDATERE VERSIONSNUMMER
\version{0.2}
\eta{$n$ minutter}
\status{Ikke færdig}

\title{En helt almindelig dag}
\author{Uhd, Maja, nogen andre sikkert}

\begin{document}
\maketitle

\begin{roles}
\role{B}[Madss] Bruger
\role{P}[Uhd] Program
\role{V}[Guldfisk] Virus
\end{roles}

\begin{props}
\prop{Regnvandstønde}
\prop{A3 papirer}
\prop{Bord x2}
\prop{Stol}
\prop{(Inter)net}
\prop{Papkasse til oversættelse}
\prop{Sprittusch}
\prop{Kaffekop}
\prop{Tastatur}
\prop{Forum skærmbilleder}
\prop{Kortspil}
\prop{404 skilt}
\prop{Djævlehorn}
\prop{Gummikylling}
\prop{Stativ til billede}
\prop{Pæn side9 pige}
\prop{Grim side9 pige}
\prop{Kazoo}
\prop{Vækkeurslyd}
\prop{reklamer}
\prop{post}
\prop{spindelvæv}
\prop{T-shirt: ``I Love You'' (logo fra P\&T-Museum?)}
\end{props}
 
\begin{sketch}                                                                  
  \scene{ En gut sidder på kanten af scenen foran en stor imaginær
    computer. Resten af scenen er det, som er inden i computeren, og
    en mand ligger inde i den i nattøj under en dyne. Han er
    processoren.  Generelt når han viser ting skal han svinge dem ud
    forbi publikum så de kan læse dem}
                                                                                
  \scene{I det ene hjørne står en -stor- kasse med skriften ``lager''.
    Der er også et bord ovre i computerafdelingen (skrivebordet) og en
    regnvandstønde med et skraldespandsikon.}
                                                                                
  \scene{B sætter sig foran computeren}
                                                                                
  \says{B} Ahhh... endnu en dag på arbejde. Nu skal jeg i gang
  \act{tryk på knap}
                                                                                
  \scene{P ligger og sover. Hans vækkeur ringer, han slår på
    snoozeknappen og vender sig om. B venter tålmodigt. Dette gentager
    sig en eller to gange, hvorefter C står op, strækker sig og
    langsomt finder et Velkommen skilt frem, som han holder op mod
    indersiden af ``skærmen''}
                                                                                
  \says{B} Ah, endeligt. Nu skal jeg tjekke post
                                                                                
  \scene{P finder en stak breve frem og begynder at sortere reklamer
    fra, som han smider i regnvandstønden. Lige som han skal til at
    vise dem frem, stopper han, smiler listigt og fisker en enkelt
    reklame op igen og propper den ind i stakken som han viser mod
    skærmen.}

  \says{B} Argh\ldots der er sgu sluppet noget spam igennem mit filter.
  DELETED\ldots bwahahaha! \act{trykker sletteknappen ned}. Nå, jeg må
  hellere få set på den opgave.
                                                                                
  \scene{ P finder en kæmpe stak blankt papir frem og viser det mod
    skærmen} .
                                                                                
  \says{B} Æv... nå ja, jeg skal jo også lige have kaffe
  \act{rejser sig og finder kaffe... et sted(TM)}
                                                                                
  \scene{ P Ser sit snit til at lægge sig tilbage under dynen}.
                                                                                
  \says{B}\act{kommer tilbage og taster noget}
                                                                                
  \scene{ P Rejser sig modvilligt igen, finder stakken med blankt
    papir frem og holder den frem.}
                                                                                
  \says{B} Nå, hvor kom jeg fra. Nå ja, jeg mangler at læse forum!
                                                                                
  \scene{ P Lægger stakken på bordet, og finder et par skærmbilleder
    frem, som han hurtigt viser frem efter hinanden (med de ting som
    han læser på forum: se et sjovt link, nu skal du høre en sjov
    historie, dagens side 9 pige etc. Til sidst vises en side med
    flames el.)}.
                                                                                
  \says{B}[bliver vred] Hvad satan er det for noget han skriver? Nuj
  han er en fræk fyr... nu skal han få, WHUAHAHAHA!\act{taster hurtigt
    noget}
                                                                                
  \scene{ P Finder et par djævlehorn frem, tager dem på og går over og
    råber noget vredt blabla ud mod spindelvævet. Går selvtilfreds
    tilbage til skærmen}.
                                                                                
  \says{B} Ha! Så skal vi lige have gemt hende tøsepigebarnet et
  sikkert sted... der

  \scene{ P finder billedet af side9 pigen, ligger det i en mappe og
    smider det på suspekt vis i lagerkassen}.
                                                                                
  \says{B} Næh, vent hun skal da på som baggrundsbillede..
                                                                                
  \scene{ P er lige nået tilbage. Trækker opgivende på skuldrene,
    trasker træt hen og roder kassen igennem og og stiller det på en
    stander på bordet mens han mumler surt til sig selv}.
                                                                                
  \says{B} Ja, hun er nu sød. Så, nu må jeg heller få lavet noget...
  meeeen vi skal da lige have musik til arbejdet.
                                                                                
  \scene{ P Finder en kazoo frem og fyr' dem auf!}
                                                                                
  \says{B} Ahh... og dog. Jeg må have noget bedre \act{bandet spiller
    et eller andet?}
                                                                                
  \scene{P Lægger kazooen fra sig}
                                                                                
  \says{B} Nå, NU må jeg hellere komme i gang. Jeg kan jo lige
  oversætte koden fra i går.
                                                                                
  \scene{ P Sætter en kasse op og smider resolut den store stak af
    papir ned i den og skriver ``oversæt'' på den}

\says{B} Og jeg skulle også lige researche den artikel jeg fandt\ldots

\scene{P går over og snakker lidt med internettet. I mellmtiden lister
  virus V ind og bytter den pæne side9 pige ud med en grim, samt
  vender bunden i vejuret på oversætterkassen. P trækker på
  skuldrene og trasker tilbage og viser et 404 skilt.}

\says{B} Hov, stop vent! Hvor kommer den sløje kost fra? Jeg har da
vist fået virus. \act{taste taste}

\scene{P finder en gummikylling frem og tæver V ud af scenen}

\says{B} Så ser det rigtigt ud. Men det er da vist også på tide med en
lille pause. Skulle man ikke lige tage sig et hurtigt spil.

\scene{P finder et spil kort frem og lægger nogen op. Han tripper med
  foden mens B laver sine træk, og flytter dernæst kortene rundt}

\says{B}[mumler mens han spiller spil]Bonde på dame der. Og så den
derover\ldots

\says{B} Yes! Den gik op. Jeg har forresten hørt fra ``min ven'' at
det nye afsnit af Battlestar Galactica er kommet. Mon ikke det vil
være til at finde i et S-tog et sted. Hmm...

\scene{Lys ned. P ud og skifte tøj, og ud foran tæppet. Tæppe for.}
                                                                             
\end{sketch}
\end{document}
%%% Local Variables: 
%%% mode: latex
%%% TeX-master: t
%%% End: 
