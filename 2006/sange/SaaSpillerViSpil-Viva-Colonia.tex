\documentclass[a4paper,11pt]{article}

\usepackage{revy}
\usepackage[utf8]{inputenc}
\usepackage[T1]{fontenc}
\usepackage[danish]{babel}

\revyname{DIKUrevy}
\revyyear{2006}
% HUSK AT OPDATERE VERSIONSNUMMER
\version{1.1}
\eta{$3.1$ minutter}
\status{Færdig}

\title{Så spiller vi spil}
\author{Uhd}
\melody{Hóhner ``Viva Colonia''}


\begin{document}
\maketitle

\begin{roles}
\role{S1}[Jakob] Sanger1
\role{S2}[Munter] Sanger2
\end{roles}

\begin{props}
\prop{2 mikrofoner}
\end{props}

  
\begin{song}
\scene{Orgelmusik}
\sings{S1+S2}[Langsomt og højtideligt]For en fattig studer'ne
er dagligdagen svær
Læser mange tykke bøg'r  
laver meget her
Alle vore timer går med hjernegymnastik
hvordan vi kan klare det, forstår man ik'

\scene{introstykke}

\sings{S1} Vi har så travlt til dagligt
der kræves mange svar
når opgaven er færdig
så er den næste klar
Vi får skrevet hundred' sider
i gennemsnit hver dag
men når en weekend nærmer sig, ja så
så slapper vi af

\sings{S1+S2}[Omkvæd] Så spiller vi spil, hele dagen
ved datamaterne
Vi lever og ånder
for Doom og Quake med mer'
med World of Warcrack morer vi os
her foran vor's PCer

\sings{S2} Dengang i gamle dage
da gjord' rapporter ondt
så fik vi blokstrukturen
men det var ikke sundt
Vi har stress og dårlig' nerver
Vi sveder angstens sved
Og beder tit en stille bøn til at
at DIKU går ned

\sings{S1+S2}[Omkvæd] Så spiller vi spil\ldots

\sings{S1+S2}[Omkvæd] Så spiller vi spil\ldots

\sings{S1}[B-stykke] Og pluds'lig er det morgen
så hurtigt kan det gå
Man burde måske læse
vi nu vi skal beståååååååå....
\sings{S2}[siges] men det bliver ikke... nu

\sings{S1+S2}[Omkvæd, stille og langsomt stigende] Nu spiller vi spil\ldots

\sings{S1+S2}[Omkvæd] Så spiller vi spil\ldots

\sings{S1+S2} Så spiller vi spil!

\scene{Tæppe}

\end{song}
\end{document}

%%% Local Variables: 
%%% mode: latex
%%% TeX-master: t
%%% End: 

