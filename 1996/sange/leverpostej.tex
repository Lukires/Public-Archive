\documentclass[10pt]{article}
\usepackage[utf8]{inputenc}
\usepackage{revy}
\title{Leverpostej}
\author{Søren Trautner Madsen}
\melody{``Live'n let die'' -- Paul Mcartney}

\version{1.101} % HUSK AT AJOURFØRE VERSIONSNUMMER!!

\revyyear{1995}
\parindent0pt
\parskip 1ex minus 1ex
\flushsingsright

\begin{document}
\maketitle

\begin{song}
\scene Live'n let die er et nummer, der sparker kosmisk! Det kunne derfor %
være ret fedt at lave. Her er et forslag til en tekst! Bemærk, at det mere %
skal være musikken end teksten, der driver værket!

\sings{Sanger}  Jeg gik en dag klokken 2 op på anden sal.
                Jeg sku' ha' mad i min vom\
\sings{Kor}     Den var så tom, den var så tom, den var så tom!
\sings{Sanger}  Så jeg gik lækkersulten hen til køledisken,
                men hvad lå der til mig?

                Kun leverpostej!   

\sings{Kor}   leverpostej
\sings{Sanger}leverpostej
\sings{Kor}   leverpostej
\scene (Her kommer der ca 1 minuts instrumentalmusik. I mellemtiden kunne %
der jo ske noget 'pudsigt' på scenen. Ide: Billeder af forskellige typer %
leverpostej, samt en karakter efter trettenskalaen. Selvfølgelig skal %
ingen af dem få mere end 5.)

\sings{Sanger}  Trestjernet rød salami,
                røget kalkunfilet,
                var der intet af!
                Der var ingenting jeg ville ha'!
\scene{(mere instrumentalmusik. Billeder af lækkert højt belagt smørrebrød %
og tocifrede karakterer....?)}

\sings{Sanger}  Jeg sku' ha' mad i min vom
\sings{Kor}     Den var så tom, den var så tom, den var så tom!
\sings{Sanger}  Men der var hverken Kærgården, eller Lurpak,
                Så hvad lå der til mig?
                
                Kun leverpostej!

\sings{Kor}   leverpostej   
\sings{Sanger}leverpostej
\sings{Kor}   leverpostej
\end{song}
\end{document}
% Local Variables: 
% mode: latex
% TeX-master: t
% TeX-master: t
% End: 
