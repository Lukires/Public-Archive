\documentclass[10pt]{article}
\usepackage[utf8]{inputenc}
\usepackage{revy}
\title{Cola i slowmotion}
\author{Revygruppen, indskrevet af \tt{beyond}}
\version{1.2}
\parindent0pt
\parskip 1ex minus 1ex
\flushsingsright

\begin{document}
\maketitle

\begin{roles}
  \role{Studerende} Med colatrang
  \role{Kvindelig studerende}
  \role{Kantinevagt} ~
  \role{Colahungerende} 10 personer eller så.
  \role{Stemme} Asbjørn Riis, Sørens også.
\end{roles}

\begin{sketch}

  \scene På scenen er en cola-automat. Der er ret mørkt i lokalet, så man
  kan se de 6 lysendde knapper. En studerende kommer ind på scenen og
  putter penge i automaten. Den spytter pengene ud igen.

  \says{Stud} [Råber til en person ude i kulissen] Hey, kan man få colaer i
  denne her maskine??

  \scene Kantinevagt kommer ind på scene og til FEVER??? fylder han
  automaten helt op. Imens vandrer den studerende hvileløst rundt i en
  lille cirkel eller hvad vi nu finder på. Knapperne bliver ved med at lyse.

  \says{Kantinevagt} Så nu skulle den være ordnet.

  \says{Stud} Tak, skal du ha'.

  \scene Den studerende putter penge i igen, men igen spytter den pengene
  ud og han ser endnu en gang at alle knapper lyser.

  \says{Stud} Så kan det også være lige meget!! {\em (Går sin vej)}

  \says{Voice over} Mine damer og herrer det De lige har overværet er et af
  DIKUlivets store mysterier. For at opnå forståelse for forklaringen af
  fænomenet, beder vi Dem endnu en gang se forløbet, noget af det i langsom
  gengivelse.

  \scene Sketchen spilles forfra, idet de sidste colaer fyldes i maskinen,
  går man over til slowmotion. Kantinevagten og den studerende bevæger sig
  med $1/10$ hastighed. Idet den er fyldt, skal lysene i maskinen slukkes,
  samtidig med at der fra alle sider af auditoriet kommer folk op på
  scenen, stiller sig i kø, skubber og hopper for at se, etc. 

  Hver og en trækker en cola, en enkelt hamstrer, og den kvindelige
  studerende trækker en Tampax. Det bliver de ved med en rum tid, indtil
  maskinen er tømt.

  Den allersidste cola kommer et par stykker op og slås om (stagefight???).
  De andre står omkring i en firkant, med reb, for at illudere en
  wrestling-ring.

  \says{Stemme} Men hvad sker der her?  Uuuuha, 'The Engell' er gået i
  clinch med 'Captain Cook' om den sidste cola!  Ubehagelig. Det er wirklig
  noget, der kan gi' blå mærker, det er det!  Lurer på hinanden....

  \scene (Cook giver Engell et knæ i ansigtet, han spytter tyggegummier ud)

  \says{Stemme} 'FACE INVADERS'. Fy for pyffor!

  \scene (efterfulgt af et spark i maven)

  \says{Stemme} Fulgt op med en 'BOTTOM UPPER'. Lige i maveregionen.
  Ubehageligt. Havde den ramt 10 cm længere nede, havde det været ENDNU
  mere ubehageligt.

  \scene (Engell slår Cook i hovedet med Colaflasken)

  \says{Stemme} Og så får 'Captain Cook' colaflasken lige i nyseren ... Feje
  tricks, men alt hvad kantinevagten ikke ser, det er jo tilladt.

  \scene (Cook låser Engell's arm på ryggen)

  \says{Stemme} Og så kommer 'The Engell' ind i en 'DEAD LOCK'. Wirklig
  kritisk region øh\ldots situation!

  \scene (Det vilde stunt)

  \says{Stemme} og en 'CORE DUMP'! Det er Wirklig noget, der gør ondt. Jeg
  tror ikke 'The Engell' kommer op fra den ...

  \scene (Cook sætter triumferende foden på Engell. 'ding' 'ding'. Han
  piller det blå mærke af flasken\ldots. Engell kravler ud\ldots)

  \says{Stemme} Ja, man får jo blå mærker af at drikke cola.

  \scene Til sidst forlader de alle scenen og hastigheden sættes til normal
  igen. Sketchen afsluttes som før.

\end{sketch}

\end{document}
% Local Variables: 
% mode: latex
% TeX-master: t
% End: 
