\documentclass[10pt]{article}
\usepackage[utf8]{inputenc}
\usepackage{revy}
\title{Forårsklassikeren}
\author{Hslowhand, kokdg, trauma \& marvin}
\version{0.910031}
\parindent0pt
\parskip 1ex minus 1ex
\flushsingsright

\begin{document}
\maketitle

\begin{sketch}

\begin{roles}
  \role{3 ryttere} på kontorstole. R1 er forrest i feltet, R3 til sidst. Hvis
vi kan skaffe Motorola, Coca cola \& 7-11-T-shirts ville det være godt.
  \role{Instruktor} med tilhørende propelhat
  \role{2 kommentatorer} uden for scenen
  \role{serviecevognfører, cola-henter, neger til reklame}
\end{roles}

\begin{props}
  \role{4 kontorstole} til deltagere og support.
  \role{1 kantine-service-vogn}
  \role{1 cola, evt. også en tom cola-flaske i holder}

\end{props}

\scene{Eurovision-temaet spilles, og tæppet går fra. På scenen sidder tre mand
på række med siden til og koder på virtuelle Motorola-maskiner}

\says{K1} Ja, vi byder nu velkommen til en af de helt store forårsklassikere,
det såkaldte Kerne-løb - eller mere populært en Forårsnat i Hel\ldots

\says{K2} Ja, og det ser ud til at bleve en spændende sidste etape. Alle
holdene fra sidste års objekt-orienteringsløb er med, og holdene ligger tæt -
forrest Motorola-holdet, tæt fulgt af Coca Cola, Seven-11 og Peter
plys-holdet, som jo længe har ligget på Juul\ldots

\says{K1} Ja, ifølge min monitor ser det nu faktisk ud til at Peter
Plys-holdet er udgået igen i år.

\says{K2} Der gik jo også rygter om, at de har pådraget sig en virus.

\says{K1} Men ellers går det jo godt for Motorola-holdet, som jo ellers i
sidste<ldots 

\says{K2} Og vi nærmer os nu den første nedkørsel.

\says{K1} Ja, nu gør det for alvor ned! Jeg tror snart vi kan vente det første
udbrud.

\scene{R1 kaster sig mod jorden, slår i scenegulvet og råber bandeord (``Shit,
Fuck, Smølfe-bæ, Æv, Lort'' osv), mens han krøller sin programudskrift
sammen. De andre fniser ad ham og han giver dem fingeren}

\says{K2} Hvilken gestus! Manden - fingeren - kommunikatoren!

\says{K1} Men ellers går det jo godt for Motorola-holdet, som jo ellers i
sidste uge\ldots

\says{K2} Ja, og det har jo været en stor publikumssucces - store hobe af
mennesker har stået i kø i stakkevis!

\scene{R2 tager en cola-flaske frem fra holderen, vender den på hovedet,
kigger på den og smider den bagud, hvorefter han trækker ud til siden, sakker
bagud. En supporter kommer ind på en kontorstol med en fyldt flaske cola og
rækker den til R2, som drikker af den og sætter resten i holderen. R2 trækker
tilbage i feltet og supporen sakker bagud ud af scenen. Imens alt dette sker,
snakker Kommentatorerne videre}

\says{K1} Ja, og netop dette clock-cykelløb bliver jo transmitteret over store
dele af Verden - jeg har set kollegaer fra flere af det store TV-selskaber
hernede: Euros-port, Bcc, RTS, BSR, CPU - ja selv en rapporter fra DIKUben er
hernede.

\says{K2} Og mens kørerne kæmper sig videre i det opslidende løb, kan vi jo
komme med lidt køretidsstatistik.

\says{K1} ja, det ser jo grønt ud for Motorola-holdet, som ligger i spidsen   
 efter gårsdagens exceptionelle etapesejr, der jo også sikrede deres holdleder
 De Gule Førerunderbukser - det er dem med fartstriberne - eller som de siger på
engelsk: Fart Stripes. 

\says{K2} Til gengæld klarede Seven-11-team'et sig jo mindre godt i samme
etape, som jo gik gennem primtalsørkenen\ldots

\says{K1} Men ellers går det jo godt for Motorola-holdet, som jo ellers i
sidste uge kom i\ldots

\scene{R3 vælter og slår sig meget. En bestikvogn kommer ind mens
kommentatoren siger:}

\says{K2} Og se så dér: Dagens første styrt! Heldigvis er service-vognen {\em
(service udtales selvfølgelig som bestik)} hurtigt fremme.

\says{K1} Ja, det ser ud til at være en punktering\ldots

\scene{R3 humper lidt (hvis der er tid) og roder med et af hjulene på
kontorstolen} 

\says{K2} Ja, jeg frygtede et øjeblik at vi skulle se en gentagelse af den
store katastrofe i '87, hvor der midt på ruten var glemt en havelåge

\says{K1} Det var et slemt massestyrt - men da synderen blev fundet, blev der
også gjort kort proces.

\scene{R3 sætter sig op på stolen og koder videre. Service-vognen kører
bagud. R3, der også er kommet bagud, kæmper sig langsomt op mod resten af
feltet} 

\says{K1} Men det ser ud til at rytten er i sadlen igen og sender sin vogn
retur. Men han er kommet bagud og må KÆMPE sig op.

\says{K2} Det er en ensom kamp. Manden mod maskinen. Der er krise nu.

\says{K1} Ja, man kan ikke påstå, han driver den af. Han kører som en ren
Thorkild Turing. Hver en time-slice bliver brugt. Nu er det tid til et
prioritetsskifte. 

\scene{R3 rækker hånden op og en instruktorvagt ruller op på siden af ham}

\says{K1} Han signalerer efter hjælp.

\says{Instr.} [Kigger på koden] MOVE, MOVE. Og brug klokken!

\says{K1} Det ser ud til at være en ringe hjælp, men han arbejder virkelig for det. Det her kunne godt blive et breakpoint i hans karriere.

\says{K2} Han er oppe i feltet igen. De svende kroppe arbejder sammen som \'en
stor muskel. Mænd er mænd og kvinder er noget fra HCØ eller KUA\ldots

\says{K1} Men ellers går det jo godt for Motorola-holdet som jo ellers i
sidste uge kom i vanskeligheder, da\ldots

\says{K2} Og så tror jeg, at det er tid til en besked fra vores sponsor!

\scene{En neger kommer ind og synger ``Always Motorola''}

\says{K2} Ja, så er vi tilbage igen i Kerneløbet. Coca-cola-holdet er nu opppe
i feltet igen efter at have været tæt på at blive hægtet af.

\says{K1} Men ellers går det jo godt for Motorolaholdet, som jo ellers i
sidste uge kom i vanskeligheder, da flere af rytternes siddesår kom\ldots

\says{K2} Og det skal også blive spændende at se, hvem der er med, når der afholdes enkeltstart i eksamenslokalet.

\says{K1} Men nu går kørslen ind i sin afgørende fase. Det er en kritisk
region, så løbsledelsen lukker for afbrydelser.

\scene{En mand kommer ind og holder et skilt med teksten ``Hold Kæft!'' op.}

\says{K1} Ja, det var jo på denne strækning gennem det grønne helvede, at
flere af deltagerne valgte at gå i skoven.

\says{K2} Det skyldtes jo nok de mange orme, der som en undtagelse fik mange
af deltagerne til at tage bussen hjem.

\says{K1} Men ellers går det jo godt for Motorola-holdet, som jo ellers io
sidte uge kom i vanskeligheder, da flere af rytterne\ldots

\scene{Alle rytterne rækker hånden op og står ef stolen. En mand i kittel
kommer ind.}

\says{K2} Men hvad er det? Der bliver signaleret efter lægehjælp!

\scene{Lægen deler sedler ud og rytterne går ud af scenen}

\says{K2} Det er meget mystiskt. Alle rytterne forlader løbet her hvor der
ellers blev lagt op til en fabelagtig spurt.

\says{K1} Men ellers gik det jo meget godt for Motorola-holdet, som jo i
sidste uge\ldots

\says{K2} Jeg får netop nu at vide, at løbet er blevet udsat til næste uge
p.gr.a. sygdom.

\scene{Tæppet går for}

\says{K1} Ja, og det er derved op til os at underholde jer frem til kl. 22.15,
hvor der kommer ``Havelågen i Bogen''.

\says{K2} Og som man kan se uden for billedet, er der nok at se på. {\em
(Bliver mere og mere ivrig)} Det er jo nogle smukke, veldefinerede omgivelser,
der nok kan fordreje hovedet på de fleste. Se foreksempel på den smukke
kantine på den anden side af vejen, hvor der VIRKELIG sker noget. Og hvis vi
ser længere bort, ja så kan vi jo se på den skønne Fælledpark, hvor....{\em
(osv.)} 

\says{K2} Og vi siger tak herfra - tilbage til revyen.

\end{sketch}

\end{document}
% Local Variables: 
% mode: latex
% TeX-master: t
% End: 
