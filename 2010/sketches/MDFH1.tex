\documentclass[a4paper,11pt]{article}

\usepackage{revy}
\usepackage[utf8]{inputenc}
\usepackage[T1]{fontenc}
\usepackage[danish]{babel}

\revyname{DIKUrevy}
\revyyear{2010}
% HUSK AT OPDATERE VERSIONSNUMMER
\version{1.6}
\eta{$1$ minut}
\status{Brugbar}

\title{Manden der forklarer humor}
\author{Brainfuck, Phillip}

\begin{document}
\maketitle

\begin{roles}
\role{M}[Brainfuck] Manden der forklarer humor
\end{roles}

\begin{props}
\prop{En tavle med humorflowchart}[Brainfuck/Phillip] Vi skal finde ud af hvad 
der sker være på dette diagram, så teksten kan skrives!
\prop{Æble}[Brainfuck/Phillip]
\prop{Banan}[Brainfuck/Phillip]
\end{props}

\begin{sketch}

\scene{M kommer ind på scenen igen.}

\says{M}[Direkte mod publikum] Okay, i denne sketch står to personer og taler
i munden på hinanden, og når til sammen konklusion.  ``Hvordan kan det lader
sig gøre?'', spørg I nok.  Det er helt enkelt; \emph{ordspil}.

\says{M} Se, nogle ord, selvom de har forskellige betydninger, så lyder de helt
ens!  De kan faktisk også staves forskelligt!  \act{Grinende} Og det sjove er så at
de tror de taler om det samme, men det gør de slet ikke. \act{Triller en tåre 
af grin.}

\scene{M står og mætter sig selv, indtil han opdager at de \emph{endnu} ikke
har forstået det.}

\says{M} Ved I hvad, jeg tror simpelthen slet ikke I forstår det, det her er
håbløst.

\scene{M går ud af scenen, så det ligner det er færdigt, men trækker så istedet
et humordiagram op.  På diagrammet er der skrevet ``iPatter'', mens det er delt
op i ``iPad'' og ``patter'' nedenunder.}

\says{M} Lad mig give jer et eksempelt på et klassisk ordspil.  Ser I, ``pat'',
altså ``patter'' i ental, lyder som ``pad'' i Apple-produktet ``iPad''.  Hvis
man sætter ``iPad'' i flertal, så får man ``iPatter''. Og \emph{alle} ved at 
grundlaget for \emph{al} humor... det' kønsdele.

\says{M} ``Hvorfor flertal?'' spørg I nok.  Vi ved alle at iPad'en ikke
kan multi-\emph{taske} \act{bruger hænderne til at illustrere ``taske''}.
Derfor skal man bruge to.  \act{Smilende} Og så får man \emph{iPatter}.
Og så kan man multitaske.  Ligesom kvinder.

\scene{M griner lidt af sig selv.  Men publikum forstår den stadig ikke.}

\says{M}[Næsten opgivende] Okay, jeg havde ikke lyst til det her, men I giver
mig intet andet valg, jeg vil koge humor ned til det banale. \act{Tager et æble
op af den ene baglomme} Ser I det her æble?  Det er ikke særlig sjovt, vel?
Dette er \emph{ikke}-humor.  Men se så her. \act{Han tager en banan op af den
anden lomme.}

\scene{Lys ud.}

\end{sketch}
\end{document}

%%% Local Variables: 
%%% mode: latex
%%% TeX-master: t
%%% End: 
