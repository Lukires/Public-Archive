\documentclass[a4paper,11pt]{article}

\usepackage{revy}
\usepackage[utf8]{inputenc}
\usepackage[T1]{fontenc}
\usepackage[danish]{babel}

\revyname{DIKUrevy}
\revyyear{2010}
% HUSK AT OPDATERE VERSIONSNUMMER
\version{1.0}
\eta{$2$ minutter}
\status{Færdig}

\title{Værdikamp}
\author{Munter}

\begin{document}
\maketitle

\begin{roles}
	\role{I}[Julie] Interviewer
	\role{JL}[Allan] Jesper Langballe i præstekjole. Åben kropsholdning. Armene ud til siden. Langsomme rolige bevægelser.
	\role{PL}[Mikkel] Peter Lundin i spændetrøje. Sammenkrympet. Pludselige bevægelser i ryk.
	\role{X}[Klaes] Instruktør
\end{roles}

\begin{props}
	\prop{Mikrofon}[Rekvisitgruppe] Til Interviewer
	\prop{Præstekjole}[Rekvisitgruppen] Til Jesper Langballe
	\prop{Spændetrøje}[Rekvisitgruppen] Til Peter Lundin
\end{props}


\scene Spilles foran tæppet.
  
\begin{sketch}

\scene lys op.

\scene I står på scenen.

\says{I} Velkommen tilbage til debatserien om kontrovierselt tøj.
\says{I} I dag fokuserer på en beklædning, som på trods af ganske få brugere, har vakt stor opstandelse i samfundet.
\says{I} Som taler imod den omstridte beklædningsgenstand har vi inviteret folketingsmedlem og præst i den danske folkekirke: Jesper Langballe

\scene JL kommer frem foran tæppet.

\says{I} Og som fortaler og hverdagsbruger af beklædningen har vi initeret Peter Lundin.

\scene PL kommer frem foran tæppet.

\says{I} Ja, sådan en spændetrøje ser jo egentlig ganske tilforladelig ud. Hvad er det der får hele danmarks befolkning til at tage sådan på vej Hr. Langballe?

\says{JL} Tilforladelig?! Vor herre bevares! Det skal jeg skam sige dig. Han kan jo ikke køre bil med sådan en på! Det burde forbydes!

\says{I} Hømmende altså. Interessant. Hvad mener Hr. Lundin?

\says{PL} Hæmmende, nej nu må jeg le. \act{Psykopatisk high-pitch grin}. Jeg kan sagtens køre bil med spændetrøje. Der er 100 procent udsyn
\says{PL} Det her viser jo bare endnu engang hvordan hverdagsdanmark ikke kan eller vil forstå folk med andre idéer, og blot prøver at forbyde dem.

\says{I}[opklarende] Hr. Lundin henviser her til det lovforslag de nyligt har fremlagt i folketinget Hr. Langballe.

\says{JL} Ja det ved gud han gør. Og heldigvis er hele folketinget enige om at det forslag bare skal vedtages.
\says{JL} Ingen ved deres fulde fem vil jo tage en så anstødelig beklædning på frivilligt. Der er nogen der har tvunget ham, og det skal forbydes. I frihedens og danskhedens navn!

\says{PL} Frihed? Hvad ved du om frihed. Spændetrøjen udtrykker min selvtillid, min bevidsthed og min egen unikke identitet.
\says{PL} Det lovforslag der fratager mig spændetrøjen hæmmer min frihed! \act{Vrider sig i spændetrøjen}

\says{JL} Nu har jeg da aldrig hørt mage. Du beskylder en præst i den danske folkekirke for at undertrykke minoriteter?

\says{PL} Det kan du bande på jeg gør. Og hvad med dig selv? Er det måske ikke folkekirken der foreskriver at bære præstekjole?

\says{JL}[forarget] \act{Gør korsets tegn} Gud fader bevares...

\says{I} Og her bliver jeg vist nødt til at afbryde før Hr. Langballe bevæger sig alt for meget ind på næste uges emne: Præstekjole - Teologi eller pædofili?

\scene Lys ud.

\end{sketch}
\end{document}

%%% Local Variables: 
%%% mode: latex
%%% TeX-master: t
%%% End: 

