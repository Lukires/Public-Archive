\documentclass[a4paper,11pt]{article}

\usepackage{revy}
\usepackage[utf8]{inputenc}
\usepackage[T1]{fontenc}
\usepackage[danish]{babel}

\revyname{DIKUrevy}
\revyyear{2010}
% HUSK AT OPDATERE VERSIONSNUMMER
\version{1.5}
\eta{$1$ minut}
\status{Brugbar}

\title{Manden der forklarer humor}
\author{Brainfuck, Phillip}

\begin{document}
\maketitle

\begin{roles}
\role{M}[Brainfuck] Manden der forklarer humor
\end{roles}
  
\begin{sketch}

\scene{M lister ind på scenen, mens han kigger på tæppet.}

\says{M}[Kigger stadig på tæppet; griner til sig selv] Det var godt nok sjovt.  
\act{vender sig mod publikum} Synes I ikke også? \act{Ligegyldigt hvad
reaktionen er så tager han det som et misforstået svar.} Ikke?  \act{Eftertænktsom}
Hm, så er det jo nok fordi I ikke forstår det.  Men lad mig hjælpe.

\says{M} Det sjove i denne sketch er at en person uden situationsfornemmelse
prøver at tale til et publikum der ikke gider lytter på dem. \act{Griner lidt
til sig selv; men opdager at publikum ikke helt har forstået det.} Hm, men jeg
tror ikke \emph{helt} I har forstået hvorfor det var sjovt.

\says{M} Ser I, man placerer en datalog i en situation som
nødhjælpsmedarbejder.  Så opnår man en situation hvor han kun gør situationen
værre, fordi han prøver at fikse problemerne på den måde han normalt løser
datalogiske problemer. \act{Kigger på publikum og forventer smilende
et forstående svar.}

\says{M}[Intet sådan et svar kommer; sukkende] Jeg tror simpelthen ikke at I
forstår det. \act{Roller med øjnene.} Det her går ikke.  Jeg giver op.

\scene{M går ud af scenen, lys ud.}

\end{sketch}
\end{document}

%%% Local Variables: 
%%% mode: latex
%%% TeX-master: t
%%% End: 
