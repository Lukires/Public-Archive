\documentclass[a4paper,11pt]{article}

\usepackage{revy}
\usepackage[utf8]{inputenc}
\usepackage[T1]{fontenc}
\usepackage[danish]{babel}

\revyname{DIKUrevy}
\revyyear{2010}
\version{1.0}
\eta{$3$ minutter}
\status{Færdig}

\title{Datababy}
\author{Dada, Søren}
\melody{Joan Javits/Philip Springer/Fred Ebb: ``Santa Baby''}

\begin{document}
\maketitle

\begin{roles}  
\role{S}[Amanda] Sanger
\role{D}[Mark] Stum datalog
\end{roles}

\begin{props}
	\prop{Slips}[] Et slips D kan trækkes i.
	\prop{Forførende outfit}[] Til S. A la 20-30'er stil, Marilyn Monroe i rød kjole og netstrømper.
	\prop{Datamat}[] Inkl. skærm, tastatur, mus, kabler.
	\prop{Bord}[] Bord som D sover på fra forrige sketch.
\end{props}

\scene D ligger over et bord og sover. Musikken går igang og han vågner.

\begin{song}
  \sings{S} Dataloger, jeg er helt ny her på DIKU,
Hjælp mig, jeg kan ikke kode,
Hvis du lyster, så kom her hen og kod lidt med mig

  \sings{S} Rekursioner, strukturer, typecheck og fun fact
  Eksplosioner, i mit lille hoved,
  Vil du ikke, kom her hen og kod lidt med mig.

  \sings{S} Jeg er helt alene her
  Hvordan skal jeg ku' tyde al' de tegn jeg ser
  Jeg er bare så hjælpeløs
  Hvis du gider hjælpe skal du nok få et kys

  \sings{S} Dataloger, ensom pige søger lidt hjælp,
  Tror du, du kan klare mosten
  Hvis du tør det, så kom her hen og kod lidt for mig

  \scene S stripper en datamat, mens bandet spiller instrumentalt (est. 30-60 sekunder).

  \sings{S} Datababy, uh hvor ku' jeg brug' en pause
  Måske, vi skulle fortsæt' derhjem'
  Sig nu ja pus, kom herhen og kod lidt med mig
  Kom med hjem og kod lidt min skat
  Kom med hjem og kod - i nat.

  \scene S trækker D ud fra scenen i slipset.

  \sings{D} Pause!
\end{song}

\end{document}

