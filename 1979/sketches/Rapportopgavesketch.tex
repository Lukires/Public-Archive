\documentclass[a4paper,11pt]{article}

\usepackage{revy}
\usepackage[utf8]{inputenc}
\usepackage[T1]{fontenc}
\usepackage[danish]{babel}

\revyname{DIKUrevy}
\revyyear{1979}
\version{1.0}
\eta{$n$ minutter}
\status{Færdig}

\title{Rapportopgavesketch}
\author{BOK, TH, HVO}

\begin{document}
\maketitle

\begin{roles}
\role{y}[] Studerende 1
\role{x}[] Studerende 2
\role{z}[] Studerende 3
\end{roles}

\begin{sketch}
\scene{Planlæggende møde i projektgruppen Mandag d. 4/3.}

\says{y} Hvordan skal vi så planlægge rapporten denne gang?

\says{z} Vi skal i gang fra starten. Planlægge vores tid ordentligt, 
ikke noget med natteroderi på DIKU den sidste nat sådan som vi plejer.
Hvis bare vi disponerer vores tid og ressourcer
ordentlig, så kan vi nok nå det. En simulaoversætter
med tilhørende databasesystem til PDP-11 skulle da være en
smal sag på 3 uger.

\says{x} Vi får travlt af helvedes til. \act{:rep. 1:}

\says{y} Skal vi ikke prøve at være færdige Fredagen før,
så kan vi lige pudse rapporten af i weekenden.

\says{z} Ja selvfølgelig. Det kan vi sagtens. Vi skal bare igang i
ordentlig tid. Vi kan sagtens få 10, hvis vi bare planlægger
ordentligt.

\says{y} Skulle vi denne gang prøve at lave brugervejledning og
problemanalyse inden vi begynder at kode.

\says{x} Det bliver en ordentlig omgang. \act{:rep.2:}

\says{z} Ja, og denne gang gider jeg fandme ikke at lave problemanalyse.
Det er altid mig. Vi må dele afsnittene ud på en anden måde end vi plejer.
Det lærer vi jo osse meget mere af.

\says{x} Jeg forstår ikke en skid af det hele \act{:rep.3:}

\says{y} Ja, og jeg gider ikke kode det hele denne gang. Skal vi ikke
dele det ud sådan at du \act{peger på z} får brugervejledning og test,
og du \act{peger på z} tager programmering og dokumentation, så tager
jeg problemanalysen.

\says{x} Rep.2

\says{z} Skal vi ikke mødes på torsdag, så stiller vi hver med en 
råskitse af de ting vi skal lave.

\says{x} Rep.3

\scene 2. akt. Første arbejdsmøde. Torsdag d. 7/3.

\says{z} Nå har I fået lavet noget. Ja, for jeg har ikke nået noget,
men jeg regner med at I har lavet jeres og når jeg ser det så vil
jeg hurtigt kunne lave mit.

\says{y} Jeg fik sku heller ikke lavet noget. Det var ellers min
mening at lave noget i går, men så kom Heæmer med en flok bajere
og så blev det ikke rigtigt til noget. Men Jeg KAN hurtigt få det
færdigt, jeg har det hele i hovedet.

\says{x} Rep. 3. Hvad er det egentlig jeg skal lave.

\says{z} OK, så mødes vi igen på tirsdag, og så skal vi altså have
lavet vores afsnit.

\scene 3. akt. Andet arbejdsmøde. Tirsdag d. 12/3.

\says{z} Ja, jeg startede på brugervejledningen i går, og jeg fik
lavet noget, men så kom jeg altså til det her problem: \act{bladrer}
Se \act{viser} her står, at man skal gøre sådan, og her \act{bladre meget mere}
står der det stik modsatte. Hvad fanden er meningen?

\says{x} Det er fordi det her \act{bladrer tilbage til første ex}
er taget fra opgaven fra sidste \act{pause} og det her \act{andet ex}
er fra opgaven fra forrige år.

\says{y} Hvad! Det har jeg sku ikke set. Det er da mærkeligt. Det forstår
jeg ikke. Det ved jeg sgu ikke, det bliver vi nødt til at spørge nogle
andre om. Står der virkelig det. Dubiøst.

\says{z} Ja vi kan også vente til vi har kodet det, så ved vi hvordan
det bliver. \act{til x} Hvordan går det med kodningen?

\says{x} Rep. 3 af det hele. Jeg har formuleret nogle af problemerne,
så synes jeg ligesom bedre at jeg kan komme igang.

\says{y} Det kigger vi på senere. Nu skal I se, hvad jeg har lavet. 
Ja, jeg har kodet lidt. Jeg synes ikke rigtigt jeg kunne skrive
problemanalysen før jeg havde lavet programmet. \act{pinlig pause}

\says{z} Skal vi ikke gå i biografen?

\says{x} Rep. 3.

\says{y} Hold da helt kæft, det har vi også. Vi kan lige nå 
9-forestillingen, hvis vi skynder os.

\scene 4. akt. Fredag inden aflevering. Fredag d. 22/3
Y sidder ved en terminal, Z sidder og tænker, X sidder og skriver på maskine

\says{z} Nu er der sgu kun 3 dage til vi skal aflevere, nu må vi se og komme
i gang.

\says{x} Rep 1.

\says{y} Det virker, det virker, kom og se \act{vinker af x og z, som vader hen
til terminalen}

\says{x} \act{efter et øjebliks pause} Er 8 et primtal?

\says{y} Ooooohhh, sikke noget lort, der er faneme altid noget i vejen med dig.
Nå, men det virker for de inddata, som det skal virke med iflg opgaven.
Der er nok ingen der lægger mærke til den lille smutter.

\says{z} \act{Kommer glad hen og viser X, hvad han/hun har lavet} Se hvad jeg har
lavet. Er det ikke smukt. Brugervejledningen er færdig. Der er simpelthen aldrig
set en bedre brugervejledning.

\says{x} \act{bryder sammen i krampegråd} Oooooohhh, hva er det for noget bavl du
har lavet. Det er lige præcis det jeg har skrevet at sådan skal det ikke gøres.
Nu går det hele galt. Om lidt vil jeg gå hen og sætte mig på mine briller.
Så kan jeg ikke se en skid. Er det ikke fantastisk. Det er det mest tåbelige
jeg nogensinde har været med til.

\says{z} Hvordan fanden skal det så være. Det kan altså ikke laves om, der er ikke tid.

\says{x} Så lad os stryge det afsnit. Det gør mere skade end gavn sådan som det er.
Så kan du skrive noget af al det vi mangler.

\says{y} \act{jamrende hyl} Nu glemte jeg at save igen \act{improviser}

\says{z} Jeg kan altså ikke rigtig lave testen før programmet er færdigt.
Jeg ved jo ikke hvordan programmet er lavet. Ka' du ikke lave testen, det
plejer du.

\says{x} Okay, så stik mig den så skriv brugervejledningen ind i stedet for.
Rep 1 og Rep 2.

\scene{y jammer.  Slut. Har afleveret.}

\says{y} Det var sgu rart at blive færdige.

\says{x} Vi får jo ikke 10 for det her.

\says{z} Egentlig er alt over 6 jo spildt arbejde.

\says{y} Men næste gang, da skal vi igang i ordentlig tid. Planlægge det hele
fra starten. Så kan vi blive færdige i ordentligt tid.

\says{z} Ja, det er noget rod at sidde hele natten. Næste gang blir det
anderledes.

\says{x} Det bli'r en ordentlig omgang.
\end{sketch}

\end{document}
