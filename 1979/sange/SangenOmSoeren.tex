\documentclass[a4paper,11pt]{article}

\usepackage{revy}
\usepackage[utf8]{inputenc}
\usepackage[T1]{fontenc}
\usepackage[danish]{babel}

\revyname{DIKUrevy}
\revyyear{1979}
\version{1.0}
\eta{$n$ minutter}
\status{Færdig}

\title{Sangen om Søren}
\author{QL}
\melody{?: ``Poul sine høns i haven lod flyve''}

\begin{document}
\maketitle

\begin{roles}
\role{S1}[L] Sanger 1
\role{S2}[H] Sanger 2
\role{S2}[J] Sanger 3
\role{S2}[B] Sanger 4
\end{roles}

\begin{song}
  \scene{Synges firstemmigt a capella.}

  \scene{``lulalalulalalulalalum'' synges efter 1. linie, efter 2. linie samt efter 5. (=sidste) linie.}

  \scene{Efter hvert vers synges ``Lulalalulalalulalalulalalulalalulalalulalalulalum''}

\sings{J}%
Her på DIKU er der for mange der studerer,
der må sørges for, at der ikke kommer flere.

\sings{H}%
Det skal ske hurtigt, helst lige på minuttet,
derfor har vores organer besluttet,
at Søren Lauesen bestyrer instituttet. 

\sings{B}%
Søren sine ideer på DIKU la'r flyve,
han skæ'r DATnul ned til to gange tyve.

\sings{L}%
Derefter gørhan DATet noget svæ're,
der er mange ting, som han synes de skal lære,

\sings{B}%
det vil være mere end Naur kan bære.

\sings{H}%
Og hvis de ikke danser efter hans pibe,
dumper DATtoerne alle på stribe.

\sings{L}%
Hvis vi så flere kandidater producerer
og vi til andendelen adgangen blokerer,

\sings{H}%
så er der ingen studerende mere.

\sings{J}%
Når så instituttet er tømt for studenter,
så kan han starte sine eksperimenter:

\sings{B}%
først får vi det indre af huset raseret,
hele anden sal bliver ganske udraderet,

\sings{J}%
og hos bestyr'en blir en tronstol placeret.

\sings{B}%
Sørens ideer er fulde af visioner,

\sings{L}%
han har nemlig mange honette ambitioner,

\sings{H}%
men hvis man forsøger at ændre hans planer,
for eksempel tænke i alternative baner,

\sings{J}%
så bli'r man kuppet førend man aner.
\end{song}

\end{document}

