\documentclass[a4paper,11pt]{article}

\usepackage{revy}
\usepackage[utf8]{inputenc}
\usepackage[T1]{fontenc}
\usepackage[danish]{babel}

\revyname{DIKUrevy}
\revyyear{1979}
\version{1.0}
\eta{? minutter}
\status{Færdig}

\title{DIKU-Guiden}
\author{CH}
\melody{?: ``Magge-duddi og jeg''}

\begin{document}
\maketitle

\begin{roles}
\role{S1}[] Sanger 1
\end{roles}

\begin{song}
\sings{S1}%
Her i Sigurdsgade ligger
ja, jeg er sikker - 
et institut.
Og her findes mange men'sker
ja, selv en svensker
fra Tvigtut.
Men hvad laver jeg så der?
Vi går indenfor og ser
måske vi træffer en student
som er en gammel god bekendt.

\sings{S1}%
Først vi ser en vindeltrappe,
så en mærk'ligt farvet mur.
Så vor dejlige og skrappe
og rappe Gurli
som sidder i sit bur.
Og vi hører noget støje,
det er en, der ta'r kopier
det er ham den flotte høje,
der laved' Algol til Gier.

\sings{S1}%
Førstesalen, den er venlig
og meget hjemlig,
Den gør mig glad.
Og hvis jeg en dag er snavset - 
bliv ej forbavset,
for der er bad!
Jeg må vaske hele min krop
blot jeg husker at tørre op.
Der er bruser og med mer',
der ka vi alle li' at vær'.

\sings{S1}%
Allerbedst er dog kantinen,
det er DIKU's dagligstue.
Og ved opvaskemaskinen
der ser vi Hjørdis - 
ja, hun er husets frue.
Snog han æder DIKU-platter
Søren spejler sig et æg,
og ved bordet høres latter
for PJo har frø i sit skæg.

\sings{S1}%
DIKU's mange kloge hoveder
de går og roder
på anden sal.
Man kan høre en loppe gø
der er miljø
i denne hal.
Og så spørger vi som så:
Er da tiden gået i stå?
Nej, den går og går og går - 
bornholmeren har bare sabbatår!

\sings{S1}%
Dette er så instituttet,
hvor man bliver datalog
og hvis du ej er betuttet
men fast besluttet
så bare start, værs'go!
Du kan få din egen nøgle,
du kan blive meldt ind i FLAB.
Du kan låne statens bøjle
i DIKU's fællesskab! 
\end{song}

\end{document}

