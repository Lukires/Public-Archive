\documentclass[a4paper,11pt]{article}

\usepackage{revy}
\usepackage[utf8]{inputenc}
\usepackage[T1]{fontenc}
\usepackage[danish]{babel}

\revyname{DIKUrevy}
\revyyear{1979}
\version{1.0}
\eta{$n$ minutter}
\status{Færdig}

\title{Åbningsnummer}
\author{QLM og HVO}
\melody{
  ?: ``Vær velkommen Herrens År'',
  ?: ``Feel like I'm fixing to die''}

\begin{document}

\maketitle

\begin{roles}
\role{S1}[] Sanger 1 - sopran
\role{S2}[] Sanger 2 - alt
\role{S2}[] Sanger 3 - tenor
\role{S2}[] Sanger 4 - bas
\end{roles}

\begin{song}
  \scene{Første del synges firstemmigt.}

  \scene{Mellem omkvædet og det første vers får man publikum til at råbe med på ``Giv mig et D'' osv. Her staves til DIKU.}

  \scene{Mellem omkvædet til 3. vers og starten pÅ 4. vers forsøges det igen, denne gang staves til DIQ.}

  \scene{``Vær velkommen Herrens År''. Synges a capella eller med svagt klaver.}

\sings{Alle}%
Vær velkommen til revy'n
og velkommen til by'n.
Vi vil starte med denne sang
den sætter formentlig revyen igang.
Velkommen alle, der vil more sig.

  \scene{``Feel like I'm fixing to die''. Omkvædet synges af hele koret og evt. får man publikum til at synge med. Af hovedteksten synger hvert af kvartettens medlemmer eet vers hver.}

\sings{Alle}%
Og det' derfor at vi er sammen her,
iaften er alle med,
og vi bli'r længe ved.
Du skal more dig, for nu har du
rendezvous
pÅ DIKU
whippy ... vi starter nu.

\sings{S1}%
 Efter at vi nu har drukket
vinen og vor tørst er slukket,
vommen er fyldt ganske op,
humøret det er helt i top,
revyen den begynder nu.
Whippy ... det aftenens clou.

\sings{Alle}%
Og det' derfor at vi er sammen her,
iaften er alle med,
og vi bli'r længe ved.
Du skal more dig, for nu har du
rendezvous
pÅ DIKU
whippy ... vi starter nu.

\sings{S2}%
Det er rart I efterkommer,
hvad vi påbød sidste sommer
dengang da - med noget lune
og måske en smule croone(d) -
Ole sang ``Vi ses igen''.
Whippy ... I gik på den.

\sings{Alle}%
Og det' derfor at vi er sammen her,
iaften er alle med,
og vi bli'r længe ved.
Du skal more dig, for nu har du
rendezvous
pÅ DIKU
whippy ... vi starter nu.

\sings{S3}%
Vi har anvendt mang' effekter,
heriblandt en ægte lektor,
enderim og anapester
samt et mindre rock-orkester.
Vi har sat det hele op,
værsgo at følge trop.

\sings{Alle}%
Og det' derfor at vi er sammen her,
iaften er alle med,
og vi bli'r længe ved.
Du skal more dig, for nu har du
rendezvous
pÅ DIKU
whippy ... vi starter nu.

\sings{S4}%
Kom nu stud'er, VIP'er, TAP'er
det skal høres nÅr I klapper,
der skal larmes meget mere,
I skal sgu participere.
Glem alt om din datamat,
whippy ... vi går grassat.

\end{song}

\end{document}
