\documentclass[a4paper,11pt]{article}

\usepackage{revy}
\usepackage[utf8]{inputenc}
\usepackage[T1]{fontenc}
\usepackage[danish]{babel}


\revyname{DIKUrevy}
\revyyear{2018}
% HUSK AT OPDATERE VERSIONSNUMMER
\version{1.0}
\eta{$2.5$ minutter}
\status{Færdig}

\title{Lockout}
\author{Troels, Phillip, Nikkel}

\begin{document}
\maketitle

\begin{roles}
  \role{B}[Sebbe] Revyboss
  \role{M}[Torben] Mægler
  \role{T}[Amalie] Tillidsrepræsentant
  \role{X}[Kristina] Instruktør
\end{roles}

\begin{props}
\prop{En afsluttende punchline}
\end{props}


\begin{sketch}
  \scene{Fra off-stage}

  \says{B} Hvor bliver skuespillerne af? Sketchen der forklarer hvor
  Kantinens kopper bliver af skal til at gå i gang!

  \says{M} De nægter altså at gå på scenen. De siger at de vil
  have omforhandlet deres overenskomst, ellers strejker de.

  \says{B}[frustreret] Nej, nej, nej, nej, nej\ldots

  \says{M} Jo, den er god nok. De har kaldt os til møde inde på scenen.

  \scene{OverTeX: ``Tekniske vanskeligheder''}

  \scene{Lys op.}

  \scene{M + B går ind på scenen.}

  \says{B} Hvem skal vi så forhandle med?

  \says{T}[træder på scenen] \textit{Jeg} er revyens
  tillidsrepræsentant! Og jeg skal meddele at vi strejker indtil
  vores krav mødes!

  \says{B} Hah! I kan ikke strejke---næh, I er lockout'et indtil
  I holder op med det pjat!

  \says{M}[prøver at mægle] \act{til T} Hvad er det da I vil have?

  \says{T} Bedre materiale!

  \says{B}[griner] Umuligt! Næh, fra bossgruppen kræver vi derimod
  bedre skuespil.

  \says{T} Der kan simpelthen ikke blive tale om bedre skuespil. Det
  er vi ikke villige til. Det kan ikke være rimeligt at der stilles
  højere krav til os, end til de andre revygrupper.

  Desuden hører vi at I vil afskaffe de gratis snacks i
  backstage-området og erstatte dem med en tvebak-ordning.

  \says{B} Det er de økonomiske virkeligheder!  Det er jo ikke gratis
  at leje Store UP-1. Nu vi er i gang, så kræver vi også andelen af glemte
  replikker sænket.

  \scene{M skal til at sige noget, men har glemt sin replik.}

  \says{B}[gestikulerer ud mod publikum] Se, de lider!

  \scene{B og T vender ryggen til hinanden. M virker splittet.}

  \says{M} Hvad med et kompromis?  Vi kunne jo\ldots droppe fjerde akt?

  \says{T}[vender sig lidt] Det ville jo fjerne en fjerdedel af det elendige materiale\ldots

  \says{B}[vender sig også] {\ldots}og spare snacks til en hel pause{\ldots}

  \says{M} Og nu vi er i gang med at forhandle, kunne vi så ikke også
           få publikum til at larme lidt mindre?

  \scene{Det bliver ikke godt modtaget.}

  \says{T} Okay, men så skærer vi fjerde akt.

  \scene{B og T giver hånd og går af scenen. M går ud på midten og taler til publikum.}

  \says{M}[til publikum] Ja, og jeg skal så bede jer om at gribe fyldepennen
  og strege fjerde akt ud i programmet -- den er desværre aflyst.

  \says{M} Ja, og den sketch der skulle have været hér er egentlig også aflyst, så\ldots Lad os gå
  videre til næste Disney-sang.

  \scene{Følgende skal være i programmet:

    \begin{itemize}
    \item 4. akt åbningstale
    \item Uffe Holm--gæsteoptræden (ægte tryllekunstner)
    \item Ubådssketchen, del 1: En sketch i mange dele
    \item Sangen om Prins Henrik (melodi: YMCA / noget på Pet Shop boys / New York city boy)
    \item Ubådssketchen, del 2: Den tunge luge vender tilbage
    \item Trump er dum
    \item Se den lille apt-get upgrade (melodi: se den lille kattekilling)
    \item Biologer knepper får
    \item Tror du det er for sjov jeg hacker (mel: Tror du det er for sjov jeg drikker)
    \item Ubådssketchen, del 3: Havet er skønt
    \item God som datalog (melodi: Lin-Manuel Miranda: How far I'll go)
    \item Sundhedsplatformen - more like "sygdomsplatformen"
    \item Ubådssketchen, del 4: Jeg føler mig splittet
    \end{itemize}
  }
\end{sketch}
\end{document}

%%% Local Variables:
%%% mode: latex
%%% TeX-master: t
%%% End:
