\documentclass[a4paper,11pt]{article}

\usepackage{revy}
\usepackage[utf8]{inputenc}
\usepackage[T1]{fontenc}
\usepackage[danish]{babel}

\revyname{DIKUrevy}
\revyyear{2018}
\version{1.2}
\eta{$5$ minutter}
\status{Færdig}

\title{Internet of Nips og Tingeltangel}
\author{Bette-Mikkel, Troels Henriksen, Phillip Rorchchdsh, mest Lucas Rorchchdsh}

\begin{document}
\maketitle

\begin{roles}
  \role{D0}[Brandt] Datalog
  \role{SF}[René] Skagensfisker af porcelæn - Royal Copenhagens Skagenfisker
  \role{SA}[Sebbe] SmartAsskebægret
  \role{FB}[Romeo] Fjæsbogen
  \role{T}[Niels] Smartphone
  \role{X}[Kristina] Instruktør
\end{roles}

\begin{props}
  \prop{Fjæsbog}
  \prop{Askebæger}
  \prop{Skagensfisker af porcelæn}
  \prop{En pibe}
  \prop{Et bord}
  \prop{En karton cigaretter}
  \prop{Lighter}
  \prop{Notifikations-flag}
  \prop{En paraply med prisskilt på 299}
  \prop{Tre lange ruller Terms of Service}
  \prop{Reb}
  \prop{Evt. en Amagerhylde}[Person, der skaffer]
  \prop{Trylleharpelyd}[https://www.youtube.com/watch?v=5YT9\_mnJrnA]
\end{props}

\begin{sketch}

  \scene{Vi er i D0s dagligstue. På bordet står en bog, et askebæger, og en lille porcelænsfisker.}

  \scene{D0 kommer gående ind på scenen og fløjter for sig selv.}
  
  \scene{D0 vender sig overrasket mod publikum.}

  \says{D0} Hov, er I også her?

  \says{D0} Ja, jeg stod jo lige og beundrede mine Internet-of-Tingeltangel-tingester. Jeg holder så meget af at gå på nettet og nappe noget nips.

  \says{D0} Jeg havde jo allere en e-espresso-maskine og en sound-bar. Men så købte jeg SmartAsskebægeret \act{præsenterer askebægeret}, fjæsbogen \act{præsenterer bogen}, og så den helt nye hjemmeassistent fra Fleggaard - Skagensfiskeren! \act{præsenterer porcelænsfiguren}

  \says{D0}[muntert] Hey Fisker, hvor'n skær'n?

  \scene{Trylleharpe, tryllestøv og stjernedrys mens Skagensfiskeren (SF) kommer på scenen og tager porcelænsfiskerens plads.}

  \says{SF}[bapper på sin pibe] Den æ' sgu' fin, gamm'el jas!

  \says{D0}[Stolt] Hey Fisker, fortæl os en vits!

  \says{SF}[bapper på sin pibe, er meget eftertænksom] Hvorfor har æ' isterning
  æ' nu'rn bein? \act{pause} Fordi den er wan'skabt.

  \says{D0}[Imponeret] \act{Griner højlydt!} Åhå-hå-hå, hvor er du bare vittig!

  \says{D0} Hey Fisker, hvordan bliver vejret i morgen?

  \says{SF} Åh, den er æ' for gur. Det ser uj' til regn \act{SF kigger direkte på
    D0}, men a' wille personlig anbefale dig den her paraply.

  \act{SF viser en paraply frem med prisskiltet ``299 kr'' på.}

  \says{SF} Ja den står jo normalt til 599, men \act{giver D0 en hyggealbue i
    siden} du ska' få'en til det halve.

  \scene{D0 får paraplyen af SF}

  \says{D0} Ej, så gi' mig to, tak!

  \scene{Tryllelyd igen, SA bryder ind}

  \says{SA}Var der nogen der sagde ``to-bak''? Hey hvad så, det er mig!!!

  \act{giver den op for publikum, og hoster bagefter}

  \says{D0} Åh, det SmartAsskebæger..

  \says{SA}[til D0] Er det ikke ved at være præcis syvogfyrre dage, \act{hoster!}
  tre timer og fire minutter siden du sidst nød en smøg?

  \scene{D0 virker lidt tøvende, er bange for at SA ikke længere synes han er cool}

  \says{D0} Ja..

  \says{SA} Er du stoppet? \act{falsk chokeret} er det ikke lidt.. gay?\act{HOSTER!!}

  \says{D0} Øhh, nej nej, jeg er ikke stoppet!

  \says{SF}[Henvendt til D0] Hør du, a' kender nogen der snart ska' til Tyskland.
  De ku' da godt lige ta' et par kartoner cigaretter med til dig. Det koster ik' alwerden.

  \says{D0}[skæver til SA] Øh, jo helt klart!

  \says{SF}[skriver ned] Det er bare gur' - \act{Venter lidt, og rækker så D0 en
    stang cigaretter} NÆH! \act{falsk overraskelse} de var her allerede.

  \says{D0} Wow. Det gør sørme stærkt når man har Fleggaard Prime.

  \scene{D0 snupper en cigaret fra pakken, mens han kigger på SA, der nikker
    anerkendende tilbage}
  
  \says{SA}[laver sej gestus med solbrillerne] Eow, det ser ud til at fyren er
  blevet flamme igen!

  \says{D0}\act{griner lidt nervøst lettet til SA}

  \scene{D0 får ild i sin cigaret, tager et stor hvæs og får et massivt hosteanfald}
  
  \says{D0}[med tårer i øjnene] Mmmm, dejligt.

  \scene{Tryllelyd igen. Fjæsbogen kommer til live, og den vifter med sit røde
    notifikationsskilt}

  \says{FB}[ivrigt] Hej [D0]; kig! Kig kig kig kig kig kig! Kig! Kiiiiiiig!

  \says{D0}[skynder sig hen] Ja, hvad sker der?

  \says{FB} Bolette har lige fået en ny mis!

  \scene{OverTeX: Man ser en statusopdatering med et frækt billede af Bolettes lille mis.}

  \says{D0} .. jamen.. det er da ikke specielt interessant?

  \says{FB} Nej nej! Men alligvel... \-- hov-KIG!\act{vifter med notifikationsflaget}

  \says{D0} Hvad nu?

  \says{FB} Ej hvor hyggeligt, Preben er ude og spise med alle sine venner..

  \scene{OverTeX: Man ser en statusopdatering hvor en masse mennesker morer sig ved et bord.}

  \says{FB}[pause, ny opdatering] Novra, og hende der Sanne du var så forelsket i -- hun er blevet gift!

  \scene{OverTeX: Man ser en statusopdatering med billeder fra et bryllup}

  \scene{D0 bliver lidt ramt af nyhederne. Der bliver lidt dårlig stemning.
    Der går lidt tid.}

  \says{FB} Hey, tag denne personligheds-quiz!

  \says{FB} Find ud af hvilken form for kommerciel eller politisk påvirkning du er mest følsom overfor!

  \scene{OverTeX: ``Hvordan manipulerer vi lettest dig?''-quiz}

  \says{D0} Eh, hvad?

  \says{FB} (rømmer sig): Øh, jeg mener, hvilken type ost er du?

  \scene{OverTeX: Quizzen ændres til ``Hvilken type ost er du?''-quiz}

  \says{D0} Øh...

  \says{FB} Vidste du godt, at alle i Vollsmose har glemt hvordan man taler
  dansk?

  \scene{OverTeX: Billede af etnisk menukort fuld af stavefejl.}

  \says{D0} Virkelig?!

  \says{FB} Ny forskning viser at racister er gladere mennesker.

  \scene{OverTeX: Screenshot af artikel om at Danmark er verdens lykkeligste folkefærd.}

  \says{D0} Men jeg føler mig egentlig ikke så glad for tiden. Så jeg er nok ikke racist...

  \says{FB}[improviserer] Se her: Nyopfundet forskning viser at brune mennesker er skyld i netop din livslede!

  \scene{OverTeX: [Passende billede.] F.eks. billede af Sebastian som lige har stjålet en baby og ser meget utilfreds ud, fordi den er brun.}

  \says{D0}[Overvældet] Øh, jeg kunne egentlig bedre lide det med katten.

  \says{FB}[Lidt fornærmet] Hvis du keder dig kan du da bare se de der
  tisse-bæ-film du er så glad for.

  \scene{OverTeX: Blurry billede af Sebastian der putter et eller andet i munden.}

  \scene{D0 ser utilpas og pinligt berørt ud.}

 \scene{SA byder D0 en smøg mere. D0 tager sig til hovedet i afmagt.}

  \says{D0} For faaaen.. \act{tager imod smøgen}

  \says{FB} Du er godt nok også begyndt at ryge en del, er du ikke?

  \says{SA} Jo, jo, prøv bare at se her! \act{går over til FB, viser sin notesbog frem}

  \says{SF} Ja, prøv lige at se her. Hans købsmønstre! \act{viser et fint broderet stykke stof frem til FB og SA}

  \scene{FB, SA og SF står nu sammen i en gruppe og udveksler noter.}

  \scene{På et tidspunkt stopper de op. SA peger hen på D0. De kigger alle tre, og fniser.}

  \says{D0} Vent! Er det ikke samkøring af registre? Er det ikke ulovligt?

  \says{FB} Kun siden i lørdags!

  \says{D0} Ej, hør nu! Det er altså \emph{mig} der ejer \emph{jer}!

  \scene{FB, SA og SF griner højt.}

  \says{SF} Vi er jo venner!

  \says{SA} Og venner, de ejer ikke hinanden

  \says{FB} Hør! Jeg har en sang på autoplay!

  \scene{D0 prøver febrilsk at forhindre at alle bryder ud i sang.}

  \scene{Bandet spiller ``Be Our Guest'' fra Beauty and the Beast og SF, SA og FB begynder at synge.}
  \says{SF+SA+FB} Vær vor ven! Vær vor ven!
  \says{FB} Terms of Service!
  \says{SA} Godkend dem! \act{ruller lange scrolls i hovedet på D0}
  \says{SF+SA+FB} Du skal aldrig vær' alene eller kede dig igen!
  
  \says{SF} Jeg er grisk!
  \says{SA} Jeg er ond!
  \says{FB} Jeg er gratis, hvorfor mon?
  \says{SF+SA+FB} Du skal ikke være bange, men nu ta'r vi dig til fange. \act{ruller D0 ind i reb}
  
  \says{D0} STOP!
  \scene{Musikken stopper}

  \says{D0} Det her er simpelthen for meget. Sluk så!

  \says{FB} Okay!

  \scene{FB, SF og SA gør et stort nummer ud af at lægge sig til at ``sove'',
    ved at lægge hænderne for ansigtet ligesom når Bamse skal sove.
    D0 går lidt frem på scenen, og får sig en smøg mere. \\
    D0 står med ryggen til de tre ting, som kigger glad ud fra hænderne og
    blinker indforstået til publikum}

  \says{D0}Jeg må have luft.. \act{søgende} hvor var min mobil..

  \scene{D0 finder sin smartphone / T springer frem fra kulissen}

  \says{D0}Åh, her var du.

  \says{T}[fistbumper D0] Hvad så dér champ, skal vi ud?

  \scene{D0 trækker T med ud af scenen. FB, SF og SA får T's opmærksomhed, og laver
    telefontegn til ham. T giver thumbs up og laver telefontegnet tilbage.}

  \scene{Tæppe}

  
\end{sketch}
\end{document}
