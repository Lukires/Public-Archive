\documentclass[a4paper,11pt]{article}

\usepackage{revy}
\usepackage[utf8]{inputenc}
\usepackage[T1]{fontenc}
\usepackage[danish]{babel}


\revyname{DIKUrevy}
\revyyear{2018}
% HUSK AT OPDATERE VERSIONSNUMMER
\version{1.0.1}
\eta{$2.5$ minutter}
\status{Færdig}

\title{studieforslag.dk}
\author{Erik, Nathalia, Sebbe}

\begin{document}
\maketitle

\begin{roles}
\role{I}[Ejnar] Institutsbestyrer, ond
\role{S0}[Torben] Studerende
\role{S1}[Mads] Studerende
\role{S2}[Æmilie] Studerende
\role{X}[Simon] Instruktør
\end{roles}

\begin{props}
\prop{Pappedatamat}[]
\prop{højt cafeebord bord}[]
\prop{Kontor}[]
\prop{Strikkerøj}
\end{props}


\begin{sketch}

\scene{S* står og brokker sig (i munden på hinanden) til I inde på Is kontor.}
\says{S0} Blablabla Studiepladser blablabla
\says{S1} Blablabla Jyrki blablabla
\says{S2} Blablabla CompSys blablabla

\says{I} Rolig nu, rolig nu: En af gangen!
   I ved jo godt, at hvis I har et forslag til ændring af studiet,
   så skal det gå igennem vores nye system: studieforslag.dk
   Så må I komme tilbage når I har nok underskrifter.

\scene{I vifter S* ud af sit kontor}
\scene{S* går over i anden side af scenen, S0 hiver pappedatamat frem}
\scene{I begynder at strikke}

\says{S1}[taster, mumler for sig selv] studieforslag.dk
\scene{S* kigger S0 over skulderen}
\says{S0}[læser fra skærmen] Bla bla{\ldots} indsende forslag{\ldots} bla bla{\ldots} Hvad!? 50000 underskrifter?
\says{S1} Fra aktive studerende på DIKU!?
\says{S2} Det kan de da ikke mene?
\says{S0} Har vi så mange?
\says{S1} Ah, lad os bare starte underskriftsindsamlingen, så må vi se hvad der sker.

\scene{S0 udfylder forslaget med ``TODO: Udfyld forslaget'' eller lignende}
\scene{Antal studerende begynder at blive talt op på OverTeX}
\says{S2} Hmm{\ldots} Altså{\ldots} det var jo en ret stor årgang i år, så vi har 250 allerede dér.
\scene{AV: næste}
\says{S0} Ja, de fleste er jo ikke nået at droppe ud endnu.
\says{S2} Ja, og så er der resten af DIKU{\ldots} Der har vi sikkert 250 mere.
\scene{AV: næste}
\says{S1} Hmm. Ja, det er jo 1\%. Hvad mere{\ldots} Bum, bum{\ldots}
\says{S0} 1 down, 99 to go!
\says{S2} Kan vi bruge store-O notation?
\says{S1} Arh, der er nok for store konstanter.
\says{S2} Hvad med de gamle røvhuller?
\says{S0} Joh; de er vel ikke blevet smidt ud endnu, så de kan vel godt tælle som aktive.
\scene{AV: næste}
\says{S1} Og hvis vi nu er lidt frie med det, kan vi så ikke inkludere sidefagsstuderende?
\says{S0} Jo, god ide! De er nok også mere aktive end vores studerende.
\scene{AV: næste}
\says{S2} Og fysikerne de ender jo også med at kode!
\scene{AV: næste (dårlig pling lyd)}
\says{S1} Aaarh, altså dem der ikke bliver gymnasielærere.
\says{S0} Jojo, men det stadig 5 ekstra underskrifter!
\says{S1} Nå, hvor mange underskrifter er vi oppe på?
\says{S0} Vi er på{\ldots} 3000. Hmm{\ldots}
\says{S2} Okay, hør her: Der står jo, at man bare skal være aktiv studerende der har \emph{gået} på DIKU.
    Altså, det det \emph{kan} jo tolkes som, at man bare skal have sat en fod indenfor DIKUs vægge.
\scene{AV: næste}
\says{S1} Ja, og HCC er jo teknisk set end del af DIKU - så kan vi ikke sige, at alle der går på KUA{\ldots}
\says{S0} {\ldots}går på DIKU!
\scene{AV: næste}
\says{S2} Uh, og ITU er jo lige ved siden af KUA!
\says{S1} Og de laver næsten det samme!
\says{S0} Ja, som på KUA! Det er godt nok til mig.
\scene{AV: næste}
\says{S2} Og når vi først har sagt \emph{I}TU, så må man også sige \emph{D}TU.
\scene{AV: næste}
\says{S0} Ja, og så kan vi ligeså godt også tage de andre universiteter med.
\scene{ tal er på 49999 }
\says{S2} Ja, og når vi har universiteterne, kan man vel også sige RUC
\says{S0} Ha! så ville jeg hellere tage min chatbot med
\says{S1} Fint!

\scene{AV: Ding! overtex: Congratulations you have 50.000 votes virus popup box}
\says{S0} Uh, vi har nok nu!

\scene{S* tager datamaten og går ind til I igen}

\says{S0} Så har vi underskrifter nok!
\says{S1} Ja, så nu \emph{skal} du høre på vores forslag!
\says{I} Ja det kan jeg godt se. Jamen, så fortæl mig det!
\scene{S* begynder at snakke i munden på hinanden.}
\says{I} Ro på, nu må I vælge: Hvad vil I gerne snakke om?
\scene{S* går i rundkreds for at snakke med hinanden om det.}

\scene{De snakker, og afbryder hinanden}
\says{S0} Vi har brug for mere
\says{S1} Jyrki
\says{S0} Studiepladser
\says{S1} er et problem og
\says{S2} Kantinen
\says{S1} Skal fjernes med det samme

\end{sketch}
\end{document}
