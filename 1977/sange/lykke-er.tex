\documentclass[a4paper,11pt]{article}

\usepackage{revy}
\usepackage[utf8]{inputenc}
\usepackage[T1]{fontenc}
\usepackage[danish]{babel}


\revyname{DIKUrevy}
\revyyear{1977}
\version{1.0}
\eta{$n$ minutter}
\status{Færdig}

\title{Lykke er...}
\author{HHK, Jan Cl.}
\melody{``Stars and Stripes'' og ``Lykke er ...''}

\begin{document}
\maketitle

\begin{roles}
\role{S}[] Sanger
\end{roles}


\begin{song}
\scene{Melodi: ``Stars and Stripes''}
\sings{S}%
Vors lands sikkerhed, det er os.
EDB gi'r os svar med det samme.
Hvis kommunerne byder os trods,
ryger byrådet ind på en fil.

Vi programmerer perfekt,
der er aldrig så meget som en løkke,
løkke, løkke, løkke,....

\scene{Melodi: ``Lykke er...''}
\sings{S}%
Lykke er, når løkken den er lykkedes,
lykke er, når løkken den er lykkedes......

Lykke er den første dag på HCØ
Lykke er at skulle stille sig i kø
Lykke er at kende bare én, når man skal på ruskursus.

Lykke er at finde Auditorium 1
Lykke er at se sin første over-head
Lykke er at kunne hør' hvad Steensgaard si'r i mikrofonen -- eller hvad?

At skrive galdt i sin rapport
At tabe alle sine kort
At finde Snog på anden sal
Når man skal kør' Pascal.

Lykke er at kunne ta' det lidt fra oven
Når man ser, at Mat-1 er helt i skoven
Lykke er at få en rædselsfuld eksamen suspenderet -- og hvad mer'?

\underline{Lykke er}
\end{song}
\begin{itemize}
\item at få forklaret, hvordan programtromlen på hullemaskinerne skal anvendes.
\item at turde stille et spørgsmål ved forelæsningen og at få et fornuftigt svar.
\item at se sit første program uden syntaksfejl og uden at vide, at det først er
  nu, problemerne begynder.
\item at få afskaffet de obligatoriske ugeopgaver på Dat-0 og
\item at få dem genindført på Dat-1.
\end{itemize}
\begin{song}
Man mærker knap at tiden går
Før borte er det første år
Gad vidst om man er forberedt
Til dette her Dat-1?

Lykke er at lære EXEC-8 sprog
Lykke er at se på Oles one-man show
Lykke er at tro, at det man lær' kan bru's til noget.

Lykke er når lampen lyser grøn og kæk
Lykke er når filen ikke er ble't væk
Lykke er når ens program på tusind linier næsten kører -- sådan da!

At starte på Mat-1 med gru
At tænke på Mat-D i smug
At se PJo gå helt i stå
Når formlerne bli'r rå.

Lykke er et kæmpe ambitionsniveau
Lykke er at kode PDP-macro
Lykke er at tro, at fra nu af kan alting kun bli' bedre -- og hvad mer'?

\underline{Lykke er}
\end{song}
\begin{itemize}
\item at have sin første kantinevagt uden at kaffen løber over.
\item at få sit eget symbol på agurkregnskabet.
\item at gi' en Dat-0'er besked på at skrive kommentarer i sit program og at
  lade være med at bruge ``\underline{go to}''.
\item at lægge testudskrifter ind i sit program og få 300 sider ubrugeligt output.
\item at få dispensation til at læse eskimologi som bifag
\end{itemize}
\begin{song}
Man mærker knap at tiden går
Før borte er et antal år
En pause ville være go'
Den ta'r man på Dat-2.

Men...

Lykke er at fatte, hvad der står i Gries
Lykke er at lave alting fasevis
Lykke er at finde sig en terminal og bruge mange penge.

Lykke er, når der er knas med automaten
Lykke er at skrive sin rapport om natten
Lykke er, når statistikken gi'r problemer i Brinch-Hansen -- og hvad mer'?

Når teorien bli'r så stor
Uden man forstår et ord
Når DIKU-lad 3.4's bud
Gi'r nervesammenbrud.

Lykke er at løbe sur i dope-vektorer
Lykke er at gå i stå i semaforer
Lykke er, når S-udtrykket har for mange parenteser -- og hvad mer'?

\underline{Lykke er}
\end{song}
\begin{itemize}
\item at Torben Zahle indfører forbrugskontrol 2 timer inden man skal aflevere
  sin opgave.
\item at aflevere sidste fase af sidste opgave på det sidste 1.delskursus.
\item at tage på Bakken efter den sidste opgave og falde i søvn i toget på vejen derud.
\item at skulle lære LISP på 3 kvarter uden at have en lærebog.
\item at skulle begynde forfra på sit bifag -- igen.
\end{itemize}
\begin{song}
Man mærker knap at tide går
Før borte er endnu et år
Nu er der gået en evighed
Kan det mon vare ved?

Lykke er, når SU siger stop
Lykke er, når man skal ud og ha' et job
Lykke er at spille ølagurk til over midnat.

Lyke er at få en plads på anden sal
Lykke er at tro, at man er genial
Lykke er at skule finde på et navn til sit speciale -- og hvad mer'?

Man seksten hundred' timer får
I løbet af en halv snes år
Så står man der en skønne nat
Er pluds'lig kandidat.

Lykke er at få et brev fra FLAB tilsendt
Lykke er at blive støtte-abonnent
Lykke er, når man kan fange vitsen i revyen.

Lykke er at sende tankerne tilbage7
Lykke er at snakke med om gamle dage
Lykke er at vide helt bestemt, at det er overstået.
\end{song}

\end{document}
