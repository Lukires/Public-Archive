\documentclass[a4paper,11pt]{article}

\usepackage{revy}
\usepackage[utf8]{inputenc}
\usepackage[T1]{fontenc}
\usepackage[danish]{babel}


\revyname{DIKUrevy}
\revyyear{1977}
\version{1.0}
\eta{$n$ minutter}
\status{Færdig}

\title{Datalogi-2}
\author{HHK, Snog}
\melody{Ukendt melodi, men nok en kendt en}

\begin{document}
\maketitle

\begin{roles}
\role{S}[] Sanger
\end{roles}


\begin{song}
\sings{S}%
Hver eneste Dat-2'er
Som tror han kan sit kram
Bestiller ikke andet end
At teste sit program.

\sings{S}[omkvæd]%
Man ta'r en terminal
Og bli'r der dagen lang
Man editerer, oversætter
Mapper gang på gang.

\sings{S}%
Man taster og man tamper
I sindsforvirret takt
Og sine fingerspidser
Får man ganske ødelagt.

\sings{S}[omkvæd]

\sings{S}%
Når nogen nævner satsvis
Forstår jeg ikke spor
De står jo op og venter
-- De har ho'det fuldt af jord!

\sings{S}[omkvæd]

\sings{S}%
Man holder fri til frokost
Men efter en agurk
Der får man glød i blikket:
Nu skal skærmen ha', den skurk!

\sings{S}[omkvæd]

\sings{S}%
Programmet er nu færdigt
Det synger rent og pænt
Nu mangler vi at teste kun
De sidste 80 procent.

\sings{S}[omkvæd]

\sings{S}%
En listning skal man have
Og SIMULA med M
Gi'r nyd'lig overtrykning
Men papiret før's ej frem.

\sings{S}[omkvæd]

\sings{S}%
På afleveringsdagen
Man møder sammenbidt
Men Hjørdis er jo aldit flink
Det gi'r os lidt respit.

\sings{S}[omkvæd]

\sings{S}%
Nej vi vil råbe hurra
Og det skal ske med smæld
Når Peter Naur prøver på
At bruge RECKU selv

Han ta'r en terminal
Og bli'r der dagen lang
Han editerer, oversætter
Mapper gang på gang.
\end{song}

\end{document}
