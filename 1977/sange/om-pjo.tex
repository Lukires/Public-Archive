\documentclass[a4paper,11pt]{article}

\usepackage{revy}
\usepackage[utf8]{inputenc}
\usepackage[T1]{fontenc}
\usepackage[danish]{babel}


\revyname{DIKUrevy}
\revyyear{1977}
\version{1.0}
\eta{$n$ minutter}
\status{Færdig}

\title{Om PJo}
\author{Banana, Mette og Erik}
\melody{``I den gamle pavillon''}

\begin{document}
\maketitle

\begin{roles}
\role{S}[] Sanger
\end{roles}

\scene{Erik M. \& 3 piger}

\begin{song}
\sings{S}%
Jeg er en af DIKU's to kustoder
de kvikke ho'der
i perioder!
Men snart er jeg mange kilometer
fra den anden Peter
og fra dat ét.
Jeg bli'r emigrant
-- dog hele livet ej --
ovre på det blå Hawaiiiii.
Snart tnæker jeg tilbage på det gamle institut
andensalen hvor jeg sad
og bladred' lidt i andet bind af ham der Donald Knuth
til jeg ku' det udenad.
Det var nu ikke noget jeg blev bidt a'
det føltes som om tankerne slog smut
som oftest smutted' tankerne til middag
på det gamle institut.

Her på Fælle'n har jeg læst og skrevet
har mangt oplevet
er doktor blevet.
Jeg har også været i Sorbone
lært allehånde
slags mat'matik.
Og p åD.T.H.
underviste jeg
til Edda kom og sa' til mig:
``Vi ønsker dig tilbage på det gamle institut
kom og lær os lidt af dit.
Vi mangle automater, én til kaffe, én til sprut
til at ta' de tørre bit.''
I ved at Edda alle mænd besnærer
det virkede som nøden var akut.
Og sådan gik det til at jeg blev lærer
på det gamle institut.

Langt derovre, midt i Stillehavet
skal jeg få lavet
no't ret begavet.
Jeg skal lære de forened' stater
om automater
og sådan no't.
Jeg kan lige se
hvordan det vil gå
jeg skal bare tænke på:
en autoforelæsning på det gamle institut
kan I huske hvo'n den lød?
Den skal måske serveres med en ekstra smule krudt
for at få den rette glød.
og skulle nogen blive uregerlig
og si' det hele er en slem gang prut.
Så si'r jeg: ``Alle syntes den var herlig
på det gamle institut.''

\scene{Her kan passende indskydes et ``Aloha''-kor}

Fra Hawai går tankerne tilbage
til gamle dage
og Ilwo kage.
Ofte tænker jeg på danske svampe
og fodboldkampe
som på dat ét.
Og når det bli'r jul
er det særlig slemt
man bli'r ganske vemodsstemt.
Gad vidst hvordan de har det på det gamle institut
mon de tænker lidt på mig?
Og hvem skal holde talen ved FLAB's jule-hutlihut
mon en anden melder sig?
Man ser hvad hjemve det kan forårsage:
Jeg rejste ud som nervesammenbrudt.
Nu kan jeg atter klare et par dge
på det gamle institut.
\end{song}

\end{document}
