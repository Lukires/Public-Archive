\documentclass[a4paper,11pt]{article}

\usepackage{revy}
\usepackage[utf8]{inputenc}
\usepackage[T1]{fontenc}
\usepackage[danish]{babel}


\revyname{DIKUrevy}
\revyyear{1977}
\version{1.0}
\eta{$n$ minutter}
\status{Færdig}

\title{Datalogvise}
\author{PJo}
\melody{``Darduse''}

\begin{document}
\maketitle

\begin{roles}
\role{S}[] Sanger
\end{roles}

\scene{Lene}

\begin{song}
\sings{S}
Kom, lyt til visens formål
og hør på mig i dag
besvar mig dette spørgsmål
hvoraf består vort fag?
hvordan kan det mon være
det er så svært at lære
maskiner at besnære?
hvoraf består vort fag?

Hvordan skal man beskrive
en algoritmes skridt?
et sprog det må det blive
så når vi meget vidt:
og dat kan vi ændre
når sprogets elementer
vi ud af sindet henter
så når vi meget vidt

En programmør han kender
sit sprog og sit problem
han begge endevender
og kender hvert system
vil du hans håndværk lære
du regne må på svære
erfaringer at bære
og kende hvert system

maskinen er jo grunden
til hele det ståhej
deraf er faget runden
derfra vi kom i vej
men hvorhen fører vejen?
hvordan mon ender legen?
vil vi gå over stregen?
derfra vi kom i vej

En algoritme, er er
en voldsom abstraktion
mon ret den eksisterer?
den bruges skal af nogen
med brugen sammenvokset
og ud af paradokset
datalogi er vokset
det bruges skal af nogen

Hvis ej der findes grænser
hvortil vor verden går
hvilke konsekvenser?
hvoraf vort fag består?
hvordan kan man dog bære
foruden grænser være?
hvordan skal vi nu lære
hvoraf vort fag består?
\end{song}

\end{document}
