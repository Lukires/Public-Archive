\documentclass[a4paper,11pt]{article}

\usepackage{revy}
\usepackage[utf8]{inputenc}
\usepackage[T1]{fontenc}
\usepackage[danish]{babel}


\revyname{DIKUrevy}
\revyyear{1977}
\version{1.0}
\eta{$n$ minutter}
\status{Færdig}

\title{Kantinen}
\author{NG}
\melody{``Kasper og Jesper og Jonathan''}

\begin{document}
\maketitle

\begin{roles}
\role{S}[] Sanger
\end{roles}

\scene{Niels \& Krølle \& Snog}

\begin{song}
\sings{S}%
Vi lister os afsted på tå,
Når vi går ind og æder.
Vi æder alt det, vi kan nå,
Og dertil ganen væder.
På DIKU er det skik og brug
At fylde alting i sin bug.
Vi drager af gårde med sult og med tørst:
Både stud'er og tap'er med læren' først.

Det er jo næppe blot fordi,
Vi ikke vil betale.
Kun visse ting dem husker vi
Og glemmer de banale:
Hvem kan vel huske femten fad,
Man selv har skyllet ned så glad?
Når nøden er størst, da er så vores tørst
Båd' hos stud'er og tap'er med læren' først.

Med knivens gamle, sløve stål
Han flænser kagesiden.
Det lille centimetermål
Er blevet glemt med tiden.
Det er vel ikke illegalt
At regne stør'lsen liberalt,
For det er en sport at få skivet den størt
Båd' hos stud'er og tap'er med læren' først.

Med luften er det ilde fat,
Og vi bli'r meget syge,
Skønt alle skilte siger, at
Man ikke her må ryge.
Et forbud gælder alle mand,
Men nogen her i DIKU-land
De ryger og damper så mangen en smøg
Både stud'er og tap'er og lærer', føj!

Oprydning er en kilden sag,
Vi huser mange grise.
De vasker næsten aldrig af;
Nej, de vil hel're spise.
Det nytter næsten ikke, at
``Motion og Samvær'' tager fat.
Vi prøver med vagter, så godt vi formår,
Men båd' stud'er og tap'er og lærer' går.

Hvis du besøger DIKU før,
Kantinelyset slukker,
Så ser du kun det harske smør
Og brød og ost, der bukker.
Hvis du den slags fortære vil,
Så skal der ingen enge til.
Vi venter på takken for det vi har spist
Både stud'er og tap'er med læren' sidst.

Vi frigter dybt i vores sind,
At DIS vil få kantinen.
Så er det slut med spild og svind,
Vi kommer i maskinen,
Og priserne bli'r højere,
Og vor forplejning sløjere,
For vor appetit reduceres med ét
Båd hos stud'er og lærer' med tap'en midt.
\end{song}

\end{document}
