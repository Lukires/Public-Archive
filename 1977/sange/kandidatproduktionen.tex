\documentclass[a4paper,11pt]{article}

\usepackage{revy}
\usepackage[utf8]{inputenc}
\usepackage[T1]{fontenc}
\usepackage[danish]{babel}


\revyname{DIKUrevy}
\revyyear{1977}
\version{1.0}
\eta{$n$ minutter}
\status{Færdig}

\title{Kandidatproduktionen}
\author{SR}
\melody{``Man kan vel ikke gøre for ... charme''}

\begin{document}
\maketitle

\begin{roles}
\role{S}[Mette] Sanger
\end{roles}


\begin{song}
\sings{S}%
Vi ka vel ikke gøre for, at vi har varme
og rent magnetisk drager stud'er og HF.
Ja, revl og krat tar vi imod med åbne arme,
og alle blir her -- ka det vær et sammentræf?
Vi er så mange, at selv guder sig forbarme.
Vi ved nu knapt, hvad vi skal gøre af os selv.
Og det er blot, frodi vi har en smule varme.
Vi ka da ikke gøre for det -- ka vi vel?

Når adgangsgrænser andre steder folk forflytter,
så tager DIKU da ekspert som dilettant.
Vi mod de kolde vinde altid os beskytter.
Vi er erhvervs- og arbejdsmarkedrelevant'.
Afgangsbegrænsning er et middel, vi benytter.
Vi slipper aldrig folk, og de går ikke selv.
Og det er blot, fordi vi ingen grænser flytter.
Vi ka da ikke gøre for det -- ka vi vel?

Andendelsmødet det er virk'lig nog't der trækker.
Det drager mangen en studerende herhen.
Der altid skaffes må nog'n ekstra stolerækker,
men slap blot af, for folk forsvinder straks igen.
Mon kurserne igen i år til alle rækker?
Ja, gudskelov der kommer kun en tredjedel.
Er det da blot, fordi kun an'delsmødet trækker?
Vi ka da ikke gøre for det -- ka vi vel?

Her kandidaten altid bliver i sin vorden,
og herom kan der siges både meg't og mangt.
Hvis til eksamen det sku ske, at der er fjorten,
har gang på stedet bragt os meget, meget langt.
Men producer' vi mer -- vi sprænger da akkorden.
Så ka vi ikke længe rpasse på os selv.
Lad os til evig tid forblive i vor vorden.
Vi vil da ikke ændre på det -- vil vi vel?
\end{song}

\end{document}
