\documentclass[a4paper,11pt]{article}

\usepackage{revy}
\usepackage[utf8]{inputenc}
\usepackage[T1]{fontenc}
\usepackage[danish]{babel}


\revyname{DIKUrevy}
\revyyear{1977}
\version{1.0}
\eta{$n$ minutter}
\status{Færdig}

\title{Tandlægens sang}
\author{Pjo}
\melody{Melodi}

\begin{document}
\maketitle

\begin{roles}
\role{Tidl. patient}[] Tidligere patient
\role{Fru Hansen}[] Assistent
\role{Tandlæge}[] Tandlæge
\role{Patient}[] Patient
\end{roles}

\scene{\textbf{Note fra digitalisator:} I sangen er det nogle steder uklart hvem
  der synger hvad.}

\scene{Snog, Mette \& Krølle}

\begin{sketch}
\scene{Dediceret til min tandlæge med tak for inspirationen}

\scene{Tandlægeklinik, stol og instrumenter.  Tidligere patient netop færdig,
  holder sig for kinden, ``glasagtige'' øjne, en smule ketchup skjult i munden.}

\says{Tandlæge (hvid kittel)} Se her er et glas piller \act{giver stort glas}
her er mit telefonnummer, her er numrene til ambulancen og til kapellet. Med
hensyn til blødninger skal de skifte hovedpudebetræk tre gange om dagen. ring
hvis der skulle være noget.

\says{Tidl. patient} \act{gurgler}

\says{Tandlæge} spyt her
\act{holder skål frem, tidl. patient spytter ketchup ud}
\act{lukkes ud stiv gang}

\scene{Nu følger sangen, når intet andet angives synger tandlægen}
\end{sketch}

\begin{song}
\scene{AGEREN OG EFFEKTER}

\sings{Tandlæge}
mon nu fru hansen der er klart
til at ta' en til i en fart? \act{fru hansen nikker}
godt, lad os ta' den næste
værs'go \act{fru hansen til dør, lukker op for patient}
\sings{Fru Hansen} goddag goddag
\sings{Patient} goddag goddag
\sings{?}
har De haft noget ubehag? \act{patient med bind om hovedet,}
det var jo ej det bedste
\sings{Patient} nej

\sings{Patient}
på vej herhen det bedre blev
og jeg har faktisk dårlig tid
det kommer til at knibe
\sings{Tandlæge} nej
det si'r de alle sammen, men \act{tandlægen sidste tre linjer, henvendt mod publikum}
når først han her er kommet men
jeg la'r ham ikke slippe
nej

\sings{Tandlæge}
trefire dages ubehag \act{charmerende, overtalende}
og De vil helt ha' glemt den sag
nej De kan intet tabe
mon? \act{patient tvivlende}
sæt De Dem trygt i stolen op
og hvil kun Deres lange/lille/trætte krop \act{afhænger af skuespiller}
og giv Dem til at gabe \act{patient sætter sig, tildækkes for overkrop og hår og
  øjne, overarme spændes fast derefter}
gab

\sings{Patient}
mon ondt det gør når De taŕ fat? \act{patient første linje,}
nej, det sker yderst sjældent at \act{derefter tandlæge}
man blot det mindste mærker
men nerven \act{denne linie skjult mod publikum}
Nu skal vi lige ha' et stik \act{sprøjte (gerne stor)}
et til, et til, endnu et prik \act{manøvreres}
nu får De ingen smerter
før i morgen \act{samme, mod publikum}

\sings{Tandlæge}
Jeg lægger en tampon herind \act{assistenten gør en slang klar}
for nemt at holde tand fra kind \act{sugelyd fra bånd,prøvende, slukkes}
den skal de ikke sluge \act{istedet for tampon lægges ketchup ind og små sten}
vi lige skal ha' kødet væk \act{her føres knive (pap! og store)}
et snit, et til, et lille træk \act{ud og ind ad munden, dyppes i}
Fru hansen vil De suge? \act{ketchuppen i munden}
sug \act{der suges}

\act{tandlæge saver og borer, lydeffekter på bånd!}

\sings{Tandlæge}
nu har jeg fræset tanden fri
om lidt det siger knæk fordi
jeg fjerner knoglelåget
knæk \act{her brækkes der med en tang,}
men når jeg ind til nerven når \act{en stump træ tages ud, rød}
så tror jeg sikkert De forstår
det her var ikke noget.
\sings{Patient} gurgl

\sings{Tandlæge}
hvis halsen snøres sammen nu \act{henvendt til patient}
og vædske løber ned, sig RRU
vi blodet straks vil fjerne
RRU (pat) \act{fru hansen suger støjende}

se tanden ligger meget skidt \act{bekymret}
mon ikke jeg sku hjælpe lidt?
jeg flækker den jo gerne \act{hjælpsomt}

\sings{Tandlæge}
en hel patient, men tanden halv \act{fremsiges så tæt som muligt som citat fra
  sagaerne}
er bedre end det andet valg
derfor jeg tand vil flække \act{der saves}

nu saver jeg, jeg bru'r mit spejl
nu brækker jeg, mon jeg ta'r fejl? \act{der brækkes}
nej, jeg kan kronen knække
bræk act{lydbånd, tang tager stump ud (sten) og kaster støjende i spand}

\sings{Tandlæge}
hvis roden fjernes li'så let
det bliver som en menuet
fra ``skæbnens magt'' af Verdi
nej \act{replik efter mislykket træk}
når roden ej vi brække kan
jeg flække vil den halve tand
jeg skal nok blive færdig \act{trøstende til patient}
\sings{Patient} gurgl \act{skal være ``tak'' med munden åben og fuld af blod}

\sings{Tandlæge}
nu må den være løs men nej
jeg save må, forstår det ej
nu prøver jeg at brække \act{det lykkes heller ikke}

jeg tror en krølle roden har
jeg saver først, så fat jeg taŕ
for næste gang at trækkke \act{nu trækkes der og tanden ryger ud lydeffekt}

\sings{Tandlæge}
åh, endelig, der har vi den \act{lettet og stolt}
nu skal der ryddes op igen \act{beslutsomt}
vær glad at den er ude \act{til patient}
gurglende ja fra patient
såret lukkes effektiv
med nål og tråd og to tre hiv
et sakseklip, en knude

\sings{Tandlæge}
jeg lægger et kompres herind
og såret lukker sig gesvind
nu skal De sammen bide \act{munden lukkes af tandlægen}
\sings{Patient} støn
et glas med piller, tag blot to
hver gang det gir den bedste ro
DET VAR DEN ANDEN SIDE! \act{sidste linie patient, peger på anden side af
  hovedet}
\act{det skal tilpasses så det er første gang patienten kan få lov at tale}
\end{song}

\end{document}
