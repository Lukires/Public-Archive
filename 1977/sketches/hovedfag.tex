\documentclass[a4paper,11pt]{article}

\usepackage{revy}
\usepackage[utf8]{inputenc}
\usepackage[T1]{fontenc}
\usepackage[danish]{babel}


\revyname{DIKUrevy}
\revyyear{1977}
\version{1.0}
\eta{$n$ minutter}
\status{Færdig}

\title{Datalogisk hovedfagseksamen}
\author{HHK, EM}

\begin{document}
\maketitle

\begin{roles}
\role{E}[] Eksaminator
\role{V}[] Vejleder
\role{K1}[] Konsulent
\role{K2}[] Konsulent
\role{K3}[] Konsulent
\end{roles}


\begin{sketch}

  \scene{Scenen er auditorium 18. Publikum er tilhørere til
    hovedfagseksamen. Ved nogle borde lige foran publikum idder -- med ryggen
    til -- diverse lærere og censorer.}

\says{V} Næste!

\says{E} \act{kommer ind fra kulissen} Ja, det er vist mig nu. Kan jeg begynde
på mit fordrag?

\says{V} Ja, gå endelig i gang.

\says{E} \act{transparent med foredragets titel: ASMIC-77. Et programsystem til
  løsning af komplekse transaktionsproblemer.} Dette programsystem udgøres af et
større system af programmer, specielt designet med henblik på, at kunne være i
stand til at udføre transaktionerne af en så høj grad af kompleksiset, at de kun
kan behandles af ASMIC-77. I programopbygningen er der lagt vægt på modularitet
og pålidelighed. Her ses det overordnede systemdiagram: \act{Transparent: PPM
  $\rightarrow$ CSM (ASMIC-77) $\rightarrow$ PPM}

\says{E} Vi ser heraf, hvordan datafolw'et i systemet er. Det ses hvordan de
såkaldte nddatatransaktioner kommer ind i billedet. Disse inddatatranskaktioner
bliver nu overført til centralsystemet, hvor de væsentligste transaktioner
foregår. Efter endt behandlig af ASMIC-77 centralsystemmodulet, går de
transformerede transaktioner videre til ASMIC-77 efterbehandlingsmodul (ASMIC
seventy seven post processing module, PPM) hvor der sker en yderligere
transformation, hvorman for produceret uddata på humanoid-læsbar form.

Ser man en smule mere detaljeret på systemet ser det sådan ud: \act{Rullende
  transperant med 1 m rutediagram}

\says{E}[fortæller om rutediagrammet under vild bladren frem og tilbage] Resten
af foredraget vil omhandle en central del af dette systemdiagram, nemlig dette
her \act{peger}. Vi har skrevet denne centrale del om i lineær notation, som
vist her \act{ny transparent med div. programstumper}. Der er udformningen taget
udbredt hensyn til modulariteten og programbevisførelsen. Læg mærke til, at der
ikke finder forsøg sted på at tage et flag ned, der slet ikke er oppe.
Bemærk de principper for mudulær programmering, der er taget i
anvendelse. Modulet udgøres i dette tilfælde af en sætning. Til slut vil jeg
fremhæve, at implementeringen ikke er på påbegyndt endnu. På grund af det
metodiske forarbejde forventer jeg dog ikke væsentlige problemer i denne
forbindelse. Ja, jeg er færdig nu, er der nogen spørgsmål i denne forbindelse?

\says{V} Ja, der var en af pilene på systemdiagrammet, som jeg ikke helt
forstod. \act{der bladres rundt i transperanten}. Den lange ud til venstre.

\says{E} Åh, ja det er rigtig. I virkeligheden kommer man aldrig denne vej. Men
jeg har bevist, at den ikke skader!

\says{V} Ja, så tror jeg ikke, at jeg har flere spørgsmål. Skal vi så gå over
til eksaminationen?

\says{K1} Ja, først er der praktikprojektet. Jeg kunne godt have lyst til at
stille dig følgende spørgsmål: Hvad \underline{følte} du egntlig, da du løste
dette projekt?

\says{E} Jeg følte mig egentlig ganske godt tilpas (bortset fra en lettere
svimmelhed om morgenen).

\says{K2} Ja, jeg sidder her med dit projekt i administrativ databehandling. I
har med vilje udeladt en af de centrale problemstillinger i forbindelse med
projektet, udvalgsproblemet. Kan du begruunde det lidt nærmere?

\says{E} Ja, det blev simpelthen for stort. Alt for stort. Så vi udelod det
problem.

\says{K2} Jatak, jeg har ikke flere spørgsmål

\says{V}[til en person -- åbenbart en censor -- der sidder og sover. Prikker ham
i siden]

\says{NN} Hva', 10!! \act{lægger sig til at sove igen}

\says{K3} Ja dette arbejde handler jo om mikroprogrammering. I indledningen til
rapporten skriver du, at en binær datamat har to tilstande, ja, nej og
ved-ikke. Kan du komme med en forklaring på det?

\says{E} Ja----nej----ved-ikke-------

\says{V} Jatak, det skulle vist være det hele. Kan vi få karaktererne: \act{som
  ved skøjtekonkurrencer} 8.8,9.7,9.9,9.9,9.7

\says{V} Og det kunstneriske indtryk: 9.9,9.6,8.7,5.3,10.0

\says{V} Ja, det bliver vist et tital. Til lykke!

\scene{NÆSTE!}

\end{sketch}
\end{document}
