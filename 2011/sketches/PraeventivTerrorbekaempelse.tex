\documentclass[a4paper,11pt]{article}

\usepackage{revy}
\usepackage[utf8]{inputenc}
\usepackage[T1]{fontenc}
\usepackage[danish]{babel}

\revyname{DIKUrevy}
\revyyear{2011}
% HUSK AT OPDATERE VERSIONSNUMMER
\version{0.6}
\eta{$6$ minutter}
\status{færdig}

\title{Præventiv Terrorbekæmpelse}
\author{Guldfisk, Spectrum, Søren, Sune}

\begin{document}
\maketitle

\begin{roles}
\role{P}[Johan] Pressemedarbejder for en regeringspolitiker
\role{J}[Jenny] Ny og ganske grøn journalist
\role{A}[Sune] Alfonso, Goon i mafia outfit og solbriller
\role{X}[X Munter] Instruktør
\end{roles}

\begin{props}
\prop{2 glas}[Person, der skaffer]
\prop{1 whiskyflaske}[Person, der skaffer]
\prop{1 mafiosogevær}[Person, der skaffer]
\prop{1 bordlampe}[Person, der skaffer]
\prop{1 kontoragtig glaskaraffel}[Person, der skaffer]
\prop{1 notesblok}[Skaffer selv]
\end{props}


\begin{sketch}

\scene J vises ind af A. P står og kigger ud af vinduet.

\says{J} Hej.

\says{P} Goddag. Er du den nye journalist?

\scene J giver P hånden. P tører hånden af med en klud.

\says{J} Ja, jeg er lige startet på Christiansborg i går. Du er
faktisk den første pressemedarbejder, jeg taler med.

\says{P} Jamen, velkommen til. Sæt dig ned. \act{taler til sig selv}
Ny på borgen, det skal fejres.

\scene P tager en flaske whisky fra hylden/bordet frem og hælder op i
to whiskyglas. I det følgende drikker J af glasset hver gang P
skænker op imens P blot nipper til glasset. J når i løbet af sketchen
at få drukket del.

\scene P har nu ryggen til J.

\says{P}Ja, det var en skam, det der skete med din forgænger.

\says{J}Hvad skete der da med ham?

\says{P}Øh, ingenting. Nå, du havde nogle spørgsmål?

\says{J}[Trækker på skuldrene] I har foreslået at udvide beføjelserne
for de såkaldte "sladresider", hvor borgere kan angive hinanden for
socialt bedrageri, mistænkelige indkøb, brug af open source
styresystemer... og så står der "samt andre former for terror".

\says{P}Ja. Det er jo bedst at stoppe en terrorist \textit{før} han
begår terror. Nu går vi skridtet videre og stopper terroristen, før
han \textit{tænker} på at begå terror.

\says{J}Hvordan kan du vide, hvad folk tænker på?

\scene P rejser sig og henter et glas mere.

\says{P}Behaviourisme. Vi kortlægger folks adfærd. Hvis man køber
lærebøger i fysik og bruger sin fritid på at bygge bomber, så er man
fysiker og så er man på vej til at blive \ldots gymnasielærer. Køber
man en Mac, er man humanist - og dermed homosexuel - og dermed
terrorist. Køber man derimod en ny lækker Alienware-PC\ldots er man en
nørd. Men hvis man på PC'en kører Linux med crypt-fs -- er man datalog
\ldots og dermed terrorist.

\scene P skåler med J's glas: ``klink''.

\says{J}Crypt-fs? Det bliver man da ikke terrorist af.

\says{P}Crypt-fs bruger man kun, hvis man har noget at skjule - for
eksempel terrorplaner! Hvis man ikke har noget at skjule, har man
jo heller ikke noget at frygte.

\says{J}Så I vil undersøge alle der opfører sig mistænkeligt?

\scene J hoster lidt som reaktion på alkohollen.

\says{P}Præcis. Hvis vi havde haft denne forebyggelseslov noget før
kunne vi have fået Osama bin Laden udleveret \textbf{den første dag}
han tog sin turban på.

\scene P hælder op igen.

\says{J}Hvor mange terrorister har I så pågrebet med den nye lov?

\says{P}Camilla Broe!

\says{J}Hun var ikke terrorist -- hun var en enlig mor fra Rødovre.

\says{P}Ja, er det ikke smart? Nu kan vi bruge terrorlovgivningens
beføjelser til ikke-terrorsager. Det giver større retssikkerhed for
folket\ldots og børnene.

\scene P rejser sig op og stiller sig bag sin stol.

\says{J}Da ikke for den anklagede.

\scene Holder på ryglænet af stolen.

\says{P}Det er ofrene, jeg bekymrer mig om. Det gør du måske ikke? Er
du på terroristernes side?

\says{J}Nej\ldots

\scene{P går foran bordet.}

\says{P}Så kan du godt se, at vi bliver nødt til at registrere alle
mistænkelige aktiviteter. Derfor indfører vi logningsbekendtgørelsen
og sladresider om socialt bedrageri.

\says{J}Sociale bedragere er da ikke terrorister.

\says{P}Hør nu her. Camilla Broe blev udleveret vha. terrorlovgivningen, ikke sandt?

\says{J}Jo\ldots

\scene{P taler nu nærmest filisofisk ud mod publikum}

\says{P}Er hun mere terrorist end en social bedrager?

\says{J}Nej, ingen af dem er vel...

\scene P stiller sig ved siden af J og abryder med:

\scene{P stiller sig ved siden af J}

\says{P}Godt. Sociale bedragere er lige så store terrorister. Altså
kan man også anholde dem vha. terrorlovgivning. For hvis man ikke har
noget at skjule ...

\scene{P tager et langt skridt om bag J -- ud af hendes synsfelt}

\says{P} har man jo heller ikke noget \act{hvisker J i øret} \textit{at frygte}\ldots

\scene P støder sit glas mod J's: ``klink''.

\says{J}Er der ikke et problem med basale borgerrettigheder?

\scene P går bag om bordet på vej mod sin egen plads.

\says{P}Vi beskytter selvfølgelig de personfølsomme oplysninger. De er
tophemmelige.

\says{J}I en database som alle statsansatte har adgang til?

\says{P}Jaja, som sagt, det er fortroligt. Den er så at sige låst med
en a-nøgle.

\scene J står nu bag sin stol.

\says{J}Og hvad står ``a'' for her? Alle og enhver?

\scene P sætter sig ned.

\says{P}Nej\ldots A for a-a-a-fterretningstjeneste.

\scene P hælder op til J

\says{J}OK. Oppositionen har klaget over at de kun fik ganske kort tid
til at læse lovforslaget.

\scene Nu bliver P irriteret.

\says{P}Sniksnak. Vi behøver jo ikke bruge lang tid på at finde ud af,
at de røde lejesvende er imod. De er jo terroristernes nyttige
idioter. Se blot på Blekingegadebanden.

\says{J}OK, men forslaget blev fremlagt kl. 12, og fristen for at
protestere var kl 15. Det er kun tre timer!

\says{P}Herregud. Det er jo kun 1600 sider.

\says{J}Men det var jo først kl 13 de fik at vide, at fristen var kl 15.

\says{P}Ja. Det giver dem også \textit{fire} timer: Tre timer fra 12 til 15
og en time fra 12 til 13!

\says{J}Øh\ldots nej?

\says{P}Give or take\ldots det er det man kalder en off-by-one-fejl.

\says{J}NEJ! Din matematik giver ikke mening!

\scene P rejser sig op og kigger ud mod publikum.

\says{P}Visse-vasse. Matematikken skal bare nulres lidt til den giver
det \textit{rigtige} resultat. 35\% af alle kvinder og 45\% af alle
mænd i Danmark synes godt om lovforslaget. I alt er 80\% af alle
danskere altså enige med os. Det gør vores politik legitim.

\scene Pause

\says{J}Det er jo svindel!

\says{P}[Trækker på skuldrene, affejer] Pfffff.

\says{J}Folk skal da kende sandheden!

\scene P løsner på slipset

\says{P}[ryster på hovedet] Du har sådan et ufleksibelt forhold til
sandheden. Ligesom din forgænger. God politisk kommunikation handler
ikke om at fortælle sandheden. Det er først løgn hvis nogen kan bevise
det.

\scene P sætter sit glas, sætter sig ned og læner sig tilbage.

\says{J}Du kan da ikke lyve for folk.

\says{P}Det gør vi nu heller ikke. Vi vælger blot demokratisk at holde
visse ting hemmelige, så de ikke kan misbruges af terrorister. Hvad
folket ikke ved har de ikke ondt af. Vil du da gøre folket ondt?

\says{J}Nej da.

\says{P}[intenst til J] Du må forstå, at et høringssvar kræver, at de,
der bør høre høringssvaret kan høre det, men hvis man hører
høringssvaret efter høringsrundens ophør, har man ikke hørt noget. Så
hvis man sørger for at høringssvaret kun kan høres, når ingen kan høre
noget eftersom høringsrunden er ophørt, så er alt som sig hør og bør
med hensyn til høringsmuligheden i forbindelse med en høringsfrist.

\says{J}[Lost] øøh?

\says{P}[Til publikum] Tager jeg måske fejl?

\says{J}Nnnnej\ldots

\says{P}Så kan du nok se, at det at sprede information skal
straffes. Alle der har lækket et hemmeligt dokument til wikileaks,
redigeret en artikel på wikipedia, eller udfyldt et \textbf{anonymt
  spørgeskema} er terrorister!

\says{J}Men det er jo at undertrykke den frie presse! I ender jo med
at afskaffe alle borgerrettigheder og dømme alle for at være
terrorister!

\says{P}Sludder. Nu lyder du som en konspirationsteoretiker. Du er
måske en af dem der tror, at månelandingen var fup, og at Elvis stadig
lever? Ja, du tror måske sågar, at Penkowa selv har sin
doktordisputat.

\says{J}Aha! Interessant. \act{rejser sig og taler til publikum:}
Yderst interessant. Det involverer jo selveste Ministeren. Det ryger
direkte til redaktionen. Jeg kan se det for mig. Cavlingprisen er i
hus. \act{til P:} Velkommen på forsiden, du gamle!

\scene{J skåler med P's glas: ``klink''}

\says{P} Hør, min ven. \act{trækker notesblokken over på sin side,
  rejser sig og går over til J} Vi to er vist kommet skævt ind på
hinanden. \act{holder om J} Lad mig byde dig på en middag. Jeg kalder
på min chauffør. Alfonso!

\scene Alfonso kommer ind

\says{P} Jeg havde også sådan en \textit{hyggelig} middag med din
forgænger.

\scene P giver tegn til Alfonso skal føre J ud.

\scene P står alene tilbage på scene, dufter til sin whisky og smiler
tilfredsstillende.

\scene{Lys ud}

\end{sketch}
\end{document}

%%% Local Variables:
%%% mode: latex
%%% TeX-master: t
%%% End:

