\documentclass[a4paper,11pt]{article}

\usepackage{revy}
\usepackage[utf8]{inputenc}
\usepackage[T1]{fontenc}
\usepackage[danish]{babel}


\revyname{DIKUrevy}
\revyyear{1976}
\version{1.0}
\eta{$?$ minutter}
\status{Færdig}

\title{Dat2 sang 76}
\author{hhk}
\melody{Den Olderburgske Kongerække}

\begin{document}
\maketitle

\begin{roles}
\role{S}[] Sanger
\end{roles}

\begin{song}
  På DIKU har man gået tvende
år og er nu fyldt med mod.
Nu kan man faktisk se en ende
og lugte sejrens gulerod.
Enfoldigt gik man rundt og tro'de
at nu sku' man ha' fred og ro.
Man var så vel og glad til mode:
Nu sku' man bare ha' dat 2!
Det blev en gyder og af de slemme
de to vi sent vil glemme:
Åh -- Naur og BP.

Der står han da vor ny professor
med transparent og mikrofon.
Om Simula han forelæser
om blokstruktur og addition.
Han talte tit for døve øren
thi med en sindrig konstruktion
han satte fluks en lås for døren
og talte strengt om præcision.
Imens de andre de nød mimikken
fik jeg gerne læst kronikken
fra dagens Information.

Han forelæste helt ubændigt
om "`if"' og "`else"' i timevis.
Den lille Peter gik behændigt
uden om alt stof fra Gries.
Om præcedensrelationer
regulære grammatik-
-ker hørte vi i små portioner
på bånd fyldt op med god musik
For run-time stakke og en dope-vektor
de passer fint til Hektor

Som autoforelæsningslytter
man hyppigt måtte sige STOP!
Jeg ikke tror det mere nytter
blot at stille formler op.
Jeg ku' det sikkert ha' forstået
hvis jeg ku' se hvad det betød.
Men det blev aldrig gennemgået
så ud'n mat 1 så gik jeg død.
Deterministiske automater
er mad for mons for dispensater
Vi har jo så dårlig plads.

Med mellemrum får vi projekter
som skal ku' køre inden jul.
Man både mad og søvn sig nægter
og taber næsten alt sit sul.
Opgaveindkøringskaos
gi'r RECKU mere isenkram;
og Naur si'r til BP: "`La' os
nu gi' de folk et ord'nligt slam!"'
Rent ortografisk, der skal det holde "=
resten spiller ingen rolle
for Naur og BP.

I studienævnet højt man klager
det er en pinlig situation.
Hver uge bringer nye sager
man vil jo nødig støde no'n.
Det hele klares diplomatisk
med udvalg som sig hør og bør.
Det virker nok lidt bureaukratisk
men skaber ro til sagen dør.
Mens kværulanterne går og råber
sidder jeg her og råber
på fjerde gangs dispensation.
\end{song}

\end{document}

