\documentclass[a4paper,11pt]{article}

\usepackage{revy}
\usepackage[utf8]{inputenc}
\usepackage[T1]{fontenc}
\usepackage[danish]{babel}


\revyname{DIKUrevy}
\revyyear{1976}
% HUSK AT OPDATERE VERSIONSNUMMER
\version{1.0}
\eta{$?$ minutter}
\status{Færdig}

\title{Datamatens ruin}
\author{?}

\begin{document}
\maketitle

\begin{roles}
\role{FS}[] Falskspiller
\role{FM}[] Fodermester
\role{D}[] Datamat
\end{roles}

\begin{sketch}

  \scene{Lige før fodermestersangen kommer Falskspilleren ind.  Hun
    ser sig forundret omkring, trækker så opgivende på skuldrene og
    sætter sig ved det runde bord.  Hun trækker en lommelærke op af
    inderlommen og et glas ud af ærmet.  Hun drikker for sig selv
    under fodermestersangen.  Når denne er færdig, rejser hun sig og
    går hen og henter to manualer, som studeres interesseret under de
    næste numre.  Under spillet med datamaten følger hun dette og
    sammenligner med "`passende"' steder i manualerne.

  Efter Blueberry Hill...}

\says{FS}[rejser sig] Hm hm... Mon ikke at det her skulle kunne gå
\act{hen til skrivemaskinen, slår et par "`takter"'.  En lampe bliver
  rød på centralenheden og på kortlæseren.}

\says{FM} Lad dog være med det..., se nu...

\says{FS} Ta' det roligt, den blir snart grøn igen \act{først skifter
  centralenhed, dernæst kortlæser} Ja ja!  Og så skal vi vist ha' et
spil kort.  \act{Hulkort ud af ærmet, hen til spillemaskinen} Tra la
lej \act{blander og hælder hulkort i spillemaskinen}

\says{D} Sort es!  Retvendt!

\says{FS} ES... og Es.

\says{D} 2

\says{FS} ESS!

\says{D} ... fem.

\says{FS} Konge!!

\says{D} Konge, Dame og jeg får BO.

\says{FS} Knægt!!!

\says{D} .n..nej! \act{lampe skifter til rødt}

\says{FS} Og så var der vist betalingen \act{slår på siden af
  spillemaskinen, jackpot rasler ud og alle lamper på datamaten er
  skiftet til rødt} Så har vi vist fred resten af dagen.  \act{Vender
  sig mod de øvrige} Et parti makro d'herrer?

\end{sketch}
\end{document}
