\documentclass[a4paper,11pt]{article}

\usepackage{revy}
\usepackage[utf8]{inputenc}
\usepackage[T1]{fontenc}
\usepackage[danish]{babel}


\revyname{DIKUrevy}
\revyyear{1990}
% HUSK AT OPDATERE VERSIONSNUMMER
\version{1.0}
\eta{$n$ minutter}
\status{Færdig}

\title{Nyhederne}
\author{Ukendt}

\begin{document}
\maketitle

\begin{roles}
\role{S}[Ole] Nyhedsoplæser
\end{roles}

\begin{sketch}

    \says{S} Godaften, her er TV2 med nyhederne. \act{Sludder, sladder}

    \says{S} Vi er blevet bedt om at bringe en meddelselse fra brandvæsnet.
             Det er ikke tilladt at ryge i det store Dormitorium på Datalogisk Institut.
             Hvis det værst tænkelige uheld skulle ske, henviser vi til flugtvejene der, der og der. \act{pegende rundt på døre}

    \says{S} Og nu vi er i Universitetsverdenen:

             En undersøgelse foretaget af Gallup for TV2 viser, at de fleste af de nye studerende på universitetet hvert år har meget
             svært ved at forstå, hvorledes universitetet er opbygget. De studerende har bedt om at få gjort emnet lettere tilgængeligt,
             og det vil TV2 nu sørge for.

    \scene{Her kommer sangen om universitetets opbygning}

    \says{S} Dansk Sprognævn har nu taget konsekvensen af den voksende brug af EDB-maskiner i det danske samfund.
             Det danske alfabet bliver udvidet med bogstaverne \{, |, \}, [, \textbackslash, ]. \act{Udtales: æ, ø, å, Æ, Ø, Å}
             Det betyder alternative stavemåder for danske ord. \act{Overhead med "Majon\}se", "Jeppe [kj\}r elsker |llebr|d"}

    \says{S} Datalogistudiet på universiteterne bliver mere og mere populært. Men da studiet er temmelig hårdt og svært, bliver
             de studerende yngre og yngre. Datalogisk Institut ved Københavns Universitet tager nu konsekvensen af dette, og
             arrangerer introduktion til datalogistudiet for de helt unge.

    \scene{Her kommer Besøg på DIKU}

    \says{S} På kurset Datalogi 1M er der nu blevet indført et nyt begreb: Kernedødskriteriet.
             Det giver nye muligheder for årets Datalogi 1M studerende, idet der åbnes op for kernetransplantation. Og nu over til sporten:

    \scene{Her kommer 1. dels stafet}

    \says{S} Datalogisk Studienævn har nu endelig fundet ud af, hvad de nye datalogi-bachelors skal undervises i.
             Det viser sig, at emnet ``lazy garbage collection'' giver mulighed for ansættelse i renovationsbranchen.

    \says{S} Det påstås, at undervisningen i Datalogi i Danmark er dårligere end i de lande, vi normalt sammenligner os med.
             Lærerne på Datalogisk Institut ved Københavns Universitet er helt uenige i dette.
             Vi har haft et reportagehold på besøg hos Jørgen Sand, en specielt udvalgt, tilfældigt ansat lærer på Datalogisk Institut.

    \scene{Her kommer sangen Den lille Jørgen}

    \says{S} Jørgen Sand nævnte desude, at der i øjeblikket arbejdes med at revidere undervisningsplanen for de første 3 år på datalogistudiet.

    \scene{Her kommer 1.dels revision}

    \says{S} I forbindelse med sammenlægningen af de 2 Tysklande til et samlet StorTyskland kskal der ske en udbygning af Datalogisk Institut
             ved Københavns Universitet. Udbygningen foregår ved at tilføje en vestfløj og en østfløj til de alledere eksisterende
             nord- og sydfløje på instituttet. Men for at skabe sammenhæng i byggeriet vil der også blive bygget en forbindelsesfløj
             ud fra den eksisterende midterfløj. Udbygningen vil sikre, at DIKU bliver klar til EF's Indre Lager i 1993 (92?).

    \says{S} Vi har endnu en nyhed med tilknytning til Datalogisk Institut: Der er for nylig sket et lederskifte på instituttet:
             Jens Clausen trådte tilbage efter 2 år som Institutbestyrer. Han blev afløst af Eric Jul, der som ny Super- undskyld -
             Institutbestyrer skal lede og fordele arbejdet på instituttet de næste par år. Vi har haft ham på besøg i vores universitetsstudie:

    \scene{Hvis Gnyf har noget klar, kan Superbestyreren vikles ind her, sammen med musikken til Indiana Jones.}

    \scene{Her kommer Bestyrelsesvisen}

    \says{S} Her til sidst i Nyhederne har vi 1.dels udsigten.

             Dat0 bliver hård,

             Dat1P bliver hård,

             Dat1M bliver hård,

             og Dat2 bliver rigtig dejlig hård.

             Og det var slut på nyhederne for i dag - I pausen \act{sidste linje mangler}
\end{sketch}
\end{document}
