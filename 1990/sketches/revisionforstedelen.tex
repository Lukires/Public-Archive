\documentclass[a4paper,11pt]{article}

\usepackage{revy}
\usepackage[utf8]{inputenc}
\usepackage[T1]{fontenc}
\usepackage[danish]{babel}


\revyname{DIKUrevy}
\revyyear{1990}
\version{1.0}
\eta{$n$ minutter}
\status{Færdig}

\title{Revision af 1. delen}
\author{Gnyf \& Holger, ide af Null}

\begin{document}
\maketitle

\begin{roles}
% Medvirkende: Holger, Jolanta, Niels Ull, KDJ, John
% Rollebesætning ukendt
\role{S1}[] Studerende 1
\role{S2}[] Studerende 2
\role{L1}[] Lærer 1
\role{L2}[] Lærer 2
\role{H}[] Bertel Haarder
\end{roles}

\begin{sketch}

\scene{På scenen står der 3 stabler af kasser påtrykt forskellige emner fra dat0, dat1 og dat2.
    \begin{description}
        \item[Dat0] Afprøvning, sortering \& søgning, håndkøring, regelstyret indlæsning, Pascal
        \item[Dat1] Datastrukturer, databaser, systemering, parallel-programmering, maskinstruk., kerne, operativsystemer
        \item[Dat2] Beregnelighed og uafgørelighed, Gries, oversætter, logik prog., funktionsprog.
    \end{description}

    hvor stablen med Dat1 er højere end en person, og de andre er næsten lige så høje som en person. Øverst på stablerne er der talkum (støv)
}

\scene{S1+S2 kommer ind og kigger på stablerne, undersøgende.}

\says{S1} \act{banker ovenpå de små stabler hvorved talkummen støver, han hoster} Det er velnok en gammel, støvet 1.del!

\says{S2} \act{på den anden side af Dat1-stablen. Kigger med åben mund og polypper op på stablen} Jahh\ldots, og den er helt uoverskuelig.

\says{S1} \act{konstaterende} Det må vi gøre noget ved!

\says{S2} \act{ivrig} Ja, vi må lave en revision af 1. delen. \act{begynder at vælte stablerne sammen med stud1, så kasserne ligger i små bunker.}

\says{S1} Nu skal vi lave en 1. del som ikke bliver for stor, vi skal kun have det vigtigste med. Den skal være sammehængende og struktureret.

\says{S2} Ja, vi starter med Pascal! Det er ihvertfald struktureret \act{tager kassen med Pascal og starter på en ny stabel}.

\says{S1} Jeps, og så må vi også have håndkøring og afprøvning med, det hører sammen \act{sætter de to kasser ovenpå}.

\says{S2} Man må jo nødvendigvis have regelstyret indlæsning med, om ikke andet af princip \act{stabler kassen ovenpå}

\says{S1} \ldots og sortering \& søgning kan man jo ikke undgå \act{kassen ovenpå}.

\says{S2} \act{Står og kigger lidt mistænksomt på stablen, som er den samme Dat0-stabel som før} \ldots Hmm. Har du også en følelse af déja vù?

\says{S1} \act{Er fuldt i gang med næste stabel, og mumler ved sig selv} \ldots lad mig nu se \ldots nu samler vi de ting, som hører sammen.
          Her har vi kernen i det hele \act{tager kassen med "KERNE"}

\says{S2} \act{Stabler "operativsystemer", "maskinstruktur" og "parallel-programmering" ovenpå, ser kassen med "systemering", ryster på hovedet
                og smider den væk, alt imedens han fløjter "Marken er mejet"}

\says{S1} \act{Starter på en ny bunke. Stabler "databaser", "logik prog." og "funktionssprog" ovenpå hinanden.
               Tager kassen med "beregnelighed og uafgørelighed", viser den til S2, kigger opgivende i vejret, og smider kassen over skulderen}
          Den passer ikke ind i strukturen.

\says{S2} Nej, jeg fattede heller aldrig sammenhængen. \act{Stiller "oversætter"-kassen for sig selv, tager den næstsidste kasse,
          man kan ikke se hvad der står på den. Ud af kassen tages en pose flæskesvær, S1 ser fornøjet ud, viser nu "Gries"-siden frem til publikum,
          og kaster kassen væk.}

\scene{Der er nu følgende kasser skubbet væk: "Systemering", "Beregnelighed og uafgørlighed" og "Gries".}

\says{S1} \act{Tager kassen med "datastrukturer"} Hmm\ldots den kan ikke rigtig hægtes bag på, men vi kan jo lige liste den ind her
          \act{klemmer den ind mellem 2 kasser i Dat0-bunken} Nu er den der snart!, vi mangler bare at sætte bunkerne sammen.

\says{S2} \act{kigger op på uret, og udbryder bestyrtet:} Klokken er tyve minutter i otte! Vi skal have forelæsning.

\scene{De 2 studerende forlader scenen. Lige efter kommer en lærer listende ind. Han går rundt om de 4 bunker}

\says{L1} Kalder de det en 1. del? Der er jo ingen udfordringer!! Det der er slet ikke fagligt forsvarligt.
          Det må vi gøre noget ved. \act{går hen til sceneindgangen og vinker til en person udenfor scenen}
          Gider du lige komme med vores små ekstra godter !!

\says{L2} \act{kommer ind, storsmilende, selvtilfreds, slæbende på en stor sæk} Hvor bliver de glade for os, jeg elsker at spille Jul-emand.
          \act{Tager en kasse op} Hvad har vi nu her?? "Brugergrænseflade", det må Dat0'erne kunne lide, de er stadig så naive \ldots

\says{L1} \act{ved Dat1 P-bunken} Der skal jo også være noget, man kan BRUGE til noget! \act{lægger "kompleksitet" på bunken}

\says{L2} Denne bunke skal også have noget en lille gave. Lad os nu \emph{C} \act{viser kassen med "C" til publikum}
          - det ER så sundt. \act{Selvsikker som kun Jul} Man ved jo hvad man gør!

\says{L1} Hm \ldots hvad har vi her? Algebraiske strukturer, tjah\ldots så lærer de måske lidt gruppearbejde \ldots
          \act{Lægger kassen ovenpå "oversætter"-kassen}

\says{L2} \act{har fundet bunken med de bortkastede "B \& U", "Gries" og "Systemering", grædefærdig sentimental}
          SE, hvad der er smidt ud - kastet bort som en gammel sutsko. Dem MÅ vi da kunne få plads til.

\says{L1} Snøft. \act{tørrer sine øjne med en karklud fra Kantinen} Kan du huske dengang, der var rigtig datalogi til?
          Vi 'smutter' den lige ind \act{"B \& U" oven på kompleksitet}, og denne her ind her \act{"Systemering" på Dat2-bunken}.

\says{L2} \act{Står med "Gries"-kassen} Gries - det navn har jo altid haft en grim smag i ørene på de studerende.
          Hvis vi nu lige ændrer navnet \act{vender kassen til "Programkonstruktion"}, og flytter den ned på Dat0 - så er der såmænd
          ikke nogle, der lægger mærke til den lille, bitte smule.

\says{L1} \act{kigger på de 4 bunker, hvor Dat2-bunken ikke er lige så stor som de andre} Nu mangler vi bare sammenhængen -
          de skal selvfølgelig bare være lige store. Har vi ikke flere godter? \act{roder i sækken, og udbryder triumferende}
          Aha! Projekt X, den passer lige \ldots

\scene{De 2 lærere står og beundrer bunkerne, da de 2 studerende kommer ind igen}

\says{S1} \act{Kigger forfærdet på de store bunker} Hvad skal det forestille? New York Skyline?!!

\says{L2} Tag det nu roligt, I har jo bestået 1. delen.

\says{S2} Dette er jo ikke for vores skyld, vi gør det! Det er princippet der tæller.

\says{L1} Vi har stor sympati for jeres principper \act{tager et skilt op af sækken: "100 mundtl."}

\says{S2} \act{kigger kort på skiltet} Nej! Det vil vi aldrig gå med til, vi har større idealer.

\says{L1} Jamen, vi kan da også tænke større \act{tager et større skilt op af sækken: "400 skr."}

\says{S2} \act{tøvende} Jaah\ldots jeg\ldots NEJ, jeg er stadig ikke overbevist.

\says{L2} Vi MÅ da kunne finde en kompromis-løsning \act{tager to store skilte op: "Licentiat", og rækker et til hver studerende.}

\says{S2} Man har jo et standpunkt til man får et nyt \ldots

\scene{rækker hinanden hånden, smilende}

\scene{TROMMEVIRVEL eller uhyggelig lys, ind på scenen træder Bertel Haarder med en STOR kasse}

\says{H} HeHeHe \act{Hånlig, ond og ikke mindst forrykt latter (á la Nickolson i BATMAN)}
         Jeg har haft en VISION! \act{ryster pakken} ALLE på 1.del skal have et halvt års projekt\ldots
         HeHeHe \act{smider kassen oveni de andre, så bunkerne vælter. Haarder danser henrykt ud, og de 4 på scenen kaster
         kasser efter ham, rasende og opgivende}

\end{sketch}
\end{document}
