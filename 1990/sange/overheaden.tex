\documentclass[a4paper,11pt]{article}

\usepackage{revy}
\usepackage[utf8]{inputenc}
\usepackage[T1]{fontenc}
\usepackage[danish]{babel}


\revyname{DIKUrevy}
\revyyear{1990}
\version{2.0}
\eta{$n$ minutter}
\status{Færdig}

\title{Den Lille Jørgen med overheaden}
\author{Kristian Damm Jensen}
\melody{Den lille Ole med paraplyen}

\begin{document}
\maketitle

\begin{roles}
    \role{S}[Bodil] Sanger 
\end{roles}

\begin{song}
  \sings{S} Den lille Jørgen med overheaden
            han kender alle men mest dat1'en
            han spreder Sand i de øjne små
            når de vil flydende tasl forstå

            Dog vil han først komplementet vise
            og dernæst regne på en mantisse.
            Det hele flyder i drømmenes sal
            når Jørgen leger med sine tal.

            Når Jørgen opgaverne skal stille
            eksaminanten i håret pille
            de siger alle den er for svær
            men lille Jørgen ka' ik' la' vær'.

            Nu er han flyttet til andre vove
            nu kan dat1'eren roligt sove
            for nu skal Jørgen på første år
            forhindre at de dat0 består.
\end{song}

\end{document}

