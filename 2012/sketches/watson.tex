\documentclass[a4paper,11pt]{article}

\usepackage{revy}
\usepackage[utf8]{inputenc}
\usepackage[T1]{fontenc}
\usepackage[danish]{babel}

\revyname{DIKUrevy}
\revyyear{2012}
% HUSK AT OPDATERE VERSIONSNUMMER
\version{0.2}
\eta{$2.0$ minutter}
\status{Færdig}

\title{Watson}
\author{Troels, Phillip, Nana}

\begin{document}
\maketitle

\begin{roles}
\role{W}[Brainfuck] Watson
\role{F}[Klaes] Forsker
\role{A}[Morten] Studerende
\role{B}[Mikkel] Studerende
\role{C}[Sune] Studerende
\role{X}[NB] Instruktør
\end{roles}

\begin{props}
\prop{Kasse til Watson}[]
\prop{Nogle kroketkøller}[]
\prop{Kroketkugler}[]
\prop{Øl}[]
\prop{Racerbile-bord}[]
\prop{Kittel}
\end{props}

\begin{sketch}

\scene{Der sidder tre mennesker (A-C) samt en stor kasse med
  IBM-WATSON-logoet omkring et Caféen?-bord.}

\scene{B og C pakker læredet sammen}

\says{C} Det er også de forpulede lukke vagter der ikke kan finde ud af at ryde op efter sig!

\scene{B og C sætter sig ved bordet sammen med A og WATSON}

\says{A} Ha! Du kørte af banen. Drik ud!

\scene{De andre hælder en øl ned i en tragt i kassen}

\says{W} [robotstemme] MIN TUR.

\scene{Der flyver et par terninger ud af et hul i siden af kassen.}

\scene{De andre kigger på terningerne.}

\says{W} JEG ER I MÅL. VIL I SPILLE IGEN?

\scene{De andre ser meget fulde og trætte ud.}

\says{B} Igen? Den nye rus spiller fandme igennem!

\scene{Mand i kittel (F) kommer ind på scenen, og går hen til en af
spillerne.}

\says{F} Hej, jeg kan se at I har mødt WATSON.

\says{B} Kender du ham? Hvad læser han? Er han biokemiker? Det ville give mening.

\scene{F og B går frem mod scenekanten. Tæppe går for bag dem.}

\says{F} Nej min ven. Kan du ikke huske WATSON? Den datamat der vandt
Jeopardy sidste år?

\says{B} Øehmf... Næhøe.... Hvad?

\says{F} Netop! Jeopardy er nemlig røvkedeligt, så folk var
ligeglade. Derfor har IBM slået sig sammen med nogle af DIKUs
topforskere, for at give konceptet lidt mere jazz.

\says{W}[Fra bag tæppet] WATSON GEARER ALTID OP.

\says{F} WATSON kan omsætte 3 liter alkohol i timen, hvilket gør ham
voldsomt svær at konkurrere med.

\says{W}[Fra bag tæppet] NÅR WATSON SKÅLER, SKÅLER ALLE. SKÅÅÅÅÅÅL.

\says{B} Men hvad med strategi.. og øeh....

\says{W}[Fra bag tæppet] BUND ELLER RESTEN I HÅRET.

\says{F} Som det typiske drukspil går frem, bliver strategi mindre og mindre
relevant. Vi har faktisk bare monteret processoren fra et syngende
fødselsdagskort på en opvaskebalje i en papkasse. Hvilket er rigeligt til at udmanøvrere
enhver Caféen?-gænger.

\says{F} Maskiners intelligens er altså endelig nået op på siden af menneskets.
Ikke fordi maskinerne er blevet så meget klogere, men fordi vi indså at
mennesker faktisk er pænt dumme.

\says{W}[Fra bag tæppet] OTTE SEKSERE.

\says{A}[Fra bag tæppet] Årh hvad man, du slår godt, Watson!

\says{W}[Fra bag tæppet] LILLEMEYER.

\says{F} Nå, jeg tror at de har trukket sig ud på græsplænen. Lad os gå hen og se
hvad de laver.

\scene{Tæppe fra. Der spilles ølkroket. Der stikker en kroket-kølle ud
  af WATSON. Den skyder til en bold, der ryger ind under bandscenen.}

\says{A:} Ha! Du er i boksen!

\says{W} BUNDER STRAKS.

\scene{C hælder en øl ned i tragten.}

\says{C} Er du sikker på, at du kan holde til det?

\says{W} DET SAGDE HUN OGSÅ I GÅR.

\says{C} Hvem sagde det?

\says{W} DIN MOR.

\scene{Lys ud}

\end{sketch}
\end{document}
