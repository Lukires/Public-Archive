\documentclass[a4paper,11pt]{article}

\usepackage{revy}
\usepackage[utf8]{inputenc}
\usepackage[T1]{fontenc}
\usepackage[danish]{babel}

\revyname{DIKUrevy}
\revyyear{2012}
% XXX: HUSK AT OPDATERE VERSIONSNUMMER
\version{1.0}
\eta{$4$ minutter}
\status{Færdig}

\title{Udlicitering}
\author{Kristine Slot, Jakob Jensen}

\melody{Journey - Don't stop believing}

\begin{document}
\maketitle

\begin{roles}
  \role{S1}[Ronni] Sanger
  \role{S2}[Amanda] Sanger
  \role{X}[NB] Instruktør
  \role{Y}[Lund] Koreograf
\end{roles}

\act{Sangen vil fungere bedst med en up beat version som denne:\\
  \texttt{http://www.youtube.com/watch?v=xIoSTbPt\_PI}}

\begin{song}

  \sings{S1}
  (Jeg) sidder på min pind - koder over stok og sten
  Jeg tænker intet selv, men det' helt okay
  Mange hundred K, får jeg for at finde på
  en ny og klar idé - farver knappen blå!

  \sings{S2}
  Vi lever fedt i sus og dus
  du gi'r mig baad og sommerhus
  Bruger penge på mærketøj
  skal ha' mer' og mer' og mer' og mer'

  \sings{S1+S2}
  Dataloger - rene Michelangeloer
  Du/jeg laver kodepoisi
  Made in China, får man ikke godt design af.
  Indien, koder uden pli

  \sings{S1}
  Men en dag, går alting galt
  andre tænker or'ginalt
  Ukraine tager alle job
  - jeg bli'r sagt op.

  \sings{S2}
  Du mister b\aa d - sommerhus
  faar job i den lokale brugs
  jeg gaar vaek og du bli'r her
  vil ha' mer' ja mer' og mer' og mer'

  \sings{S1+S2}
  Dataloger - stadig Michelangeloer
  men andre koder nu som mig
  Made in China, får man pluds'lig godt design af
  livet er sat paa stand-by

  \sings{S1+S2}
  Udlicitering
  u-lands programmering
  De ta'r dit job

  \sings{S1+S2}
  Udlicitering
  u-land(e)
  De ta'r dit liv

\end{song}
\end{document}
