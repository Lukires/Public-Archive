\documentclass[10pt]{article}
\usepackage[utf8]{inputenc}
\usepackage{revy}
\title{Hvad skal jeg med DKD?}
\author{Pernille Rasmussen \& Tore Green}
\melody{Look at me, I'm Sandra Dee}

\version{1.3} % HUSK AT AJOURFØRE VERSIONSNUMMER!!

\revyyear{1995}
\parindent0pt
\parskip 1ex minus 1ex
\flushsingsright

\begin{document}
%\twocolumn[ % alt hvad der står imellem [] bliver skrevet i hele sidebredden
            % Hvis man ikke vil have to spalter, fjernes `\twocolumn[' og `]'
\maketitle

\begin{roles}
  \role{M1} Maries veninde
\end{roles}
%\begin{props}
%  Ingen særlige, tror jeg...
%  \prop{skraldespand} til skraldemanden
%\end{props}

\scene
Maries DKD-venner har endelig fået slæbt hende med på
fredagsbar. Talen falder på DKD, og Marie siger nej til at blive
medlem.
Da Marie går (på toilettet?) synger M1 i 'rollen' som Marie.


%]
\begin{song}
\sings{1}     Hvad ska' jeg med D -- K -- D?
               Nej det kan jeg ikke se
               Bowling og Barbie og rigtige mænd
               Sådan er D -- K -- D

\sings{2}     Marie er mit navn -- det' sandt
               (Mit) job er studierelevant
               Kælderens mørke med kodning og blå
               (Jeg) mit kald i kæld'ren fandt
               
\sings{}[tale] DKD dag og nat
               Sikke noget pjat
               Jeg skal kode og pudse mit bat
               For hvis ikk' jeg' på Uni
               Hvem skal så lukke konti?
               Se på {\tt yacc}-okser til jeg bliver mat

\sings{3}[sang]På caf\' e -- det' meget godt
               (Men) jeg skal altså spille Mud
               Embla og Ask -- ja, det ved man hvad er
               Jeg' ikk' til D -- K -- D

\sings{}[tale]    Fyre, fyre -- hold jer væk!
                         Jeres stolthed får et knæk
\sings{}[sang]   Det er da klart -- jeg er ikk' desperat
\sings{}[tale]    Ha! Gu' er hun så!

\end{song}
\end{document}
% Local Variables: 
% mode: latex
% TeX-master: t
% End: 


