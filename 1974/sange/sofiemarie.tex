\documentclass[a4paper,11pt]{article}

\usepackage{revy}
\usepackage[utf8]{inputenc}
\usepackage[T1]{fontenc}
\usepackage[danish]{babel}


\revyname{DIKUrevy}
\revyyear{1974}
\version{0.1}
\eta{$n$ minutter}
\status{Ikke faerdig}

\title{Sofie Marie}
\author{}
\melody{?}

\begin{document}
\maketitle

\begin{roles}
\role{S}[] Sanger
\end{roles}

\begin{song}
  Porten er lukket
  og natten er lang
  med nøglen optrukket
  Lilholt nynner en sang.
  Han tænker ikke just på dem
  som gik hjem da klokken blev fem
  Sofie, Marie, er navne til de to PDP'er
  Og ét kan man sige, at det dem begge klær'

  Han mindes de nætter,
  de sammen har haft,
  hvor intet ham træffer,
  og tapper hans kraft.
  Han åbner porten lidt på klem
  og tænker tiden delta t frem
  Sofie, Marie, til lidt af hvert, de er parat
  men ét må man vide, hvordan man trykker start.

  Men Lilholt ved netop
  hvordan det skal gør's
  Så er du i tvivl
  Er det ham du ska spø'r
  Det synes som om de oplagte bli'r
  for modstanden falder, skønt spændingen sti'r
  Sofie, Marie, nu spinder de som misser små
  nu vil det glide, hvad Lilholt vil kan han få

  De ta'r sig en trekant
  det hurtigt bli'r hedt
  da morgen oprandt
  de stadig holdt trit.
  Da resultatet forelå
  så var det no'ed der ku' forslå
  Sofie, Marie, de sukker efter fuldført dåd
  og lidt kan man sige, de også selv har få'et

  Da Lilholt er færdig
  han tænder en smøg
  ser på Marie
  og ta'r så sit tøj.
  Han tænker på, hvor tiden fløj
  og går, da han hører morgenstøj
  Sofie, Marie, nu venter de med spænding på
  hvad dagen vil give, hvad selskab de vil få.
\end{song}

\end{document}

