\documentclass[a4paper,11pt]{article}

\usepackage{revy}
\usepackage[utf8]{inputenc}
\usepackage[T1]{fontenc}
\usepackage[danish]{babel}


\revyname{DIKUrevy}
\revyyear{1974}
% HUSK AT OPDATERE VERSIONSNUMMER
\version{0.1}
\eta{$n$ minutter}
\status{Ikke færdig}

\title{Ressourcebesparing}
\author{?}

\begin{document}
\maketitle

\begin{roles}
\role{FL}[] Operatør
\role{PO}[] Porter
\role{NA}[] Peter Naur
\role{A}[] Studerende
\role{B}[] Studerende
\end{roles}

\begin{sketch}

  \scene{Scene 1: FL kommer hen til portneren, der står bag en
    skranke.  FL har en kuffert i hånden.}

  \says{FL} Er dette Naur's ressourcebesparende programmeringskursus?

  \says{PO} Ja, dette er {\em professor} Naur's ressourcebesparende
  programmeringskursus.

  \says{FL} Jeg er den nye operatør, hvor skal jeg gå hen?

  \says{PO} Lige et øjeblik, nu har De vel ingen forbudte varer med?

  \says{FL} Forbudte varer?  Må man da ikke ryge elle drikke spiritus?

  \says{PO} Det var ikke det jeg spurgte om.  Har de noget printerpapir eller hulkort med?

  \scene{Portneren åbner kufferten og tager et hulkort frem, kigger bebrejdende på FL.}

  \says{PO} Hvad er dette?

  \says{FL} Det er bare et kort fra min søster, der er på ferie.

  \says{PO}[Kigger igen på kortet] Jamen, der står jo kun noget i
  kolonne 1-79.  Der er altså en hel ubrugt kolonne.  Det skal
  afleveres til professoren.  Har de andre ting.

  \scene{Portneren roder kufferten igennem og trækker en julestjerne
    op.}

  \says{PO} Det er jo en ubrugt strimmel, den kan bruges.

  \scene{PO} Portneren demonterer stjernen.

  \says{FL} Jamen, den var bare til pynt.

  \scene{FL pakker kufferten sammen og går.}

  \scene{Scene 2: Naur sidder bag skrivebordet.}

  \says{NA} Lad mig nu se, dagens ration skal være: Øh, 3 hulkort
  ooooog 10cm printerpapir og 12 $\mu{}$sec CPU-tid.  Nej, vent nu lidt,
  3. hulkort er jo alt for meget.  2 må være nok.

  \scene{FL banker på.}

  \scene{FL kommer ind.}

  \says{FL} Goddag, jeg er den nye operatør.

  \says{NA} Velkommen, nu skal de høre om stedet her.

  \says{NA} Det er her på stedet vor fornemste opgave at give de
  studerende en fornemmelse af, hvor vigtigt det er, at spare på vore
  ressourcer, både når det gælder maskintid og papirforbrug.  Derfor
  er det vort håb at de ville hjælpe med at overholde reglerne.

  \says{FL} Hvilke regler?

  \says{NA} Hver student må kun bruge 2 hulkort, 10cm printerpapir og
  2 $\mu{}$sec CPU-tid om dagen.

  \says{FL} HA HA Hvad skal de bruge resten af tiden til? HA HA

  \says{NA} Det kan være at De har ret, 2 $\mu{}$ er nok i overkanten.
  Vi skærer ned til 1.  Tak skal de ha', de har den rette ånd, De skal
  nok falde til her.

  \scene{Begge ud}

  \scene{Scene 3.  Tom scene.  Skilt: 1 uge senere.}

  \scene{Printerpapir, hulkort og strimler kastes op over bagtæppet.
    Operatøren skynder sig ind på scenen, samler tingene sammen og
    putter dem i en papkasse.  Operatøren ud.  Skilt: Samme aften.}

  \scene{FL ved datamaten.  Det banker på døren.}

  \says{FL} Dagens kodeord?

  \scene{(Papkasse med to huller, et til at lægge hulkort i; et til at
    trække print ud igennem.)  Alle står omkring papkassen og svælger
    i printerpapir med påtrkt EVA'er.}

  \says{A} Sådan en dejlig dame fik vi ikke lov til at lave ved kurset
  i eftermiddag, det blev bare til sådan en \act{viser 10cm
    printerpapir med print over det hele}.

  \says{FL} Det bliver lad mig nu se: 50 hulkort á kr. 1, det er 50kr;
  10 sider printerpapir á kr. 10, så er vi oppe på 150kr; plus entre
  50kr.  Ja det bliver lige 200kr.

  \says{A} Det er dyrt men det er dejligt.

  \scene{Det banker på døren, operatøren går hen til den.}

  \says{FL} Dagens kodeord?

  \scene{Døren bliver lukket op.}

  \says{NA} Hvad foregår her?

  \says{FL} Jeg prøver kun at redde disse stalker fra et nervesammenbrud.

  \says{NA}[ophidset] Her prøver man at give folk en forståelse for
  tings værdi og så kommer De og ødelægger det.

  \says{FL} Tag det roligt kammerat, når de er færdige her, vil de
  alligevel bruge løs af ressourcerne, derfor er der ikke sket større
  skade.

  \says{FL} De får tilfredsstillet deres trang til perfektionisme, og
  jeg tjener gode penge.

  \says{NA} OK, men det vil vise sig at mine idéer vil sejre, når de
  er blevet forgiftet af dette frådseri længe nok \act{begynder at gå
    hen mod døren}.  Jeg vil sejre, jeg vil sejre.

  \scene{De andre står stadig ved papkassen.  Derefter alle ud med
    deres udskrifter knuget ind til sig.}

  \says{B} Og ved De hvad han gjorde i dag?  Han delte et hullet
  hulkort ud til os hver.  \act{Deler hulkort til dem, der sidder på
    føste række}.  Så tog han en kasse med huller \act{tager en
    papkasse} og sage "`Nu kan I klistre dem her på, så kan I bruge
  dem igen"' \act{tager en håndfuld huller og smider ud over
    publikum}.

\end{sketch}
\end{document}
