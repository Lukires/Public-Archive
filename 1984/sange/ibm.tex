\documentclass[a4paper,11pt]{article}

\usepackage{revy}
\usepackage[utf8]{inputenc}
\usepackage[T1]{fontenc}
\usepackage[danish]{babel}


\revyname{DIKUrevy}
\revyyear{1984}
\version{0.1}
\eta{$n$ minutter}
\status{Færdig}

\title{IBM}
\author{BO}
\melody{Bjørn Tidmand: ``Nu ta'r jeg til Dublin''}

\begin{document}
\maketitle

\begin{roles}
\role{S1}[JC] Studerende
\role{S2}[JT] Studerende
\role{S3}[J] Studerende/sanger
\end{roles}

\begin{sketch}

\scene{2 studerende ved et bord er flyder med papirer, 1 studerende med en kaffekop.}

\scene{De 2 studerende sidder og mumler og roder i deres papirer, studerende 3 kommer ind med en kaffekop og sætter sig.}

\says{S3} Nå, hvordan skær?

\says{S1} Vi har travlt af helvede til.

\says{S3} Ja, det er en ordentligt omgang, men der er da lang tid til vi skal aflevere, hvad er det I stresser for?

\says{S2} For saten, følger du da aldrig med i hvad der sker uden om dig.  Du går sgu' altid rundt med skyklappen på.

\says{S3} Nå, nå små slag. \act{Drikker lidt af sin kaffe.}

\says{S1} Hej pas på med den kaffe.  Du spilder på indkaldelser til institutrådsmødet.

\says{S3}[tager et papir op] Er det her en indkaldelse til et institutrådmøde.  Så er det sågar mere spændende end jeg havde forestillet mig.

\says{S2}[uden at se op] Hvad snakker du om.

\says{S3} Jo, hør her.

\scene{S3 rejser sig op og synger med stor følelse.  S1 og S2 sidder og roder videre, men lidt efter lidt bliver S2 opmærksom.}
\end{sketch}

\begin{song}

\sings{S3} Vil du frem i verd'nen så brug EDB
  Men hos IBM der er der bedst (men hos IBM er bedst)
  For man glemmer alt hvad man har lært
  Om en verden udenfor

\sings{S3} Der er ingen brug for Lo
  Og en/at strejke det er out
  Det er dejligt nemt at være her
  I et firma der kan mer'

\sings{S3}[omkvæd] Her hos IBM der går det altid godt
  Vi har altid gjort det flot
  Og det lykk'es os at være først
  Det er derfor vi er størst

\sings{S3} Der er sommerskolen for de unge
  Vi må ha' eliten frem
  Det en pragtfuld chance for at fjerne/finde
  De får der tænker selv

\sings{S3} IBM er mere end jobbet
  Der stribevis af tilbud
  Du kan bruge al din fritid her
  I et firma der kan mer'

\sings{S3}[omkvæd] Her hos IBM der går det altid godt
  Vi har altid gjort det flot
  Og det lykk'es os at være først
  Det er derfor vi er størst

\sings{S3} Og når nu vor stat skal spare
  Er det godt med IBM
  For vi ser en chance til at gi'
  Jer et nyt maskineri

\sings{S3} Der er alt hvad I skal bruge
  Der er terminaler nok
  Og det er en dejlig gave fra
  Det firma der kan mer'

\sings{S3}[omkvæd] Her hos IBM der går det altid godt
  Vi har altid gjort det flot
  Og det lykk'es os at være først
  Det er derfor vi er størst

\end{song}

\begin{sketch}

\says{S2} Hov, lad mig se hvad det er. \act{Tager papiret ud af hånden på S3.}  Det må være noget af det der REVY-flip.  Det er ihvertfald ikke vores, det er jo til at brække sig over.

\says{S3} Åhr, det svingede da meget godt. \act{nynner lidt}

\says{S1}[kommer også frem] Ved du egentligt hvad der skete?

\says{S3} Næh \ldots \act{bliver afbrudt af S1}

\says{S1} Nå så hør her!

\scene{Her skal så komme en monolog der kort og koncist fortæller hvad der skete.  Det skal siges så S3 bliver fønfølget.}

\says{S3} Nå, sådan.  Jamen skal vi så ikke lave den lidt om.  Prøv at høre her.

\sings{S3}[synger lidt af et vers] Der er ingen plads til IBM
  De skal bare ud herfra

\says{S3} Nej, det er ikke helt godt.  Nu har jeg det.

\sings{S3} Vi vil ikke ha' en IBM'er her
  Vi har set hvordan det gik NEUCC
  Der er meget andet vi skal ha'
  Og vi vil sgu' vælge selv

\says{S1+S2}[i munden på hinanden] Fedt nok.

\scene{Alle 3 går syngende ud.  De skal gå i takt med en hånd ud mod publikum (i show stil).} 

\end{sketch}

\end{document}

