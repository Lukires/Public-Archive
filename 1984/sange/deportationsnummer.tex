\documentclass[a4paper,11pt]{article}

\usepackage{revy}
\usepackage[utf8]{inputenc}
\usepackage[T1]{fontenc}
\usepackage[danish]{babel}


\revyname{DIKUrevy}
\revyyear{1984}
\version{0.1}
\eta{$n$ minutter}
\status{Færdig}

\title{Deportationsnummer}
\author{HVO}
\melody{Woody Guthrie: ``Deportee''}

\begin{document}
\maketitle

\begin{roles}
%role{VO1}[Skuespiller] Voice Over 1
%role{VO2}[Skuespiller] Voice Over 2
\role{P}[LOL] Pigen
\role{N}[J] Naur
\role{B}[JM] Bertel
\role{L}[KJ] Lærer
\role{LK0}[IB] Korlærer
\role{LK1}[CH] Korlærer
\role{LK2}[CE] Korlærer
\end{roles}

\begin{sketch}

\scene{Tæppet er for.  Over højtaleranlægget lyder radiovisens kendingssignal.  Derefter (evt. båndet):}

\says{VO1} DIKU-nyt, godaften.  På grund af kapacitetsproblemer på datalogisk institut ved Københavns Universitet - det såkaldte ``DIKU'' - og især på grund af ringe kandidatproduktion har man i dag i undervisningsministereiet holdt et møde, hvor DIKU's øjeblikkelige situation er blevet drøftet.  Vi stiller om til Frederiksholms kanal. \act{Der høres bølgelyde.}

\says{VO2} Ja, her i undervisningsministeriet er mødet netop slut, og Bertel, hvilke tiltag er man kommet frem til?

\says{B}[Voice-Over] Vi har fundet fremt il en ordning, hvorunder noget af belastningen på DIKU overføres til Århus Universitet, DTH og AUC.  På den måde håber vi, at kandidatproduktionen kan blive forbedret - for slet ikke at tale om forskningen.

\says{VO2} Kan vi få detaljer?

\says{B}[Voice-Over] Ordningen går i al sin enkelhed ud på, at de lærere, der producerer for lidt eller for ringe forskning, overføres til en anden læreanstalt.  På den måde bliver kun de bedste tilbage.  Og det har jo altid været vores mål, at DIKU skulle havde de bedste.

\says{VO2} Ja, det var jo positivt, det bedste er ikke for godt til DIKU.

\says{Kor af stemmer} Det bedste er ikke for godt til DIKU!

\scene{Tæppe fra.  Lærer (L) kommer frem og synger.}

\end{sketch}

\begin{song}

\sings{L} Jeg var lærer på DIKU og en af de første
  Sad som var min røv limet fast til min stol
  Vel har jeg da aldrig hørt til blandt de største
  Skønt jeg badede mig lidt i 70'ernes sol
  Men selvom der er over 1000 studenter
  Der står her hver dag foran lærenes dør
  Så er det som om DVU de forventer
  At der forskes det samme - eller mere end før

\sings{L}[omkvæd] Farvel da til DIKU, farvel kære venner
  Farvel HCØ og de steder, jeg kender
  Jeg er vejet og fundet for let til at vær' her
  Jeg er blevet deporteret op til AUC

\scene{Under de sidste linier af andet vers kommer et ``kor'' af lærere ind og synger med på de sidste omkvæd.  Naur kommer (ovre fra musikken) og vinker dem med sig ud.}

\sings{L} Fra direktoratet, der kom cirkulærer
  ``Jeres årsproduktion vil blive vejet og målt
  og Gud se i nåde til de dovneste lærere
  For slapt sløseri vil sgu' ikke bli' tålt
  Den, hvis produktion er på under en meter
  Kan godt se sig om efter nyt institut''
  Så tre fjerdedel har måttet følge med Peter
  Naur, som selv var så klog og sige slut

\sings{L+LK}[omkvæd] Farvel da til DIKU, farvel kære venner
  Farvel HCØ og de steder, jeg kender
  Jeg er vejet og fundet for let til at vær' her
  Jeg er blevet deporteret op til AUC

\sings{L+LK}[omkvæd] Farvel da til DIKU, farvel kære venner
  Farvel HCØ og de steder, jeg kender
  Jeg er vejet og fundet for let til at vær' her
  Jeg er blevet deporteret op til AUC

\end{song}

\end{document}

