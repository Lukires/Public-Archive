\documentclass[a4paper,11pt]{article}

\usepackage{revy}
\usepackage[utf8]{inputenc}
\usepackage[T1]{fontenc}
\usepackage[danish]{babel}


\revyname{DIKUrevy}
\revyyear{2013}
% HUSK AT OPDATERE VERSIONSNUMMER
\version{1.0}
\eta{6 minutter}
\status{Færdig}

\newcommand{\question}[3]{\fbox{\parbox{10cm}{
\begin{textrm}
#1
\end{textrm}

\begin{tabular}{c|c}
  0. &         1.\\\hline
  #2 & #3
\end{tabular}
}}}

\title{Hvem vil være kandidat?}
\author{Troels, Nana, Phillip, Brainfuck, Søren Pilgård}

\begin{document}
\maketitle

\begin{roles}
\role{I}[Troels] Instruktør
\role{V}[Spectrum] Vært
\role{P}[Simon] Preben (en rus)
\role{T} Torben Mogensen (præindspillet voiceover)
\end{roles}

\begin{props}
\prop{Musik og jingles fra "`Hvem vil være Millionær?"'}[AV]
\prop{Spørgsmåls-overheads}[AV]
\end{props}


\begin{sketch}

  \scene{To stole og et bar-bord. Talk-show-agtigt. Frit udsyn til
    projektor/bagvæg. V sidder i den ene stol.}
\scene{Jingle}

\says{V} Godaften, og velkommen til "`Hvem vil være kandidat?"'

\says{V} Det er første år med den nye studieordning. Vi glæder os til at se
hvad vores nye studerende er i stand til. Og lad os byde ham hjertelig
velkommen.  Tag rigtigt godt imod... Simon!

\says{P}[lidt sky] Hej, hej...

\says{V} Nå, Simon. Hvad bringer dig her?

\says{P} Jeg har læst på HTX de sidste 3 år, og jeg har kodet en smule Delphi.

\says{V} Og du vil gerne være kandidat. Hvad kunne du tænke dig at lave bagefter?

\says{P} Netcompany har vist mig sådan et organisationsdiagram, og
det har virkelig inspireret mig.

\says{V} Det lyder... spændende. Lad os se at komme i gang.

\scene{Jingle}

\scene{Underlægnings-tema \#1 begynder}

\scene{Første spørgsmål ses på bagvæggen:}


\question{Hvilket af disse to binære tal er størst?}{}{}

\says{V}[læser højt fra bagvæggen] Hvilket af disse to binære tal er størst?

\says{P}[ivrig] Det ved jeg godt, det er selvfølg--

\says{V}[afbryder P] Og svarmulighederne er...

\scene{Svar-mulighederne dukker op på bagvæggen og bliver læst højt af
V, én efter én, fra venstre til højre:}

\question{Hvilket af disse to binære tal er størst?}{1}{0}

\says{V} ..."`0: 1"'. Eller er det "`1: 0"'?.

\says{P} Øh... 1?

\says{V} Du vælger 1?

\says{P} Ja!

\says{V} Og det er dit endelige svar?

\says{P} Det er det største tal, så det er mit endelige svar.

\scene{Svar \#1 (højre søjle, værdi "`0"') bliver markeret. \scene{Jingle}}

\says{V} Svarmulighed nummer 1 er blevet valgt.

\says{P} Nej, vent! Jeg mente--

\scene{Jingle}

\scene{Den valgte svarmulighed bliver markeret som korrekt (m. grøn
  eller lign...)}

\scene{Jingle}

\says{V} Og den rigtige svarmulighed er... 1! ...altså "`0"'. Tillykke. Du har
nu 7.5 ECTS.

\says{P} Men... men...

\says{V} Ja, der er selvfølgelig tale om to-kompliments-repræsentation, så
det drejer sig om henholdsvis -1 og 0. Der var du hurtig.

\says{P} ...

\says{V} Næste spørgsmål!

\scene{Jingle}

\scene{V læser spørgsmål og svarmuligheder op efterhånden som de
bliver vist på skærmen:}

\question{KEN er en...}{Opvaskemaskine}{Desinficeringsmaskine}

\says{P} Åh, men KEN er da en opvaskemaskine!

\says{V} \act{stikker P en lussing}

\says{P} Av!  Øeh... Jeg mener desinficeringsmaskine.

\says{V} Og det er dit endelige svar.

\says{P} Ja.

\says{V} Det var ikke et spørgsmål.

\scene{Svar \#1 ("`Desinficeringsmaskine"') bliver markeret markeret som korrekt. \scene{Jingle}}

\says{V} Du er nu oppe på 22.5 ECTS. Det er længere en de fleste
kommer her på studiet. Vi går videre til næste spørgsmål.

\scene{Jingle}

\scene{V læser spørgsmål og svarmuligheder op efterhånden som de
bliver vist på skærmen:}

\question{Hvis din pegerpeger peger på noget andet end en peger, så er pegerpegerens pegning blevet hvad?}
{Gal}{Forkert}

\says{P}[tænker et stykke tid] Den er svær. Jeg vil gerne bruge en
livline og ringe til min vejleder.

\says{V} Vi ringer til din vejleder.

\scene{Riiiiing, riiiiing, riiiiing}

\says{T} Hej, det er Torben...

\says{P} Hej Torben! Jeg sidder lige...

\says{T} ... jeg kan desværre ikke komme til telefonen lige nu, da jeg er
inde og se DIKUrevy. Men læg en besked efter tonen, så ringer jeg
tilbage om et par timer.

\scene{Man hører et par takter af musikken fra Monkey Island,
hvorefter der bliver lagt på.}

\says{V} Vil du bruge endnu en livline?

\says{P} Hmm... Jeg tror faktisk jeg kan huske noget fra det faglige
indhold på rusturen.

\scene{P tager en rusturssangbog frem og bladrer lidt i den...}

\says{P} Ah, her er den! \act{Synger} Hvis din peger-peger peger på
noget andet end en peger, så er peger-pegerens pegning gået hen og
blevet gal...

\scene{Svar \#0 ("`Gal"') bliver markeret som korrekt.  Jingle.}

\says{V} Hvis du dropper ud nu, så får du lov at gå hjem med 75 ECTS.

\says{P} Jeg vil gerne lige se det næste spørgsmål.

\scene{V læser spørgsmål og svarmuligheder op efterhånden som de}
bliver vist på skærmen:

\question{Hvilken fil har vi \texttt{cat}'et ind i \texttt{/dev/audio}?}{/dev/random}{~jyrki/paper.pdf}

\scene{Man hører højttalerstøj.}

\says{P} (lytter intenst til støjen) Mmm... Det er... Jeg synes
overhovedet ikke det giver mening.

\scene{Svar \#1 ("`Jyrki"') bliver markeret.  Jingle.}

\says{V} Og det er dit endelige svar?

\says{P} Hvad?

\scene{Den valgte svarmulighed bliver markeret som korrekt.  Jingle.}

\says{V} Tillykke!  Du har nu 120 ECTS. Lad os gå videre til næste spørgsmål.

\scene{Jingle}

\scene{V læser spørgsmål og svarmuligheder op efterhånden som de}
bliver vist på skærmen:

\question{Hvad er den objektivt bedste editor?}{Emacs}{VI}

\says{P} Tja, det skal jo være den \textit{objektivt} bedste editor,
så jeg må hellere spørge publikum.

\says{V} Vi spørger publikum.

\scene{Publikumsjingle!  Der kommer en "`stemmetæller"' frem på
  bagvæggen. Publikum råber. Stemmetælleren fluktuerer, men ligger
  nogenlunde 50/50 på Emacs og \texttt{vi}. Indimellem dukker der
  søjler for andre editorer op (gedit, \texttt{ed}, Eclipse, ...)}

\says{V} Uafgjort?

\says{P} Jeg synes at jeg hørte flest Emacs.

\scene{Svar \#0 ("`Emacs"') bliver markeret.  Jingle.}

\says{V} Husk, hvis du svarer rigtigt nu, får du 180 ECTS og er bachelor. Men
hvis du svarer forkert, mister du det hele. Måske er det på tide at trække sig?

\says{P} Publikum har talt.  Jeg vælger Emacs!

\scene{Den valgte svarmulighed bliver markeret som korrekt. Jingle.}

\says{V} Tillykke, du er nu bachelor i datalogi!

\says{V} Du har nu sikret dig 180 ECTS uanset hvad. ...medmindre KU smider
dine resultater væk, selvfølgelig. Har du lyst til at fortsætte på
kandidaten?

\says{P} Jeg er klar.

\scene{Underlægningstema \#3 begynder.}

\scene{Jingle}

\scene{V læser spørgsmål og svarmuligheder op efterhånden som de
bliver vist på skærmen:}

\question{Hvilket af de følgende 2 udsang er korrekt?}  {$P =
  \textrm{NP}$}{$P \neq \textrm{NP}$}

\says{V} Hvis du svarer rigtigt på dette spørgsmål bliver du
kandidat. Det er alt eller intet!

\says{P} Hmm... Jeg vil gerne bruge min sidste livline.

\says{V} Fornuftigt.  Datamat, fjern ét af de forkerte svar

\scene{Jingle.}

\says{V}[utålmodigt] Jeg sagde: Datamat, fjern ét af de forkerte svar.

\scene{Jingle}

\says{V} Datamat?

\scene{Jingle}

\scene{V og P venter.}

\scene{Jingle}

\scene{V og P venter.}

\scene{Jingle}

\scene{V og P venter.}

\scene{Jingle}

\scene{Lys ud, tæppe for}

\end{sketch}
\end{document}
