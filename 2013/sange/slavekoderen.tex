\documentclass[a4paper,11pt]{article}

\usepackage{revy}
\usepackage[utf8]{inputenc}
\usepackage[T1]{fontenc}
\usepackage[danish]{babel}


\revyname{DIKUrevy}
\revyyear{2013}
\version{0.1}
\eta{$4$ minutter}
\status{Mangler et par ord og scenespil}

\title{Slavekoderen}
\author{Guldfisk}
\melody{Four Jacks: ``Mandalay''}

\begin{document}
\maketitle

\begin{roles}
  \role{I}[KØK] Instruktør
  \role{S1}[Rasmus] Sanger 1
  \role{S2}[Peter] Sanger 2
  \role{S3}[Arinbjörn] Sanger 3
  \role{S4}[Andreas] Sanger 4
\end{roles}

\begin{song}
  På et mørkt kontor på bryggen
  sidder der en datalog
  Han har kodet hele natten
  Han får aldrig, aldrig ro
  For han deadline var i forgårs
  han er tem’lig ramt af stress
  Kom igen, du kodeslave
  for din chef er utilfreds.

  Kod, du slavekoder nu
  Arbejd hele natten du
  Sæt dig ned ved datamaten
  Du skal kode lidt endnu.
  For du slavekoder, uh!
  det ’ din lod som datalog.
  Du skal sidde på din flade
  til din backend går itu.

  På en bænk i kongens have
  sidder der en datalog.
  Han er stakkels og forhutlet
  men engang, der var han go'
  Ak, hans job, det røg til Østen
  da konturerne gik ned
  Kom igen, du kodeslave
  det’ der ikk’ nog’t at gøre ved. (looooose fraseret...)

  Stakkels kodeslave, nu
  der er ingen jobs mer nu.
  Det er slut med datamaten
  Men du lever her endnu.
  For du slavekoder, uh!
  det ’ din lod som datalog.
  Du skal sidde på din flade
  til din backend går itu.

  På et mørkt kontor i østen
  sidder der en datalog
  Der på væggen er et webcam
  Det er her, han nu skal bo
  Han får tjent til dag’n og vejen
  han kan li’ det, kan du tro
  Han får end’lig brugt sin backend
  (og) du kan tro, at han er go’

  Og den kodeslave, du!
  (han) fester hele natten nu
  Det er slut med stress og kode,
  og han tjener penge nu
  For vor's slavekoderven
  han er end’lig glad igen
  han får sex af en hr. Cheng Weng
  der kan lide hans backend.
  <kort solo>
  (for) Det er nu, den tredje Verden,
  der kan * *  os, min ven :)
\end{song}

\end{document}

