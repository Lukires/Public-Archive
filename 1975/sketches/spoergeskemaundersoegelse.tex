\documentclass[a4paper,11pt]{article}

\usepackage{revy}
\usepackage[utf8]{inputenc}
\usepackage[T1]{fontenc}
\usepackage[danish]{babel}


\revyname{DIKUrevy}
\revyyear{1975}
% HUSK AT OPDATERE VERSIONSNUMMER
\version{0.1}
\eta{$n$ minutter}
\status{Færdig}

\newcommand{\kasse}{\hspace{1em}\fbox{\vrule height 5pt width 0pt\hspace{10pt}}\hspace{0.5em}}
\newcommand{\plads}{\vspace{.4cm}}

\title{Spørgeskemaundersøgelse}
\author{Solveig Rydahl}

\begin{document}
\maketitle

\begin{roles}
  \role{A}[Solveig Rydahl] Skuespiller
\end{roles}

Sketchen er bygget op omkring et spørgeskema, der er blevet
uddelt/besvaret på ukendt tidspunkt.  Det er reproduceret nedenfor.

{\bf Instruktorundersøgelse}

\begin{enumerate}
\item Hvilken markeringsmåde ønsker du at benytte?

  \kasse x \kasse o \kasse andet

\item Hvilket kursus har du fulgt?

  \kasse dat0 \kasse dat1 \kasse dat2 \kasse andet

\item Hvordan har instruktoren været?

  \kasse flink \kasse jævn \kasse ringe

\item Hvor ofte kom instruktoren til øvelserne?

  \kasse hver gang \kasse hveranden gang \kasse aldrig

  Hvor ofte kom du til øvelserne?

  \kasse hver gang \kasse hveranden gang \kasse aldrig

\item Kom du når instruktoren gav øl?

  \kasse ja \kasse nej

\item Hvis nej under 6: Er du afholdsmenneske?

  \kasse ja \kasse nej

\item Hvor mange små opgaver har du regnet i årets løb?

  \kasse 4 \kasse 1 \kasse ingen

\item Hvis ingen under 8: Har du gået på det under 2 nævnte kursus?

  \kasse ja \kasse nej

\item Hvor mange gange har du været hjemme hos din instruktor?

  \kasse flere \kasse 1 \kasse færre

\item Hvor mange var I?

  \kasse flere \kasse 2 \kasse færre

\item Hvad lave instruktoren mest vægt på?

  \kasse teori \kasse praksis \kasse andet

\item Kan du læse at læse datalogi?

  \kasse ja \kasse nej

\item Hvilket spil foretrækker du?

  \kasse bridge \kasse master mind \kasse skak \kasse reversi \kasse
  kryds og bolle \kasse go \kasse olsen \kasse agurk \kasse andet

\item Hvilken øjenfarve har du?

\kasse brun \kasse blå \kasse andet

\item Hvilken hårfarve har du?

\kasse rød \kasse grå \kasse andet

\item Hvad er dit navn?

\kasse Jens Jensen \kasse Anne Andersen \kasse andet

\item Gav instruktoren øl, mens du udfyldte dette skema?

\kasse ja \kasse nej

\item Ved du, hvor containeren er?

\kasse ja \kasse nej

Underskrift:

\item Har du lagt din besvarelse i containeren?

\kasse ja \kasse nej
\end{enumerate}

Følgende opføres antageligvis til revyen.  Eller måske udleveres det?
Hvem ved?

\begin{sketch}
{\bf Begrundelse for spørgsmål}

\begin{enumerate}
\item Der har så tit været brok over at man skulle sætter krydser -- eller at man ikke måtte.

\item For at kontrollere at det samlede antal besvarelser på et kursus
  ikke overstiger studentertallet.

\item Et af de centrale spørgsmål.

\item Vi må jo påse, at de ikke overskrider de normerede 306 timer.

\item Endnu et centralt spørgsmål.

\item Et spørgsmål, der giver rige fortolkningsmuligheder.

\item En vigtig oplysning.  Hvis der svares "`nej"' på både 6 og 7, må
  der være noget galt med de sociale kontakter.  Et "`ja"' kunne tyde
  på, at vi har at gøre med en afviger, og må behandle besvarelsen
  under hensyntagen hertil.

\item Et glimrende udgangspunkt for registrering og kontrol af de studerende.

\item En ekstra sikkerhed.

\item Den går igen på de sociale kontakter.

\item "`2"' indikerer en noget ensidig påvirkning og en {\em for} høj
  grad af arbejdsiver.

\item Jeg ved altså ikke hvad det kan bruges til, men nogle mente, det
  var nødvendigt at have det med.

\item Under de givne forhold har vi ikke råd til at have utilfreds
  iblandt os.  De vil straks få et strafpoint, for klager tager for
  lang tid.

\item Af hensyn til den videre udvikling i faget kunne det være
  nyttigt allerede på 1. de lat få fat i folk med nye epokegørende
  interesser.

  \noindent De tre næste spøgsmål kan være til stor gavn ved en
  kødannelse til næste år.

\item Som bekendt er mennesker med brune øjne upålidelige -- det har
  jeg selv læst i en damebrevkasse (som det så smukt hedder) --
  blåøjede personer må blive et politisk spørgsmål som højere instans
  må tage sig af.  De vil nok blive for lette at indoktrinere.  På den
  anden side er de ikke til så meget besvær.

\item Rødhårede er som bekendt koleriske -- de vil blive til for meget
  besvær.  Gråhårede må vi henvise til førtidspension, idet de
  alligevel vil blive pensionerede, inden de bliver færdige.

\item Vi kan jo ikke klare registreringen, hvis der er alt for mange
  med samme navn.

\item Hvis der svares "`ja"', må bevarelsen lades ude af betragtning.
  Påvirkede personer ka nnaturligvis ikke tages for gode varer.

\item Her begik vi vel nok en fejl ved formuleringen, idet vi skulle
  have oplyst til nejsigerne, at de kunne spørge Kurt eller Gurli
  eller Lissi.
\end{enumerate}

\textbf{Opgørelse}

Der indkom 414 besvarelser -- måske lidt mindre end vi havde håbet,
men dog alligevel et passende stort materiale.

Det viste sig desværre, at vi var nødt til at kassere nogle af
besvarelserne:

\begin{itemize}
\item der var 64 der havde brugt forkert notation.  Findes der
  virkelig nogle her, som ikke ved, at vi kun kan acceptere krydser?

\item der var 41, som havde skiftet notation under besvarelsen.  Mage
  til inkonsekvens!

\item der var 3 som ikke havde fulgt et førstedelskursus

\item der var 1, der svarede "`ja"' til spørgsmål 7

\item den ekstra kontrol under i var helt på sin plads, idet det viste
  sig at 3 overhovedet ikke var sikre på hvilket kursus de havde
  fulgt.

\item der var 2 som svarede "`2"' under 11, og om der er nogen
  sammenhæng skal jeg ikke kunne sige, men i hvert fald er der tale om
  påvirkning.

\item der var 57 med brune øjne

\item der var 217 som havde været under påvirkning -- spørgsmål 18.

\end{itemize}

Så er der 30 tilbage.  Heraf må 28 kasseres fordi underskriften ikke
stemmer overens med det under 17 opgivne.  Det er skuffende at så
mange har villet gøre grin med en så gennemgribende alvorlig
undersøgelse.  Man har forsøgt at narre os -- men det er sværere end
som så -- som for eksempel ved at skrive Joan Jensen, Jens Petersen,
Anders Nielsen, o.l.

Af de to resterende besvarelser havde den eneste svaret "`aldrig"' til spørgsmål 4, og den anden "`aldrig"' til spørgsmål 5.

Der er altså ingen af besvarelserne, der har relevans i en
undersøgelse af instruktorers mulighed for at opnå stemmeret.

Med hensyn til en oprydning i studenterbestanden og en mulighed for
udskillelse til næste år kan det med sikkerhed fastlås, at der er
rigeligt af folk:

\begin{itemize}
\item med brune og blå øjne
\item med rødt og gråt hår
\item der hedder Jens Jensen
\item der hedder Anne Andersen
\item og især dem der hedder andet
\end{itemize}

Ca. halvdelen svarede at de kunne lide at læse datalogi.  Den anden
halvdel må vi vel bortrationalisere.

Det meget centrale spøgsmål 14 gav et skuffende resultat, idet kun én
angav andet spil end de nævnte -- nemlig gris, og det spiller vi jo
alligevel så tit.

Det viser med al tydeligvis, at vi må gøre os de alleryderste
anstrengelser for at skaffe folk med nye interesser, så hvis jeg må
konkludere det derhen uden for mange omsvøb, så ville jeg mene, og jeg
håber at Institutråd og Studienævn er enige i denne sag, at vi ved
fremtidige stillingsoplag gør opmærksom på præcis hvilke -- hvis jeg
må sige det på den måde -- ? (slut)

\end{sketch}
\end{document}
