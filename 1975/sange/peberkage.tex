\documentclass[a4paper,11pt]{article}

\usepackage{revy}
\usepackage[utf8]{inputenc}
\usepackage[T1]{fontenc}
\usepackage[danish]{babel}


\revyname{DIKUrevy}
\revyyear{1975}
\version{0.1}
\eta{$n$ minutter}
\status{Ikke faerdig}

\title{Kantinesangen}
\author{Tom Skovgaard, Jens Hammerum}
\melody{Når en peberkagebager...}

\begin{document}
\maketitle

\begin{roles}
\role{S}[Anker Helms Jørgensen] Sanger
\end{roles}

\begin{song}
Når som ny kantinebørge
du skal for det hele sørge
ta'r du først et kilo kaffe
og en lille ekstra sjat
derpå varmer du med møje
fire liter i det høje
som du hælder ned på kaffen
og så går det hele glat

Når det sidste vand er løbet
fra det høje ned i dybet
ta' kontakten ud af væggen
og sæt kaffen hen på plads
derpå henter du så fløde
og så noget til at søde
som man hælde kan i kaffen
så det bliver noget stads

Derpå tænder du for vasken
under megen larm og pjasken
spiser basser, brød og snitter
til den viser firs og tres
når det lyser i den fjerde
ta' en teske sæbe mere
hvis det halve bliver renset
kan du være godt tilfreds

Klokken tolv så tar' du sedlen
og går ned til bag'ren med den
klokken to du henter mælken
og ta'r vand og øller frem
Og når det er blevet sent
si'r du bare der er pænt
skriver ik' et ord på sedlen
og ta'r nøglen med dig hjem.
\end{song}

\end{document}

