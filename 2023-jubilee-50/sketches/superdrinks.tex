% Oprindeligt filnavn: RekursivDrink.tex
\documentclass[a4paper,11pt]{article}

\usepackage{revy}
\usepackage[utf8]{inputenc}
\usepackage[T1]{fontenc}
\usepackage[danish]{babel}

\revyname{DIKUrevyens 50 års jubilæum}
\revyyear{(2002)}
\version{2.0}
\eta{7 min.}
\status{Færdig}

\title{Superdrinks}
\author{Uffe Christensen, Uffe Friis Lichtenberg, Jonas Ussing, Niels H.
  Christensen, Torben Æ. Mogensen, Jørgen Elgaard Larsen}

\newcommand{\sd}{\textbf{superdrinks}}

\begin{document}
\maketitle

\begin{roles}
  \role{1}[Simon] Videnskabsmand 1, i hvid kittel
  \role{2}[Sebbe] Videnskabsmand 2, i hvid kittel
  \role{N1}[Johnson] Ninja til flipover
  \role{N2}[Niels] Ninja til rengøring
  \role{X}[Troels] Instruktør
\end{roles}

\begin{props}
  \prop{En flip-over}
  \prop{8 * flip-over-plancher med tegninger (se nedenfor)}
  \prop{Et bord der kan spildes på}
  \prop{2 * Ginflasker}
  \prop{4 * Glas}
  \prop{Kande/kolbe}
  \prop{2 * Hvide kitler}
  \prop{2 * Lemon flasker}
  \prop{Pegepind}
  \prop{Spand}
  \prop{Suppeske}
  \prop{Tøris}
  \prop{Optagelse af voice-over}
\end{props}

\scene{Flip-over'en står på scenen med et blankt stykke papir forrest
  ved siden af bordet. 1 og 2 kommer ind}

\begin{sketch}

  \says{1+2} Uffe! \act{slår ud med armene}

  \says{1} Goddag, mit navn er Dr. Phil. Uffe Skjerning Linneberg\ldots

  \says{2} \ldots{}og jeg er Dr. Phil. Uffe Paaske Tørholm.

  \says{1} Vi arbejder for regeringens alkohol-taskforce, kendt fra TV
  som ``Alkoholdet''\ldots

  \says{2} \ldots{}eller ``The A-Team''.

  \scene{Det står og ser lidt selvfede ud.}

  \says{1} Som det nok er jer bekendt vil regeringens fest-tænke-tank
  snart offentliggøre opskriften på en \sd. Denne drinks er endnu ikke
  blevet testet i praksis, men er indtil videre blot et akademisk
  teoristykke.

  \says{2} Det har været vores opgave at afprøve drinksen under
  realistiske forhold. Det har vi gjort ved at præsentere studerende
  fra forskellige studieretninger for opskriften på \sd.

  \says{1} Som det ses indeholder opskriften på en \sd:

  \scene{Flipover: ``Superdrinks: 1 del gin, 3 dele lemon, 2 dele
    superdrinks''}

  \says{2} 1 del gin

  \says{1} 3 dele lemon (2: du er jo sindssyg!)

  \says{2} og 2 dele \sd

  \says{1} Det første sted vi tog hen var matematisk institut og hvad
  fandt du ud af der, Uffe?

  \says{2} Matematikeren valgte at betragte opskriften som en ligning, og løste
  derefter ligningen symbolsk, som man kan se her:

  \scene{Flipover: ``$S = \frac{1}{6}G+\frac{3}{6}L+\frac{2}{6}S
    \Rightarrow \frac{4}{6}S = \frac{1}{6}G+\frac{3}{6}L \Rightarrow S
    = \frac{1}{4}G+\frac{3}{4}L $''}

  \says{1} Umiddelbart herefter skreg matematikeren "QED", og faldt i søvn på
  tavlesvampen.

  \says{2} Men hvordan forholdt det sig så hos fysikerne, Uffe?

  \says{1} Fysikerne ``lånte'' noget \sd{} vha. kvantefluktationer:

  \scene{Flipover: Feynmann-diagram}

  \says{1} og afleverede det tilbage da blandingen var færdig.

  \says{2} Det viste sig naturligvis at være ganske udrikkeligt.

  \says{1} Ja, så vi tænkte at kemikerne måtte være de rette at søge
  hjælp hos. Og hvad sagde de så, Uffe?

  \says{2} Kemikerne brokkede sig over at de ikke havde 100\% kemisk rent gin
  eller lemon:

  \scene{Flipover: billeder af en kolbe med stort GIN-mærkat med en
    rød streg over og en anden kolbe med et stort LEMON-mærkat med en
    rød streg over.}

  \says{1} \ldots{}hvorpå de drak hjenen ud i finsprit.

  \says{2} Så havde vi dog \emph{lidt} bedre resultater hos biologerne, ikke sandt, Uffe?

  \says{1} Jo, eller, og dog\ldots{} Det viser sig at biologerne er
  nogle tålmodige mennesker:

  \scene{Flipover: et billede af et træ (biologisk).}

  \says{2} Ja, biologen ville pode et fyrretræ med citron-DNA og så
  drikke sig standervissen på Biobar i 20 år, indtil træet er vokset
  sig stort nok til at man kan tappe \sd{} direkte fra stammen.

  \says{1} Men så var det at vores søgen bragte os på sporet af
  datalogerne, Uffe?

  \says{2} Ja, og de havde rigtigt nok et interessant løsningsforslag,
  Uffe?

  \says{1} Datalogen valgte at anskue det som et klassisk rekursivt
  problem og satte sig for at løse det vha. approximation. Algoritmen
  går altså som følger, Uffe\ldots

  \says{2} Antag at du har en færdigblandet \sd. Tag 1 del gin, 3 dele
  lemon (1: du er jo sindssyg!) og bland det med 2 dele af din antagede \sd.

  \says{1} Da vil man for hvert skridt nærme sig ideal\sd{}en.

  \scene{Flipover: med et ML program ``fun superdrinks(0) = V |
    superdrinks(n) = 1/6*G + 3/6*L + 2/6*superdrinks(n-1)''.}

  \says{2} Nulte approximation er altså\ldots{} vand! \act{hælder op i
    2 glas, 1+2 smager på det og væmmes kraftigt.}

  % Evt. tilføje en joke?

  \says{1} Næste skridt bliver så at blande 1 del gin \act{2 blander
    løbende i 2 glas}, 3 dele lemon (2: du er jo sindssyg!)
    med 2 dele af vores nulte approximations \sd. \act{de smager på den færdige blanding og væmmes
    lidt mindre end før}

  \says{1} Det er jo stadig ikke rigtig godt.

  \says{2} Nej, men med næste skridt blander vi så 1 del gin \act{1
    begynder at blande}, 3 dele lemon (1: du er jo sindssyg) \act{1 holder op} og 2
  dele\ldots

  \says{1} \act{samme fagter som senere} Hov, stop, vent! Det her går
  alt for langsomt. Hvad nu hvis nulte approximation var gin?

  \says{2} \act{pause} Nåh... gin!

  \says{1+2} Oooookaaay\ldots

  \says{1} Altså vi starter med gin i vores \sd.

  \scene{2 hælder gin op i 2 glas, de smager på det hver især, bunder
    det og 2 hælder gin op igen}

  \says{2} \emph{Meget} bedre. \act{1 blander mens 2 snakker} Og så
  hælder vi 1 del gin, 3 dele lemon (1: du er jo sindssyg!) og 2 dele \sd{} op her.

  \scene{de smager på det, bunder den}

  \says{1+2} Aaaahhh\ldots

  \says{1}[lettere pløret] Godt så, så tager vi altså 1 del lemon, (2: du er jo sindssyg) 3
  dele gin og 2 dele\ldots{} øh\ldots \act{2 blander, men går i stå da
    approximationen mangler}

  \says{1} \ldots{}hvad er der blevet af vores
  \textbf{superdrink\ldots{}sss\ldots}?

  \says{2} \act{pause} Nå, så er det godt vi har forberedt os
  hjemmefra!

  \scene{de bunder begge og tager derefter en spand med boblende
    superdrinks op}

  \says{1}[vældigt pløret] Gosså, vi tager altså 1 lemon gin\ldots

  \says{2}[ditto] øh\ldots{} og 3 dele\ldots{} spand

  \says{1} og 2 sindssyge de er jo dele!

  \says{2} Så har vi en alletiders \sd. \act{vælter omkuld og falde i
    søvn}

  \scene{1 begynder at drikke af spanden}

  \says{VO} Moralen er altså at datalogerne måske ikke er vanvittigt
  præcise, men de opnår i det mindste et brugbart resultat.

  \scene{tæppe}




%   \says{F} Goddag. Jeg er dr. phil. Bjarne Hansen. Idag vil jeg
%   fremlægge de nyeste forskningsresultater fra min undersøgelse af
%   universitetsstuderende. Til brug for undersøgelsen har jeg fået
%   følgende opskrift fra Cafeen:

%   \scene{Flipoveren bladres. Der står opskriften
%     Super-drink:
%     1 del gin
%     3 dele lemon
%     2 dele super-drink}



%   \says{F} Som det ses, er der tale om en forholdsvis simpel opskrift.
%   Jeg har forelagt denne opskrift for forskellige studerende, og de
%   har forholdt sig til den på vidt forskellig vis\ldots

%   \scene{Flipover:

%     0. Vand

%     1. 1/6 gin + 3/6 lemon + 2/6 vand

%     2. 1/6 gin + 3/6 lemon + 2/6 \em{1}

%     3. 1/6 gin + 3/6 lemon + 2/6 \em{2}

%     \ldots

%     }

%   \says{F} En datalog vil gøre det rekursivt på følgende måde:

%   Nulte approksimation er at Superdrink er vand.

%   Første approksimation er at Superdrink er 1/6 gin + 3/6 lemon + 2/6
%   vand

%   Anden approksimation er at Superdrink er 1/6 gin + 3/6 lemon + 2/6
%   første approksimationsblanding.

%   Tredie approksimation er at Superdrink er 1/6 gin + 3/6 lemon + 2/6
%   anden approksimationsblanding.

%   Og så fremdeles.  Han vil aldrig lave en helt præcis Superdrink, men
%   han kan komme arbitrært tæt på.

%   Efter 4. approksimation vil han definere, at problemet er løst, og
%   gå over og bestille en superdrink i Cafeen.

%   \scene{Flipover: En flaske med en streg over}

%   \says{F} Kemikere vil klage over, at de ikke har 100\% kemisk rent
%   gin eller lemon på laboratoriet.

%   \says{F}Fysikere vil "låne" noget Superdrik via kvantefluktuationer,
%   og aflevere det tilbage, når blandingen er færdig.

%   \scene{Flipover: Et træ (sådan et biologist et)}

%   \says{F} Biologen vil pode et fyrretræ med citron-arveanlæg og så
%   drikke sig standervissen på Biobar i tyve år (indtil træet er vokset
%   sig stort nok til at høste af).

%   \scene{Flipover: En pige, der holder en flaske. Stort spørgsmålstegn
%     ovenover}

%   \says{F} Humanisten vil stille sig med en flaske gin og en flaske
%   lemon i hånden og læse definitionen en tyve-tredive gange, hvorefter
%   hun giver op og går ud at købe nogle "Bacardi Breezers".

%   \scene{Flipover: En mand med core\ldots øh, kårde}

%   \says{F} Teologen vil finde definitionen uransagelig, drikke ginen
%   rent og derefter trave Indre By igennem i kjole og med påspændt
%   kårde.

%   \scene{Flipover:

%     S = 1/6 G + 3/6 L + 2/6 S

%     (masser af plads nedenunder)}

%   \says{F}Matematikeren vil løse ligningen på følgende måde:

%   \says{F} [skriver]  => 4/6 S = 1/6 G + 3/6 L

%   \says{F} [skriver]  =>  S = 1/4 G + 3/4 L

%   \says{F} Derefter vil han være så udmattet, at han vil falde i søvn
%   på tavlesvampen.

%   \says{F}Moralen er altså, at dataloger måske ikke er vanvittig
%   præcise, men i det mindste opnår de et brugbart resultat. Tak for
%   iaften.

%   \scene{F bukker og går ud}

%   \scene{Tæppe}

\end{sketch}
\end{document}

%%% Local Variables:
%%% mode: latex
%%% TeX-master: t
%%% End:
