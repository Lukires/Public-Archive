% Oprindeligt filnavn: datalog_the_debugging.tex
\documentclass[a4paper,11pt]{article}

\usepackage{revy}
\usepackage[utf8]{inputenc}
\usepackage[T1]{fontenc}
\usepackage[danish]{babel}


\revyname{DIKUrevyens 50 års jubilæum}
\revyyear{(2014)}
\version{1.0}
\eta{$4.25$ minutter}
\status{Justérbar}

\title{Datalog: The Debugging}
\author{Troels, Phillip, Ejnar, Mia, Maya}

\begin{document}
\maketitle

\begin{roles}
\role{GM}[Ejnar] Spilstyrer for KUAs rollespilsklub
\role{N}[Marie] Ny humanistisk rollespiller der endnu ikke har stereotype
forestillinger om dataloger, spiller \textbf{Sigurd}
\role{G0}[Amira] Garvet humanistisk rollespiller der går efter indlevelse,
spiller \textbf{Preben}
\role{G1}[JoJo] Garvet humanistisk rollespiller der går efter at vinde, spiller
\textbf{Turing}
\role{G2}[Kim] Garvet humanistisk rollespiller der går efter at have en helt
fantastisk baggrundshistorie, spiller \textbf{Bella Bit-svane}
\role{X}[Niels] Instruktør
\end{roles}

\begin{props}
\prop{Video med Torben}[Niels]
\prop{Bord}[]
\prop{Terningeslag-lydeffekt}[]
\prop{Pawel-billede}[]
\prop{PC-billeder}[]
\prop{Datalog-regelbog}[]
\prop{Datalog-monsterbog}[]
\end{props}

\begin{sketch}

\scene{Lys op.  Der står et bord på scenen.  GM, G0 og G2 sidder ved bordet.
G1 og N kommer ind.}

\says{G1}[til N] Velkommen til KUAs rollespilsklub.  Har du nogensinde spillet
Datalog før?

\says{N} Neej, men jeg har kigget lidt på regelbogen.

\says{G1} Ja, vi har jo både spillet førsteudgaven, andenudgaven,

\says{G0}[fortsætter] tredjeudgaven, fjerdeudgaven,

\says{G2}[fortsætter] femte-

\says{GM}[afbryder G2, G1 og G0, henvender sig til N] Har du fået lavet dig en
datalog-karakter endnu?

\says{N} Ja, her \act{rækker karakterark over}.  Han hedder Sigurd.

\scene{De andre kigger på N's karakterark.  Illustration på OverTeX.  Pegen og
gestikuleren.}

\says{G1} Hmm, høj Charisma?  Hvad vil du bruge det til?

\says{G0} Og Hobby: Fodbold?

\says{N} Altså, jeg tænkte at jeg ville lave en velafrundet karakter.

\says{G1} Du har misforstået!  Dataloger er ikke velafrundede, bare... runde.

\says{G0} Og hit points?  Dataloger har slet ikke så meget liv. \act{trækker
regelbogen frem} Ifølge Mogensen så har den gennemsnitlige datalog hverken
venner eller kærester.

\says{GM} Så, lad os nu bare komme i gang.  Kan I andre ikke lige præsentere
jeres karakterer?

\scene{På OverTeX vises billeder af karakterne efterhånden som de bliver
introduceret.}

\scene{Billede af Bella Bit-svane.}

\scene{G2 tager en fabulous paryk på.}

\says{G2} Jeg spiller Bella Bit-svane!

\scene{G2 tager 100 siders papirer frem.  På det øverste er der måske stjerner
og glitter mens der står 'Min BG'.  G2 går begejstret i gang med at fortælle den
fantastiske baggrundshistorie.}

\says{G2} Min karakter nedstammer direkte fra Ada Lovelace, Dijkstra og Steve
Jobs.  Hendes forældre blev slået ihjel af en compilerfejl.  Og som baby faldt
hun ned i radioaktiv cola og nu ved alle at hun er den mest fantastiske og
smukkeste datalog nogensinde.

\says{G0}[tørt, ikke begejstret] Ja... vi kan godt se at du er speciel.

\says{GM}[afbryder] Kom nu videre.

\scene{Billede af Preben.}

\says{G0} \textit{Min} karakter er tro over for kildematerialet.  Jeg spiller
Preben, en datalogistuderende på 11. år, som bor i sin mors kælder.  Preben er
rigtig glad for at kode i C, og indenterer altid med to mellemrum, og så kan han
godt lide at spise Pringles med barbecue-smag.  I kantinen kan han...

\says{GM} Jaja, næste karakter!

\says{G1} Min karakter hedder \textit{Turing} \act{de andre sukker}, og er
designet som den \textit{ultimative} datalog.  På sin Github har han hver dag to
tusinde commits og-

\says{G2}[afbryder] Men han kan jo ikke engang lave sin egen toast!

\says{G1} Til gengæld får han +2 til SML!

\says{GM} Godt, lad os nu bare komme igang.  Bella Bit-svane og Turing er på vej
ind til gruppeeksamen.  Jeres eksaminator er... \act{ruller
terning}... \act{slår op i regelbogen} Pawel!

\scene{På OverTex vises Pawel-siden fra en monsterbog.}

\says{G2} Han er sgu da alt for høj challenge rating!

\says{G1} Men tænk på hvor meget ECTXP vi får ud af det!

\says{GM} Pawel har initiativ.  Han spørger: Kan I bevise det?

\says{G2} Bella Bit-svane forklarer at hun har 400 års nedstamning fra
Turing-vindere og at historiens \textbf{mest optimale} datalog er hævet over den slags.

\says{GM} Slå nepotisme.

\says{G2} Jeg lægger min renblodede Dijkstra-æt til.

\scene{De andre rollespillere sukker.}

\says{GM} Pawel er immun, han fortæller at Dijkstra stadig skylder ham en øl.
Din karakter falder.  Hvad gør Turing?

\says{G1} Jeg vil gerne prøve... at lave beviset.

\says{GM} Slå for det.

\says{G1} Turing har +31 i quicksort-algoritmer.

\says{GM} Pawel vil \textit{kun} høre om konvekse hylstre.

\scene{G1 slår en dårlig terning.}

\says{N+G0+G1+G2} Piß!

\says{GM} I mellemtiden sidder Preben og russen Sigurd i kantinen og koder.  Det
er nu blevet eftermiddag, og Bella Bit-svane og Turing kommer ind i kantinen.  I
kan se at de er dumpet.

\says{G0} Preben siger ``DNUR''.

\says{G1} Turing siger ``bliv nu færdig, gamle røvhul''.

\says{GM} Random encounter!  \act{slår en terning, synes resultatet er sjovt}
Politikerne har lavet en reform.  I er alle for langt bagud med studiet, og
bliver smidt ud!

\says{N} Men jeg er rus!

\says{GM} Så må du skrive en dispensationsansøgning.  Slå Bureaukrati.

\scene{N slår en terning.}

\says{N}[slår terning] Yes, 20!

\says{G2} Ej, se, den ligger skævt.

\says{N} Så må jeg slå om. \act{skal til at tage fat i terningen}

\says{G0}[tager fat i N's hånd] Nej!  Man må ikke røre terningen når den er
landet, det bringer uheld! \act{G1 ryster også på hovedet over N}

\says{N} Men den er skæv!

\says{G1} Din mor er skæv!

\says{N} Jeg kan ikke se hvordan det er relevant.

\says{G2}[har taget regelbogen] Okay, hvad med det her nummer til en
terning-hotline bag i bogen?

\says{GM} Jeg prøver at ringe.

\scene{Telefonen ringer op, og man ser på OverTeX en video af Torben med en
fastnet-telefon.}

\says{TM} God...

\scene{Torben ruller en terning.}

\says{TM} ... aften, og velkommen til Torbens terninge-hotline.

\says{GM} Ja, hej, vi kan ikke blive enige om hvorvidt denne terning ligger
skævt.

\says{TM} Sig mig at I ikke har flyttet på den!

\says{G0} Nej!

\says{TM} Godt.  Så er jeg mere rolig. Den slags bringer uheld.

\scene{Torben ringer rundt for at få hjælp.}

\says{TM} Det er en 7'er!

\says{G1} Ej!  Den ligger jo mellem to og tyve, altså, faktum er jo at hvis man
kigger på det-

\says{TM}[afbryder] Torben har talt!

\scene{Der er en outro på videoen.}

\scene{Lys ned.}

\end{sketch}

\end{document}
