% Oprindeligt filnavn: Traadsketch.tex
\documentclass[a4paper,11pt]{article}

\usepackage{revy}
\usepackage[utf8]{inputenc}
\usepackage[T1]{fontenc}
\usepackage[danish]{babel}

\revyname{DIKUrevyens 50 års jubilæum}
\revyyear{(2008)}
\version{1.5}
\eta{$6$ minutter}
\status{Færdig}

\title{Trådsketchen}
\author{Sandfeld, Troels, Mikkel}

\begin{document}
\maketitle

\begin{roles}
    \role{F}[Ejnar] Metafortæller
    \role{RH}[Amira] Rødhætte
    \role{U}[Niels] Store stygge ulv
    \role{G1}[JoJo] Første lille gris
    \role{G2}[Marie] Anden lille gris
    \role{G3}[Brandt] Tredje lille gris
    \role{B}[Amanda] Bedste (voice over)
    \role{N}[Marcus] Ninja til at sætte puder på U uden for scenen
    \role{X}[Sejer] Instruktør
    \role{Y}[Simon] Bonusinstruktør
\end{roles}

\begin{props}
    \prop{Hus af strå}[JoJo]
    \prop{Hus af pandekager}[Marie]
    \prop{Bedstes hus}[Amira/Brandt]
    \prop{Et træ?}[??] Som RH kan gå rundt om (''ind i skoven'')
    \prop{Semafor}[Amira] 2 flag, rødt og grønt
    \prop{Gaffertape}[Ejnar]

    \prop{Dessertpandekager}[Marie]
    \prop{Ulvemaske/-hat}[Niels] Gerne med munden fri
    \prop{3 grisetryner}[JoJo, Marie, Brandt]
    \prop{Blå overalls}[Brandt]
    \prop{Sømandsbukser el. lign.}[JoJo, Marie]
    \prop{Blond paryk}[Amira]
    \prop{Rød hætte}[Amira]
    \prop{Bedstes kyse}[Niels]
    \prop{Fyldig bondepigekjole}[Amira] Gerne med underkjole
    \prop{Fletkurve med pandekager}[Amira]
    \prop{Fletkurv med værktøj}[Marie]
\end{props}


\begin{sketch}

\scene F står på scenen og gestikulerer begejstret
\scene{\textbf{Alternativt}: F sidder med en pibe el.lign. og er meget selvfed}

\says{F} Mine damer og herrer! Jeg vil nu introducere en verdensnyhed.
\says{F} Vi har, for at optimere DIKUrevyen, valgt at udnytte de moderne muligheder for ekstrem multiprogrammering og kan dermed udføre revyen på hele to kerner, samtidigt!
\scene{F sætter en stribe gaffertape midt på scenen. Bemærk, at det er publikums højre og venstre, der menes i det følgende.}
    \says{F} Den første kerne er på venstre side af scenen, og den anden kerne er på højre side af scenen.
\says{F} Det har neturligvis også givet nye muligheder for besparelser, såsom delte ressourcer.
    Delte ressourcer kan dog være tricky, da man skal holde øje med race-conditions, starvation og ikke mindst de frygtede deadlocks.
    Derfor er det godt at den ene delte ressource er mig, så skulle sketchen være i trygge hænder. \act{Blinker til publikum}
\says{F} Læn jer nu blot tilbage og ny simultaneventyret ``Rødhætte og de tre små grise''.

\scene Tæppe fra
\scene Lys: Højre side

\says{F} Der var engang 3 små grise...

\scene Lys: Venstre side
\scene F løber over på venstre side

\says{F} og en lille Rødhætte. Det hed hun fordi hendes hætte var så rød.

\scene Lys: Højre side
\scene F løber over på højre side

\says{F} De små grise var rejst hjemmefra og skulle hver især bygge deres eget lille hus.
\says{Grise} Vi skal hver bygge vores eget lille hus.

\scene Lys: Venstre side
\scene F løber over på venstre side

\says{F} Rødhætte var sendt afsted med en kurv af godter til sin syge bedstemor, som boede langt ude i skoven.
\says{RH} Jeg skal ud til min syge bedstemor, som bor langt ude i skoven. \act{Traller rundt i skoven}

\scene Lys: Højre side
\scene F løber over på højre side

\says{F} Den første lille gris byggede sit hus af strå.
\says{G1} Jeg tror jeg vil bygge mit hus af strå!
\says{F} ... for det virkede som en god idé.

\scene G1 bygger sit hus.
\scene Ulven kommer ind på scenen.

\says{F} Inde i skoven på vejen til Bedstemors hus boede en stor styg ulv, som var meget sulten og meget fæl.
\says{U} Jeg er meget sulten og meget fæl.
\says{F} Ulven, som var så sulten, kom forbi den første lille gris' hus.
\says{U} Luk mig ind, luk mig ind lille gris. Eller jeg vil puste og pruste til dit hus falder omkuld.
\says{F} Og ulven pustede og prustede til huset væltede omkuld ...
\scene{U puster til huset. Da det ikke vælter, sparker U i stedet 'diskret' til huset, så det vælter.}
\says{F} ... og spiste den første lille gris.
\scene{U jæger G1 om bag bagtæppet. Der høres et skrig, før ulven kommer tilbage, nu med en pude e.l. under maven.}

\scene Lys: Venstre side
\scene F løber over på venstre side

\says{F} Inde i den mørke skov stødte Rødhætte på den store stygge ulv.

\scene Ulven går truende over mod Rødætte

\says{U} Goddag Rødhætte.
\says{RH} Goddag ulv.

\scene Lys: Højre side
\scene F løber over på højre side

\says{F} Den anden lille gris var mere snedig og havde en kurv med værktøj, så han kunne bygge et rigtigt hus.
\says{G2} Jeg har en kurv med værktøj så jeg kan bygge et rigtigt hus.
\scene G2 stiller værktøjskurven midt på scenen

\scene Lys: Venstre side

\says{U} Hvad er det du har under forklædet?
\says{RH} Det er pandekager.

\scene Rødhætte stiller kurven midt på scenen lige ved siden af værktøjskurven
\scene F løber over i venstre side

\says{F} Ulven tænkte
\says{U} ``Det var da en lækker lille sag''
\says{F} ... og udtænkte en udspekuleret plan.

\scene G2 samler Rødhættes kurv op i stedet for værktøj

\says{U} Tag og se dig omkring engang. Se hvor dejlige blomsterne er. Nyd dem, duft til dem.
\says{RH} Jeg tror jeg vil plukke en smuk buket til min bedste.
\says{F} Men straks fandt Rødhætte endnu flottere blomster længere væk og kom længere og længere ind i skoven.
\says{F} Imens gik ulven den \emph{korteste vej} til bedstes hus.

\scene Lys: Højre side
\scene F løber over på højre side

\says{F} Ulven kom nu forbi den anden lille gris, som havde bygget sit hus af... øh... Pandekager?
\says{U} Luk mig ind, luk mig ind lille gris. Eller jeg vil puste og pruste... mens jeg æder mig mæt i pandekager.
\says{F} Og ulven pustede og prustede... og åd pandekager.

\scene Lys: Venstre side
\scene F løber over på venstre side

\says{F} Imens havde Rødhætte plukket en smuk buket og kom nu til bedstes hus, hvor hun fandt døren åben... øh... eller rettere lukket?

\scene Rødhætte går op til bedstes hus hvor døren er lukket
\scene Fortælleren ser forvirret ud
\scene Rødhætte banker på bedstes dør

\says{B} Kom ind!
\says{F} Hov! Stop! Vent! Frys kerne 1!

\scene Rødhætte fryser.

\says{F} Du må vist hellere vente på denne her semafor.

\scene Fortælleren giver Rødhætte to flag og sætter hende i semafor-position

\scene Lys: Højre side
\scene F løber over på højre side

\scene Fortælleren japper sig hurtigt igennem det næste stykke for at indhente kerne 1
\says{F} Som sagt: Og ulven pustede og prustede til huset væltede omkuld og spiste den anden lille gris.

\scene Lys: Venstre side
\scene F løber over på venstre side

\says{F} Og fordi ulven havde gået den \emph{korteste vej} kom han først til bedstes hus, og bankede på døren.
\scene Ulven kynder sig over op den anden side af scenen
\says{U} Jeg kom først til bedstes hus.
\scene Ulven går fordi den frosne Rødhætte og skuler til hende
\scene Ulven banker på døren
\says{B} Kom ind.
\says{F} Ulven gik nu ind og fandt bedste i sengen, og fordi han var så sulten, slugte han hende i én mundfuld.
\says{F} Ulven tog nu bedstes kyse på \act{U tager kysen på}, for det var en del af hans snedige plan.
\says{U} Det er en del af min snedige plan.

\scene Lys: Højre side
\scene F løber over på højre side

\says{F} Imens havde den tredje lille gris, som var den snedigste af de tre, bygget sit hus af mursten.
\says{G3} Jeg har bygget mit hus af mursten.
\says{F} Men ulven kom nu forbi den tredje gris' lille hus.
\scene Ulven kigger ud af bedstes hus
\says{U} Hvad?
\says{F} Jeg sagde: Men ulven kom nu fordi den tredje gris' lille hus.

\scene Ulven slæber sig propfyldt ud af bedstes hus fra venstre side og over på højre side af scenen

\says{F} ... og var stadig Ååh så sulten.
\says{U} Jeg er stadig Ååh så sulten...

\says{F} Den tredje lille gris løb ind i sit hus og låste \emph{alle} døre.
\scene Den tredje lille gris løber ind i bedstes hus fra højre side og låser dørene
\says{G3} Du får mig aldrig!

\says{U} Luk mig ind, luk mig ind, lille gris! Eller jeg vil \act{pust} puste og pruste til dit hus falder omkuld.
\scene Ulven pruster af mæthed
\says{G3} Bare kom an store stygge ulv.

\scene Rødhætte vifter med semaforen
\says{RH} Hej hej, hallå altså!
\says{F} Ja okay, bare fortsæt kerne 1.

\scene Lys: Hele scenen

\says{F} Ulven pustede og prustede... og Rødhætte kom nu til bedstes hus, hvor hun fandt døren åben... øh... \act{fuck}, eller låst.
\scene Rødhætte banker på den låste dør
\says{RH} Hallåå Bedste, luk mig ind, lissom!!
\says{G3} Nej, jeg lukker dig ikke ind!
\says{U} \act{Pust} \act{Prust}!

\scene Rødhætte banker på den låste dør
\says{RH} Hallåå Bedste, luk mig ind, lissom!!
\says{G3} Nej, jeg lukker dig ikke ind!
\says{U} \act{Pust} \act{Prust}!

\scene Rødhætte banker på den låste dør
\says{RH} Hallåå Bedste, luk mig ind, lissom!!
\says{G3} Nej, jeg lukker dig ikke ind!
\says{U} \act{Pust} \act{Prust}!

\scene Fortælleren forsøger at redde sit skind
\says{F} Og de levede lykkeligt til deres dages ende...
\scene Fortælleren skynder sig at løbe sin vej

\scene Rødhætte banker på den låste dør
\says{RH} Hallåå Bedste, luk mig ind, lissom!!
\says{G3} Nej, jeg lukker dig ikke ind!
\says{U} \act{Pust} \act{Prust}!

\scene Rødhætte banker på den låste dør
\says{RH} Hallåå Bedste, luk mig ind, lissom!!
\says{G3} Nej, jeg lukker dig ikke ind!
\says{U} \act{Pust} \act{Prust}!

\scene{Note til TeXnikken: Skru ned for U og RH når lys går ned.}

\end{sketch}
\end{document}
