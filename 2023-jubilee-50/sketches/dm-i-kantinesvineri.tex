% Oprindeligt filnavn: dm_i_kantinesvineri.tex
\documentclass[a4paper,11pt]{article}

\usepackage{revy}
\usepackage[utf8]{inputenc}
\usepackage[T1]{fontenc}
\usepackage[danish]{babel}


\revyname{DIKUrevyens 50 års jubilæum}
\revyyear{(1989)}
\version{1.0}
\eta{$n$ minutter}
\status{Færdig}

\title{DM i kantinesvineri}
\author{?}

\begin{document}
\maketitle

\begin{roles}
\role{K1}[Sebbe] Kommentator
\role{K2}[Elbo] Kommentator
\role{F1}[Marcus] Førsteårsstuderende 1
\role{F2}[Sejer] Førsteårsstuderende 2
\role{A1}[Niels] Andendelsstuderende 1
\role{A2}[Brandt] Andendelsstuderende 2
\role{D}[Kim] Dommer
\role{R}[Torben M] Rengøringspersonale
\role{HD}[Bjørn] HD'er
\role{N}[Mathias] Ninja til at gøre rent
\role{X}[Simon] Instruktør
\end{roles}

\begin{props}
    \prop{Dommerfløjte}[Kim]
    \prop{Sølvfade}
    \prop{4 fodboldtrøjer (2 par af 2 ens, med holdnavne på)}
\end{props}


\begin{sketch}

  \says{K1} Og vi byder velkommen til de åbne mesterskaber i
  kantinesvineri, der i år afholdes her i de nærmest ideelle
  omgivelser i DIKU's kantine.

  \scene{To førsteårsstuderende kommer på scenen med lidt tallerkner,
    tomme kiks, teposer, mm.}

    \says{K2} Deltagerne er så småt ved at indfinde sig. På vores højre side
    har vi et hold bestående af førsteårs datalogistuderende, der overraskende
    nok har formået at kæmpe sig vej frem til finalen. Trods deres ringe
    erfaring i international sammenhæng lægger de en forbavsende entusiasme for
    dagen.  På vej til finalen har de slået så stærke mandskaber som
    Vestforbrændingen og Kommune-Kemi - og hvem husker ikke den forrygende
    semifinalekamp mod det blandede hold fra politiets uropatrulje med udvalgte
    BZ'ere, hvor de førsteårsstuderende måtte ud i forlænget svinetid for at få
    kampen afgjort.

  \scene{To andendelsstuderende kommer på scenen medbringende
    alm. svineri (tallerkner, cigaretter, tom ølflaske, rødbedemad,
    mm.)}

  \says{K1} På den anden side har vi sidste års nummer
  fire: Et blandet hold af andendelsstuderende, der i deres semifinale
  tog revanche fra nederlaget i sidste års kvartfinale til Kemisk Værk
  Køge - kampen blev stoppet før tid på teknisk knock-out da de
  andendels-studerendes fiskefrikadeller brændte på netop under
  røg-alarmen og et hold røg-dykkere med skumsprøjter måtte tilkaldes
  før de andendels-studerende kunne få deres velfortjente sejr.

  \says{K2} Og kampen er så småt ved at gå i gang - dommeren kigger på
  sit ur

  \says{K1}[afbryder] Dommeren er for øvrigt en nuværende forelæser på
  datalogisk institut. Han var med på det vindende hold bestående af
  datalogi-forelæsere i '88, men har vundet tre år i træk og er derfor udelukket
  i år.

  \says{K2} For øvrigt en skam, da der virkelig var tale om et
  stilskabende og nytænkende hold indenfor kantinesvinerisporten -
  hvem husker ikke den sojakage de bagte i finalen i '86 mod de
  københavnske kloakarbejdere...

  \scene{Dommeren fløjter og afbryder K2 - deltagerne rumsterer med deres ting.}

  \says{K1} Begge holds deltagere lægger forsigtigt ud - de venter
  afventende på modpartens første træk.  De krummer diskret og
  rutinepræget - eller skyldes det blot gammel vane... der kommer
  kampens første træk - de førsteårsstuderende skubber forsigtigt en
  tom pakke kiks på gulvet

  \says{K2} Ja, de andendelsstuderende har ikke noget
  modtræk... Jo... jo! Anføreren stiller en tom ølflaske om bag sit ene
  stoleben.

  \says{K1} Den tomme ølflaske er et af de klassiske elementer i
  kantinesvinerisporten, og havde det ikke været en offentlig
  turnering kunne den have blevet stående i månedsvis.

  \says{K2} Ja, årevis! Hvem husker ikke den Star-flaske der blev fundet sidste år med Peter
  Naur's afslørende fingeraftryk på - den blev dateret ved hjælp af
  kulstof-14 metoden til at være en 1973'er.  Den var i en sådan grad groet
  ind i miljøet at den flyttede med til disse nye omgivelser.

  \says{K1} De unge håb har forladt deres plads, tilbage står en stabel tallerkner og en udsøgt
  samling gamle æbleskrog - iblandt disse skelner jeg et Cox Orange
  fra efteråret 1971 netop indkøbt i HCØ's kantine.

  \says{K2} Meeen... Hvad ser jeg ovre på pensionisternes plads... en
  {\em Kernighan \& Ritchie}

  \says{K1}[til K2] Den slags smudslitteratur kan dommeren
  da umuligt tillade...

  \says{K2} Og nu, NU ser dommeren det... han fløjter... de andendelsstuderende
  idømmes Dat 1 kursusbog 7...

  \says{K1} Den dom er lige så uforståelig som bogen.

  \says{K2} Men hvem er dog det der dukker op på banen?  En HD'er.

  \scene{Dommeren fløjter.}

  \says{K2}[holder øje med HD'eren] Denne sportsgren er trods alt ikke for professionelle
  svinere.

  \scene{Dommeren idømmer kursusbog 5.}

  \says{K2} Og ganske rigtigt - han idømmes kursusbog 5 - hård men fortjent!

  \says{K1}[har holdt øje med de andendelsstuderende] Mens dommeren var distraheret har de andendelsstuderende tilkæmpet
  sig adgang til det aflåste skab med ekstra kopper, og nu skodder de i dem.  Et smukt og aggressivt træk!

  \says{K2} De førsteårs-studerende er desperate. Det er tid til det helt store
  skyts... Hvad er det?? De har fundet en mælk... fra brugerkøleskabet!!  Og de hælder
  den ned i koppen! \act{hælder grøn, klumpet mælk i kaffen}

  \says{K1} Umenneskeligt.

  \says{K1} Kampen står hermed lige på point...

  \says{K2} ... og nu rykker de andendelsstuderende...

  \scene{En andendelsstuderende åbner et "`tomt"' syltetøjsglas, den
    anden smider en "`rødbedemad"' på gulvet}

  \says{K1} ... det er vist bananfluer det der, det må være længe siden
  holdet har været på DIKU, dem har vi jo i forvejen.

  \says{K2} Men i øvrigt en smuk detalje med rødbedemaden!!!

  \scene{Førsteårsstuderende vader på maden.}

  \says{K1} ...uha !  De førsteårsstuderende svarer straks igen...

  \says{K2} Bemærk med hvilken elegance rødbedemaden tværes ud over og ned i gulvet.

  \scene{Rødmedemadsvaderen overfaldes af rengøringspersonalet, der slår ham oven i
    hovedet med kost}

  \says{K1} ... Men hvad sker der... uha, det er rengøringspersonalet!! De studerende flygter...

  \says{K2} Dommeren fløjter kampen af og trækker sig tilbage for at votere...

  \says{K1} Men vent, rengøringen stopper ham og stiller ham til ansvar...

  \says{K2} Ja, det er jo ham der har booket kantinen.

  \says{K1} Og dermed, mod alle odds, hiver han sin fjerde sejr hjem!

  \says{K2} Og tillykke til årets vinder.
\end{sketch}
\end{document}
