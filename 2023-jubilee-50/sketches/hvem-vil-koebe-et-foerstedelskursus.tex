% Oprindeligt filnavn: salgafkurser.tex
\documentclass[a4paper,11pt]{article}

\usepackage{revy}
\usepackage[utf8]{inputenc}
\usepackage[T1]{fontenc}
\usepackage[danish]{babel}


\revyname{DIKUrevyens 50 års jubilæum}
\revyyear{(1978)}
\version{1.0}
\eta{$n$ minutter}
\status{Færdig}

\title{Hvem vil købe et førstedelskursus?}
\author{EM, JBC}

\begin{document}
\maketitle

\begin{roles}
\role{S}[Kim] Sælger
\role{L1}[Torben M] Lærer
\role{L2}[Brandt] Lærer
\role{X}[Troels] Instruktør
\end{roles}

\begin{props}
  \prop{2 kasser øl}[]
  \prop{En flaske Gammel Dansk}[]
  \prop{AV: et dias}[Troels]
  \prop{AV: billeder til den dårlige hundejoke}[Troels]
\end{props}


\begin{sketch}

  \scene{Et bord med vore lærere omkring.  En sælger i passende
    mundering er ved at indlede sin salgstale.}

  \says{S} Før jeg begynder på en detaljeret gennemfang at mit
  compagnys produkter, vil jeg godt sige, at jeg -- og hele mit firma
  -- er meget happy for at have fået lov til at komme her til
  datalogisk institut, og jeg er proud over, at hele lærerforsamlingen
  vil lisne til, hvad jeg har at sige.

  Vores firma lever af at sælge 1. delskurser til vore kunder.  Ikke
  blot kurserne, vi er også leveringsdygtige i studiekoordineringer,
  bifag (de har måske set nogle af vores reklamer) forelæsninger, ja
  alt til faget hørende.  Efter denne lille indledning, vil jeg så
  sige velkommen til vores workshop.

\says{L1} Voksok?  Hvad betyder det?

\says{L2} Det er nok en butik, hvor man sælger arbejdere.

\says{S} Dagens emne er fornyelse af undervisningsmateriale på Deres
undergraduate kurser.

\says{L2} Er vi nu også undergraduerede?  Jeg troede kun vi var underbemandede.

\says{S} Vores organisation har lavet et program specielt sammensat
til deres behov.  Jeg kan belyse det med en case.

\scene{Planche:}

\begin{verbatim}
case kursus of
  dat0: instroduction to computing
  dat1: advanced computing
  dat2: management information systems
esac
\end{verbatim}

\says{S} Som det fremgår, lægger vi op til en radikal sanering af
deres virksomhed.

\says{L1} Det kan ikke blive før i '85, for før er mine transparenter
ikke slidt op.

\says{L2} Og jeg har allerede lavet eksamensopgaverne til næste år; jeg
tager blot dem fra i år og retter fejlene.

\says{L1} På mit kursus har jeg en mikrofilm af en afskrift af en kopi af
original til de reviderede autoforelæsninger.  Så der kan vi jo ikke
lave meget om.

\says{S}[fortsætter uforstyrret] Som svar på et udpræget consumer survey har vi
endvidere gjort vores product solution mere jordnær og dagligdags.  Hunde!  Vi
har udvalgt en hund til at illustrere hvert emne: engelsk pointer til pointers,
irsk setter til oversættere, og read-ehvalp-paw-print-loop til fortolkere.

\scene{Evt. AV: Billeder på OverTeX.}

\says{L2} Jeg er hunderæd for, at disse her planer er lidt for vov'ede.

\says{L1} Desuden kan vi da ikke anvende en så løs og udenlandsk terminologi.

\scene{Spredt mumlen.  Lærerne synes ikke om oplægget.}

\says{S} Jeg kan forstå på d'personers holdning, at De ville
foretrække en prøve på vores gamle danske terminologi. \act{Trækker en
  flaske Gammel Dansk op, L1 og L2 er pludselig interesserede og
  trækker snapseglas frem fra inderlommerne. S skænker.}

\says{L2} Det var måske ikke så tosset at indøve nogle bottom-up varianter.

\scene{L1 og L2 skåler og bunder.}

\says{S} Så vil det måske også interessere Dem at se vores
undervisningsmateriale til illustration af matricer? \act{Haler en
  kasse øl frem.}

\says{L2} Den matrix giver en god forståelse i flydende regning. \act{Tager en øl op.}

\says{L1} Bare det nu ikke er en tynd matrix? \act{Tager også en øl
  op.} Kan du se om flaskene er tomme?

\says{L2} Det må gøres til genstand (!) for en algoritmeanalyse.

\says{L1} Det skulle ikke være svært at finde flaskehalsen.
\act{Knapper flaskerne op.}

\scene{L1 og L2 skåler og drikker.}

\says{S} Ja, ja, men den kan meget mere.  Her kan vi også øve os i at
fjerne etiketter \act{demonstrerer på en ølflaske} og programmøren kan
jo dårligt hoppe nogen steder hen uden etiketter.

\says{L2} Glimrende, jeg kan allerede nu se resultater af
undervisningen.  Jeg er for eksempel ikke nu i stand til at hoppe
nogen steder.

\says{S} Ofte vil det være nødvendigt med dobbelt buffer for at få
et tilstrækkeligt hurtigt gennemløb i systemet.  \act{Haler endnu en
  kasse øl frem.}

\says{L1} Han er nu slet ikke så tosset, ham den sælger. Se nu hvordan
jeg kan sortere de røde fra de grønne hutigere end du kan søge
$O(n\log(n))$. \act{Flytter rundt på flaskerne i kassen.}

\says{L2} \act{Har en øl i hver hånd.} Så kan du måske samtidig lære
at kende forskel på en venstre Tuborg og en højre Tuborg?

\says{S} Mens d'personer lader indtrykkene synke til bunds, vil jeg
tillade mig en hurtig og koncis gennemgang af rækkereduktion.

\scene{S tager favnen fuld af flasker.  Lys ned hurtigt.}

\end{sketch}
\end{document}

%%% Local Variables:
%%% mode: latex
%%% TeX-master: t
%%% End:
