\documentclass[a4paper,11pt]{article}

\usepackage{revy}
\usepackage[utf8]{inputenc}
\usepackage[T1]{fontenc}
\usepackage[danish]{babel}

\revyname{DIKUrevyens 50 års jubilæum}
\revyyear{(2002)}
\version{1.0}
\eta{2 min.}
\status{Færdig}

\title{WAP hastighed}
\author{Jørgen Elgaard Larsen}

\begin{document}
\maketitle

\begin{roles}
\role{K}[Elbo] Kaptajn
\role{F}[Sebbe] Fændrik
\role{X}[Troels] Instruktør
\end{roles}

\begin{sketch}

  \scene{F og K står til venstre (set fra publikum) i midtergangen,
    foran døren. Spot på. Star Trek-intromelodi.}

  \says{K} Fændrik!

  \says{F} Javel hr. kaptajn!

  \says{K} Tag os til WAP-hastighed!

\says{F} Javel!

\scene K og F går ekstremt overdrevet langsomt ud af scenen

\scene \textbf{teknik}: F og K skal gå meget langsomt på scenen. Dette
vil medføre råben og skrigen. Så snart råben og skrigen tager af, så
skal lyset ned og personerne forsvinder ud bag tæppet. Hvis teknikken
ikke selv vil afgøre dette, så vil instruktøren eller en af bosserne
give tegn til tæppemesteren.

\end{sketch}
\end{document}

%%% Local Variables:
%%% mode: latex
%%% TeX-master: t
%%% End:
