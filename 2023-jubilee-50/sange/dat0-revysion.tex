% Oprindeligt filnavn: dat0Revision.tex
\documentclass[a4paper,11pt]{article}

\usepackage{revy}
\usepackage[utf8]{inputenc}
\usepackage[T1]{fontenc}
\usepackage[danish]{babel}

\revyname{DIKUrevyens 50 års jubilæum}
\revyyear{(1994)}
\version{0.95}
\eta{$n$ minutter}
\status{Færdig}

\title{Dat0-revysion}
\author{Martin Koch m.fl.}
\melody{Folk og røvere i Kardemommeby}

\begin{document}
\maketitle

\begin{roles}
  \role{L1}[JoJo] Underviser i Common Lisp
  \role{L2}[Lukas] Underviser i ML
  \role{L3}[Sejer] Underviser i Emerald
  \role{S}[Brandt] Stakkels studerende til sceneshow
  \role{X}[Niels] Instruktør
\end{roles}

\begin{song}
\scene{(Koncept for sceneshow: Når en underviser ikke synger, giver vedkommende overdimensionerede bøger til S.  S får med tiden for mange bøger og kollapser helt under dem.)}

\scene{L1, L2 og L3 går løbende ind på scenen i takt med at de synger deres dele.  Til sidst står de samlet på scenen.}

\sings{L1}%
Vi reviderer det dat0,
I alle sammen kender. \act{kig direkte ud på publikum}
\sings{L2}%
Og no'en vil sikkert vær' imod,
men vi har mange venner.
\sings{L3}%
For vi er nemlig overklog',
vi køber ej en lærebog.
\sings{Alle}%
Vi sparer de penge og laver den selv
båd' til Emerald, Lisp og så S-M-L.

\scene{S kommer ind med en tegneserie og er glad.}

\sings{L1}%
Dat0'erne har længe haft
det alt for nemt i starten.
\sings{L2}%
Vi syn's Pascal det er for let,
vi ta'r det af plakaten.
\sings{L3}%
Og re-gel-sty-ret ind-læs-ning
er yt, det er en væm'li' ting.
\sings{Alle}%
Til håndkøring vil vi nu sige farvel
og kør' Emerald, Lisp og så S-M-L.

\scene{Under næste vers går L2 hen for at fange S, og L3 går hen for at finde Common Lisp-bogen.  L2 skubber S hen til L3's fundne bog, og S begynder interesseret at læse i den i stedet for sin tegneserie.}

\sings{L1}%
Vi lægger ud med Common Lisp
og grafisk grænseflade.
Dat0'erne de skriger gisp!
Men vi er ligeglade.
Til prøven kræves et bevis,
ved induktion --- naturligvis.
Med CONS og med QUOTE og med CDR og CAR
så Dat0'erne ikke en chance har.

\scene{Under næste vers går L1 ud for at finde ML-bogen, som er i et andet område af scenen.  L3 får fjernet Common Lisp-bogen ud af S' hænder og skubber S hen til L1.  L1 finder til sidst bogen og giver den direkte til S, men denne bog er en del større og tungere, hvilket overrasker S.  S kan dog stadig kigge i/på den.}

\sings{L2}%
Et typet sprog det skal der til,
for det er godt at kende.
Så de må lære om ML,
hvor er vi ellers henne?
Hvis de har typedisciplin,
bli'r koden ikke helt til grin.
Og mønstre, exceptions og polymorfi
det er noget Dat0'erne vil ku' li'.

\scene{Under næste vers går L2 tilbage ud og henter Emerald-bogen fra et tredje gemt sted bag et tæppe el.lign.  L1 går hen til S og skubber ML-bogen ud af S' hænder.  L2 overrækker Emerald-bogen til S, og S vælter med bogen over sig.}

\sings{L3}%
Dat0'erne skal også kunne
tænke i objekter.
Det' vigtigt at de lærer om det
uden brug af hægter.
Så valget faldt på Emerald.
Jeg kender det til hudløshed.
Nu håber jeg bare det ikke går ned,
for så vil de ik' la' mig vær' i fred.

\scene{L1, L2 og L3 synger deres sidste linjer mens S ligger på scenen og synligt har det skidt mens han prøver at slippe ud fra den tunge bog.}

\sings{L1}%
En fordel har vor revision,
det må vi ikke glemme.

\sings{L2}%
PC'erne skal på pension,
og det er bedst det samme.

\sings{L3}%
For vi har altid ønsket os,
at DIKU ville droppe DOS.

\sings{Alle}%
Der spilles for meget på vores PC'r
mens vor UNIX forhindrer at dette sker.
\end{song}

\end{document}
