\documentclass[a4paper,11pt]{article}

\usepackage{revy}
\usepackage[utf8]{inputenc}
\usepackage[T1]{fontenc}
\usepackage[danish]{babel}


\revyname{DIKUrevyens 50 års jubilæum}
\revyyear{(2016)}
\version{1.1}
\eta{$4$ minutter}
\status{Shaky koreografi}

\title{Sangen om Signe}
\author{Sebastian Paaske Tørholm}
\melody{Barry Manilow: ``Copa Cabana''}
% Originalversionen: https://www.youtube.com/watch?v=uZejMyIrghc
% Det cover jeg originalt skrev på: https://open.spotify.com/track/1suik7QZIdVrJIVfxLBOxC / https://www.youtube.com/watch?v=MDHiyZwc8yQ
% Den live-udgave jeg endte med at skrive på: http://open.spotify.com/track/0C9lJEO06RetDEOvF3Qc3r
% Rimelig straight lip-synched live-udgave: https://www.youtube.com/watch?v=i2dlXZo348w
% En anden god live-udgave: https://www.youtube.com/watch?v=cDzV0xeV5pA
% Jeg synes ikke om denne live-udgave: https://www.youtube.com/watch?v=HLaSDLM5Ehc
% Faktisk endte jeg med at synes bedre om live-udgaven end den jeg startede med at skrive på. Mere gang i den.

\begin{document}
\maketitle

\begin{roles}
\role{S}[Lukas] Sanger
\role{HD}[Amira] Hot date
\role{T}[Arinbjørn]  Tjener
\role{G1}[Mads] Restaurantgæst 1
\role{G2}[JoJo] Restaurantgæst 2
\role{G3}[Johnson] Restaurantgæst 3
\role{G4}[Sejer] Restaurantgæst 4
\role{X}[Sejer] Instruktør
\end{roles}

\begin{props}
  \prop{Stole til date + andre gæter}
  \prop{Et bord til date + evt. til andre gæster}
  \prop{Duge til alle borde}
  \prop{To glas}
  \prop{En kande med rigtigt vand i}
  \prop{To maracas med ris i}
  \prop{Tre store colaflasker (Coca, Pepsi og Harboe)}
  \prop{Netto-pose med Harboe Cola}
\end{props}

\begin{song}
\scene{Gæster sidder på restauranten og spiser. T går rundt og servicerer dem.}
\scene{S og HD kommer ind sammen. S trækker stolen ud for HD og begynder så at synge.}

\sings{S}%
        Hun hedder Signe
        fandt hende på Tinder
        Vi swipede begge samme vej
        og nu skal hun ud med mig

        \scene{S sætter sig til rette ved siden af HD}

\sings{S}%
        Vi skal i byen
        ja, ud at spise
        Tjeneren han stod ganske klar \act{T kommer ind på scenen fra siden}
        spurg't sku' vi ha' drikkevar'?


\sings{S}[forklarer til publikum]
        Men valget det var gjor'
        Tvivler ej, ikke spor
        For når jeg skal slukke tørsten
        (Er) der' kun ét der dur!

\scene{T rækker indbydende S en maracas, Coca-Cola-flasken kommer frem, og S og T danser latin-dans frem og tilbage på scenen}

\sings{S}[omkv]%
        Ja en cola (cola)
        en Coca-Cola (en Coca-Cola)
        for alt hvad jeg drikker er soda- (<{\ldots}>)
        -vand, en cola (cola)
        en Coca-Cola
        Den slukker tørsten når tørsten er størst, den
        er en cola
        {\ldots}den var udsolgt{\ldots}
        \act{S surmuler og sætter sig igen}
        (Åh da! Ikke mer' cola{\ldots})

\scene{T forlader scenen for at hente en kande vand}

\scene{S samler fatningen og liver op igen. I dette vers danser S muntert og synger midt på scenen. S synger om HD, men ignorerer samtlige af HDs tilnærmelser fra bordet}

\sings{S} Men hend' der Signe
        hun er da dejlig
        tænk dog at hun vil ud med mig
        hun er da godt nok en steg
        Og mens jeg nyder
        synet af Signe
        Står tjeneren foran mig, fortabt \act{T kommer ind og hælder vand i HDs glas}
        kigger på mig og får sagt
        ``Vil du så bar' ha' vand?''
        ``Nej! Hvad med\ldots Pepsi, mand?'' \act{Får en god idé}

\scene{S rejser sig igen og taler til publikum}

\sings{S}%
        Det er ikke en Coca-Cola
        Men den kan gå an!

        \scene{S rejser sig og danser med T}

\sings{S}[omkv]%
        Ja en cola (cola)
        en Pepsi Cola (en Pepsi Cola)
        for alt hvad jeg drikker er soda- (<{\ldots}>)
        -vand, en cola (cola)
        en Pepsi Cola
        Den slukker tørsten når tørsten er størst, den
        er en cola
        {\ldots}den var udsolgt{\ldots}
        (Åh da! Ikke mer' cola{\ldots})
        (Åh da! Ikke mer' cola{\ldots})

        \scene{Langt mellemspil. G1-G4 rejser sig og deltager i dansen, som bliver mere avanceret. Til sidst laver S en \emph{pose}, hvorefter HD kaster sit vandglas udover S, og stamper vredt ud ad bagtæppet.}
        \scene{S skænker det IKKE en tanke.}


        \scene{Efter mellemspillet sætter S sig til rette ved bordet. S er sulten!}

\sings{S}%
        Nå, men hend' Signe
        er da charmer'nde
        ja, hun kan få mig til at grin'
        og hendes kjole er da fin \act{gestikulerer over mod den tomme plads}
        Men jeg er ved at
        være lidt sulten \act{T kommer ind}

\sings{S}[til T]%
        Hør tjener, kan du sige mig (sunget mar')
        hvornår maden dog er klar?

\sings{T}%
        Nu må du slappe af
        Har spurgt hva' du vil ha'
        Du kan bare ta' i Netto
        der har de col-a

        \scene{En Netto-pose fyldt med Harboe-cola kommer ind fra sidetæöpet og uddeles til gæster og publikum.}
        \scene{S, T og G1-G4 danser sammen}

\sings{S+T}%
        Ja en cola (cola)
        en Harboe Cola (en Harboe Cola)
        for alt hvad jeg drikker er soda- (<{\ldots}>)
        -vand, en cola (cola)
        en Harboe Cola
        Den slukker tørsten når tørsten er størst, den
        er en cola
        -- BREAK I MUSIKKEN -- \act{S tager en tår af sodavanden}
        {\ldots}den var ikk' god
        {\ldots}den var ikk' god

        {\ldots}den var ikk' gooood

\scene{Lys ned.}
\end{song}

\end{document}
