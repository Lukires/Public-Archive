\documentclass[a4paper,11pt]{article}

\usepackage{revy}
\usepackage[utf8]{inputenc}
\usepackage[T1]{fontenc}
\usepackage[danish]{babel}

\revyname{DIKUrevyens 50 års jubilæum}
\revyyear{(2017)}
\version{1.0}
\eta{$4.5$ minutter}
\status{Færdig}

\title{Rus laver toast}
\author{Sebastian Paaske Tørholm}
\melody{Ray Dee Ohh: ``Jeg Vil La' Lyset Brænde''}
% https://open.spotify.com/track/1rRcFwE1ydO2Y5jSmk1UIs

\begin{document}
\maketitle

\begin{roles}
\role{R}[Eva] Rus
\role{K1}[Amanda] Kor (omkvæd)
\role{K2}[Sejer] Kor (omkvæd)
\role{D}[Mathias] Toastdanser
\role{T}[Torben] Torben Olai Milhøj
\role{X}[Niels] Instruktør
\role{N}[] Ninja til at sætte flammer på D
\role{Kg}[Morten] Koreograf
\end{roles}

\begin{song}
\scene{Lys op.  R står på scenen og musikken går i gang.  Under næste vers går R ud og henter D, som smådanser lidt.}

\sings{R}%
Maven rumler lidt igen
Tror jeg laver mig en toast
Laver den (da) telefonen den ringer
Pluds'lig lugter her lidt brændt\ldots

\scene{D får sat flammer på sig af N.  D danser rundt som man gør når der er ild i en.}

\sings{R}%
Jeg vil la' toasten brænde, la' DIKU fyldes med røg
Så de ikke ved jeg har glemt den, så de ikke anklager mig
Jeg vil la' toasten brænde, la' DIKU fyldes med røg
Peger fingre ned mod rus-enden, skyder skylden langt væk fra mig

\scene{Torben kommer ind.  D laver en lille dans kun for Torben, så Torben rigtig kan dufte til D.}

\sings{R}%
Torben kigger hen på mig
``Er det dig der laver toast?''
(Pa)nikken rammer, ryster stille på hoved't
Skammer mig mens han går væk

\scene{Torben forlader meget skuffet scenen.}

\scene{I næste omkvæd embracer D sine flammer og vil ikke længere slukkes, men vil bare sprede flammerne til alle.}

\sings{R}%
Jeg vil la' toasten brænde, la' DIKU fyldes med røg
Så de ikke ved jeg har glemt den, så de ikke anklager mig
Jeg vil la' toasten brænde, la' DIKU fyldes med røg
Peger fingre ned mod rus-enden, skyder skylden langt væk fra mig

\scene{Resten af revyen kommer stille og roligt ind og prøver at finde kilden til lugten.  Folk går rundt og kigger forvirret rundt, og R skjuler sig helt ovre ved bandscenen.  Folk begynder at vise hinanden rene toastskiver for at bevise at det i hvert fald ikke er dem der har brændt nogen toast på.  Folk begynder også at vise deres toastskiver til R, men R vifter dem bare væk.  Folk må også gerne tygge lidt på deres toastskiver.  Til sidst bliver R nødt til at indrømme sin skyld og bryder ud i sang igen midt på scenen.  Alle folk vifter i takt med deres toastskiver.}

\sings{R}%
For jeg lod toasten brænde, lod DIKU fyldes med røg
Men nu vil de finde en synder, og de er på sporet af mig
Ja jeg lod toasten brænde, lod DIKU fyldes med røg
Men nu ved de at jeg har glemt den, venner hav nu nåde for mig

\scene{R falder ned på knæ og får frataget sin mikrofon.  D giver sine flammer til R.}

\sings{Alle}%
Vi vil la' russen brænde, la' DIKU fyldes med røg
Vi vil statuer' et eksempel, vi vil vise russen på vej
Vi vil la' russen brænde, la' DIKU fyldes med røg
Indtil du forstår at vi mener, toasteren er ikke til dig
\end{song}

\end{document}
