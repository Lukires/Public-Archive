\documentclass[a4paper,11pt]{article}

\usepackage{revy}
\usepackage[utf8]{inputenc}
\usepackage[T1]{fontenc}
\usepackage[danish]{babel}

\revyname{DIKUrevy}
\revyyear{2023}
\version{1.0}
\eta{Estimeret tid}
\status{Work in progress/Almost Done/Done}

\title{TED-Talk: Sortering i konstant tid}
\author{Bertil}

\begin{document}
\maketitle

\begin{roles}
\role{E}[Eva] as herself
\role{B}[Bertil] as himself
\role{T} [Nicholas] Toaster
\role{S} [Kristoffer] Stol
\role{V} [Markus] Ven (ølautomaten)
\role{X}[Elbo] Instruktør
\role{N}[Lotte]


\end{roles} 

\section*{Noter}
Det refereres både til en tavle og til AV i sketchen, det skal nok laves så det kun er et AV powerpoint show, siden sidste års revy viste os at tavler er svære at bære ind og ud af scenen.

\begin{props}
\prop{Tavle F skal skrive på ? }
\prop{AV Slideshow}
% add props as needed
\end{props}

% sketch titel: En Smartere K@ntine

\begin{sketch}
\scene{lys op. T, S, og V starter inde på scenen.}

% introsketch ide: Smart Kantineappliances.

\says{V} Nøjjj, tænk at DIKU endelig har betalt for at alle os kantineting er blevet smart!

\says{T} Ja, nu kan jeg sende dig en notifikation så snart din toast er brændt på! 

\says{V} Og jeg kan udtrykke meget tydeligere hvorfor jeg ikke vil give dig en øl!

\says{S} Og jeg kan sige 'av' når du sætter dig på mig !

\says{T} Hmmm, altså ølautomat, jeg forstår godt hvorfor du er blevet smart. Og jeg forstår også godt hvorfor jeg er blevet smart. Men stol, hvorfor er du egentlig blevet smart?

\says{S} Jo nu skal du høre ...

\scene{Bandet begynder på 'Havet er skønt' mens stol siger replikken ovenfor}

\says{S} Os stole bli'r set som dumme
\says{S} i forhold til andre ting
\says{S} men borde er meget dummer'

\scene{todo: skriv et par linjer mere til stolen, så der er mere at synge mens E afbryder}

\scene{I tredje linje kommer E på scenen og råber Bertil, først uden mikrofon og tager så backup mikrofon og...)}

\says{E} BERTIL

\says{E} Jeg tror sgu jeg har fundet ud af det!

% fyld mere præcist ud hvad replikkerne er her
\says{B} Vilt brormand. Jeg kommer ned med det samme!

\says{E} Hey i der, jeg skal bruge scenen!
\scene{Ligger backup mic, går om og tager tavle fra ninja bag bagtæppe, og håndholdt.}

\scene{F begynder at skrive op på tavle}
\says{E} Lad mig forklare: givet et array A af størrelse n er det muligt at sortere A i konstant tid!
\says{B} Umuuuuuligt brooormaaaaand!
\says{E} Vi starter med at lave et nyt array af størrelsen \textbf{MAX\_SIZE} - altså den maksimale størrelse et array kan have - hvilket for eksempel er omkring 2 147 483 647 i Scratch. Dette nye array kalder vi A’, og alle dets værdier er nil.

\says{B} Neeeej hvaaaaad maaaand, de kan da ikke aaallesaaammen være nil!? Det er jo for vaaaanviiittiiiigt!?

\scene{AV viser A’ = \textbf{new nil array(MAX\_SIZE)}}

\says{E} Nu indsætter vi A i A’, så de første n værdier i A’ svarer til A.

\says{B} Neeej, hvor er det viiildt, mand!

\scene{AV viser A’[0:A.length] = A}
\says{E} Og nu sorterer vi A’ med for eksempel mergesort, der har en køretid på O(n log n), men da A’ er af størrelsen \textbf{MAX\_SIZE}, der er en konstant får vi O(\textbf{MAX\_SIZE } log \textbf{ MAX\_SIZE}) = O(1).

\says{B} Fuuuck maaand, hvor er det fucking geniaaaalt maaand! Så sorterer de bare med mergesort med n som en konstant???

\scene{AV viser mergesort(A’) og tilhørende matematik ved siden af}

\says{E}Vi kan nu hente A ved at tage de n første elementer af A’

\says{B}Nej, nej, neeeej!!! Hvor det fucking VILDT mand! Band, hører I det samme som mig!?

\scene{AV viser return A’[0:A.length]}

\says{E} Og dermed har vi sorteret A i konstant tid!

\says{B} BWA-... 
\scene{R abryder sig selv og bliver eftertænksom}
\says{B} Øhm, men kommer det ikke til at tage flere timer at køre algoritmen?
\says{E} Jo!
\scene{lys ned}
\end{sketch}

\end{document}
