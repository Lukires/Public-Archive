\documentclass[a4paper,11pt]{article}

\usepackage{revy}
\usepackage[utf8]{inputenc}
\usepackage[T1]{fontenc}
\usepackage[danish]{babel}

\revyname{DIKUrevy}
\revyyear{2007}
% HUSK AT OPDATERE VERSIONSNUMMER
\version{1.1}
\eta{$3.25$ minutter}
\status{Færdig}

\title{Amortiserede omkostninger ved kritiske studievalg}
\author{Uffe og Uffe Productions\texttrademark}

\begin{document}
\maketitle

\begin{roles}
\role{A}[Allan] Bitter studerende
\role{B}[Bo Ælling] Studerende
\end{roles}


\begin{props}
\prop{Datalogtøj}[Rekvisitgruppen] Placeholder 1 
\prop{Datalogtøj}[Rekvisitgruppen] Placeholder 2
\end{props}

\begin{sketch}
\scene{To studerende møder hinanden på en gang på DIKU.}

\says{A} Hej! Så er du ved at forberede dig til Maskinarkitektur?

\says{B} Ah, nej, ikke rigtigt\dots Jeg havde sgu lidt tænkt mig at droppe det.

\says{A} Hvad! Har du virkelig \emph{råd} til det?

\says{B} Øh, hvad snakker du om? Jeg kan jo bare tage det igen næste år. Det
koster jo ikke noget.

\says{A} Haha! Hvor naivt! Har du tænkt på at hver gang du dropper et kursus
forlænger du dit studie med halvanden måned?

\says{B} Ja, det er da ikke noget problem!

\says{A} Jamen, det betyder jo at når du engang bliver færdig får du først
løn halvanden måned senere end du ellers ville have\dots

\says{B} Ja, men det er da heller ikke noget problem.

\says{A} Ej, hør nu her\dots når du lige er færdiguddannet får du\dots hvad? 35?
40.000 om måneden?

\says{B} \act{smiler} Ja, det lyder meget fint.

\says{A} Så halvanden måned koster dig altså 60.000 kr\dots

\says{B} Øh, okay, det havde jeg måske ikke lige tænkt på, men det er da
ikke noget stort problem\dots

\says{A} Nej, men havde du så overvejet at du i virkeligheden også har
halvanden måned mindre arbejde inden du går på pension?

\says{B} Øh? Hvad har det med sagen at gøre?

\says{A} Altså, vi må antage at din løn vil stige støt indtil du går på pension ikke?

\says{B} Jo, det lyder meget fint\dots

\says{A} Så når du er pensionmoden får du, hvad? 55? 60.000 om måneden?

\says{B} (bekymret) Ah, altså okay, jo, hvis jeg ikke er blevet forfremmet?

\says{A} Jaja, vi tager et worst-case scenarie her!

\says{B} (lettet) Okay okay!

\says{A} Så i virkeligheden snyder du dig selv for yderligere 90.000 kr!

\says{B} Wow\dots Det var alligevel en sjat, men i det store billede er det
ikke et rigtigt alvorligt problem\dots

\says{A} Ha! Så du mener at de 90.000 plus de 60.000\dots altså 150.000 kr\dots
at de ikke er noget værd?

\says{B} Nej\dots Altså men fordelt over 35 år er det jo småpenge\dots

\says{A} Nja\dots det siger du nu\dots MEN hvad nu hvis du investerede dem\dots de
ville trække renter over 35 år\dots aktier giver i gennemsnit 10% pro
anno\dots Det er næsten en fordobling på 7 år, så det er 5
fordoblinger\dots Altså du får pengene 32 gange igen\dots

\says{B} \act{gisp}

\says{A} Altså \dots regne regne\dots 5 mio kr!!!

\says{B} Øh\dots øh\dots

\says{A} Når de penge bliver investeret har de jo skabt drift i samfundet og
medvirket til den videre økonomiske vækst her i landet\dots så de 5 mio
investeret har måske afstedkommet 50 mio kr merværdi i det danske
samfund!!!

\says{B} \dots

\says{A} Det kunne f.eks. være bedre plejehjem eller hospitaler\dots eller
KRÆFTFORSKNING!

\says{B} Jamen\dots

\says{A} SÅ DET JEG SIGER\dots er\dots (pust)\dots hvis du dropper mit
kursus DØR DU FOR TIDLIGT AF KRÆFT!!!!

\says{B} Ej hør nu her\dots

\says{A} \dots jo jo\dots den er god nok\dots

\scene{pause}

%\says{B} Nå\dots men hvordan gik det så i Funktionsprogrammering?
\says{B} Nå\dots men hvordan går det så med specialet?

\says{A} \dots huh?

%\says{B} Har du ikke lige fået din rapportopgave tilbage?
\says{B} Skulle du ikke aflevere imorgen?

\says{A} Æhh\dots Jeg har fået en udsættelse

\says{B} Er det ikke din tredje?

\says{A} \dots jo\dots

%\says{B} Hvordan gik det?

%\says{A} \dots jeg\ldots øhm\dots dumpede

\says{B} HAR DU VIRKELIG RÅD TIL DET?!?!?!

(TFH)



\end{sketch}
\end{document}

%%% Local Variables: 
%%% mode: latex
%%% TeX-master: t
%%% End: 

