\documentclass[a4paper,11pt]{article}

\usepackage{revy}
\usepackage[utf8]{inputenc}
\usepackage[T1]{fontenc}
\usepackage[danish]{babel}

\revyname{DIKUrevy}
\revyyear{2007}
% HUSK AT OPDATERE VERSIONSNUMMER
\version{1.0}
\eta{$3$ minutter}
\status{Færdig}

\title{ÅbningsDRM}
\author{Uffe, Kristine og Daniel}

\begin{document}
\maketitle

\begin{roles}
\role{S1}[Munter] Sælger
\role{S2}[Bo Ælling] Sælger
\end{roles}

\begin{props}
\prop{Sælgeragtigt tøj}[Rekvisitgruppen] Bo Ælling
\prop{Sælgeragtigt tøj}[Munter] Munter
\prop{Klap til øje}[Rekvisitgruppen] Bo Ælling
\end{props}

  
\begin{sketch}
% Evt. kan der indflettes noget med licens for at have mulighed for at
% se revyen.
\says{S1} Velkommen til DIKU revy 2007.

\says{S2} At I er her i aften, er jo tegn på at I har truffet det rette
valg og købt en DRM-beriget DIKU revy 2007 oplevelse.

\says{S1} I år har vi virkelig lagt os i selen for at bringe jer det
ypperste indenfor revyer.

\says{S2} Takket være den nye, og forbedrede, DRM revy 2007, vil det
være muligt at tilbyde dig, brugeren, en totalt skræddersyet event.

\says{S1} Ja du vil opnå hidtil uhørte højder af tilfredsstillelse.

\says{S2} F.eks er det nu muligt at købe en revybillet med ret til at
se revyen med begge øjne. For den studerende på det skrabede budget,
vil det også være muligt at købe en billet der tillader at se revyen
på udelukkende højre \act{Sætter klap for venstre øje} \ldots{} eller venstre øje \act{sætter klap for højre øje}. Klap for øjet købes separat.

\says{S1} Maksimer din personlige oplevelse idag!

\says{S2} Husk at for at holde prisen på revybilletterne så lav som
mulig må revyen ikke kopieres. Det er derfor kun tilladt at
viderefortælle revyen i en forringet version.

\says{S1} Du skal derfor enten være påvirket af alkohol når du ser
revyen eller når du viderefortæller den, og helst begge dele.

% Kan evt. slettes
\says{S2} Alternativt kan du bare genfortælle Fysikrevy 2005-7.
% Kan evt. slettes
\says{S1} Ahr, nu er der heller ingen grund til at gå til yderligheder.

\says{S2} Som specialtilbud \dots

\says{S1} \dots Kun i dag \dots

\says{S2} \dots kan vi give jer ubegrænset mulighed for at klappe og
grine.

\says{S1} Det er fuldstændig uhørt, har vi virkelig råd til det?

\says{S2} JA! Med DRM revy 2007 er der ingen grænser for hvad vi kan.

\says{S1} For at bringe prisen helt i bund, har vi undladt at
inkludere åben ild og mobiltelefoni i revybilletten. 

\says{S2} Det er derfor strengt forbudt at have tændt mobiltelefon
eller benytte åben ild \ldots

\says{S1} \ldots alle versioner \ldots

\says{S2} \ldots under revyen.

\says{S1} Til gengæld har vi fået hele 6 \ldots

\says{S2}[Viser 6 fingre]

\says{S1} \ldots jeg gentager: 6, nødudgange med i prisen.

\says{S2} Arh, det lyder for godt til at være sandt, hvordan vil du få
plads til dem her i auditoriet.

\says{S1} Jo der er hele to oppe bag ved \act{peger}, to i siderne og to bag ved scenen.

\says{S2}[Jappende] Det er fantastisk hvad man kan med DRM. Men for at al denne DRMmagi kan
lykkes skal vi gøre opmærksom på at DIKU revy
2007 ikke er egnet for børn under 4 år, og at ledelsen ikke påtager
sig ansvar for eventuelle dødsfald undervejs i revyen. Alle ejendele
og deres ejermænd er indehaveren af billettens eget ansvar. Positive
og/eller negative oplevelser, erfaret i forbindelse med DIKU revy
2007 kan ikke på nogen måde refunderes og ligeledes kan revyens
deltagere og/eller ledelse, familie, kæledyr, husstøvmider holdes til ansvar.

\says{S1} Puha, den var godt nok lang \act{pause, venter på at
  publikum råber ``Det sagde hun også igår''}

\says{S1} Det minder mig om at der med DRM må råbes lige så højt I vil
\act{vent på at publikum råber lige så højt de vil}. Dog er vores
råbelicens kun dimensioneret til 20 samtidige råbere. Der er desuden
enkelte fraser som ikke kan håndteres af licensen.

\says{S2} Det gælder for eksempel alle sætninger indeholdende ordene:
``langsommere''\act{pause, vent på publikum}, ``Tøjet''\act{pause,
  vent på publikum} og ``Skål''\act{pause, vent på publikum}.

\says{S1} Hermed vil vi ønske jer en fantastisk DRM revy 2007. 

\says{S2} Og husk at der kan købes tillægslicenser i pausen.


\scene{Tæppe}

\end{sketch}
\end{document}

%%% Local Variables: 
%%% mode: latex
%%% TeX-master: t
%%% End: 

