\documentclass[a4paper,11pt]{article}

\usepackage{revy}
\usepackage[utf8]{inputenc}
\usepackage[T1]{fontenc}
\usepackage[danish]{babel}

\revyname{DIKUrevy}
\revyyear{2007}
% HUSK AT OPDATERE VERSIONSNUMMER
\version{1.2}
\eta{$5.5$ minutter}
\status{Færdig}

\title{Den grimme KUA-rus}
\author{Daniel, Uffe, Uffe, Lars, m.fl.}

\begin{document}
\maketitle

\begin{roles}
\role{F}[Allan] Fortæller
\role{C}[André] Censor
\role{G}[André] Grønlændere
\role{N}[André] Dataloger
\role{D}[Madss] Den grimme KUA-rus
\role{R}[Hanne] Humanistrus
\role{V}[Guldfisk] Rusvejleder
\role{P}[Guldfisk] Politibetjent
\role{H}[Bo Ælling] Humanist
\role{A}[Bo Ælling] Autonome
\end{roles}

\begin{props}
\prop{H.C.Andersen tøj}[allan] til Placeholder 1
\prop{høj hat}[allan] til Placeholder 1
\prop{Bog}[Rekvisitgruppen] til Placeholder 1
\prop{Lange røde strømper}[Rekvisitgruppen] Placeholder 2
\prop{svømmefødder}[andre] Placeholder 2
\prop{KUA-rus tøj}[madss] til Placeholder 3 
\prop{KUAner tøj}[hanne] til Placeholder 4
\prop{Elefanthue}[Rekvisitgruppen] til Placeholder 7
\prop{Politistav}[Rekvisitgruppen] til Placeholder 8
\end{props}

  
\begin{sketch}
\scene{Baggrund med KUA}
\says{F} Der var så deiligt ude på Landet; det var Sommer og der gik
Censor på sine lange, røde Been og snakkede ægyptisk, for det Sprog
havde han netop eksamineret i. Rundtom på Amager var der store
Bygninger, og midt i Bygningerne dybe Auditorier; jo, der var
rigtignok deiligt derude på Landet!

\says{F} Midt i Solskinnet var et kuld af nyst udklækkede Studenter. Endelig
var de optagne på Universitetet og hjemkomne fra deres rustur.

\says{R+H} "Blah blah!"

\says{F} Alle studenterne var Humanister og deres rusvejleder viste dem rundt.

\says{V} "Blah! Blah!"

\says{F} Og så ævlede de alt hvad de kunde, og såe til alle Sider.

\says{R} "Hvor dog universitetet er stort!"

\says{F} Thi de havde nu rigtignok ganske anderledes Plads, end da de var i Gymnasiet.

\says{V} "Troer I, KUA er hele universitetet! ja, det er det faktisk også! -
I ere her dog vel Allesammen! nei, jeg har ikke alle! der mangler een
på Listen; hvor længe skal det vare! nu er jeg snart kjed af det!".

\says{F} Endelig ankom den sidste Rus.

\says{D} "Blip! Blip!"

\says{F} Han var så stor og styg.

\says{V} "Det er da en forfærdelig stor Rus den! ingen af de andre see
sådanne ud! 

%\says{v} det skulde dog vel aldrig være en Jurist! nå, det skal
%vi snart komme efter!"

%\says{F} Og vejlederen fremdrog Alkoholiske lædskedrikke.

%\says{V} "Nei, det er ingen Jurist! see hvor hurtigt han bunder Breezers!
%det er min egen Rus! igrunden er han dog ganske kjøn, når man rigtig
%seer på ham! 
\says{V} ``Blah! blah! - kom nu med mig, så skal jeg præsentere
Jer for Instituttet, men hold Jer altid nær ved mig, at I ikke kommer
ud i noget empirisk efterviseligt!''

\says{F} Og det gjorde de: men de andre humanister rundt om såe på dem.

\says{H} "See så! nu skal vi have det Slæng til! ligesom vi ikke vare nok
alligevel! og fy, hvor den ene rus seer ud! ham ville vi ikke tåle!"

\says{F} Strax fløi der en humanist hen og bed ham i Nakken.

\says{V} "Lad ham være! han gjør jo Ingen noget!"

\says{H} "Ja, men han er for stor og for aparte! og så skal han nøfles!"

\says{F} Så den stakkels rus, som sidst var ankommet til KUA, og såe så
fæl ud, blev bidt, puffet og gjort Nar af.

\says{H+R} "Han er for stor!"

%\says{F} Og en Professor Emeritus, der var født i en kendt familie fra
%Holdbergs tid, pustede sig op som et Fartøi for fulde Seil, gik lige
%ind på ham og så pludrede han og blev ganske rød i Hovedet. 
\says{F} Den stakkels rus vidste hverken, hvor han turde ståe eller
gåe, han var så bedrøvet, fordi han såe så styg ud og var til Spot for 
hele KUA.

\says{F} Så løb ja nærmest fløj han henover lange-bro; de små Grøndlændere
på bænken foer forskrækket i Veiret;

\says{D} "det er fordi jeg er så styg,"

\says{F} Han lukkede Øinene, men løb alligevel afsted. Så kom han ud på
Jagtvei, hvor de unge vilde boede. Her gik han rundt hele Natten, han
var så træt og sorrigfuld.

\scene{Baggrund udskiftes til Jagtvej}

\says{F} Så kom der to Autonome og de vare ganske uvaskede.

\says{A} "Hør Kammerat! Du er så styg at jeg kan godt lide Dig! vil Du
drive med og være Voldspsykopat! Tæt herved er der nogle
søde velsignede Autonome, allesammen Frøkener, der kunne sige: fuck!
fuck! Du er istand til at gjøre din Lykke, så styg er Du!" - -

\says{P} "Klokken er mange og I er taget til fange"

\says{F} Begge Autonomerne blev lagt i Strips og ført bort. 

\says{F} Så blev vinter, og vinteren blev så kold, så kold; 
Men det vilde blive altfor bedrøveligt at fortælle al den Nød 
og Elendighed, han måtte prøve i den hårde Vinter 

%\says{F} - han sad på en
%Parkeringsplads i udkjanten af Nørrebroe, da Solen igjen begyndte at
%skinne varmt; Lærkerne sang - det var deiligt Forår.

\says{F} En aften, Solen igen skinnede på kalv og på kid (og marken
duftede af vår), kom der en heel Flok deilige store studerende ud fra
Caféen, Russen havde aldrig seet nogen så smukke; det var Dataloger,
de bredte deres prægtige Laptops ud og loggede ind på Servere i varmere Lande! 
\says{D} Blip! blip!
\says{F} De kodede så smukt så smukt og den store grimme Rus blev så
forunderlig tilmode; han besluttede sig at Følge efter dem.

\scene{Baggrund skiftes til DIKU}
\says{F} Før han ret vidste det, var han i et stort Rum, hvor Terminalerne
stode på Borde, hvor Netkablerne duftede og hang lige ned imod de
bugtede Kabelbakker! Og lige foran, ud af Bygningen, kom tre deilige,
store dataloger; de disputerede værdiorienteret programmering. Russen
kjendte de prægtige studerende og blev betagen af en forunderlig Sørgmodighed.

\says{D} "Jeg vil gå hen til dem, de kongelige studerende! og de vil
annullere min punkt.ku brugerprofil, fordi jeg, der er så styg, tør
nærme mig dem! men det er det samme! bedre at slettes af dem, end at
nappes af humanisterne og lide ondt om Vinteren!"

\says{F} Og han for hen imod de prægtige Dataloger.

\says{D} "Blip! Blip!"

\says{F} - men hvad såe han i DIKUS spejlblanke dobbeltdøre! han såe sit eget
Billed, men han var ikke længere en kluntet, sortgrå Rus, styg og
fæl, han var selv en Datalog.

\says{F} Han tænkte på, hvor han havde været forfulgt og forhånet, og
hørte nu Alle sige, at han var den deiligste af alle deilige
dataloger; og Netvjærket havde rigelig med kapacitet og DIKU's
systemer havde langt rigeligt af kørselstid, da indtastede han på
eengang sit Brugernavn og Løsen og af Hjertet jublede han:

\says{D} "Så megen Lykke drømte jeg ikke om, da jeg var den grimme KUA-rus!"

\says{F} Se det gjør ikke noget at have spildt tid på KUA, når blot man til
sidst viser sig at være Datalog!

\scene{Tæppet begynder at blive trukket for}

\says{F} Og den grimme KUA rus studerede lykkeligt på DIKU til sine dages ende. 

\scene{Tæppe}

\end{sketch}
\end{document}

%%% Local Variables: 
%%% mode: latex
%%% TeX-master: t
%%% End: 
