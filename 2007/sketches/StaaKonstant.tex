\documentclass[a4paper,11pt]{article}

\usepackage{revy}
\usepackage[utf8]{inputenc}
\usepackage[T1]{fontenc}
\usepackage[danish]{babel}

\revyname{DIKUrevy}
\revyyear{2007}
% HUSK AT OPDATERE VERSIONSNUMMER
\version{1.2$\iota$}
\eta{$6.6667$ minutter (+/- 0.0001)}
\status{Færdig}

\title{Stå Konstant}
\author{Uffe og Uffe Productions\texttrademark}

\begin{document}
\maketitle

\begin{roles}
\role{A}[Uffe] Dr. Uffe
\role{B}[Marvin] Dr. Uffe
\end{roles}

\begin{props}

\prop{Kittel}[Rekvisitgruppen] Uffe
\prop{Kittel}[Rekvisitgruppen] Marvin
\prop{Beskyttelsesbriller}[Rekvisitgruppen] Uffe
\prop{Beskyttelsesbriller}[Rekvisitgruppen] Marvin
\prop{Hjelm}[Rekvisitgruppen] Uffe
\prop{Hjelm}[Rekvisitgruppen] Marvin
\prop{Solbriller}[marvin] Marvin
\prop{Solbriller}[Rekvisitgruppen] Uffe
\prop{Sandaler}[Rekvisitgruppen] Uffe
\prop{Sandaler}[Rekvisitgruppen] Marvin
\prop{Shorts}[Rekvisitgruppen] Uffe
\prop{Shorts}[marvin] Marvin
\prop{Hvide tennisokker}[Rekvisitgruppen] Uffe
\prop{Hvide tenissokker}[Rekvisitgruppen] Marvin
\prop{Flipover}[Rekvisitgruppen] 
\prop{A1 ark til flipover}[Rekvisitgruppen] , DIKUs omlægning - før, nu og i fremtiden
\prop{1 A1 ark til flipover}[Rekvisitgruppen] 2,4,8,16 kyllinger
\prop{1 A1 ark til flipover}[Rekvisitgruppen] flowchart
\end{props}

  
\begin{sketch}
\scene{evt. foran tæppet med flipover}
\says{A} Goddag, mit navn er Dr. Uffe.

\says{B} Og jeg er Dr. Uffe.

\says{A} Vi er her i dag for at afprøve vores oplæg om "DIKUs omlægning -
før, nu og i fremtiden" som vi skal forelægge på mandag.

\says{B} Ja, vi har forsket seriøst de sidste 3 år og har valgt at bruge
DIKU revyens publikum som prøveklud da vi mente at jeres
intelligensniveau på nuværende tidspunkt var nogenlunde
sammenligneligt med Folketingets Uddannelsesudvalg.

\says{A} Hov, nu er der jo ingen grund til at svine publikum til!

\says{B} Oh, undskyld, ah, sådan var det jo heller ikke ment \dots

\says{A} Under alle omstændigheder... Regeringen har pålagt DIKU at omlægge
undervisningen da det er kommet frem at STÅ-produktionen her på stedet
er konstant!

\says{B} Ja. Altså, i 2003 blev der optaget 210 studerende som producerede
60 STÅ på første år. I 2004 blev der optaget 165 studerende, som også
producerede 60 STÅ i løbet af deres første år. Og i 2005 blev der
optaget 93 studerende, som igen producerede 60 STÅ på deres første år.

\says{A} Så, vi har altså en god videnskabelig basis for at fastlægge at
mængden af producerede STÅ er statistik uafhængig af antallet af
studerende.

\says{B} Vi er så blevet pålagt at komme med forslag til hvordan vi kan
udnytte dette for at få profitmaksimeret DIKU!

\says{A} Regeringens første udspil var at vi skar ned til 12 optagne. Dermed
ville de 60 STÅ blive til 5 STÅ per næse og dermed producere 12
kandidater på blot 1 år.

\says{B} Vi følte dog ikke at denne fremgangsmåde var helt visionær nok.
F.eks. er det jo oplagt at vi så kan skære ned til kun 1 optaget og så
ville vi jo producere 1 kandidat per måned!!!

\says{A} Men det var jo slet ikke nok! Vores undersøgelser viser at en
kandidat har tilbagelagt ca. 1000km til fods på instittutet igennem
sin studietid.

\says{B} Hvis nu vi antager de studerende kan bevæge sig med lysets
hastighed kan denne strækning jo tilbagelægges på blot $\frac{1}{300}$ sekund.

\says{A} Det ville altså snildt give 8.640.000 kandidater om dagen eller
1.728.000.000 kandidater på et studieår... Altså lige under 10 mia STÅ
om året...

\says{B} Men vi ved jo at STÅ er konstant! Så for at få de 10 mia STÅ til at
blive til 60 STÅ skal vi dividere med 166.666.666,6667 

\says{A}( +/- 0.0001)

\says{B} hvilket giver at vi altså skal optage 0.000000006 studerende om året

\says{A} (cirka)

\says{B} Da vi allerede har (mindst) 1 studerende tilknyttet behøves vi
altså ikke optage nogle studerende før om 166.666.666 år, 8 måneder,
17 minutter og 16.8 sekunder.

\says{A} Vi håber selvfølgelig at vores kollegaer på fysik snart kan bryde
den der fjollede "lysets hastighed" grænse, så vi kan få ENDNU bedre
resultater!

\scene{Kort pause}

\says{B} Det stiller selvfølgelig nogle krav til kursusudbuddet, da 1 kursus
jo er på 7.5 ECTS hvilket er $\frac{1}{8}$ STÅ.

\says{A} Så når 1 studerende består 1 kursus optjenes der altså 0.125 STÅ.

\says{B} Men når 0.000000006 studerende består 1 kursus optjenes der altså
kun 0.00000000075 STÅ!

\says{A} For at dette kan blive til de 60 STÅ skal der altså udbydes 80 mia
kurser om året.

\says{B} Desværre giver de nuværende 60 STÅ om året kun penge til de
eksisterende 30 undervisere, så hver underviser skal altså afholde
2.666.666.666,6667 kurser om året
\says{A} (+/- 0.0001) 

\says{B}  eller en gennemsnitlig kursuslængde på 2.16 ms 

\says{A} (inklusiv forberedelse og eksamination)

\scene{Kort pause}

\says{B} Lærerstaben har dog udtalt at de mener at dette lægger et unødigt
højt niveau af stress på dem og er derfor kommet med et modspil.

\says{A} Hvis vi istedet for optager flere studerende end i dag kan vi
slippe afsted med at holde færre kurser og dermed mindre stress.

\says{B} Hvis vi nu f.eks. optog uendeligt mange studerende, ville vi kun
skulle afholde uendeligt få kurser... Og hvis uendeligt mange
studerende tager 1 kursus genererer de altså $\infty \times 7.5$ ECTS eller
$\frac{\infty}{8}$ STÅ.

\says{A} Så vi skal finde ud af hvor få kurser vi skal holde om året for at
få genereret de 60 STÅ!

\says{B} Altså *mumler* $\frac{\infty}{8}$ gange 60 STÅ *mumle* noget med limit når
stud går mod uendelig *mumle*

\says{A} Ej, hør her, du kan jo ikke optage uendeligt mange studerende...

\says{B} Øv? Hvorfor ikke?

\says{A} Altså, hvis du skal presse så mange studerende ind i DIKUs
bygninger så vil du allerede ved 439.200.000.000 studerende opnå en
massefylde så stor at instituttet spontant vil kollapse til et sort
hul!

\says{B} Oh det havde jeg ikke tænkt på... kan vi ikke få fysikerne til at se på det?

\says{A} Hm måske, men her og nu må vi jo holde os til realistiske tal!

\says{B} Ok ok... Så vi antager at vi kun optager ca. 400 mia studerende...

        så 60 STÅ ialt gange med 8 kurser per STÅ divideret med de 400 mia
studerende giver... \act{mumler ``regne regne''} $1.2*10^{-9}$ kurser per år...

        Huhm altså...

        Så hver forsker skal altså afholde $0,04*10^{-9}$ kurser om året...
eller 1 kursus hvert 25 mia år\dots

\says{A} Vi kan godt se at det muligvis vil øge forskernes arbejdsbyrde en
lille smule i forhold til i dag, men vi føler at det er vigtigt at
alle er med til at løfte byrden for at opnå regeringes mål.

\says{B} Huhm, altså... en forsker underviser jo ikke mere end typisk 70-80 år...
så egentlig behøves vi kun at lade 1 forsker ud af 333.333.333,3333 foretage egentlig undervisning.

\says{A} (+/- 0,0001) \ldots DET ER JO GENIALT!!!!

\says{B} Ja, ikke?

\says{A} ...jamen, hvem skal det så være?

\says{B} Eftersom det kun skal være \emph{en} underviser, så må vi hellere vælge en de
studerende godt kan lide...

\says{A} Ja, en der laver rigtig god forskning!

\says{B} Ja og som alle respekterer.

\says{A} En stor pædagog!

\says{B} EN VISIONÆR FORSKERGUD!!!!

\says{A} Ja!

\says{B} Det kan kun være...

\says{A} Ja!

\says{A+B} GEORG STRØM

\scene{Tæppe}

\end{sketch}
\end{document}

%%% Local Variables: 
%%% mode: latex
%%% TeX-master: t
%%% End: 

