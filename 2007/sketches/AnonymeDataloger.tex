\documentclass[a4paper,11pt]{article}

\usepackage{revy}
\usepackage[utf8]{inputenc}
\usepackage[T1]{fontenc}
\usepackage[danish]{babel}

\revyname{DIKUrevy}
\revyyear{2007}
% HUSK AT OPDATERE VERSIONSNUMMER
\version{1.0}
\eta{$4.5$ minutter}
\status{Færdig}

\title{Anonyme Dataloger}
\author{Sandfeld, Mikkel, Lars samt Uffe og Uffe Productions\texttrademark}

\begin{document}
\maketitle

\begin{roles}
\role{P}[Kristine]  Psykopater
\role{D1}[Johan] Peter - ny ADer
\role{MM}[Daniel] Morten - Wannabe ADer
\role{D3}[Allan] Ole - ADer
\role{D4}[Munter] Søren - ADer
\end{roles}

\begin{props}
\prop{5 stole}[Rekvisitgruppen] En til hver skuespiller
\prop{Dannebrog slips}[Rekvisitgruppen] MM
\prop{Hvid skjorte}[Rekvisitgruppen] MM
\prop{Mørke bukser}[Rekvisitgruppen] MM
\prop{Striktrøje}[Rekvisitgruppen] P
\end{props}

  
\begin{sketch}

\scene{En rundkreds med 4 ADer og en psykopater}


\says{P} Goddag og velkommen til Anonyme Dataloger. Det er godt at se
at der er kommet så mange idag.

\says{P} Og jeg synes vi skal starte med vores mantra: 

\says{Alle} Computeren er roden til alt ondt.

\says{P} Vi har en ny med os idag: Peter.

\says{P} Og Peter kan du så fortælle os lidt om dig selv.

\says{D1} Hej, jeg hedder Peter, og jeg.... \act{siger med skyldfølelse} jeg er
datalog.

\scene{Alle klapper.}

\says{P} Ja, det er rigtigt, du skal erkende dit problem for at komme
videre. Det har taget meget mod at komme her idag. Fortæl os hvordan
du fandt ud af du var kommet for langt ud.

\says{D1} Jeg levede egentlig et meget normalt liv. Jeg var arbejdsløs
og surfede en del på nettet, men det er jo meget normalt. Men så en
nat, jeg tror det var fuldmåne, tog det overhånd. Jeg ved ikke helt
hvad der skete, men jeg vågnede op med hovedet på tastaturet. Der lå
tomme cola-flasker og pizza bakker alle vegne. Min kæreste havde
forladt mig, uden at efterlade sig nogen spor; det var næsten som om
hun aldrig havde eksisteret!

\says{D1} På skrivebordet lå et brev med en fed jobkontrakt og på
skærmen blinkede kurseren på sidste linje efter 4297 slutparenteser,
tilsyneladende havde jeg skrevet en ny Skype-client i Lisp.

\says{D1} Det var der jeg indså jeg var datalog.

\scene{Alle klapper.}

\says{P} Ja hvis ikke man passer på, kan det hurtigt gå galt. Det er
flot at du tør komme her i dag Peter. Lad os lige give ham mantraet
med på vejen:

\says{Alle} Computeren er roden til alt ondt.

\says{P} Nå men hvem har lyst til at fortælle hvad de har lavet siden sidst?

\scene{D2 stikker armen i vejret, undervejs ser det \emph{næsten} ud
  som om han heiler}

\says{D2} Jeg har ikke heilet eller sunget nazi-sange i to måneder nu!

\says{P}Lad nu vær' Morten! Du ved udmærket godt at anonyme
nazi-sympatisører er om torsdagen!

\says{D2} Men der har jeg jo travlt med at tilbede min Piaplakat.

\scene{D3 rækker hånden i vejret}

\says{P} Ja Ole, fortæl os hvad du har lavet.

\says{D3} Jeg var ude i fælledparken i lørdags, der mødte jeg en sød
pige. 

\scene{D'ere klapper.}

\says{P} Flot Ole og hvad skete der så?

\says{D3} Vi snakkede sammen og på et tidspunkt kom der nogen af
hendes venner. De var mægtig rare og var allesammen klædt i flotte
orange kåber og spillede på lystige instrumenter mens de messede ting
jeg slet ikke forstod.

\says{P} Øhhh \dots

\says{D3} Hun spurgte mig om jeg også ledte efter lykken og meningen
med livet. Da jeg sagde ja tog hun mig med over i en park ved siden
af. 

\scene{De andre D'ere er meget begejstrede, de kan næsten lugte den
  kommende saftige sexscene}

\says{D3} Hun inviterede mig indenfor et privat sted. Hun sagde at jeg
bare skulle tage den med ro, der var ikke noget at være nervøs
for. Hun vidste udmærket godt hvor forvirrende det kunne være første
gang. Det havde det også været for hende, sagde hun. 

\says{P}[Ivrig] Ja, JA! Og hvad skete der så?

\says{D3} Hun tog tøjet tøjet af mig \dots og så meldte hun mig ind på Matematik.

\says{D'ere}[Forfærdede] Uuuuhh.

\says{P}[strengt] Ja, der kan du se! Du skal ikke lade dig samle op af
tilfældige piger i parken! Matematik er jo forløberen for al datalogi!

\says{D'ere} Computeren er roden til alt ondt.

\says{P} Søren, jeg håber du har klaret dig bedre.

\scene{D4 ser brødbetynget ud}

\says{D4} Jeg er kommet til at bruge en algoritme.

\says{P} Nej!

\says{D4} Jo, og det var bucketsort.

\says{P} Bucketsort? For fanden, det er jo en lineær køretids
sorteringsalgoritme! Hvad helvede lavede du, din klovn?

\says{D4} Jo, jeg skulle vaske tøj, og så \dots og så tog jeg hvert stykke
tøj, et af gangen, og hvis det var lyst puttede jeg det i hvidvask, og
hvis det var farvet puttede jeg det i farvevask.

\says{P} Nej nej nej Søren. Du ved jo godt du skal holde dig væk fra
algoritmer. Du kan jo ikke styre det. Du husker jo nok hvad der skete
sidst!

\says{D4}[Svagt, klynkende] \dots ja.

\says{P} Kan I andre fortælle Søren hvad han skulle have gjort?

\says{D3} Brugt Bubblesort?

\scene{P stikker D3 en lussing.}

\says{D2} Vasket det hele sammen, og hvis noget gik galt bare skyde
skylden på de fejlfarvede?

\says{P} Hold nu kæft Morten! 

\says{D2} Jeg sagde ikke noget.

\says{P} Ej Søren, nu gik det ellers lige så godt. Du havde klaret dig
gennem de 3 ugers nedtrapning med PHP-programmering og du var næsten
kommet gennem de 4 ugers udslusning i supporterjobbet. Nu har du
totalt ødelagt din chance for at indgå som en velfungerende normal
og accepteret del af samfundet som arbejdsløs.

\says{D'ere} Computeren er roden til alt ondt.

\says{P} Men vi er her jo for at støtte hinanden, så jeg synes at vi
alle sammen skal give Søren et knus.

\scene{Kæmpe gruppekram på Søren. Midt i gruppekrammet kommer D2s
  heilende arm op af mængden.}

\says{P} Gå nu hjem Morten.

\says{D2} Men Caféen er jo lukket.

\says{P} Det var alt for idag. Nu er der så sommerferie, men for at vi
kan holde kontakten og støtte hinanden når computeren lokker med sine
ugudelige fristelser, så har jeg oprettet det her forum på MySpace dot
com hvor næste
møde også vil blive annonceret.

\scene{Alle ADerne ser målløse ud.}
\scene{TFH\texttrademark}

\end{sketch}
\end{document}

%%% Local Variables: 
%%% mode: latex
%%% TeX-master: t
%%% End: 

