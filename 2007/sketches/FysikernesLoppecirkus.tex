\documentclass[a4paper,11pt]{article}

\usepackage{revy}
\usepackage[utf8]{inputenc}
\usepackage[T1]{fontenc}
\usepackage[danish]{babel}

\revyname{DIKUrevy}
\revyyear{2007}
% HUSK AT OPDATERE VERSIONSNUMMER
\version{1.2$\Upsilon$}
\eta{$5$ minutter}
\status{Færdig}

\title{Loppecirkus}
\author{Munter og Guldfisk}

\begin{document}
\maketitle

\begin{roles}
\role{VO}[Johan] Voiceover
\role{HK}[Guldfisk] Hvid Klovn
\role{GU}[André] Gustav
\role{R1}[Daniel] Revyst 1
\role{R2}[Johan] Revyst 2
\end{roles}

\begin{props}
\prop{Kittel}[Rekvisitgruppen] André
\prop{tøj til under kittel, tak}[André] André
\prop{klovneparyk}[Rekvisitgruppen] André
\prop{Rød klovnenæse}[Rekvisitgruppen] André
\prop{Rød klovnesko}[Rekvisitgruppen] André
\prop{Fez4}[Rekvisitgruppen] André
\prop{Sygeligt stor tryllestav}[Rekvisitgruppen] André
\prop{miniature togbane}[Andre] Placeholder 3, André har, men den
skal monteres på en plade.
\prop{miniature vippe}[Rekvisitgruppen] Placeholder 3
\prop{Hundesnor}[Rekvisitgruppen] André, mindst 4m, m. karabinhage. Gerne træk-ud-model
\prop{hamsterhjul}[Rekvisitgruppen, André] Placeholder 3, André har en motor

\prop{Kittel}[Rekvisitgruppen] Guldfisk
\prop{tøj til under kittel, tak}[Guldfisk] Guldfisk
\prop{Hvid fez}[Rekvisitgruppen] Guldfisk
\prop{Kittel}[Rekvisitgruppen] Daniel
\prop{tøj til under kittel, tak}[Daniel] Daniel
\prop{Fez5}[Rekvisitgruppen] Daniel
\prop{Kittel}[Rekvisitgruppen] Johan
\prop{tøj til under kittel, tak}[Johan] Johan
\prop{Fez6}[Rekvisitgruppen] Johan
\prop{Fez1}[Rekvisitgruppen] Hustler Fez 1, nummer påtrykt
\prop{Fez2}[Rekvisitgruppen] Hustler Fez 2, nummer påtrykt
\prop{Fez3}[Rekvisitgruppen] Hustler Fez 3, nummer påtrykt
\prop{Lydeffekt: Klovnemusik}[farmand] Gustav
\prop{bord}[Rekvisitgruppen] Placeholder 2
\prop{afskærmning til bord}[Rekvisitgruppen] Placeholder 2
\end{props}

  
\begin{sketch}
\scene{Partiklen symboliseres ved at tommel og pegefingre rører hinanden}
\scene{Der står et bord som er afskærmet så man ikke kan se under eller bag det. Under det er miniaturerne. Ovenpå er Fez 1+2+3.}

\says{VO}[Går fra meget seriøs til cirkussprog når teksten bliver absurd] Og den næste ansøger til forskningsmidler fra fonden ``Helge Sanders lommepenge'' er to højt estimerede videnskabsmænd. Efter at have rejst tusindevis af kilometer, den falske vej omrking jorden, hele vejen fra HCØ: Cirkus FYSIK!

\scene{Tæppe fra, klovnerne kommer løbende ind på scenen. Cirkusmusik.}

\says{HK} Tak skal I have, tak skal I have. I aften vi vil præsentere et 
STUR STUR Forskning: TELEPORTATION! 

\scene{Gustav jubler.}

\says{HK} I aften I skal se hvordan vi kan teleportere en partikel fra 
et sted, til ET HELT ANDET STED! Gustav! \act{Klapper to gange}

\says{GU} Kom lille partikel. Kom kom! \act{holder Fez1 i hånden og lokker en partikel til} VUPTI! \act{GV vender hurtigt fez'en om på bordet, partiklen er nu fanget}. 

\says{HK} Nu vi har en partikel under Fez nummer 1. \act{Peger på Fez1 med partikel}. \act{Løfter Fez2+3} Ingen partikel her... heller ingen partikel der.

\says{HK} Og nu vi er klar til STUR STUR Forskning: Vi teleportere partikel fra Fez 1... \act{Peger på Fez1} til Fez 3. Gustav! \act{Klapper to gange}

\scene{Gustav svinger tryllestaven. Trommehvirvel.}

\says{HK} Og nu... Partikel er teleporteret til Fez nummer 3! \act{peger på Fez3}. Da Daaa!

\says{HK} Og nu... Vi teleporterer partikel tilbage igen. \act{Løfter hurtigt Fez1 og tager partiklen op for at vise den frem.} Ta Daa!

\scene{Badum-Tsch}

\says{GU} Gustav kan også teleportere! \act{snupper partiklen fra Hvide Klovn og propper under Fez1. Hustler med Fez1 og Fez3 frem og tilbage.} 

\says{GU}[Henvendt til publikum] Hvor er nu partikel? \act{Peger på Fez1 og kigger ud på publikum.} Nej...

\says{GU} Er den her? \act{Peger på Fez2 og kigger ud på publikum.} Nej...

\says{GU} Hvad med her? \act{Peger på Fez3} JA! \act{Løfter Fez3 og opdager lynhurtigt at partiklen ikke er der. Skynder sig at smække den ned igen.}

\scene{Gustav kigger med store øjne på publikum og begynder herefter desperat at løfte alle Fez'er på bordet og kaste dem væk. Mens han kigger rundt om bordet træder Hvide Klovn til.}

\says{HK} Gustav! \act{Klapper to gange}

\scene{Gustav står ret. HK løfter Gustavs Fez og tager en partikel frem fra den.}

\says{HK} Her er den. \act{Dasker Gustav over nakken}

\scene{Gustav kigger forvirret op for at se hvor det slag kom fra. Tager sin Fez på igen.}

\says{HK} Nej Gustav. Nu vi skal vise ENDNU STURRE Forskning! Vi teleportere MANGE partikler! Men hertil vi skal bruge en partikelaccelerator. Gustav! \act{Klapper to gange}

\scene{Gustav sætter en togbane frem på bordet, sætter partiklen ind i førerhuset og toget begynder at køre.}

\says{HK} NEJ NEJ NEJ Gustav! Det er konstant fart. Nul acceleration!

\scene{Gustav sætter trampolinen på bordet og flytter partiklen fra toget til trampolin. Trampolinen begynder at se ud som om partiklen hopper (snoretræk)}

\says{GU} HUJJJJJJ! Partikel har det sjovt!

\says{HK} Nej! Der er ikke konstant acceleration. \act{Klapper to gange}

\scene{Gustav sætter hamsterhjulet op på bordet og drejer rundt hurtigt.}

\says{GU} HUJJJJJJ! Partikel har det sjovt!

\says{HK} Ja! Nu vi har konstant acceleration. Så nu vi kan teleportere MANGE partikler. Og hertil vi har medbragt en revyst.

\scene{Revyst1 går op til en dør i mellemgangen}

\says{HK} Vi vil nu teleportere vores revyst fra Teleportationsmaskine nummer 1... \act{Peger på døren ved Revyst1.} Til Teleportationsmaskine nummer 2 \act{Peger på døren modsat Revyst1.} Gustav! \act{Klapper to gange}

\scene{Trommehvirvel start. Gustav danser rundt med tryllestaven.}

\says{HK} Fordi vi har nedsat lysets hastighed går dette en lille smule langsomt.

\scene{Revyst1 kommer ind af modsat dør, åbenlyst forpustet. Badum-Tsch}

\says{HK} Da Daa!

\says{GU} Gustav kan også teleportere! \act{Begynder at danse og svinge med tryllestaven.}

\scene{Revyst1 går ud af døren. Revyst2 kommer ind ad modsatte dør umiddelbart efter. Badum-Tsch.}

\says{HK} Nej nej nej! Det ligner jo slet ikke den samme revyst.

%\says{HK}[Henvendt til publikum] Vi har stadig lidt tekniske problemer. Det løses let med flere penge. Vi så hvor galt det gik da vi ville teleportere Danmarks Radio til amager.

\says{HK} Vores næste store eksperiment er at teleportere DIKU ud til Amager. Til dette formål skal vi bruge... EN FANTASILLION KRONER!

\scene{Der bliver smidt en kæmpe pose penge op på scenen fra forreste række af lysmester.}

\says{HK} Men husk nu hvor galt det gik sidst, da vi teleporterede DR til amager. 
For at være helt sikre skal vi bruge en ekstra fantasillion.

\scene{Endnu en kæmpe pose penge kommer på scenen - på posen står der: DIKU (muligvis 
overstreget)}

\says{HK} Tak Helge. Og nu kommer det fantastiske nummer!

\scene{Trommehvirvel. Gustav vifter med tryllestaven.}

\scene{Lyset går ud og en paddehattesky eksploderer. Imens løber Gustav og Hvide Klovn ud, skjult af paddehatten.}

\scene{Tom scene. Cirkusmusik. Tæppet går for.}

\end{sketch}
\end{document}

%%% Local Variables: 
%%% mode: latex
%%% TeX-master: t
%%% End: 
