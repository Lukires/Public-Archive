\documentclass[a4paper,11pt]{article}

\usepackage{revy}
\usepackage[utf8]{inputenc}
\usepackage[T1]{fontenc}
\usepackage[danish]{babel}

\revyname{DIKUrevy}
\revyyear{2007}
% HUSK AT OPDATERE VERSIONSNUMMER
\version{1.0}
\eta{$5$ minutter}
\status{Færdig}

\title{Binær Aritmetik, part 3}
\author{Uffe og Munter}

\begin{document}
\maketitle

\begin{roles}
\role{O}[Uffe] Ordstyrer
\role{B0}[Kristine] Mindst betydende bit
\role{B1}[André] bit
\role{B2}[Hanne] Mest betydende bit
\end{roles}

\begin{props}
\prop{Stol}[Rekvisitgruppen] Kristine
\prop{Stol}[Rekvisitgruppen] André
\prop{Stol}[Rekvisitgruppen] Hanne
\prop{Ensfarvet tøj}[Rekvisitgruppen] Kristine, evt. genbruge det
sorte tøj fra DIKU Bli'r
\prop{Ensfarvet tøj}[Rekvisitgruppen] André
\prop{Ensfarvet tøj}[Rekvisitgruppen] Hanne, evt. genbruge det
sorte tøj fra DIKU Bli'r
\prop{Mega stopur}[Rekvisitgruppen] Uffe
\prop{Quizzzzshow musik}[Rekvisitgruppen] Sketchen
\end{props}

  
\begin{sketch}

\scene{Der står tre stole på en række med bits på, Ordstyreren står ved stopuret} 

\says{O} Vi har observeret at de nye studerende på datalogi er markant ringere 
til at forstå to-tals systemet end de tidligere årgange har været. DIKU revyen har 
derfor i samarbejde med studienævnet udviklet kurset: 'Binær Aritmetik 
0'. Når vi nu er i en så heldig situation at vi har et fyldt auditorium, 
afholder vi kurset nu, så vi kan maksimere indlæringspotentialet. 

\says{O} To-tals systemet er ikke så meget anderledes end ti-tals systemet, 
idet hver plads repræsenterer en potens af grundtallet. Den eneste 
forskel er at vi i to-tals systemet kun har cifrene nul og et at arbejde 
med. For at demonstrere to-tals systemet, også kaldet det binære 
talsystem, har vi nogle revytter der hver repræsenterer en bit. De 
sidder i rækkefølgen fra mindst betydende bit \act{peger på B0, som
  ser ked ud af det fordi hun er mindst betydende} 
til mest betydende \act{peger på B2 som er begejstret for at være mest
betydende} bit. Når en revyt sidder ned 
svarer det til en slukket bit, eller tallet nul. Når revytten rejser sig 
op \act{B0 rejser sig op} svarer det til en sat bit, i dette tilfælde 
tallet et. 

\says{O} Vi viser kort hvordan man tæller, og i datalogi starter man jo altid 
fra nul.

\scene{B0 sætter sig}.

\says{O} et... to... tre... (op til 7).
\scene{Mens O tæller, rejser og sætter B0-2 sig så de viser de rigtige
tal} 

\says{O} Og ved 8 opnår vi så overflow og ender tilbage på 0.
\scene{B0-2 sætter sig}

\says{O} Ja, så skulle konceptet vist være klart for alle og
enhver. \act{Henvendt til B0-2} Så kan I godt gå ud bagved.

\scene{pause, B0-2 sidder stille}

\says{O} 7. \act{B0-2 rejser sig op.}

\says{O} 6, 5, 4, 3, 2, 1, 0. 

\scene{Mens O tæller ned, bevæger B0-2 sig ud af scenen mens de
  tæller med kroppen.} 

\says{O} Nu kan vi gå i gang med den spændende del af dette kursus... øvelserne. 

\says{O} Vi har på allersnedigste vis placeret publikum på rækker som hver kan 
repræsentere et binært tal. Jeg vil om lidt sige et tal som hver række 
skal danne for at vise at de har forstået dagens pensum. Vi starter med 
nogle lettere øvelser og går til sidst til en lille ekstraopgave som 
udløser en præmie til den første række der klarer opgaven. 

\says{O} Er alle klar?

\scene{Publikum råber} 

\says{O} Godt så. Det første tal der skal repræsenteres er tallet: et. Sæt i 
værk \act{Venter på at publikum nosser sig sammen til at finde ud af 
hvem der er mindst betydende}. 

(Her skal der improviseres. O retter på 1-2 rækker) 

\says{O} Her er et par rækker som tydeligvis ikke har forstået konceptet. Det 
er bitten til højre for læseren der er den mindst betydende. Det betyder 
at det er personen længs til venstre i rækken, der skal rejse sig. 

\says{O} Det næste tal vi skal have repræsenteret er: 128. Sæt i værk. 

\says{O} Til dem der stadig ikke har forstået konceptet så kig på række N
\act{peger på en række som har gjort det rigtigt}. 

\says{O} Det sidste tal vi skal repræsentere er: 2047. Sæt i værk.

\scene{Mange i publikum rejser sig op}

\says{O} Det var godt gået. Nu har de fleste helt klart forstået konceptet. Vi 
springer direkte til den lille ekstraopgave, som udløser en præmie til 
den første række der løser opgaven: Fri bar til hele rækken resten af 
aftenen. Vi har afsat nogle sekunder til at løse problemet. Er alle klar? 

\says{O} Super. Tallet vi skal repræsentere er: -0.04392 i IEEE floating point 
notation. Sæt i værk. \act{Starter det store stopur}

\scene{Der spilles stress-musik fra et quizshow.} 

\says{O} Så er tiden gået. Jeg kan se at der ikke er nogen der har løst 
problemet, så der er ingen fri bar i aften. Vi har valgt at benytte 
denne prøve som eksamen og hvis nogen skulle være i tvivl er I alle 
dumpet. Vi ses igen til næste år til re-eksamen.

\scene{Tæppe for hurtigt}

\end{sketch}
\end{document}

%%% Local Variables: 
%%% mode: latex
%%% TeX-master: t
%%% End: 

