\documentclass[a4paper,11pt]{article}
\usepackage{revy}
\usepackage[utf8]{inputenc}

\title{To gamle gnavpotter}

\version{1.5}

\author{Beyond \& Panic \& km}

\begin{document}

\maketitle

\begin{roles}

\role{Lars}

\role{Karoline}

\end{roles}

\begin{sketch}

\vspace{-3em}


\scene To ``gamle m@end'', \'a la Muppet Show sidder bagest i
auditoriet og begynder at kommentere revyen efter f|rste akt.  I
starten roser de revyen, men g@ar efterh@anden over til et
kritiserende, h@anligt tonefald.


\says{Lars}Hold da k@eft, hvor er det en god revy, hvad?

\says{Karoline} Ja, det er en helt {\em fantastisk} revy!

\says{Lars} De har virkelig gode sketches!

\says{Karoline} Og nogen virkelig gode sange, de
var {\em s@a} morsomme!

\says{Lars} Ja, jeg kunne faktisk ogs@a h@ore teksten

\says{Karoline} Jeg kunne nu ogs@a h@ore de fleste af dem

\says{Lars} Ja, der var godtnok et par stykker, som jeg havde sv@ert
ved at h@ore\dots

\says{Karoline} Jeg havde m@aske ogs@a lidt sv@ert med at forst@a
sketchene\dots

\says{Lars} Ja, der var faktisk en del af dem, som jeg ikke forstod en
hujende papfis af!

\says{Karoline} Ja, osse mig! Faktisk var der kun een enkelt vittighed
jeg forstod - og den var d@arlig!

\says{Lars} Forstod du den? Det gjorde jeg slet ikke - de var
d@arlige, alle sketchene!

\says{Karoline} Ja, sangene var ogs@a d@arlige - jeg kunne ikke
h@ore dem, og de sang falskt!

\says{Lars} Sang de falskt? Det var sgu da helt umuligt at h@ore dem!

\says{Karoline} Ja, det er en elendig revy i aften

\says{Lars} Elendig? Det er da mildt sagt! Jeg synes den var ti gange
v@erre end opgaveformuleringen af Dat2 eksamenss@ettet!

\says{Karoline} Ja, jeg synes faktisk vi tr@enger til en pause.

\says{Lars} Ja, Ud med dem!

\says{Karoline} Ja, ogs@a publikum

\says{Karoline \& Lars i kor} {\em Buuuuuuuuuhhh! @Ov! @Ov! @Ov! Uuuuuuud!}

\end{sketch}

\begin{sketch}

\scene Efter 2. akt, men f@or ekstranumrene, kommenteres der igen.

Video: Henrik Damborg.
Om og hvordan dette kan lade sig g@ore overlades til de to optr@edende
at afg@ore.


\says{Karoline} Sikke noget lort!

\says{Lars} M@ognummer!

\says{Karoline} Det hele var forf@erdeligt!

\says{Lars} Det v@erste jeg nogensinde har set!

\says{Karoline} Ja, lyden var selvf|lgelig meget god.

\says{Lars} Det var vist Steen Bartholdy, der stod for den.

\says{Karoline} Og lyset fik nu ogs@a skuespillerne til at se p@enere ud.

\says{Lars} Ja, det var jo Bo Arlif, der var lysmester.

\says{Karoline} Og Robert Stahr og Ole Lennert p@a spots.

\says{Lars} Og s@a hjalp det jo ogs@a, at de blev sminkede.

\says{Karoline} Ja, det var Nina Frederiksen.

\says{Lars} Musikerne var ogs@a ret gode

\says{Karoline} Is@er Knud Henriksen p@a trommerne.

\says{Lars} Ham p@a synthesizer var også god, han hedder Jacob S\"onnicsen

\says{Karoline} Nej, hedder han ikke S@oren Holstebroe?

\says{Lars} Nej, det er den anden

\says{Karoline} N@a, men han var da ogs@a god.

\says{Lars} Og Paal @Osterud p@a bas

\says{Karoline} Og Asger H@ogsted og Jacob Lorentzen p@a
guitarerne, de var ogs@a ret gode.

\says{Lars} Faktisk var alle musiknurene ret vellykkede.

\says{Karoline} Og morsomme!

\says{Lars} Og vinen til middagen, den var ogs@a god

\says{Karoline} Den havde du jo selv med

\says{Lars} Jamen alligevel!

\says{Karoline} Og sketchene(?) var ogs@a vittige

\says{Lars} Tekstforfatterne er jo virkelig dygtige!

\says{Karoline} Og s@a beskedne

\says{Lars} Egentlig var det jo en god revy

\says{Karoline} Ja, den var skidego'

\says{Lars} Den var fantastisk!

\says{Karoline} Genial!

\says{Begge} Mere! Jahh! Mere! Ekstranummer!

\scene (Her overtager publikum.)

\end{sketch}

\end{document}

