\documentclass[a4paper,11pt]{article}
\usepackage{revy}
\usepackage[utf8]{inputenc}

\title{Reklamesketches}

\author{Mia Henrikson}

\version{1.0} % HUSK AT AJOURFORE UDGAVE-NUMMERET!

\begin{document}

\maketitle




\begin{center}
{\bf Pampers Student}
\end{center}

\begin{roles}
 \role{En (mandlig) datalogistuderende}
 \role{evt.\ ``Prikkeren''}
\end{roles}

Rekvisitter: Evt. bleer i voksenstoerrelse (hvem ved, hvor man 
              skaffer dem?), bord og stol, evt. computer (hvis 
              den alligevel er paa scenen).

\begin{sketch}
Den datalogistuderende siger (henvendt til publikum): 
``Du kender problemet: Du sidder ved maskinen og laver
kerneopgave, men er noedt til at rende paa toilettet i tide og
utide og toemme blaeren for alt det cola, du har drukket. Det
behoever du ikke laengere: nu er ``Pampers Phases'' udviklingen
nemlig naaet til ``Pampers Student'' - den helt rigtige type ble
til dig, som kan li' at sidde doegn efter doegn ved maskinen.
Foerst, da jeg saa hvor tynd en ``Pampers Student'' er, var jeg
lidt skeptisk, men efter at have siddet med den paa i 18 timer
fik dens fantastiske sugeevne mig helt til at glemme, at jeg
havde den paa... Jeg foelte mig simpelthen toerrere! 
(Kort pause...) Engang proevede min rapportmakker et andet
maerke, men han var bare sur og tvaer under hele
rapportperioden - han blev simpelthen ikke holdt toer. Saa gav
jeg ham en af mine Swappers, og siden har han NYDT at sidde
ved maskinen i timevis. Naar jeg har det godt, skal mine
venner ogsaa ha' det godt.''
Hvis vi kan skaffe en stor ble, kan den studerende til sidst
smide noget toej og vise publikum, at han er ifoert en ble.
Nummeret kan evt. afsluttes med, at ``Prikkeren'' kommer og
sender ham vaek.
\end{sketch}

\newpage
\begin{center}
{\bf Intern besked}
\end{center}

\begin{roles}
\role{En datalogistuderende} 
\role{evt.\ ``Prikkeren''}
\end{roles}

Rekvisitter: Overhead (el. lign.?), bord, stol, computer       
             (skaerm, tastatur), cola, chipspose.

\begin{sketch}
En studerende kommer ind med en pose chips og en stor cola
under armen; saetter sig ved en terminal. Han logger ind,
aabner colaen, tager en slurk, og ser paa skaermen. (Vi skal
bruge en overhead el. lign. til at vise publikum, hvad han ser
paa skaermen). Inden han naar at synke colaen, kommer
foelgende op paa skaermen: ``Intern besked: Det er ikke tilladt
at indtage drikkevarer ved terminalen!''. Han ser sig
fortvivlet omkring og spytter saa colaen ud over skaerm og
tastatur!!!
Nummeret kan evt. afsluttes med, at ``Prikkeren'' kommer og
sender ham vaek.
\end{sketch}

\vspace{2cm}

\begin{center}
{\bf Gillette censor}
\end{center}
\begin{roles}
\role{Mandlig datalogistuderende} \role{``Prikkeren''}
\end{roles}

Rekvisitter: Barbergrej (Gillette Sensor), 
             stort skilt: ``Gillette Censor''.

\begin{sketch}
Den studerende kommer ind, viser barbergrejet frem til
publikum, og siger: ``Med Gillette Censor kan du vaere sikker
paa at faa en GLATBARBERET CENSOR til eksamen (DET er jo en
fordel, hvis censor er en kvinde!). Husk vort slogan:
``Gillette - the best a MAN can get'' !!!''
Til slut kommer prikkeren og sender ham ud.
\end{sketch}

\vspace{2cm}

\begin{center}
{\bf Intern DIKU reklame}
\end{center}
\begin{roles}
\role{En nyhedsoplaeser-type}\strut
\end{roles}

Rekvisitter: Manuskript.

\begin{sketch}
Nyhedsoplaeseren kommer ind paa scenen, og laeser langsomt og
omhyggeligt op fra manuskriptet: ``Vi bringer nu en intern DIKU
reklame: Patologisk studielevn soeger destruktorer til at
varetage undervisningen paa plat nul, plat et, og plateau!
(pause...) Efter reklamerne bringer vi ugens seks lottobits og
den ekstra paritetsbit!''. EXIT.
\end{sketch}

\end{document}
