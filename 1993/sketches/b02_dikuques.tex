\documentclass[a4paper,11pt]{article}
\usepackage{revy}
\usepackage[utf8]{inputenc}

\begin{document}

\title{DIKU quest}

\author{Stephan Dahl}

\version{1.0} % HUSK AT AJOURFORE UDGAVE-NUMMERET!

\maketitle

\begin{roles}

\role{S@elger} Reklames@elger: En rigtig TV-reklame s@elgertype.
\role{Sp} Spilleren: En typisk studerende.
\role{SM} Spillemester: En anden typisk studerende, med et sadistisk
glimt i @ojnene.

\end{roles}

\begin{sketch}




Rekv.:	Bord, to stole. Stor 8-sidet terning. En stabel b@oger (Date, Decker,
W\&E) etc. + l@osark ad libitum.

\scene	Et bord, en stol p@a hver side. S@elger kommer ind, spot f@olger ham.

Note:	Alle x'er l@eses som ``hvad der nu tilf@edigvis lige er blevet rullet p@a
terningen''.


\says{S@elger}	I har alle set DIKU (udtalt som var det en film); I har k@obt
DIKU T-shirten og de moderigtige DIKU underbukser, I har nydt DIKU
morgenmaden og DIKU snackbaren - Pr@ov nu: DIKUQUEST(tm): Rollespillet!

(Sp kommer ind, medbringende et ark papir, pen \& 8-sidet terning. Mens S@elger
forklarer, ruller Sp med terningen og noterer p@a sit ark papir)

\says{S@elger}	Spilleren (peger) skal f@orst sl@a sin SK -
StudenterKarakter - Hvis karakteren ikke er god nok, f@ar man slet
ikke lov til at spille!  DIKUQUEST(tm) SK'er er utroligt detaljerede;
f.eks skal spilleren sl@a for sin karakters UDHOLDENHED (Sp ruller),
hvilket er hvor godt maven holder til cola og kaffe, samt hvor l@enge
man kan holde sig v@agen under kernen; Han skal sl@a for BEH@ENDIGHED
(Sp ruller), hvor hurtigt man kan indtaste de sidste 105 sider af en
rapport; Der er NERVER (Sp ruller), hvor god man er til at h@andtere
netnedbrud og manglende farveb@and dagen f@or afleveringsfristen,
eller STYRKE (Sp ruller), hvilket er fuldst@endig irrelevant i normale
sammenh@enge, og UDSEENDE (Sp ruller), hvilket heller ikke er s@a
vigtigt, da der sj@eldent er nogen at g@ore indtryk p@a... (SM kommer
ind, med stabler af b@oger \& papirer, SM-sk@erm \& MatadorMix) ...
Alt dette, og mange flere detaljer, beskrives i det smagfulde
DIKUQUEST(tm) regelh@efte, der er flot gr@ont og har et fint indeks!
(S@elger vifter med std. gr@on A5-DIKU-Rapport).

(Lyset d@empes til en r@od spot p@a SM, der taler med dyster stemme til Sp, der
sidder overfor ham) 

\says{SM}	Du er lige blevet optaget p@a DIKU, og du st@ar nu p@a
trappen foran instituttet. Store bronzestatuer skuler olmt ned til
dig, men du ved, at du m@a udholde tre @ars tr@engsler \& lidelser bag
disse d@ore, f@or du opn@ar den ultimative h@eder: et Bachelor-Bevis!
(trommehvirvel).

\says{S}	Jeg sparker d@oren op. (Klir! Meget antiklimaks. Almindeligt scenelys
p@a SM \& Sp)

\says{SM}	(Gr@emmer sig) Okay, rul en terning (Sp ruller). Hm, x, du
balrer foden igennem ruden, og mens du st@ar og forbinder dit ben,
kommer dekanen forbi. Rul for EKSAMENS\-TEKNIK for at bluffe dig ud af
det (Spilleren ruller).  Hm, x, (ser @ergelig ud) Ja, du klarer det.
Dekanen hj@elper dig med forbindingen, hvorp@a han ringer til
bygningsudvalget for at f@a panserglas i d@oren. Du kommer ind til
Dat-0. Der st@ar en pr@est og messer for et tempel fuldt af zombier
... @oh (bladrer i papirer) ... undskyld, forkert eventyr, der st@ar
en LEKTOR og FOREL@ESER for en flok STUDERENDE. Hvad g@or du?

\says{S}	Jeg s@etter mig ned og lytter opm@erksomt, mens jeg flittigt tager
notater. 

\says{SM}	(ser vantro ud) Du g@or hv...? N@a, @oh, javel ja, (bladrer
febrilsk i sine papirer) ... Der g@ar fem minutter. Rul for
UDHOLDENHED for at holde dig v@agen. (Sp ruller) Ha, haa! x! Du falder
i s@ovn, og har et skr@ekkeligt mareridt. Du v@agner med et skrig,
hvilket iriterer forel@eseren grusomt. Han kalder dig ned for at
h@andk@ore Ackermann(5,8). Hvad g@or du? (griner f@elt).

\says{S}	(ser panisk ud) @Oh, @Oh, @Oh, Jeg l@ober op i kantinen!

\says{SM}	Hm, jah, rul for at SNIGE dig (rulle-rulle), x, jajada, du
slipper ud af torturkammeret, @oh undskyld, AUDITORIET. Du n@ar op i
kantinen. S@a snart du @abner d@oren sl@ar en h@orm af uvaskede kopper
og studerende dig i m@ode. Du f@oler din mave rotere (ruller
terningen), men du genser dog ikke morgenmaden.

(Lyset bl@ender op for S@elger, der afbryder)

\says{S@elger}	Var det ikke heldigt, hvad, folkens? Man ser her hvor
realistisk og detaljeret DIKUQUEST(tm) er!

(Lyset bl@ender ned igen)

\says{SM}	(Irriteret over afbrydelsen) Hm, ja, (til spilleren) Hvad g@or du?

\says{S}	Er der noget kaffe?

\says{SM}	(fniser skadefro) Du tager en kop kaffe? (Ruller en
terning, ser skuffet ud) Der er desv@erre ikke noget tilbage. --- Men
der ligger en pose kaffe fremme (forh@abningsfuldt).

\says{S}	Ok, jeg laver kaffe.

\says{SM} 	Ha!

\says{S}	Vent! Jeg tjekker for f@elder.

\says{SM}	For sent. Du br@ender dig p@a varmtvandshanen, og
mister et styrkepoint. Har du ogs@a lyst til en cola? Gn@ek, Gn@ek.

\says{S}	(ser meget snedig ud, har lige gennemskuet noget)
N@e@eh, jeg tror ...  jeg snupper mig en Bl@a!

\says{SM}	(fortr@edeligt, hvisker til publikum) Curses! Foiled
Again! (til spilleren) Hmm. Rul for HELBRED for at se om din lever kan
klare det.

\says{S}	(griner) Jaja, og husk jeg har en bror i EDB-afdelingen, det giver +2
bonus! (ruller).

\says{SM}	(opgivende) Du drikker den bl@a, og overlever chokket. En @ol medf@orer
en anden, og da du v@agner igen, er der g@aet seks @ar, og du har lige best@aet
Dat-2. Du f@ar 1100 kreditpoint for det her eventyr... (Lyset skifter til
S@elger; SM \& Sp exeunt).

\says{S@elger}	Er det ikke fantastisk, venner! Og I kan gl@ede jer til n@este @ar, hvor
DIKUQUEST II: ANDENDELEN(tm) udkommer! (exit)
\end{sketch}

\end{document}
