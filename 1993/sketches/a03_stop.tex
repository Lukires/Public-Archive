\documentclass[a4paper,11pt]{article}
\usepackage{revy}
\usepackage[utf8]{inputenc}

\title{Stoppr@ovenummer}

\author{Peter Hary Eidorff}

\version{1.0} % HUSK AT AJOURFORE UDGAVE-NUMMERET!

\begin{document}

\maketitle

\begin{roles}

\role{8 studerende}
\strut
\role{1 kommentator}\strut
\role{"prikkeren" (evt.\ en rekvisit der skal styres)}\strut
\end{roles}

{\bf Rekvisitter}

\begin{tabular}{ll}
To & borde\\
              6& terminaler/sk@erme\\
              6& stole\\
              1& mikrofon til kommentatoren\\
\end{tabular}

Jeg forestiller mig dette nummer som et af de f@orste, s@a vores
``prikker'', der prikker folk p@a skulderen n@ar de skal vige fra
scenen, kan blive introduceret.

\begin{sketch}

\scene
Bordene stilles mod hinanden, s@a de danner en stor kvadratisk
overflade. De 6 terminaler placeres j@evnt fordelt rundt om
bordene. Stolene stilles udfor terminalerne. Der skal v@ere god
plads p@a scenegulvet omkring bordene.

Kommentatoren kommer ind p@a scenen og g@ar hen til mikrofonen.

\says{Kommentator} Datalogisk Institut er blevet p@alagt at indf@ore
             stoppr@over p@a Dat-0 allerede fra n@este semester.
             Form@alet med stoppr@overne er, at de studerende
             skal vise at de er studieegnede. Studieegnethed
             er en betingelse for at m@atte forts@ette med studiet.
             Fra 95, n@ar der bliver frit optag p@a de videreg@aende
             uddannelser, vil stoppr@overne kunne bruges til at
             regulere antallet af studerende. De studerende skal
             med andre ord k@empe med hinanden om studiepladserne.

             Det diskuteres meget i denne tid, hvordan stoppr@overne kan realiseres p@a Dat-0. Revygruppen er g@aet
             i t@enkeboksen, og har fundet p@a en del sp@endende
             ideer. Vi illustrerer en af dem her.

\scene
8 studerende kommer ind p@a scenen, og stiller sig op side om side
ved scenens forkant.

\says{Kommentator} Her har vi s@a de glade studerende, der har besluttet
             sig for at l@ese datalogi. De l@eser nu p@a Dat-0.

             Halvvejs inde i semesteret skal de til stoppr@ove.
             Der er kun plads til 3/4 af dem p@a f@orstedelen, s@a de
             andre er ikke studieegnede.

             Se her hvordan det foreg@ar.

\scene Musikken begynder at spille en eller anden b@ornesang
(konstruktive forslag udbedes [hvad med ``Lille Peter Edderkop'',
red.]), mens de 8 studerende danser rundt om terminalerne. Efter ca.\
30 sekunder stopper musikken brat, og de 8 studerende skynder sig at
s@ette sig ved en af de 6 terminaler. Det m@a n@odvendigvis blive kamp
om nogen af pladserne. De 6 studerende, som har f@aet tilk@empet sig
en terminal, begynder straks at taste og se aktive ud.

\says{Kommentator} De studerende, som ikke l@engere er studieegnede,
             bliver diskret prikket p@a ryggen, og bedt om at g@a.

\scene Prikkeren prikker de to tiloversblevne studerende p@a ryggen,
hvorefter de forlader scenen.

\says{Kommentator} Princippet kan ogs@a bruges til at regulere antallet
             af studerende p@a andendelen. Lad os sige, at der kun
             er plads til 1/3 af de studieegnede, dvs.\ 1/4 af de
             studerende, der valgte datalogi til at starte med.

\scene
4 af terminalerne fjernes fra bordene sammen med 4 stole. Det kan
de studerende evt.\ selv g@ore.

\says{Kommentator} Udm@erket. Vi k@orer igen!

\scene
Musikken starter igen, og de 6 studerende danser rundt om de 2
terminaler. Efter et stykke tid stopper musikken brat, og der
k@empes om de 2 terminaler p@a samme m@ade som f@or. De 4 uheldige
bliver prikket p@a ryggen, og forlader scenen.

\says{Kommentator} Faktisk kan princippet nemt tilpasses optag p@a
licentiatstudiet, ans@ettelse af adjunkter, ans@ettelse af lektorer og
endelig udv@elgelse af professorer.

\scene
Lyset slukkes, og akt@orerne tager bordene, de to terminaler og de
to stole med sig ud.

\end{sketch}

\end{document}
