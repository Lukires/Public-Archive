\documentclass[a4paper,11pt]{article}
\usepackage{revy}
\usepackage[utf8]{inputenc}

\version{1.0}

\title{Led, ledere, ledest}

\author{Panic}

\begin{document}

\maketitle

{\large \em (enhver lighed mellem f@olgende personer er ikke tilt@enkt
- institutlederen st@ar over dem alle!)}

\begin{roles}

\role{J@orgen, Institutleder} Meget p@ent n@alestribet jakkes@et,
slips osv. (checket mappe ved siden af bordet)

\role{"Laila", Lederekret@er} H@ores over samtaleanl@eg med en
sexet stemme

\role{Carl, Lykkeand} Ikke uslidte jeans, forvasket hvid
T-shirt over en lidt kraftig mave, sort fuldsk@eg og
metalstelsbriller

\role{Henrik, Dekan} Batiksm@olfeudstyr: stor mave, hvidt
sk@eg \& h@ar, batik-poncho

\role{Rekvisitter:} Et bord bag J@orgen med en terminal p@a, 1
lederstol (l@ederstol, vippeaggregat osv.), et bord foran J@orgen med
samtaleanl@eg, telefon, sagsmappe, et skilt med ``LEDER'' og et
billede af Laila i ramme. Skraldespand, dekret, papir

\end{roles}

{\sl Med den ny Universitetslov skulle det nu v@ere muligt at drive DIKU
effektivt, med en slagkraftig, fremtidsorienteret stab af vision@ere
ledere, der har det fulde overblik, og som kan stilles personligt til
ansvar...}

\begin{sketch}

\scene J@orgen sidder i m@orke, og begynder at tale, mens lyset
langsomt bl@endes op.

\says{J@orgen}[fast, myndigt og lidt lumskt, mens han s@etter sidste
punktum p@a et papir] H@e - h@e ! S@a skal vi nok f@a sat en stopper
for de ineffektive studieparasitter, der altid sidder og bruger MINE
maskiner, uden at give mig studietrinstilv@ekster! {\em (rejser sig
op)} Lad mig nu se, om dette har den rette autoritet...  {\em
(begynder at g@a myndigt frem og tilbage, mens han l@eser op af
papiret i h@oj \& fast stemme)} "Idet der er konstateret et kraftigt
overforbrug af s@avel maskinelle som programelle og
kommunikationsm@essige ressourcer for@arsaget af sm@a 1.-delsmider,
indsk@erpes det herved, at det p@a ingen m@ade tolereres, at min -
@oh - instituttets diskplads anvendes til GIF-billeder af sexuelt
underl@odig karakt\'er. Edb-afdelingen vil gennemg@a samtlige konti
og lukke dem der m@atte indeholde s@adanne billeder. Kontiene vil
f@orst blive gen@abnet efter en audiens hos Den store - @oh -
Institutlederen.

\scene (det banker p@a d@oren)

\says{J@orgen} Kom ind!

\says{Carl}[kommer ind med et papir i h@anden] Davs, J@orgen! Jeg
ville lige...

\says{J@orgen}[bryder ind med kraftig stemme] Carl! HAR jeg ikke
sagt, at jeg er Den store Leder?

\says{Carl} @Oh, jo, undskyld, Store Leder. Jeg ville...

\says{J@orgen}[mildere] Det var bedre. Hvad vil du, min Lykkeand?

\says{Carl} Jo, jeg ville bare sp@orge, om det er rigtigt, at du vil
have @endret dit brugernavn.

\says{J@orgen} Er det m@aske ikke det, der st@ar p@a det papir... {\em
(peger pa Carls papir)} ...med - min - underskrift - p@a ???

\says{Carl}[lidt beklemt] Joda, men vi HAR jo den politik, at
brugernavne skal v@ere p@a mindst fire tegn, og...

\says{J@orgen}[bryder ind, irettes@ettende] Carl! Lederen sp@orger
dig ikke om politik, han beder dig {\em (retter sig)} BEORDRER dig til
at @endre hans brugernavn! {\em (vender sig mod publikum)} - Og
i@ovrigt synes jeg da det er meget passende, at mit brugernavn fra nu
af bliver "GUD"...

\says{J@orgen}[tager GIF-dekretet og r@ekker det til Carl] Her,
h@eng dette dekret op overalt. Og s@org for, at det kun er mig p@a
min terminal, der kan gen@abne en lukket konto!

\scene Carl nikker @erb@odigt og g@ar ud.

\says{J@orgen}[b@ojer sig ned til samtaleanl@egget, og trykker
p@a det. S@od \& indsmigrende tonefald] Laila-pus! Vil du v@ere s@od
at hente min kaffe. Og aflys lige alle mine aftaler i eftermiddag.

\says{Laila}[over anl@egget] Hvorfor det? Skal der v@ere
bestyrelsesm@ode, J@onsebasse?

\says{J@orgen} Bestyrelsesm@ode? Nej, da. Det er kun noget man
holder, n@ar man vil give folket indflydelse p@a, hvilken farve der
skal v@ere p@a undersiden af toiletbr@etterne! Nej, jeg skal i
zoologisk have med min mor...

\says{Laila}[forelsket] @Ah, J@onsebasse! Du er s@a romantisk!

\scene Henrik kommer ind - J@orgen ser ham ikke

\says{J@orgen}[stadig til anl@egget] Det er du ogs@a, min lille
pussenussetrusset@os {\em (kysser billedet)}...

\says{Henrik} Ahem!

\says{J@orgen}[springer tilbage i stolen, retter p@a slipset]
Javel, hr. Dekan - @oh - Overleder! Hvad skylder jeg @eren?

\says{Henrik} Vi skal have aftalt nogle optagelsestal for efter@aret.
Kender De det magiske tal 36?

\says{J@orgen}[l@ener sig selvsikkert tilbage i stolen] Jada, det er vores VIP-tal.

\says{Henrik} Og tallet 3.552.462?

\says{J@orgen} Selvf@olgelig! {\em (Smiler bredt)} Det er vores
driftspenge.

\says{Henrik} Hvad s@a hvis jeg siger 745?

\says{J@orgen}[grubler] - @oh - nej, det kender jeg ikke {\em
(grubler lidt, og f@ar s@a en id\'e)} - er det en 2'er potens?

\says{Henrik} Nej, det er vistnok jeres optagelsestal til august...

\says{J@orgen} Jamen, vi kan da ikke...

\says{Henrik}[b@ojer sig ned og st@otter med begge arme p@a bordet,
truende tone] hr. UNDERleder! V\'ed De, HVEM der givet Dem MAGTEN i
h@enderne? Og v\'ed De, hvem der kan afs@ette Dem???  {\em (ilter)}
Og ved De, hvad man f@ar, hvis man tr@ekker 2-i-femte fra 36??? {\em
(v@elter billedet ned i skraldespanden)}

\says{J@orgen}[lidt bange] Ih jo, selvf@olgelig. Og vi kan da
sagtens optage 745 nye studerende; vi laver bare en kvik
januar-stoppr@ove i beregnelighed og Miranda... ...og der kan da vel
snildt sidde tre studerende p@a hver stol i auditorierne!

\says{Henrik}[rejser sig op igen] Det var bedre. S@a er det en
aftale. I@ovrigt har jeg lige sat en evaluering i gang. Alle VIP'ere
der ikke passer deres forskning og undervisning vil f@a et brev med
tilbud om en MEGET lang ferie...

\says{J@orgen}[rejser sig op, sl@ar h@elene militaristisk sammen]
Selvf@olgelig! Effektivitet og Kvantitet frem for alt - jeg skal
personligt s@orge for at afskedige samtlige VIP'ere, der ikke lever
op til Fakultetets ry!

\scene Telefonen ringer.

\says{Henrik} Godt {\em (g@ar ud)}

\says{J@orgen}[tager telefonen] Hallo.

(...)

\says{J@orgen} Ja, jeg kan godt huske, jeg sendte dig den ordre.
(...) Mhm. Ja. Nej. {\em (han samler billedet op af skraldespanden og
stiller det sirligt p@a bordet igen)}

(...)

\says{J@orgen}[overlegent] Forskningsfrihed? Ja da, hr. Kj@er, De
har fuld frihed til at forske... ...enten i det emne jeg gav dig,
eller i... {\em (tager en sagsmappe og leder efter et bynavn)}
...Albertslunds bistandskontors indretning!

\scene (samtaleanl@egget summer)

\says{J@orgen}[i telefonen] Nej, hr. Kj@er, jeg synes ikke det er
irrelevant at forske i "st@ovpartiklers indflydelse p@a
computerkabinetters reflexionstal". Farvel, hr. Kj@er! {\em (sm@ekker
telefonen p@a)}.

\says{Laila}[i samtaleanl@egget] J@onsebasse! J@onsebasse!

\says{J@orgen}[til anl@egget, syngende tonefald] Ja, min
pussenussetrusset@os?

\says{Laila}[gl@edestr@alende] Der er kommet brev fra Dekanens
evalueringsgruppe. De har ogs@a lige ringet til mig, og spurgt, hvem
der skulle forel@ese i Dat2 i dag.

\says{J@orgen} Aha.

\says{Laila} Ja, og s@a sagde jeg, at det skulle du jo egentlig,
J@onsebasse, men at du havde aflyst undervisningen, fordi du skulle i
zoologisk have, ikke? {\em (ivrig)} Og ved du hvad? Nu har de sendt
dig et brev, hvor de vil bel@onne dig med en lang ferie...

\says{J@orgen}[panik] Hvad? Hvor? NEEEJ! Hvor er de
evalueringsfolk? Ring til dem! Stop! @A@a@ahnej! HJ@E@ELP!

\says{Laila} Jamen J@onsebasse, dog! Er der sket noget slemt?

\says{J@orgen}[fortabt] Jeg arme stakkel - hvad skal jeg g@ore?

\says{Laila}[har stadig ikke forst@aet det] Du kunne jo skrive et
elektronisk brev til dem, hvis du vil aflyse ferien...

\says{J@orgen}[fatter sig igen] Ja. Tak, Laila. Det vil jeg g@ore
- og s@et mig p@a Dat2-forel@esningen igen!

\scene J@orgen vender sig om til terminalen for at logge ind.

\says{Terminal}[taler med en metallisk (kvinde?-)stemme, s@a
publikum kan h@ore det] {\sf LOGIN:}

\says{J@orgen} G - U - D {\em (trykker p@a retur)}

\says{Terminal}: {\sf PASSWORD:}

\says{J@orgen} L - A - I - L - A {\em (trykker p@a retur)}

\scene (der g@ar et stykke tid)

\says{Terminal} {\sf DER ER FUNDET GIF-BILLEDER MED SEXUELT TILSNIT P@A DIN
KONTO, SOM DERFOR ER BLEVET LUKKET. DU SKAL HENVENDE DIG TIL
INSTITUTLEDEREN, DA HAN ER DEN ENESTE, DER KAN GEN@ABNE DIN KONTO.
VENLIG HILSEN EDB-AFDELINGEN}

\says{J@orgen}[skriger op i luften] CARL!

\scene Lyset slukkes.

\end{sketch}

\end{document}
