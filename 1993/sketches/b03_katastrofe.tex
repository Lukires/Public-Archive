\documentclass[a4paper,11pt]{article}
\usepackage{revy}
\usepackage[utf8]{inputenc}

\title{Den Værst Tænkelige Hændelse}

\author{Steen, Null, Harry, Henrik og Mia}

\version{1.0} % HUSK AT AJOURFORE UDGAVE-NUMMERET!

\begin{document}

\maketitle

\begin{roles}
\role{interviewer} med mikrofon
\role{2 datalogistuderende, Bent og Pia }
\role{1 fysik-dat studerende, Steen} en rigtig nørd


\end{roles}


Der er egentlig ikke brug for rekvisitter (bortset fra --- måske ---
noget der ligner varmtvandsbeholderen fra kantinen). Sketchen går ud
på, at intervieweren lader de studerende fortælle (efter tur) om den
dramatiske dag, hvor varmtvandsbeholderen i DIKUs kantine gik i
stykker, så man ikke kunne få kaffe!

\begin{sketch}
\scene
Alle står på scenen. Stemningen er alvorlig.

\says{Interviewer}[henvendt til publikum]
     Det, I nu skal overvære, er en sandfærdig beretning
     fra det virkelige liv. Den værst tænkelige hændelse
     indtraf en torsdag morgen her på jeres institut!

\scene
Interviewer holder sin mikrofon frem til Pia

\says{Interviewer}Ja, Pia, du er jo studerende på Datalogisk Institut,
     og du var øjenvidne til den forfærdelige hændelse.
     Fortæl os, hvad der skete den torsdag morgen!

\says{Pia}
Klokken var 9.03 den torsdag morgen. Jeg var lige
     kommet med bussen til DIKU, og gik op til kantinen for
     at få mig en kop kaffe, inden forelæsningen begyndte. 
     Straks, da jeg kom ind i kantinen, kunne jeg mærke, at
     noget var anderledes, end det plejer. Der var en
     underlig atmosfære i lokalet\dots Da jeg gik hen mod
     køledisken, kunne jeg se, at der var en sammenstimlen
     af mennesker henne ved varmtvandsbeholderen. Jeg hørte 
     højrøstede, ophidsede stemmer, men jeg kunne ikke se, 
     hvad der skete.

\scene Interviewer nikker forstående --- og henvender sig så til
Bent:

\says{Interviewer} Ja, Bent, du kom jo til katastrofestedet straks efter.
     Fortæl os, hvad du så!

\says{Bent}
     Ja, jeg har jo været kantinevagt her i et par år.
     Dagene plejer at forløbe ganske stille og roligt her
     på DIKU, men denne torsdag viste sig at blive ganske
     anderledes\dots Jeg var den første kantinevagt, som kom
     til katastrofestedet, så jeg var naturligvis helt klar
     over min pligt. Jeg måtte gøre alt, hvad jeg kunne for
     at komme mine medstuderende til undsætning. Straks da
     jeg så mine ulykkelige medstuderende stå ved
     varmtvandsbeholderen, vidste jeg, at der var brug for en
     mand med et køligt overblik.    

\says{Interviewer}[til Bent]
     Du forstod hurtigt, hvad der var i vejen?

\says{Bent}
     Ja, jeg fandt hurtigt ud af, at kantinens
     varmtvandsbeholder var brudt sammen. Der stod mine
     stakkels medstuderende altså en tidlig morgen uden at
     kunne få kaffe! De anede ikke, hvad de skulle gøre.
     Jeg var godt klar over, at det var en yderst katastrofal
     situation, vi var havnet i!

\says{Interviewer}[henvendt til Pia]
     Og Pia, vil du fortælle, hvad du så?

\says{Pia}
     Ja, jeg så simpelthen mine medstuderende stå omkring
     den sammenbrudte varmtvandsbeholder. De bageste stod og
     rakte deres tomme kaffekopper frem, men de forreste
     {\em havde}\/ opgivet håbet om at få kaffe. De vidste godt, at
     der ikke var noget at gøre.

\says{Interviewer}[til publikum, og derpå til Bent]
     Sekunderne tikkede afsted, men der kom alligevel en
     løsning på problemet, inden forelæsningen begyndte.
     Utroligt nok kom der hjælp udefra, så at sige, ikke
     sandt, Bent?

\says{Bent} Jo, vi fik jo hjælp af en fysikstuderende\dots

\says{Interviewer}[henvender sig til Steen]
     Ja, den fysikstuderende det var jo dig, Steen. Vil du
     fortælle os, hvordan du kom til at spille en vigtig
     rolle i denne kritiske situation?

\says{Steen}
     Ja, jeg læser fysik som hovedfag og datalogi ved siden
     af, så normalt færdes jeg mest på HCø. Denne torsdag
     morgen havde kombinationen af mit stramme budget og
     HCø-kantinens høje priser dog drevet mig over til
     DIKUs kantine. Jeg vidste ikke, at denne dag skulle
     blive skelsættende\dots For et par år siden tog jeg
     frivilligt et fysikkursus i varmelære. Jeg troede
     ærligt talt ikke, at jeg nogensinde skulle få brug for
     den viden; men denne skæbnesvangre morgen gjorde mig
     meget glad for, at jeg havde taget netop {\em det}\/ kursus\dots

\says{Interviewer}
     Vil du forklare os, hvordan det var, du hjalp, helt
     konkret?

\says{Steen}
     Ja, som sagt har jeg haft varmelære, så jeg foreslog
     simpelthen, at vi kunne tage en gryde og varme noget
     vand op i den på komfuret. Så kunne vi bruge det
     kogende vand til at lave kaffe.

\says{Interviewer}
     Ja, det må jo siges at være en genial løsning på
     det katastrofale problem, ikk' Bent?

\says{Bent}
     Jo, jeg var virkelig glad over, at der kom en
     løsning --- og så fra en så uventet side endda!

\says{Interviewer}
     Ja, Pia, du og dine medstuderende fik lavet kaffe
     alligevel\dots 

\says{Pia}
     Ja, det var utroligt at se, hvordan alle hjalp hinanden
     i denne alvorlige situation; der var et virkeligt fint
     teamwork. Nogle sørgede for at varme vand op --- under
     vejledning af den fysikstuderende, mens andre afmålte
     kaffemængden i tragten, og andre igen vaskede
     kaffekopper op og gjorde dem klar. Det gamle ordsprog
     passer virkelig: Det er i nøden, man skal kende sine
     venner!

\says{Interviewer}[nikker og holder mikrofonen frem mod Bent]

\says{Bent}     
     Det, der overraskede mig mest ved hele denne
     katastrofesituation, var faktisk, hvor roligt alle de
     implicerede tog det hele. Kun en enkelt gang blev der
     optræk til panik blandt et par studerende, men de blev
     beroliget og talt til fornuft af nogle studerende med
     psykologi som bifag.

\says{Interviewer}{henvendt til publikum}
     Ja, heldigvis endte det godt denne gang --- takket være
     en gruppe studerendes hurtige indsats. Jeg lover Jer at
     vende tilbage med en øjenvidneberetning, næste gang en
     katastrofe indtræffer her på jeres institut!

\scene

Lyset slukkes, og alle forlader scenen.

\end{sketch}

(Hvis der skal være mere tekst, må skuespillerne improvisere
sig frem til det under prøverne.)

\end{document}
