\documentclass[a4paper,11pt]{article}
\usepackage{revy}
\usepackage[utf8]{inputenc}

\title{Elefantsangen}

\author{Jacob Marquard}

\version{1.0} % HUSK AT AJOURFORE UDGAVE-NUMMERET!

\begin{document}

\maketitle

\begin{center}
{\bf Melodi: Jeg vil male himlen blå}
\end{center}
	
Kommentarer: Synges stille, der er brug for 2 store papelefanter (profil)
samt et par "levende rekvisitter", som jeg finder pa senere.

Der er en flip-over (eller en over-head) inde på scenen og en lille
flipovervender i kittel.



\begin{tabbing}
Ud fra den går mange kanter \hspace{1cm}	\=(2.) \kill 
Træet starter med en rod	\>(1. billede)\\ 
Roden sidder højest her\\ 
Ud fra den går mange kanter \>(2.)\\ 
Og på dem går elefanter \>(3.)\\ 
Her er alting - undtagen løkken\\
Jeg vil male grafen blå	\>(4.)\\
\end{tabbing}


\begin{tabbing}
Hvor der cykler elefanter	 \hspace{2cm}\=(6.) \kill
Et træ, det er en graf,\\
der ingen cykler har	\>(5.) \=(en hel cykel (udkrydset med 2 brædder)\\					\>\>bæres evt.\ henover scenen her.)\\
ingen parallelle kanter		\\
Hvor der cykler elefanter	\>(6.)\\
Disse kanter kan ha' vægte,\\
der er røde vægte på.	\>(7.)\\
\end{tabbing}


\begin{tabbing}
Træet har en samlet vægt,\\
præcis som dataloger,\\
træet vejer bare mere.\\
elefanterne er flere.		 \hspace{3cm}\=(2 elefanter pa scenen)\\
Hvad har de med træer at gøre\\
Intet --- så nu må de gå.	\>(exit elefanter)\\
\end{tabbing}

\begin{tabbing}
Første akt er nu forbi,\\
så vi holder straks lidt fri,\\
drengene de står og fjanter,\\
jeg har ingen elefanter,\\
Træet vokser ikke mere,\\
1.\ akt er sluttet nu.\\
\end{tabbing}

exit på instrumental vers.


\end{document}

