\documentclass[a4paper,11pt]{article}

\usepackage{revy}
\usepackage[utf8]{inputenc}
\usepackage[T1]{fontenc}
\usepackage[danish]{babel}


\revyname{DIKUrevy}
\revyyear{1987}
\version{1.0}
\eta{$n$ minutter}
\status{Færdig}

\title{MAC-SNAK}
\author{rag Pac \& gagRonald}

\begin{document}
\maketitle

\begin{roles}
\role{A}[] Person
\role{B}[] Person
\end{roles}


\begin{sketch}

\scene{To personer kommer gående. Den ene(B) standser den anden(A)}

\says{B} Goddag, vil de være så venlig at sige mig, hvad klokken er?

\says{A} Selvfølgelig. \act{Ser på sit ur, ryster det, og ser undrende på det}
\underline{Mac}værdigt? Det \underline{går} vist ikke.

\says{B} Det \underline{går} vel an at sige mig, hvad klokken er?

\says{A} Joh, men uret, det er noget \underline{Mac}værk.

\says{B} Ja, sådan et gammelt urværk, det er ikke til at regne med.

\says{A} Ahh, er de Mactematiker?

\says{B} Nej, Macgister.

\says{A} Bare man laver noget, for ``arbeit \underline{Mac}'t frei'' som de nok
ved.

\says{B} Har du forresten set \underline{Mac}Cloud?

\says{A} Næh...

\says{B} Nå men, det er jo også et \underline{tåget} \underline{program}.

\says{A} Jeg står faktisk og bliver sulten. \act{Tager et regnbuefarvet æble
  frem med en stor, ulækker orm.}

\says{B} \act{Ser forfærdet på ormen} Den er jo \underline{enorm}.  Skal vi ikke
hellere tage ind på \underline{Mac}donald?

\says{A} Nej, vi kan gå op i kantinen, der er noget \underline{Mac}coroni.

\says{B} Hellere \underline{Mac}krel. Er der noget smør?

\says{A} Nej, men der er \underline{Mac}garine, \act{melodisk} og det varer
sikkert længe før det gamle \underline{Mac}garine er blødt. \act{Opløftet}
Bagefter tager vi nogle \underline{Mac}kroner.

\says{B}[Begejstret] og \underline{Mac}cipan.

\says{A}[Hensat] Hvad er der ellers ... \underline{muslig}! (muesli)

\says{B} Døde mus? Hvor \underline{Mac}abert! Hvordan er de gået til?

\says{A} Jo, de blev skudt med en \underline{mus}kedonner af en
\underline{mus}selman.

\says{B} En \underline{mus}selman, hvad er det?

\says{A} Det er en \underline{mus}lim.  I øvrigt gav han musene til
Naturhistorisk \underline{Mus}eum.

\says{B} Godt! ``Ordnung \underline{mus} sein'', og derovre er jo også
\underline{mus}estille.

\says{A} Her er der heldigvis \underline{mus}ik.

\says{B} De mangler nu hende \underline{Mac}donna, hun er ellers virklig
god... i hvert fald flot, med masser af \underline{Mac}up og
\underline{Mac}scara.

\says{A} Jeg kan bedre lide Jazz, f.eks. den der Mig og
\underline{Mac}keduddi. Du skulle komme med ind på kontoret, så kan vi nyde et
glas \underline{Mac}deira. Den er \underline{double} \underline{Mac}
\underline{plus} \underline{god}.

\says{B} \underline{Mac}deira? Er det ikke en ø?

\says{A} Jo, den ligger vist nede ved det varme \underline{Mac}lorca.

\says{B} \act{Ser op med en projektør} Apropos varme så er det dog frygteligt,
som solen skinner ind af \underline{vinduet}. Er der ikke et
\underline{rullegardin} ?

\says{A} Der har været en \underline{Mac}kise, men den blev taget af en
\underline{Mac}oman.

\says{B} Det er også alle de der Macxisters skyld med deres
\underline{Mac}ximer.

\says{B} Næh, det er Macterialisterne, de klæber til \underline{Mac}ten som
\underline{Mac}neter.

\says{A} De \underline{Mac}ccokister kan så \underline{stå} bag dørene med deres
forpulede \underline{Mac}sturbation.

\says{B} Og de \underline{Mac}kerer sig \underline{Mac}simalt gennem
\underline{Mac}asinerne.

\says{A} Som burde have været smidt i Macskraldespanden forlængst... eller sågar
Mackuleret.

\says{B} Nej, nu går det ikke længere. Hvad ER klokken?

\says{A} Den er mange, og som man siger ``\underline{Mac}tid er
\underline{Mac}penge''. See you \underline{Mac}later
\underline{Mac}allegator. \act{haster ud af scenen}

\says{B} Hvornår var det nu, den natbus gik \act{går ud}
\end{sketch}
\end{document}
