\documentclass[a4paper,11pt]{article}

\usepackage{revy}
\usepackage[utf8]{inputenc}
\usepackage[T1]{fontenc}
\usepackage[danish]{babel}


\revyname{DIKUrevy}
\revyyear{1987}
\version{1.0}
\eta{$n$ minutter}
\status{Færdig}

\title{Diverse brokker til sammenhæng mellem numrene}
\author{KM}

\begin{document}
\maketitle

\begin{description}
\item[Indledning $\rightarrow$ Banansang]\hfill\\
Bestyreren fortæller lidt mere om, hvor hårdt vi har det og at vi er så kede af,
at der er mange af vores lærere, der rejser -- ja, det vil sige, det er nu ikke
altid vi er kede af det.

\item[Banansang $\rightarrow$ ££]\hfill\\
Bestyreren kommer ind igen og fortæller noget mere om vores nye hus, (``Vivi,
gider du lige komme med tegningerne over bygningen.'') fx har vi fået et vældig
praktisk £system til alle lokalerne -- der er ikke nummereret efter dørene, men
efter vinduerne, så man ikke får problemer, hvis man pludselig får lyst til at
bygge et stort lokale om til 2--3 kontorer. (``Vivi, tegningerne.'') ``Nu skal I
bare se:'' Ruth kommer ind med en planche over bygningen. Bestyreren forklarer
£systemet og slutter af med: ``og det er som sagt meget anvendeligt.'' og går
ud.

Nu kommer du to spillere end og ser planchen.

\begin{sketch}
\says{A} Sikke en sjov een.

\says{B} Tror du det er nogen, der spiller D\&D?

\says{A} Tjahh -- det ligner ikke helt -- der er også små numre på.

\says{B} Numre? Så er det måske sænke slagskibe?
\end{sketch}

\item[Sænke slagskibe $\rightarrow$ Omstillingen]\hfill\\
En af spillerne rammer omstillingen, men der sker ingenting. Dyb undren. Lyset
flyttes nu over på omstillingen, hvor omstillingsdamen enten sidder eller kommer
ind. Der står en stor raket (ca en halv meter høj) med næsen ned i
bordet. Omstillingsdamen ser på raketten med et let misbilligende blik og smider
den så i skraldespanden.

\item[Omstillingen $\rightarrow$ mødesang]\hfill\\
JC kommer ind i omstillingen og spørger: ``Det der møde, jeg skulle til, ved du,
hvor det bliver holdt?''

OD: ``Nej, men jeg skal også til frokost nu, den er allerede halv tolv.''

OD ud. JC står tilbage på scenen.

\item[Mødesang $\rightarrow$ Fejl og mangler]\hfill\\
Efter sangen går JC ud af scenen og samtidig kommer bestyreren ind.

JC: ``Hej Stig, skal du også til møde i
førstedelsplanudvalgslokalerevisionsudvalget?''

Best: ``Nej, jeg skal bare til møde med arkitekten.''

\item[Fejl \& mangler $\rightarrow$ Tapper-pavillion]\hfill\\
Den ene af sangerne er også sekretær i fejl og mangler og bliver tilbage på
scenen.

\item[Tapper-pavillion $\rightarrow$ Rabalderstræde]\hfill\\
Efter sangen kommer bestyreren ind igen og siger farvel og tak til revygruppen
fordi de ville udlåne noget af deres aften og grje til DIKU's åbent
hus. Derefter præsenterer han musikerne og teknikken (herunder Vilmar). Der skal
nu være posterete en person blandt publikum, der med høj og tydelig røst kan
råbe: ``Rabalderstræde'', hvorefter musikerne straks går igang med at spille
dette nummer.

\item[Dat-0 $\rightarrow$ Amiga]\hfill\\
Begge sangere mødes på scenen: ``Hej det er længe siden.'' ``Hvad går du rundt
og laver?'' ``Jeg læser stadig inde på DIKU'' ``Nå, hvordan er det så gået dig
på dat-0?''

\item[Amiga $\rightarrow$ Brd. Lisp]\hfill\\
``Så jeg har det faktisk helt fint uden DIKU, og der er jo også masser af jobs
at få, bare man kender lidt til det. en af mine venner er fx ansata i
konsulentfirmat 5*Kurt. lige her rudt om hjørnet.''
\end{description}


\end{document}
