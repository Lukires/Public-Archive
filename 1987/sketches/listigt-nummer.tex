\documentclass[a4paper,11pt]{article}

\usepackage{revy}
\usepackage[utf8]{inputenc}
\usepackage[T1]{fontenc}
\usepackage[danish]{babel}


\revyname{DIKUrevy}
\revyyear{1987}
\version{1.0}
\eta{$n$ minutter}
\status{Færdig}

\title{Listigt nummer}
\author{Står ikke}

\begin{document}
\maketitle

\begin{roles}
\role{F}[Står ikke] Forelæser
\role{H}[Står ikke] Head, et listehovede (blåt)
\role{q}[Står ikke] Hjælpevariable (blå)
\role{p}[Står ikke] Program (gult)
\role{E1}[Står ikke] Dynamisk element (rød)
\role{E2}[Står ikke] Dynamisk element (rød)
\role{E3}[Står ikke] Dynamisk element (rød)
\role{E4}[Står ikke] Dynamisk element (rød)
\role{new}[Står ikke] (hvid)
\role{gc}[Står ikke] garbage collector (sort)
\end{roles}

\begin{props}
\prop{Overhead projektor}[]
\prop{Et sæt paranteser, ca. 30 cm høje}[]
\prop{Ring bind til noter og transparenter}[]
\prop{Rengøringsvogn med kost og affaldssæk}[]
\prop{4 hatte, f. ex. Bowler hatte}[]
\end{props}


\begin{sketch}

\says{F} Goddag og velkommen. Sidste år lovede vi en forelæsning om kerner, men
i år har vi byttet forelæsninger, så det nu er programmel. Jeg håber ikke I har
forberedt jer forgæves, men I plejer jo slet ikke at forberede jer, så det gør
vel ikke noget.

mnet i dag er listebehandling.  Ved sidste forelæsning så vi på en stak, som var
implementeret ved hjælp af en tabel, og vi så at der opstod problemer, hvis
tabellen ikke var stor nok til at rumme stakken. I dag skal vi gennemgå en
metode som lapper lidt på detet problem uden dog at løse det principielt. Og
hermed er vi så kommet til hægterne.

Lad os starte med et simpelt eksempel: den tomme liste. Den tomme liste udmærker
sig ved ingen elementer at indeholde, men for at vide hvor vi har den, må vi
have et listehovede til at pege på den. Må jeg få et listehovede ind!

\scene{programmet henter et listehovede, som lister ind}

\scene{listehovedet peger ud mod publikum}

\says{F} Vi ser programmet hente et listehovede, og for at vise at listen er
tom, skal hovedet sættes til NIL. I øjeblikket peger det på NILs Andersen.
Bemærk at hovedet er helt blåt; det er fordi det er en statisk variabel.

\scene{programmet stikker hovedets højre hånd i dets lomme}

\says{F} Vi er nu klar til at indsætte et element. Det gøres naturligvis ved at
sætte hovedet til at pege på elementet. Men først skal vi oprette det ny
element. Elementerne tages fra en pulje efterhånden som behov opstår. Må jeg få
en pulje ind!

\scene{ind kommer en pulje bestående af 4 dynamisk elementer}

\says{F} Man siger at elementerne oprettes dynamisk. Det er derfor at de er
røde. Indtil de bliver oprettet befinder de sig i puljen og kaldes fri
elementer. Vi ser her, hvordan de går frit omkring.

Oprettelse af et element sker ved at kalde standard proceduren new som tager et
frit element fra puljen og sætter en pointer (dans jagthund) til at pege på
det. I vores eksempel er det listehovedet, som skal pege på elementet.

Vi ser her hvordan programmet sætter parentesen om hovedet og kalder.

\scene{hovedet sætter parenteserne om hovedet og kalder:``new''}

\scene{new kommer ind, tager et frit element E1, som begynder at græde. new
  tager hovedets arm op af lommen og sætter den til at pege på E1.}

\scene{exit new}

\says{F} Vi ser at det gik helt fint at prøve at indsætte et nyt element.

\scene{programmet sætter parenteser om hovedet og kalder new. new kommer ind
  næsten som før, tager fat i E2, som har hånden i lommen. E2 begynder at
  græde. new sætter head til at pege på E2, hvorved E1 tabes på gulvet,
  bogstaveligt.}

\scene{exit new}

\says{F} Hovsa det var ikke godt. Hvem gjorde det?

\scene{head og E2 peger på hinanden og siger i munden på hindanen: ``det var
  ham''}

\says{F} Vi har nu opdaget den cykliske liste, og som ved alle væsentlige
videnskabelige opdagelser skete det ved et uheld. Den cykliske liste er en meget
vigtig struktur, som vi skal vende tilbage til ved en senere lejlighed, men den
kan nu ikke løse dette problem. Her bliver vi nødt til at rette fejlen i
programmet

Vi indfører hjælpevariablen q.

\scene{E2's højre hånd stoppes i lommen}

\says{F} Vi ser nu hvorledes det går for sig. Jeg vil gerne pointere at
hjælpevariablen q betragtes som statisk selvom den er lokal, og derfor også er
blå.

\scene{programmet går ud og henter q. q spørger head hvem han peger på, og peger
  selv på den samme}

\scene{programmet sætter parenteser om head og kalder: ``new''. Unter new, som
  tager E3 og sætter head til at pege på E3. Prorammet spørger head, hvem han
  peger på, og tage dennes, dvs E3's hånd. Programmet spørger så q, hvem han
  peger på og sætter E3 til at pege på denne (E2). Programmet går ud med q og
  kommer tilbage.}

\says{F} Ja, det var jo meget bedre. Vi tager det lige en gang til så alle er
med.

\scene{det hele gentages, denne gange med E4 i stedet for E3}

\says{F} Sådan ja. Men det var jo en stak som var vores egentlige problem og
foreløbig har vi \underline{skubbet} det mere \underline{poppede} foran os.  Nu
skal vi til at udtage elementerne igen. Læg mærke til vordan de sisdte her
bliver de første. Det er en karakteristisk stakegenskab. Vi udtager topelementet
-- det som hovedet peger på -- ved følgende algoritme.

\scene{algoritmen vises på overhead}

\scene{på overhead: head:= head $\uparrow$ .next}

\scene{programmet skal prikke til head og antyde at head skal pege på det som E4
  peger på. Head skal spørge E4 hvad denne peger på, og pege på dette, som er
  E3. E4 tabes på gulvet, bogstaveligt.}

\says{F} Ja, her ser vi det igen. Et element er blevet tabt på gulvet. Hvis man
vil bruge elemnetet igen, skal man huske at pege på det før det hægtes ud af
listen; man kan sågar have en liste med elementer der ikke bruges og også puljen
kan laves på denne måde. Men i dette tilfælde skal vi ikke umiddelbart bruge
elementet til noget, så vi lader det ligge.

Vi forestiller os nu at programmet kører videre, og at det igen ønsker at føje
et element til stakken. Vi ser hvorledes dette går for sig, nu da puljen af fri
elementer er tom.

\scene{programmet henter q. q spørger hvad head peger på, og peger på det
  same. programmet sætter parenteser om head og kalder``new''. Enter new, som
  undersøger puljen og ser at den er tom, og kalder garbage kollektoren. Enter
  garbage kollektor med rengøringsvogn}

\says{F} New har nu konstateret at puljen er tom, og har kaldt garbage
kollektoren for at få eventuelle tabte elementer tilbage i puljen. Garbage
kollektoren vil nu indlede sin første fase: mærkningsfasen, hvori de aktive
elementer mærkes. Garbage kollektoren starter med at mærke de statiske variable
og alle de dynamiske som de peger på, og igen de som disse peger på og så videre
og så videre, indtil vi har nået afslutningen -- af mærkningsfasen.

\scene{garbage kollektoren påbegynder mærkningen af head, q og kompagni}

\says{F} Det kan \underline{gå} at benytte denne simple mærkningsalgoritme fordi
vi ingen \underline{cykler} har i strukturen, der som vi ser er en ren
\underline{kæde}. Garbage kollektoren har nu gennmført mærkningsfasen og går
over til komprimeringsfasen.

\scene{garbagekollektoren tager en stor kost og tjatter til E4, der sendes over
  i puljen. Garbage kollektoren rykker E3 og derpå E2 hen mod head og justerer
  hænderne. Derefter får E1 samme behandling som E4, og også guitristen forsøges
  indsamlet}

\says{F} Nej, hov.

\scene{programmet tager sig til hovedet i panik. garbage kollektoren standser og
  protesterer}

\scene{den peger på guitaristens hovede som er uden hat}

\says{F} Ja, godt nok er dette element ikke mærket, men garbagekollektion
omfatter kun programmets egne variable. Garbagekollektoren har nu udført sin
mission, og vi ser at new, som skulle oprette et element, fortsætter
ufortrødent.

\scene{garbage kollektoren giver new et håndtryk og forlader scenen. new tager
  den nylige frie E4 og sætter head til at pege på det. exit new. programmet
  spørger head hvad denne peger på, og tager dettes, dvs E4's hånd. Programmet
  spørger q, hvad denne peger på, og sætter E4 til at pege på dette,
  E3. programmet følger q ud.}

\says{F} Her vil jeg slutte gennemgangen af de hægtede stakke. Klokken er også
blevet mange. Næste gang skal vi se på hægtede multilister i
netværksdatabaser. Tak for i dag.

\scene{programmet kommer igen ind og få alle ud af scenen}

\end{sketch}
\end{document}
