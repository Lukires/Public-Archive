\documentclass[a4paper,11pt]{article}

\usepackage{revy}
\usepackage[utf8]{inputenc}
\usepackage[T1]{fontenc}
\usepackage[danish]{babel}


\revyname{DIKUrevy}
\revyyear{1987}
\version{1.0}
\eta{$n$ minutter}
\status{Færdig}

\title{Omstillingssketchen}
\author{HVO}

\begin{document}
\maketitle

\begin{roles}
\role{OD}[Står ikke] Omstillingsdame
\role{O1}[Står ikke] Opkalder
\role{O2}[Står ikke] Opkalder
\role{O3}[Står ikke] Opkalder
\role{O4}[Står ikke] Opkalder
\role{B}[Står ikke] Bestyrer
\role{S}[Står ikke] Sekretær
\end{roles}

\begin{props}
\prop{Skranke}[]
\prop{Skrivemaskine}[]
\prop{Reoler}[]
\prop{Mapper}[]
\prop{Dameblade}[]
\prop{Telefon/omstillingsbord}[]
\end{props}


\begin{sketch}

\scene{Omstilling, dvs en slags skranke, en skrivemaskine (med \underline{meget}
støv på -- brug talkum), evt diverse reoler, mapper og dameblade og naturligvis
-- en telefon/omstillingsbord. Od sidder bag skranken og laver absolut intet ud
ver at drikke kaffe, file negle, læse dameblade....

Der er fra starten ike andre personer på scenen.}

\says{OD} \act{Kigger på uret, trykker på en knap på omstillingsbordet}

Ja, godmorgen allesammen, så er klokken 9.30, og vi byder velkommen til ``De
ringer -- vi stiller om''. Det er idag lørdag d.13. juni, og publikummer med
tallet 1024 i enden kan ringe på 01396466 og blive stillet om. Og vi
... \act{afbrydes af telefon}

\scene{Telefonen ringer to gange}

\says{OD} \act{Med sukkersød stemme} Godmorgen, De ringer vi stiller om!

\says{O1}[stemme over højtaleranlæg] Øh..godmorgen, jeg vil gerne tale med
EDB-afdelingen...

\says{OD} EDB-afdelingen, javel \act{ser på uret}. Ja klokken er jo ikke så
mange, så jeg tror ikke de er kommet endnu. Men jeg prøver lige at stille dig
derned \act{trykker på nogle knapper, man hører en dut-tone over
  højtaleren. Mange gange}

\says{OD}  \act{trykker på en knap igen} Hallo!

\says{O1} Hallo?

\says{OD} Der er desværre ikke nogen i EDB-afdelingen endnu, er der nogen
besked?

\says{O1} Ja, sig at de får repareret køleanlægget om ca. 14 dage.

\says{OD}  \act{Mumler} mumle mumle om ca. 14 dage, ja. Det bliver de sikker
glade for, det er jo meget hurtigere end sidst det var i stykker.

\says{O1} Nå, men det her er reparationen af det der var i vejen sidst. Hvis det
det er i stykker igen varer det mindst 2 måneder.

\says{OD} Mumle mumle .. 2 måneder. Hvorfor det?

\says{O1} Det er svært at skaffe isterninger om sommeren.

\says{O1} Farvel.

\says{OD} Farvel \act{røret på}

\scene{Bestyreren kommer ind på scenen. OD sidder og filer negle}

\says{B} Godmargen Skat. Noget post?

\says{OD} Godmorgen. Nej jeg har ikke fået sorteret den endnu, jeg har ikke haft
tid. Nu skal jeg se om der er noget til dig \act{Tager to breve op, studerer dem
  nøje} Næh, kun post til fagrådet.

\says{B} Nåh ok. Jeg har lige været ude p åDtH, så jeg smutter ned og tager mig
et brusebad.

\says{OD} Skal jeg komme og skrubbe dig på ryggen?

\says{B} Ih ja -- gider du?

\says{OD} Ja, man skal jo holde sin Stig ren.

\scene{I mellem tiden er der komet endnu en sekretær ind på scenen. Hun kan
  foretage sig noget af det samme som OD. File negle, læse dameblade, drikke
  kaffe osv. OD smider de to breve i papirkurven. Musikken starter så småt til
  Omstillings-blues}

\end{sketch}

\begin{song}
\scene{Omstillings-blues.  Melodi: Står ikke}

\sings{Står ikke}
Søster -- de ringer på dig
søster -- ja de har øje for mig
oh søster -- vi venter på et lille klim'
de sexed'VIPér ved vel næppe
hvad vi er værd
søster -- de sku' vide hvod'n vi er
oh søster vi er jo deres sekretær

VIP'er
på os I burde sætte pris
for der mangler sku' da charme her
men tro ikke
at vi ikke ønsker mer'
så søde, super smukke VIP'er
du står der og glor
oh VIP'er -- vi kan mere end du tror
oh VIP'er -- vi prøver bare igen og igen.

VIP'er ku' du holde ud
hvis nu vi alle sagde stop
og forestil dig
hvordan du sku' ku' holde dig op
så søde, super smukke VIP'er
vi ved at du tror
oh VIP'er -- din magt er meget stor
oh VIP'er -- vi klare os ganske uden jer
oh VIP'er -- nu gir' vi sku ikke mer'
\end{song}

\begin{sketch}

\scene{Efter Omstillings-blues}

\scene{O2 ind som en studerende, der skal fotokopiere}

\says{O2} MÅ jeg låne tælleren, jeg skal kopiere en masse?

\says{OD} Ja, ja husk nu at skrive rigtigt ned \act{udleverer tælleren,
  telefonen ringer}

\says{OD} Godmorgen, De ringer vi stiller om, De ønsker?

\says{O3}[en kvindestemme] Jeg vil gerne tale med professoren.

\says{OD} Ja så gerne, tænker du på nogen særlig eller bare en af dem?

\says{=£}[henført] Ham den flotte!!!!

\says{OD}[lettere mystificeret] Den flotte? Øh jeg er vist ikke helt med. Kunne
du beskrive ham lidt nærmere?

\says{O3} Ja ham den nydelige, charmerende, atletiske ...

\says{OD} Åh.. du mener Per Brinch Hansen. Han er her ikkke her mere \act{hånden
  over røret} Heldigvis!

\says{O3} Er han der ikke mere? Den pæne mand, hvor er det
synd. Farvel. \act{rør på}

\says{OD} Farvel, farvel. \act{rør på}

\scene{O2 kommer ind med tæller og blok}

\says{O2} Jeg skal betale med 28 kroner \act{VIFTER med en 100 kr seddel}

\says{OD} Har du ikke mindre?

\says{O2} Nej.

\says{OD} Altså det er for galt, at I kommer her og kopierer, og jeg skal tage
mig af altsamen, når jeg også skal passe omstillingen og sortere post og skrive
breve \act{slår ud med hånden mod skriemaskinen, som støver synligt} Og hvis du
vidste, hvor travlt jeg i virkeligheden har, men jeg har også sagt til
bestyrelsen, at det kan ikke blive ved med at gå og jeg kan virkelig ikke
overkomme det alt-sammen vel? \act{Under denne svada hives en pengekasse frem}

\scene{OD veksler 100 kr seddelen med lethed}

\says{O2} Tak \act{går ud}

\scene{Telefonen ringer}

\says{OD} Goddag, De ringer vi stiller om! Hvad kan vi gøre for Dem?

\says{O4} Goddag, jeg vil gerne stilles om til emnegruppen på 1. sal i
nordfløjen.

\says{OD} Ja det kan sandelig ikke lade sig gøre, der er ikke telefon i
emnegrupperne, det ville bare overbelaste omstillingen med alle de studerendes
private samtaler og jeg ved ikke hvad. Farvel! \act{Rør på med et SLAM!!!}
Hrumpf! \act{Ser på uret} Ah -- halv tolv! Frokost. \act{Ud af scene}

\end{sketch}
\end{document}
