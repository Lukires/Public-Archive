\documentclass{article}
\usepackage[utf8]{inputenc}
\usepackage{revy}

\version{1.1}

\begin{document}

\title{Fangerne på Instituttet}

\author{Theo Engell-Nielsen}

\maketitle

\begin {roles}

\role{Stemme} tjaeh, hvem det nu er\ldots
\role{Vært} "Thomas Mygindput"
\role{Studievejleder} "Camilla DOStrup"
\role{Viismanden i tårnet} Det kunne måske være i skikkelse af en
Peter Naur lignende person?
\role{3 kombatanter} 2 datalogistuderende og 1 fysikstuderende (Steen).
Alle 3 er iført noget sport-outfit ligesom Fangerne på Fortet.
\role{En studenterbedømmer}Iført noget kedeligt gammelt tøj.

\role{Rekvisitter} En dør og et bord, 4 nøgler, "harddisk", 2 små
skruetrækkerne, en stor hammer, tøj.

\end{roles}

\begin{sketch}

\says{Stemme}[mens tæppet går op] Godaften. Vi blænder nu op for endnu en
  omgang ''Fangerne på Instituttet'', hvor Thomoas Mygindput og Camilla
  Sachs DOSstrup leder ugens kombatanter gennem diverse prøvelser. Sidste
  gang udspillede rædslerne sig med et hold fra Humaniora og vi husker med
  gru, hvor frygteligt den endte, da de skulle afprøve deres hjemmelavede
  flyvemaskine\dots Men nu er vi kommet til Datalogisk Institut.

\says{Thomas} Velkommen her til Fangerne på Instituttet her på Fort Boolean-ard. Og
Camilla, hvem har vi så på øvelsesholdet idag?

\says{Camilla} Jo - først har vi øvelsesholdets stærke mand, Prick
Dicker, med masser af muskler og dertilsvarende næsten ingen hjerne, som
til daglig er stærk mand i EDB-kælderen på DIKU. Han er derfor holdets
anfører - I mangel af bedre.... Som nr. 2 har vi holdets eneste pige
Miranda Bird Wadler, som også er datalogistuderende. Som nr. 3 har vi
Steen, han er med som fyld, for han er jo fysik-studerende til daglig.

\says{Thomas} Lad os se at komme igang, I har fra nu 16 minutter til at
finde de tre nøgler og jeres clue. Dem skal I bruge til at komme
igennem førstedels- og andendelsporten.  Derefter skal I finde et
password, for at komme udenom arbejdsløshedskøen.  Men det kommer vi til
senere.

\says{Camilla} OK! Det går jo fint! Er I parate?

\says{Alle kombatanter} [vilde af begejstring] Jahhh!!!

\says{Camilla} Må vi så høre kampråbet!

\says{Alle kombatanter} Nul for (alle minus en) og (alle minus en) for nul!!!

\says{Thomas} Det første der skal ske, er at vi sender øvelsesholdets
anfører op til viismanden i tårnet, for måske at få fat i den første
nøgle: Førstedelsnøglen!

\scene (Anføreren løber ivrigt med meget høje knæløftninger op til
viismanden i tårnet. Spot på viismand. Thomas går videre til næste post
imens.)

\says {Viismand} [Rigtig irriterende langsomt] Goddag - Vi er parate og klar
over, hvad det betyder for {\em vores\/} hold at få denne nøgle? {\em
(Holder en nøgle frem) \/} Vi skal gætte følgende gåde: Hvad er det, som
under {\em normale former} indeholder informationer og hvis attributter er en
høj pH-værdi?

\scene (Anføreren tænker i lang tid, mens han gnubber panden)

\says{Anfører} En bog? Et par sokker? \ldots {\em (Pause\ldots
Råber) \/} Selvfølgelig, hvordan kunne jeg være så DÅM? En
data{\em base}\/!

\says{Viismand} Det er korrekt. Værsågod her er {\em Vores\/} nøgle og
{\em Vores\/} clue! {\em (rækker ham en nøgle og et papskilt)} 

\scene (Anføreren løber triumferende ned til holdet igen)

\says{Camilla} Hvor var det godt gjort! Nu skal I høre: Vi har kun 8 minutter
tilbage. Vi skal videre. On y va!

\scene (Camilla klasker Prick Dicker bagi og alle løber over til Thomas,
som nu står foran en dør...)

\says{Thomas} Ja det gik jo fint med den første nøgle. Jeg kan kun love jer, at
den næste nøgle, andendelsnøglen, bliver meget sværere at få fat i.
Herinde bag døren finder vi et - ja nærmest {\em NP-komplet umuligt\/} -
problem, som kræver en skarpsindig og lynhurtig hjerne. Hvad siger
anføreren?

\says{Anfører} Jeg tror det skal være Miranda!

\says{Miranda} Åh nej, jeg gider ikke sådan noget...
\says{Thomas} [Hurtigt] Nu ikke så doven. Værsågod - her er en blyant.
Den får du brug for\ldots

\scene (Miranda går gennem døren....Bag den er et bord, og på den ligger en
opgave, der er skrevet på et stykke papir: Håndkør Ackermann(5,8). Hun
går igang.... Opgaven kan sættes op på en overhead.. Tiden skal også
kunne ses/høres??? Nogen tid passerer... Tiden er ved at løbe ud...)

\says{Camilla} Det går helt godt! Nu må I støtte hende derinde. Det er
ikke nemt! Jeg synes, hun så lidt DisMat ud... Kom! vi må støtte hende
lidt, kom nu!

\says{Anfører + Steen} Kom nu Miranda, clockcyklerne er ved at løbe ud!

\says{Miranda} Nu må I ikke være så strikse. Svaret er en gogoplex
minus 3-tals-logaritmen til en gogol.

{\em (KLING!! Miranda griber nøglen og papskiltet, der ligger ved siden af
  på bordet og kommer ud)}

\says{Camilla} Hvad skete der? Du nåede det, men hvorfor tog det så lang
tid? Kan vi få en partiel evaluering af forløbet?

\says{Miranda} Jeg tror, jeg fik for meget curry igår, så det tog lidt
længere tid! Men jeg fik jo 2.delsnøglen!

\says{Camilla} Det gik så fint! Nu skal I høre allesammen: vi har kun 4
minutter tilbage. Vi skal videre. On y va!

\scene (Camilla klasker Miranda bagi og alle løber over til Thomas)

\says{Thomas} Ja det gik jo fint med denne nøgle. Jeg kan kun love jer, at
den sidste nøgle, {\em Søge-nøglen\/}, bliver meget sværere at få fat i.
Herinde bag døren finder vi et problem, som kræver en lynhurtig tunge.
Hvad siger anføreren til det, hvem skal ind?

\says{Anfører} Det bliver jo desværre Steen, sådan er reglerne jo!

\says{Thomas} Ja sådan er det, vi må håbe det bedste. {\em (He he!)}

\scene (Steen går ind.. Bag døren sidder en studenter-bedømmer, bag et bord)

\says{Camilla} Nu skal I høre. D\'et Steen skal gøre, er at få hævet
karakteren på en andendelsrapport med 6. Fra et 5-tal til et 11-tal.
Rapporten handler om 3-dimensionel fourier-transformering af modeller
beskrevet med bezier-kurver og 42-grads polynomier i 3 dimensioner,
beskrivende 4 dimensionelle transformationer til generering af talesyntese!
Der er blot \'et stort problem. Steen ved ikke hvad den handler om...
Og husk I har kun 2 minuttter tilbage\ldots

\says{Bedømmer} [gnækkende] Tjæh.. Rapporten var jo ikke særlig god, og
terningerne var imod dig. Hvad har du tænkt dig, du gøre ved det, din
lille\ldots?

\scene (Bedømmeren dingler med nøglen, rundt i dens snor. Steen åbner
rapporten og begynder at læse\ldots)

\says{Steen} Jeg mener, \ldots

\says{Camilla} Tiden er ved at løbe ud, I må få ham ud

\says{Anfører og Miranda} Ud Steen!!!!

\says{Steen} Jamen - Jeg er jo lige igang med at\ldots og det ligner lidt noget
med varmelegemer.

\says{Anfører og Miranda} Ud Steen!!

\scene (Steen kommer ikke ud og sidder så fanget!!)

\says{Camilla} [Råber gennem døren] Det gik så fint, Steen! Men du sidder
fast. Og hvad kan man lære af det? Nu sidder du fanget og I fik ikke fat i
Søge-nøglen til andendelsporten!

\says{Thomas} Det ser ud til, at I skal løse endnu en opgave.
{\em (he he!)\/}Bag denne dør disekerede man i gamle dage studerende for
øjnene af mennesker. Sådan er det ikke idag, {\em desværre!\/} Vi har
brug for jer begge til denne opgave, da den er temmelig svær, for her skal I
sætte noget, der ikke virker, sammen igen.

\scene (Bag døren er en opgave, der går ud på at skulle redde en
harddisk, som er smadret. Til det formål ligger en hammer og en
skruetrækker. På en slide vises opgaven: {\em Reparer harddisken\/}. De
starter begge med at behandle den forsigtigt med de små
skruetrækkere. Efter et stykke tid kigger de på hinanden og Prick Dicker
tager en stor hammer frem, og smækker på harddisken.)

\says{Miranda} Jeg tror den fungerer nu, lad os prøve, at sætte
controleren til. {\em (Rigtig ironisk!!):\/} Det er godt vi har haft Dat1M,
hvor vi rigtig fik lov til at komme i kontakt med hardwaren!!

\scene (De sætter et stik i harddisken. Harddisken kommer med et ''BLIP!!!'')

\says{Anfører} [Råber] Ja! Den virker, hvor godt... Tag nøglen og
kodeordet!

\scene (Miranda tager nøglen og papskiltet (der ligger inde i harddisken))

\says{Miranda} Hvad med tiden? Lad os komme ud!

\scene (De løber begge ud.)

\says{Camilla} Jah! Hvor gik det godt! Var det svært? {\em (Klasker dem
begge to bagi!)\/}

\says{Thomas} [Afbryder, irriteret] Ja, nu er I kommet igennem første- og
andendelsporten. Det I nu skal, er at slippe forbi arbejdsløsheden.
Desværre {\em (for jer)\/} bliver Steen sluppet fri, for sådan er
reglerne! Det eneste I nu mangler er at finde passwordet\dots

\says{Camilla} I skal skynde jer, for I har nu 2 minutter til at finde det
hemmelige kodeord. On y va! {\em (Camilla klasker Miranda {\em og\/} Prick
  Dicker bagi!. De løber over til password-udfordringen.)\/} Det er nu at
I virkelig skal arbejde sammen. I har fundet følgende 3 clues\dots

\scene (Camilla tager et stor skilt, som hun sætter ude foran tæppet. På
skiltet står:

\begin{verbatim}
  To      Lede     Skov
\end{verbatim}
\scene Kombatanterne kigger op på dem\ldots En trommevirvel lyder, imens
tæppet langsomt trækkes for.)

\says{Thomas} Der er nu kun 10 skunder tilbage. Har I et gæt??

\says{Steen} Ligner det ikke noget med nogle varmelegemer?? {\em (Tæppet
  trækkes hurtigt helt for. Stortrommen lyder.)}

\says{Miranda og anfører} {\em (gætter febrilsk bag tæppet)} En
sendestation til en mobiltelefon\dots Økologisk datalogi\dots Nej, det er
s'gu da {\em binære sø\-ge\-træ\-er\/}!! ARGHHH!.

\scene {Lys ud\ldots}
\end{sketch}

\end{document}


