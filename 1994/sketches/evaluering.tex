\documentclass{article}
\usepackage[utf8]{inputenc}
\usepackage{revy}
\version{0.9}

\title{Revyloering}

\author{Harry}

\begin{document}

\maketitle

\sl\noindent Jeg forestiller mig at dette nummer kan bruges
til aller sidst (før ekstranumrene), hvor vores
backstage-crew skal præsenteres, og alle de medvirkende
kommer frem og bukker.

\begin{sketch}
\says{Speaker} Studienævnet har besluttet at gennemføre evaluering af
         revyen på DIKU. Evalueringen består af to dele: Første
         del går ud på at vurdere en række spørgsmål omkring revyen
         udtrykt ved udråbene: "JAAAAAAH", "joooooh" og "hmmmmmm". 
         Det gælder generelt for spørgsmålene, at "JAAAAAAH" er 
         virkelig godt og "hmmmmmm" er dårligt. Anden del er en 
         diskussion med jeres revygruppemedlem til sommerfesten, 
         der skal munde ud i et skriftligt resumé til studienævnet.

         Spørgsmålene falder i tre dele:
         \begin{description}
       \item [Generel del] Den generelle del indeholder spørgsmål 
              om revyen som helhed. 1.dels-administrationen laver 
              statistik på denne del af evalueringen. --- De ved 
              det bare ikke endnu!

       \item [Specifik del] Den specifikke del indeholder
              spørgsmål rettet direkte mod publikum og
              de medvirkende.

       \item [Evt.\ tidligere revymedlemmer] Denne del åbner
              mulighed for at publikum kan evaluere 
              tidligere revygruppemedlemmer, som måtte
              være til stedet her i aften.

         \end{description}
           

         Der vil blive afsat en halv sommerfest til diskussion
         af evalueringen. Diskussionen giver jer mulighed for
         direkte at give jeres revygruppemedlem tørt på, og
         diskutere problemer med ham/hende. Denne diskussion
         er overordentlig vigtig for revyen. Både fordi
         kritikken vil blive taget personligt, og fordi vi
         gerne vil vide, hvem vi skal hænge ud til næste års
         DIKU-revy. Diskussionen bør koncentrere sig om at
         finde de væsentligste platheder ved revyen, og
         forsøge at komme med nogle endnu mere platte
         forbedringer.

         Efter diskussionen skal I skrive et resume og
         videregive det til studienævnet. Til det formål skal
         I nedsætte en skrivegruppe på mindst 20 mand. I bør
         allerede før diskussionen afklare indbyrdes, hvem
         der skal sidde i skrivegruppen. Det er ikke sikkert,
         at I har samme opfattelse af revyen, som vi har.
         Ud over resumeet, kan I give en kort vurdering af
         evalueringen som helhed: lykkedes det at gennemføre
         en konstruktiv diskussion? Var evalueringen en
         god ide? Hvad kan gøres bedre?

         Nu til første del af evalueringen . . .

\noindent{\large\bf  Generel del:}
         \begin{description}

   \item[1.1]Har der været en god sammenhæng i revyen?
   \item[1.2]Har kommunikationen mellem musikken og de 
             optrædende fungeret godt?
  \item[1.3] Har de praktiske ting fungeret bare nogenlunde?

  \item[2.1] Har de medvirkende en god humoristisk sans?
  \item[2.2] Er revyformen god og motiverende?
  \item[2.3] Er de medvirkende selv motiverede og engagerede?
  \item[2.4] Interesserer revygruppen sig for jeres 
             faglige problemer?
  \item[2.5] Er klimaet til revyen godt publikum og de
             medvirkende imellem?
  \item[2.6] Hvordan har revyen været i forhold til de
             foregående revyer under hensyntagen til 
             publikums indflydelse og det nye scenetæppe?
         \end{description}
\noindent{\large\bf   Specifik del:}
         \begin{description}
  \item[3.1] Knud H. har slået på trommer. Ku' I li' det?
  \item[3.2] Jacob L. har spillet guitar. Ku' I høre det?
  \item[3.3] \ldots
  \end{description}
  
(O.s.v. På denne måde præsenteres musikken)
\begin{description}
  \item[3.4] Steen B. har mixet lyden. Ku' I høre sangerne?
  \item[3.5] \ldots
  \end{description}
  
(O.s.v. På denne måde præsenteres vores backstage-crew)
         
         Endelig har vi haft et meget veloplagt publikum:

\begin{description}
  \item[4.1] Har I været motiverede til at overvære revyen?
  \item[4.2] Hvor stor en del af revyerne har i overværet?
  \end{description}
  
         Det var så den xxx'te revy. Her kommer de medvirkende:

(alle de medvirkende tumler ind på scenen)
\begin{description}
     
   \item[5.1] Skal de ha' en hånd?
   \end{description}
   
(under forudsætning af et stort bifald bukker revymedlemmerne 
for publikum, og forlader derefter scenen).

  
\end{sketch}

\end{document}

