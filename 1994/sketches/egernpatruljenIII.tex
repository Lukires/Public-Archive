\documentclass{article}
\usepackage[utf8]{inputenc}
\usepackage{revy}
\title{Kantinevagten $\sqrt2$ / TV-kantinen}
\version{0.96}
\author{Jørgen Elgaard Larsen og tvind}
\begin{document}
\maketitle
\vskip-1.5\bigskipamount
\begin{roles}
\role{kantinevagt, kok, fortæller, egern}\strut
\end{roles}
\vskip-\bigskipamount
\begin{props}
  \prop{lommelygte} til at se rundt med
  \prop{gryde} med låg
  \prop{div.\  madvarer}
  \prop{bord}
  \prop{fejebakke /m  dejskraber}
  \prop{Lille spade} til at slå egernet i hovedet med
\end{props}
\vskip-2\bigskipamount
\begin{sketch}
\scene
Der er mørkt --- evt.\ lys på en filmplakat med indgangen eller lignende.

\says{Fortæller} Fra instruktøren af ``Batchman'', ``Endian Jones'',
``Xorcisten'', ``Dirty Carry'', ``Last Action Table'', ``Semantic Park'',
``Speciale Dybet'', ``Kantinevagten'', ``Store Wars'', ``Evaluelle'',
``Non-Determinator III'', ``The Lord of the Strings'' og ``Aluen --- den 8.
processor'' kommer nu storspændingsgyserthrilleren "Kantinevagten 2" ---
fyldt med hæsblæsende xsession's, farlige processorer og {\em hurtige}
laserprintere. Kommer snart til et auditorium nær dig. 

\scene Spot på kantinevagt, der går langsomt og mistænksomt over scenen.
Lyser søgende omkring sig, også mod publikum.
                
\says{Fortæller}Han var jaget. Russerne kendte ham og der var krus og
kopper overalt. Han kom fra HCØ --- fra ask'en til ilden.  {\em (Raslelarm
  fra musikken. Kantievagten lyser deroverimod og nærmer sig)\/}
Automaterne var tomme og hans konto hang i en tynd tråd. Det var hans anden
dag på jobbet og superbrugerne havde sporet ham. Han svedte angstens\ldots

\scene Spot bliver hængende på kokken. Fortæller fader ud. TV-Køkken-melodi
i baggrunden.

\says{Kok} Goddag, og velkommen til TV-kantinen. Idag vil jeg vise hvordan
jeg laver DIKU-gryde. Til dette skal man --- som rettens navn antyder ---
bruge en gryde. For at spare tid, har jeg snydt lidt og lånt en gryde i
forvejen. {\em (Stiller gryde på bordet.)\/}

\says{Kok} Først udvælges en passende grundsubstans.  Der kan her bruges
alt lige fra det traditionelle margarine til det mere eksotiske mango
chutney --- eller bare noget af det der udefinerbare fedtede stads bag
komfurerne. Til doseringen kan evt.\ bruges en fejebakke eller en
dejskraber.

\scene Kok skraber sidstnævnte ingrediens op fra gulvet eller lign.

\says{Kok}For at få den bedste fond kan en sjat af karton'en i skabet
bruges som fortynder. Man kan sikkert også bruge noget andet.  {\em (tømmer
  en Blå ud i gryden.)\/} Der skal selvfølgelig også bruges krydderier til
at give retten en behagelig modståelig kulør, og lidt emulgator til at
forbinde den kulinariske oplevelse. {\em (tømmer lidt af hvert i
  gryden.)\/} Nu kommer vi så til det spændende højdepunkt. "Kød" er jo ---
specielt blandt de mandlige studerende --- en eftertragtet vare her på
DIKU. Friskt kød er jo især en sjældenhed, så man må som regel klare sig
med, hvad der nu er for hånden (høhø). Det man finder, er jo som regel lidt
overgemt --- i hvert fald over 18 år\ldots

\scene Der er et egern i grypden. Kokken tager det, og mimer et angreb.

\says{Egern}[Egernet går mod struben på kokken] Squeeek!

\says{Kok}[Upåvirket] Og husk endelig {\em (griber egernet om halsen eller
  i ørerne)}, at kødet skal hakkes grundigt (banker egernet med en spade og
stopper det i gryden), før det stoppes i gryden og koges tilpas længe.
{\em (Holder låget nede. Bankelyde fra gryden.)} Med denne enkle opskrift
siger jeg tak for denne gang. Opskriften kan {\em ikke\/} findes på text-revy
side 42.

\scene Tæppe
\end{sketch}
\end{document}

